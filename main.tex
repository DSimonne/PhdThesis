%%%%%%%%%%%%%%%%%%%%%%%%%%%%%%%%%%%%%%%%% Preamble %%%%%%%%%%%%%%%%%%%%%%%%%%%%%%%%%%%%%%%%%

\documentclass[11pt]{report}

\usepackage{titling} % makes following commands available
\title{Catalytic properties at the nanoscale probed by surface X-ray diffraction and coherent diffraction imaging}
\author{David Simonne}
\date{\today}

%%%%%%%%%%%%%%%%%%%%%%%%%%%%%%%%%%%%%%%%%% Colours %%%%%%%%%%%%%%%%%%%%%%%%%%%%%%%%%%%%%%%%%%

\usepackage{xcolor}

\definecolor{ArgonColor}{HTML}{008FD5}
\definecolor{AmmoniaColor}{HTML}{DD65B0}
\definecolor{OxygenColor}{HTML}{FC4F30}
\definecolor{NitrogenColor}{HTML}{810F7C}
\definecolor{NitrousOxideColor}{HTML}{8B8B8B}
\definecolor{NitrogenOxideColor}{HTML}{6D904F}
\definecolor{WaterColor}{HTML}{E5AE38}

\definecolor{Ostwald}{HTML}{800080}
\definecolor{Haber}{HTML}{158466}

\definecolor{LightBlue}{RGB}{30,75,180}
\definecolor{LUBlue}{RGB}{0,0,128}
\definecolor{LUGrey}{RGB}{77,76,68}
\definecolor{Prune}{RGB}{99,0,60}
\definecolor{Important}{HTML}{980000}

\definecolor{DarkFern}{HTML}{407428}
\definecolor{DarkCharcoal}{HTML}{4D4944}
\definecolor{DarkBlue}{HTML}{23373B}
\definecolor{DarkOrange}{HTML}{EA7500}
\definecolor{LightOrange}{HTML}{F9D6B3}
\definecolor{Blue}{HTML}{13406a}

\colorlet{lightblue}{blue!60}
\colorlet{lightorange}{orange!60}
\colorlet{lightgreen}{green!60}
\colorlet{lightred}{red!60}
\colorlet{lightgray}{gray!60}
\colorlet{lightbrown}{brown!60}
\colorlet{lightpink}{pink!60}
\colorlet{lightviolet}{violet!60}

\colorlet{shadecolor}{gray!40}

%%%%%%%%%%%%%%%%%%%%%%%% Page layout %%%%%%%%%%%%%%%%%%%%%%%%

\usepackage[
    a4paper,
    width=155mm,
    headheight=110pt,
    top=25mm,
    bottom=25mm
]{geometry}

\usepackage{etoolbox} % Apply different styles
\usepackage{fancyhdr}
\pagestyle{fancy}
\renewcommand{\chaptermark}[1]{\markboth{#1}{}}
\renewcommand{\sectionmark}[1]{\markright{\thesection\ #1}}

% Define fancy style for thesis
\fancypagestyle{main}{
%    \fancyhead[L]{\chaptername \, \thechapter:  \small{\textsf{\leftmark}}}
    \fancyhead[L]{\chaptername \, \thechapter}
    \fancyhead[R]{\small{\textsf{\rightmark}}}
    \fancyfoot[R]{\small{\thepage}}
    \fancyfoot[L]{\small{\theauthor}}
    \fancyfoot[C]{}
    \renewcommand{\headrulewidth}{0.4pt}%
    \renewcommand{\footrulewidth}{0.4pt}%
}
\appto\mainmatter{\pagestyle{main}}

% Define plain style for table of contents and acknowledgements
\fancypagestyle{plain}{
    \fancyhf{}
    \fancyfoot[C]{\thepage}
    \renewcommand{\headrulewidth}{0pt}%
    \renewcommand{\footrulewidth}{0.4pt}%
}
\appto\frontmatter{\pagestyle{plain}}

\usepackage{sectsty} % To colour sections, raise an error on underbar
\chapterfont{\color{Prune}}
\sectionfont{\color{LUGrey}}
\subsectionfont{\color{LUGrey}}
\subsubsectionfont{\color{LUGrey}}

%%%%%%%%%%%%%%%%%%%%%%%%%% Citation %%%%%%%%%%%%%%%%%%%%%%%%%%

\usepackage[utf8]{inputenc} % Use accents in source code, load before biblatex
\usepackage[T1]{fontenc} % Show accents in output

\usepackage[style=authoryear,sorting=nyt,mincitenames=1, maxcitenames=2]{biblatex}
%    nty Sort by name, title, year.
%    nyt Sort by name, year, title.
%    nyvt Sort by name, year, volume, title.
%    anyt Sort by alphabetic label, name, year, title.
%    anyvt Sort by alphabetic label, name, year, volume, title.
%    ynt Sort by year, name, title.
%    ydnt Sort by year (descending), name, title.
%    none Do not sort at all. All entries are processed in citation order.
\addbibresource{/home/david/Documents/PhD/PhdThesis/references.bib}

%%%%%%%%%%%%%%%%%%%%%%% Other packages %%%%%%%%%%%%%%%%%%%%%%%
\usepackage{lipsum} % Latin blabla
\usepackage{lmodern} % to avoid errors in font size
\usepackage[british]{babel}
\usepackage{setspace} % Space between lines
\usepackage{mathtools}
\usepackage{tikz} % Drawing
\usetikzlibrary{shapes.geometric, arrows, positioning, calc}
\usepackage{colortbl} % Colour the lines in the tables
% \usepackage{minted} % Format code
\usepackage[version=4]{mhchem}
\usepackage{animate} % for animations in 3D
\usepackage{pdfpages}
\pdfminorversion=6 % to avoid errors when including pdf
\usepackage{epigraph} % for quote

\usepackage{textcomp} % To avoid warnings in gensymb
\usepackage{gensymb} % for degree symbol

\usepackage{siunitx} % Units
\DeclareSIUnit\angstrom{\text{Å}}
\DeclareSIUnit\barn{b}
\DeclareSIUnit\photons{\text{photons}}
\DeclareSIUnit\bar{bar}
\usepackage{wasysym} % for diameter symbol

\usepackage{graphicx} % Insert images
\graphicspath{{/home/david/Documents/PhD/Figures/}}

\usepackage[hypcap=true,font={small,it},justification=centering]{caption}
\captionsetup{belowskip=2pt,aboveskip=2pt}

\usepackage{scrextend} % Forcer la 4eme  de couverture en page pair
\usepackage{mdframed} % Entourer un paragraphe
\usepackage{multirow} % Mettre un texte sur plusieurs rangées
\usepackage{multicol} % Mettre un texte sur plusieurs colonnes
\usepackage{array} % Create tables as arrays
\usepackage{csquotes}
\usepackage{booktabs} % for \midrule and such
\usepackage{lscape}
\usepackage{amsmath} % for maths
\usepackage{amssymb}

\usepackage[
    colorlinks = true,
    linkcolor = LightBlue,
    urlcolor  = LightBlue,
    citecolor = LightBlue,
    anchorcolor = LightBlue,
]{hyperref} % load last to avoid errors

%%%%%%%%%%%%%%%%%%%%%% Commands %%%%%%%%%%%%%%%%%%%%%

\newcommand{\argon}{\ce{Ar}}
%{\textcolor{ArgonColor}
\newcommand{\ammonia}{\ce{NH_3}}
%{\textcolor{AmmoniaColor}

\newcommand{\nitrousoxide}{\ce{N_2O}}
%{\textcolor{NitrousOxideColor}

\newcommand{\water}{\ce{H_2O}}
%{\textcolor{WaterColor}

\newcommand{\nitrogen}{\ce{N_2}}
%{\textcolor{NitrogenColor}

\newcommand{\nitricacid}{\ce{HNO_3}}
%{\textcolor{DarkOrange}

\newcommand{\nitricoxide}{\ce{NO}}
%{\textcolor{NitrogenOxideColor}

\newcommand{\nitrogendioxide}{\ce{NO_2}}

\newcommand{\carbondioxide}{\ce{CO_2}}

\newcommand{\dioxygen}{\ce{O_2}}
%{\textcolor{OxygenColor}

\newcommand{\ptthreeofour}{\ce{Pt_3O_4}}
%{\textcolor{Important}

\newcommand{\ammoniumnitrate}{\ce{NH_4NO_3}}
%{\textcolor{Blue}

\newcommand{\urea}{\ce{CO(NH_2)_2}}

\newcommand{\yes}{{\textcolor{green}{0}}}

\newcommand{\no}{{\textcolor{red}{x}}}
\newcommand{\ra}[1]{\renewcommand{\arraystretch}{#1}}

%%%%%%%%%%%%%%%%%%%%%% Document %%%%%%%%%%%%%%%%%%%%%%

\begin{document}

% Title page
    \begin{titlepage}

\newgeometry{left=6cm,bottom=1cm, top=1cm, right=1cm}

\tikz[remember picture,overlay] \node[opacity=1,inner sep=0pt] at (-13mm,-135mm){\includegraphics{logo/Bandeau_UPaS.pdf}};

% fonte sans empattement pour la page de titre
\fontfamily{fvs}\fontseries{m}\selectfont

%*****************************************************
%******** NUMÉRO D'ORDRE DE LA THÈSE À COMPLÉTER *****
%******** POUR LE SECOND DÉPOT                   *****
%*****************************************************

\color{white}

\begin{picture}(0,0)

\put(-110,-743){\rotatebox{90}{NNT: }}
\end{picture}

% \vspace{-12mm}
% \flushright \includegraphics[height=2.1cm]{logo/cea.jpg} \hspace{7mm}
% \includegraphics[height=2.1cm]{logo/SOLEIL.png}

%*****************************************************
%******************** TITRE **************************
%*****************************************************

\flushright
\vspace{10mm}
\color{Prune}
\fontfamily{cmss}\fontseries{m}\fontsize{22}{26}\selectfont
Catalytic properties at the nanoscale probed by surface x-ray diffraction and coherent diffraction imaging

\normalsize
\color{black}
\Large{\textit{Propriétés catalytiques à l'échelle nanométrique sondées par diffraction des rayons X de surface et imagerie de diffraction cohérente}}
%*****************************************************

\vspace{1.5cm}
\normalsize
\textbf{Thèse de doctorat de l'Université Paris-Saclay}

\vspace{15mm}

École doctorale n$^{\circ}$ 564, physique en Île-de-France (PIF)\\
\small Spécialité de doctorat: Physique\\
\footnotesize Graduate School: Physique\\
\footnotesize Unités de recherche: Synchrotron Soleil (Université Paris-Saclay)\\
\footnotesize et CEA Grenoble
% \footnotesize Référent: : COATI Alessandro\\
\vspace{15mm}

\textbf{Thèse présentée et soutenue à Paris-Saclay,\\ le 12/01/2024, par}\\
\bigskip
\Large {\color{Prune} \textbf{David SIMONNE}}

%************************************
\vspace{\fill} % ALIGNER LE TABLEAU EN BAS DE PAGE
%************************************

\bigskip
\flushleft
\scriptsize
\begin{tabular}{|p{7cm}l}
\arrayrulecolor{Prune}
{\footnotesize \textbf{Composition du jury}}\\
& \\
\textbf{Thomas Cornélius} & Rapporteur \\
Directeur de recherche, IM2NP, Marseille & \\
\textbf{Andreas Stierle} & Rapporteur \\
Professor, University of Hamburg / DESY, Hamburg & \\
\textbf{Gerardina Carbone} & Examinatrice \\
Researcher, Lund University, Lund & \\
\textbf{Virginie Chamard} & Examinatrice \\
Directrice de recherche, Institut Fresnel, Marseille & \\
\textbf{Sylvain Ravy} & Examinateur \\
Directeur de recherche, LPS, Orsay & \\

\end{tabular} 

\medskip
\begin{tabular}{|p{7cm}l}
\arrayrulecolor{Prune}
{\footnotesize \textbf{Direction de la thèse}}\\
& \\
\textbf{Alessandro Coati} & Directeur \\
Dr., Synchrotron SOLEIL & \\
\textbf{Andrea Resta} & Co-directeur \\
Dr., Synchrotron SOLEIL & \\
\textbf{Marie-Ingrid Richard} & Co-directrice \\
Directrice de recherche, CEA Grenoble & \\

\end{tabular} 

\end{titlepage}

% Optional Chapters
    %\chapter*{Dedication}
    %\chapter*{Declaration}
    % \chapter*{Acknowledgements}
    % I extend my deepest gratitude to Andrea for his unwavering support throughout the duration of this thesis.
You have contaminated me with your love of Linux and open source technology, and set a strong example in terms of commuting to work by bicycle, which I failed to follow in rainy and cold days.
Thank you very much for all the feedback that you gave me about this thesis, down to the details in the figure placement, fontsize, linewidth, and sentence length.
Pursuing your interest in the study of the ammonia oxidation has allowed me to explore a large subject.
This work has given me a missing sense of purpose, as it feels like my contribution is ever so slightly adding to the global endeavour in combating global warming and climate change.

I am immensely grateful for the shared beamtimes, the camaraderie during strenuous long shifts in summer or winter, skipping breakfast or dinner, compiling data while monitoring the sample heater.

Thank you Alessandro for being engaged in everything you do, at work and outside, for taking care of all the tedious work so that I could have the best possible experience, for inviting me to have meals, for the nice chats during commute, and for providing the SixS beamline with good Italian coffee.
I am most importantly grateful for all the crucial scientific feedback you have provided during this thesis, for accompanying me during beamtime, and for your attention to detail down to the many index of the many equations illustrating this thesis.

Thank you Marie-Ingrid for being so driven by your love of science that everybody around wants to participate, for arranging my stay in Grenoble and bringing all of the padawans to conferences, for always being in a good mood and ready to discuss new challenging experiments.
Thank you for the support regarding especially BCDI, a technique with a data analysis process quite difficult to decipher in the first months, and that always brings new questions and new experiment ideas.

Thank you to my three advisors for being good mentors, allowing me to teach, to participate in schools and numerous beamtimes, for having my best interest in mind at all times and for trusting me.
I have learned an incredible amount from each of you and particularly appreciated the beamtime at Diamond, during which the four of us collected many electrons, but also shared meaningful conversations around a pint of beer and delicious stew.

I am very grateful to the research staff of SixS, to Alina and Yves for helping me when I was lost in the reciprocal space, to Benjamin without which none of the experiments could be possible, and to Michèle for her numerous advice.
Thank you also Frédéric for spending a lot of time in helping us regarding BCDI and SXRD code development.
Thank you to all the young researchers at SOLEIL for creating a group so that Saclay did not feel too remote and secluded.
Thank you for the direction of SOLEIL who has supported my going to different conferences, abroad and in France, as well as for providing a good working environment.

I of course have a specific thought for the Grenoble legions, half of whom I don't know half as well as I should like, and half of whom I know from half my stay spent in beamtime.
Maxime, for introducing me to the world of political satire videos during ammonia oxidation cycles at SixS; Corentin, for the delightful culinary explorations in Grenoble, and for lending me the ID01 bicycle; Clément, for your constant positivity and our conversations about the promising future of PhD students; Nikita and Mattia, for the shared moments over a refreshing beer and the intense MSSBB games during Hercules; Ewen, for representing Bretagne alongside me; Edoardo, for the initiative of LAN parties, and the unforgettable Quake games; and Noor, for being an exceptional office (and coffee) mate in Grenoble.

I would also like to thank Steven for his great advice and for kind conversations, as well as Tobias and Joël for facilitating my stay in Grenoble for a few months during this thesis.
Thank you Vincent for letting me participate in the engaging ESRF tutorials, and Jérôme for starting the large work of bringing our BCDI community together in code development.
My heartfelt appreciation to Sarah and Stéphane for their warm hospitality in Marseille and their engaging discussions.

I feel very grateful towards Virginie and Christian who have followed me during the thesis by participating in the CSI, even available on weekends so that I would not miss registrations deadlines.

Thank you Andrea, Davide, Elisa and all the others from the university of Torino for making me want to pursue an academic path.
Thank you Davide again for hosting me in Grenoble after the ESRF user meeting because my train got cancelled, thank you Alessia and Giorgio for welcoming me.

Thank you to all the MaMaSELF members without which I would never have kept studying physics and material science, and thus missed on studying in many countries, discovering new cultures, and mindsets.
Thank you Michael for bringing me to the world of neutron diffraction at Munich, and for letting me participate in my first beamtime at the ILL with Ulrike, a master and PhD student that had to learn a lot fast, but sadly not fast enough to handle the beam polarisation well.

Thank you to all the professors from Rennes that transmitted me their love of physics, especially Pr. Guérin that permitted me to study one year in Japan, thank you to Pr. Iwai for accepting me as part of his group in Sendaï.

Finally, to my family, whose unwavering support has bolstered my every endeavour, to Lucas, whose shared passion for science has been a source of motivation, and to my parents, for their trust and encouragement in my global pursuits.
Thank you to all my friends that have seen me being projected in numerous states during those three years, and without which I would probably not have kept the same level of sanity.

I extend my thanks to the esteemed members of the jury for their invaluable feedback and constructive critique, shaping the course of this work.

Furthermore, I acknowledge the invaluable support and resources provided by synchrotron SOLEIL, the CEA, and the ERC Carine, without which this research endeavour would not have been feasible.

Lastly, I express my gratitude to all those who have, in various capacities, contributed to the completion of this thesis.

%************************************
\vspace{\fill} % ALIGNER EN BAS DE PAGE
%************************************

\newpage\thispagestyle{empty}\null\newpage

\frontmatter
{\hypersetup{linkcolor=black}
    \tableofcontents
    \listoffigures
    \listoftables
}

\mainmatter
% Introduction
    \chapter{Introduction}
    The use of catalysts has several advantages such as faster, selective, and more energy-efficient chemical reactions, directed towards producing higher amounts of the desired product while reducing unwanted byproducts.
Over the years, scientists have developed specialised catalysts for various real-world applications, today \qty{90}{\percent} of chemical processes involve catalysts in at least one of their steps \parencite{Weiner1998, DeVries2012}.
Notable advancements in catalysis have led to the production of biodegradable plastics, novel pharmaceuticals, and eco-friendly fuels and fertilisers \parencite{Fechete2012}.

\begin{figure}[!htb]
    \centering
    \includegraphics[width=\textwidth]{/home/david/Documents/PhD/Figures/ammonia/ParisNO2English.png}
    \caption{
        $NO_2$ levels in Paris near the main traffic roads remained on average twice superior to the annual limit of \qty{40}{\ug \per \m^3} \parencite{AirParis} between 2012 and 2018 despite a global decrease of since 2012.
    }
    \label{fig:NO2Paris}
\end{figure}

Several new challenges have emerged in the field of catalysis related to improving efficiency, reducing environmental impact, and developing sustainable processes.
First, environmental challenges concern minimising and/or managing by-products, reducing contamination in effluents/wastewaters, and using sustainable sources of raw materials \parencite{Ludwig2017, Lange2021} and energy supplies.
Secondly, economical challenges imply using cheaper, readily available raw materials, increased productivity, and decreased lag-time between discovery to commercialisation \parencite{Keisuke2019, Gunay2021}.
For example, recent studies suggest that alternative, more economical catalysts, such as non-noble metals \parencite{Zhong2021, Ruan2022} and other derived metal-based compounds, need to be tested as possible substitutes for the most frequently used noble metals, which are very efficient but expensive.

Catalysis also has a role to play to combat pollution and create cleaner energy with for example the development of efficient water-splitting technologies \parencite{Ahmad2015}, and enhancing the use of biomass and other energy vectors such as ammonia \parencite{Fang2022}.
Finally, challenges also arise in automotive exhaust where catalysts participate in the reduction of the emissions of toxic gases and nanoparticles \parencite{WHOAirPollution, Heck2001, Gandhi2003}.
Some of the major air pollutants such as nitrogen oxides, (\ce{NO_x}), and particulate matter (PM) are emitted by road traffic (\qty{65}{\percent} of \ce{NO_x}, \qty{\approx 35}{\percent} of PM), mainly by diesel vehicles, and directly inhaled by nearby major city inhabitants.
To set a striking example, in Paris in 2018, 700 000 inhabitants were exposed to \nitrogendioxide concentrations exceeding the regulations (fig. \ref{fig:NO2Paris}), 60 000 inhabitants for PM$_{10}$, and all Parisians were concerned by exceeding the World Health Organisation (WHO) recommendations for PM$_{2.5}$ \parencite{AirParis}.
Air quality is the main environmental concern of Ile-de-France residents (\qty{65}{\percent} of total mentions) ahead of climate change (\qty{63}{\percent}) and food (\qty{38}{\percent} - \cite{AirParis}).

\section{The oxidation of Ammonia}

\subsection{The Haber-Bosch and Ostwald processes}

\epigraph{Today, about \qty{50}{\percent} of the world population relies on nitrogen-based fertilisers to produce the food necessary to their alimentation.}%{\textit{Nature Catalysis editorial, \cite*{NatureEditorial2018}.}}

The story of ammonia begins in the early $20^{th}$ century with the discovery in 1902 of the Ostwald process that permitted the synthesis of fertilisers from ammonia.
Wilhelm Ostwald later received the Nobel price in 1909 \textit{"in recognition of his work on catalysis and for his investigations into the fundamental principles governing chemical equilibria and rates of reaction"}.
Seven years later, in 1909, Fritz Haber designed a process for the synthesis of ammonia which was later improved by Carl Bosch.
It is today known as the Haber-Bosch process and is at the origin of the mass production of ammonia using metallic catalysts \parencite{Hosmer1917, Parsons1919}.
Fritz Haber received the Nobel prize in 1918 \textit{"for the synthesis of ammonia from its elements"} \parencite{Alexander1920} and Carl Bosch in 1931 \textit{“in recognition of his contributions to the invention and development of chemical high pressure methods”}.

\begin{figure}[!htb]
\centering
    \begin{tikzpicture}
        \node (image) [anchor=south west, inner sep=0pt] {\includegraphics[width=0.95\textwidth]{/home/david/Documents/PhD/Presentations/Slides/PhdSlides/Figures/worldindata/world-population-with-and-without-fertilizer.png}};
        \begin{scope}[x={(image.south east)}, y={(image.north west)}]
            \node [text width=4cm,align=right] at (0.18, 0.7) (OP) {\textcolor{Ostwald}{Ostwald} process\\(1902)\\ \textrightarrow Nobel prize\\(1909)};
            \draw [-latex, ultra thick, Ostwald] (0.10,0.75) to (0.10,0.12);
            \node [text width=4cm,align=right] at (0.19, 0.45) (HBP) {\textcolor{Haber}{Haber-Bosch}\\process (1908)\\ \textrightarrow Nobel prize\\(1918)};
            \draw [-latex, ultra thick, Haber] (0.13,0.4) to (0.13,0.12);
        \end{scope}
    \end{tikzpicture}
    \caption{
    Since the discovery of the Ostwald and Haber-Bosch process that allowed the mass production of nitrogen-based fertilisers, the world population increase has been relying on their production and use for agriculture.
    Today, about \qty{50}{\percent} of the worlds population relies on nitrogen-based fertiliser to produce the food necessary to their alimentation.
    Taken from \cite{WorldDataFertilizer}.
    }
    \label{fig:FertilizerWID}
\end{figure}

The oxidation of ammonia can be described by three equations that, depending on the stoechiometric ratio between \ce{NH_3} and \ce{O_2}, have different products.

\begin{align}
    \label{eq:AmmoniaOxidationNitrogen}
    4 \ammonia (g) + 3 \dioxygen (g) & \rightarrow 6 \water (g) + 2 \nitrogen (g) \\
    \label{eq:AmmoniaOxidationNitrousOxide}
    4 \ammonia (g) + 4 \dioxygen (g) & \rightarrow 6 \water (g) + 2 \nitrousoxide (g) \\
    \label{eq:AmmoniaOxidationNitricOxide}
    4 \ammonia (g) + 5 \dioxygen (g) & \rightarrow 6 \water (g) + 4 \nitricoxide (g)
\end{align}

The first equation (eq. \ref{eq:AmmoniaOxidationNitrogen}) yields nitrogen (\nitrogen), a naturally occurring gas that does not pollute the environment nor shows any toxic behaviour towards humans.
The second equation (eq. \ref{eq:AmmoniaOxidationNitrousOxide}) yields nitrous oxide (\nitrousoxide), a powerful greenhouse effect gas, and thus an often unwanted by-product.
The third equation (eq. \ref{eq:AmmoniaOxidationNitricOxide}) yields nitric oxide, also called nitrogen monoxide (\nitricoxide), which is the main desired product for the subsequent production of nitrogen-based fertilisers\textit{via} the synthesis of nitric acid with the Ostwald process.
The characteristics of the many different gases linked to oxidation of ammonia are recapitulated in tab. \ref{tab:NitrogenGases}

Known side reactions to the oxidation of ammonia are the recombination of nitric oxide with unreacted ammonia that leads to the production of water and nitrogen (eq. \ref{eq:SideReactions1}), and the thermal decomposition of nitric oxide (eq. \ref{eq:SideReactions2}), that both lower the total yield of the reaction when aiming at the production of nitric oxide.
It is also possible to observe the dissociation of ammonia resulting in the production of \ce{N_2} and \ce{H_2} (eq. \ref{eq:SideReactions3}).

\begin{align}
    \label{eq:SideReactions1}
    4 \ammonia (g) + 6 \nitricoxide (g) & \rightarrow 5 \nitrogen (g) + 6 \water (g)\\
    \label{eq:SideReactions2}
    2 \nitricoxide (g) & \rightarrow \nitrogen (g) + \dioxygen (g)\\
    \label{eq:SideReactions3}
    2 \ammonia (g) & \rightarrow \nitrogen (g) + 3 \ce{H_2} (g)
\end{align}

Reactions \ref{eq:AmmoniaOxidationNitrogen}, \ref{eq:AmmoniaOxidationNitrousOxide}, \ref{eq:AmmoniaOxidationNitricOxide} and \ref{eq:SideReactions1} are strongly exothermic, leading to a significant increase of the catalyst temperature during the reaction \parencite{Hatscher2008}.

The second (eq. \ref{eq:Stage2}) and third (eq. \ref{eq:Stage3}) stages of the Ostwald process stem from the production of \ce{NO} \textit{via} the first stage, \textit{i.e.} the oxidation of ammonia (eq. \ref{eq:AmmoniaOxidationNitricOxide}).

\begin{align}
    \label{eq:Stage2}
    2 \nitricoxide (g) + \dioxygen (g) & \rightarrow \nitrogendioxide (g) \\
    \label{eq:Stage3}
    3 \nitrogendioxide (g) + \water (l) & \rightarrow 2\nitricacid (aq) + \nitricoxide (g)
\end{align}

Nitrogen dioxide (\nitrogendioxide) is produced from \ce{NO} which then reacts with water to form nitric acid (\nitricacid), an important actor in multiple industrial process (tab. \ref{tab:NitrogenGases}), including fertilisers such as ammonium nitrate (eq. \ref{eq:AmmoniumNitrate}).

\begin{align}
    \label{eq:AmmoniumNitrate}
    \nitricacid + \ammonia & \rightarrow \ammoniumnitrate
\end{align}

\section{The importance of heterogeneous catalysis}\label{sec:AmoOxiHC}

\subsection{Industry conditions and catalysts}

Important selectivity towards the production of \ce{NO} must be achieved during the ammonia oxidation (\textit{i.e.} the first stage of the Ostwald process) when aiming at the production of fertilisers (eq. \ref{eq:AmmoniaOxidationNitrousOxide}).
However, at low temperature or in the absence of catalyst, the production of \ce{N_2} is favoured (eq. \ref{eq:AmmoniaOxidationNitrogen}), which is at the origin of the selective, heterogeneous catalytic oxidation of ammonia towards the production of nitrogen oxide.

Since the 1930s, the presence of rhodium (\qty{\approx 10}{\percent} Rh) in knitted Pt-Rh gauzes catalysts at favourable reaction conditions (e.g. \qty{900}{\degreeCelsius} - \qty{5}{\bar} - excess \dioxygen) allowed a \qty{98}{\percent} \ce{NO} yield to be achieved, \qty{4}{\percent} higher than for pure platinum, while also increasing the catalyst lifetime, and decreasing the material loss \parencite{Ostwald1908, Kaiser1909, Handforth1934, Heck1982}.

In order to understand the reaction mechanism occurring on the catalyst surface, and the role of Pt and Rh in the catalyst stability and selectivity, scientist have been studying the reaction by different methods since the beginning of last century.
A comprehensive review of the ammonia oxidation is given by Hatscher et al. \parencite*{Hatscher2008}, important literature findings from the last 100 years will be resumed below.

% roughening, oxides
The importance of high temperature for the selective production of \ce{NO} was demonstrated early by temperature dependant studies \parencite{Nutt1968, Pignet1974, Li1997}, while a restructuring of the catalyst, also called catalytic \textit{etching}, was put into evidence by the means of \textit{ex-situ} SEM imaging of industrial samples \parencite{McCabe1974, FlytzaniStephanopoulos1979, McCabe1986}.
This \textit{activation process} leads to an increase selectivity towards \ce{NO} while the gauzes undergo a transformation towards a roughened surface, composed of pits, facets, and large \textit{cauliflower} patterns (fig. \ref{fig:Gauzes}).

\begin{figure}[!htb]
    \centering
    \includegraphics[width=0.49\textwidth]{/home/david/Documents/PhD/Figures/sample/EtchedGauze.png}
    \includegraphics[width=0.49\textwidth]{/home/david/Documents/PhD/Figures/sample/ReconstructedGauze.png}
    \caption{
    SEM images of Pt-Rh reconstructed gauzes with cauliflower patterns after use in industry, taken from Bergene et al. \parencite*{Bergene1996}.
    The horizontal bar is \qty{0.1}{\mm} wide.
    }
    \label{fig:Gauzes}
\end{figure}

The deactivation of Pt-Rh industrial catalysts after long exposure times have been explained by the presence of rhodium oxides \parencite{Bergene1996}.
However, a deactivation process was also reported on pure Pt catalyst \parencite{Ostermaier1974}, linked to the presence of platinum oxides \parencite{Ostermaier1976}.

The role of volatile surface oxides in the roughening and etching of the catalyst surface was theorised by Wei et al. \parencite*{Wei1996} and confirmed experimentally by Nilsen et al. \parencite*{Nilsen2001}.
The transportation of Pt and Rh was found to be permitted by the presence of oxygen and surface oxides \parencite{Hannevold2005a}, decreasing with increasing Rh content and decreasing oxygen pressure, only for high temperature gradient, \textit{i.e.} \qtyrange{1400}{800}{\degreeCelsius} over \qty{100}{\mm}.
\textit{Ex-situ} characterisation of Pt and Pt-Rh industrial gauzes by x-ray powder diffraction and electron microscopy allowed the identification of defect sites at the origin of high temperature gradient areas during reaction.
The existence of such areas allows the restructuring process to occur nearby initial defects by the formation of \ce{PtO_2} and \ce{RhO_2} oxides, depositing metallic atoms on colder regions, and also leading to the loss of some of the precious metals constituting the catalyst \parencite{Hannevold2005}.
The replacement of the material loss constitutes the second biggest expense in the production of fertilisers after the production of ammonia \parencite{Hatscher2008}.

The progressive deactivation of the catalyst due to the ever increasing presence of \ce{Rh_2O_3} \parencite{McCabe1986} was refuted by Hannevold et al. \parencite*{Hannevold2005}, explaining that such oxide could only form during the cooling of the catalyst, which is a good example of the limitation of \textit{ex-situ} works.
A \ce{Pt_3O_4} catalyst used for the oxidation of ammonia was proven to be unstable at working temperature above \qty{690}{\degreeCelsius}, decomposing into a Pt phase after \qty{7}{\hour} of operation \parencite{Zakharchenko2001}.

Overall, the presence of rhodium in the catalysts limits the loss of platinum during operation, thus reducing the cost of industrial scale ammonia oxidation \parencite{Hatscher2008}.

\subsection{Reaction mechanism}

First studies performed at low working pressure but at different temperatures have supported a Langmuir-Hinshelwood mechanism during which both reactants are decomposed and adsorbed on top of the catalyst surface \parencite{Nutt1969, Pignet1974, Ostermaier1974, Pignet1975, Gland1978a}, the final pathway towards the production of nitrogen or nitrous oxide depending on the ratio between the \ce{O_2} and \ce{NH_3} partial pressures.

The importance of adsorbed atomic oxygen and \ce{OH} in the dissociation of \ce{NH_3} on both Pt (111) \parencite{Mieher1995} and Pt (100) surface was put into evidence \parencite{Bradley1995, Bradley1997, vandenBroek1999, Kim2000}.
High oxygen coverage on Pt (100) has shown to reduce the production of \ce{N_2}, mainly produced by the dissociation of \ce{NO}, favoured at lower temperatures over \ce{NO} desorption \parencite{Bradley1995}.
Asscher et al. \parencite*{Asscher1984} have reported the existence of a mechanism involving oxygen in the gas phase reacting with adsorbed NH species on Pt (111) for the production of \ce{NO}.
At UHV, a rotated hexagonal reconstruction on top of Pt (100) was found to impinge \ce{NH_3} dissociation at low temperature (\qty{-123}{\degreeCelsius}) \parencite{Bradley1997}, the ammonia oxidation stabilising the (1x1) phase \parencite{Rafti2007} by the formation of \ce{NH_x} intermediates between \qty{125}{\degreeCelsius} and \qty{350}{\degreeCelsius}.

% steps
The importance of atomic steps in the catalytic activity was first revealed by Gland et al. \parencite*{Gland1978, Gland1980}.
A more recent study of the oxidation of ammonia on several model catalysts (Pt(533), Pt(443), Pt(865), Pt(100), Pt foil) to investigate the structure selectivity by Yingfeng \parencite*{Yingfeng2008} linked the presence of steps and kinks with higher catalytic activity in the \qtyrange{1e-9}{1e-5}{\bar} range.
Moreover, the production of \ce{N_2} was reported to be promoted by lower temperatures and a reduced \ce{O_2}/\ce{NH_3} ratio in the incoming gas flow, while higher temperatures and an elevated \ce{O_2}/\ce{NH_3} ratio tend to result in a higher selectivity towards the formation of \ce{NO} \parencite{Zeng2009}.
No production of \ce{N_2O} was observed within the studied pressure range.
The importance of steps at ambient pressure was revealed to be most important when aiming at producing \ce{N_2} at low temperature, but could not be correlated to an increase in \ce{NO} production at high temperature in a comparative study of the Pt(111) and Pt(211) surfaces.
It was shown that the adsorbed \ce{NH_3} hopping rate is close to its desorption rate on Pt (111) terrace sites, making it unlikely to reach the steps where it may react rather than desorb from the catalyst surface \parencite{Borodin2021}.

% tap, kinetic studies
Rebrov et al. \parencite*{Rebrov2002} detailed the reaction kinetics and mechanism with a 13 step, temperature dependent model, the parameters of which have been refined with data collected from the reactant and product partial pressures evolution in a micro-reactor.
A wide range of conditions was explored, including ambient pressures of \ce{NH3} (\qtyrange{0.01}{0.12}{\bar}) and \ce{O_2} (\qtyrange{0.10}{0.88}{\bar}), in a large temperature range (\qtyrange{250}{400}{\degreeCelsius}).
A dual adsorption site mechanism was proposed, with a preference for hollow site for oxygen species, and for top or bridge sites for nitrogen species, while \ce{NO} species have been reported to exist on both top and bridge sites.

Pérez-Ramirez et al. \parencite*{PerezRamirez2004} have also attempted to study the kinetics of the reaction by directly analysing the selectivity of catalysts used in industry (\textit{i.e.} Pt-Rh and Pt gauzes) above \qty{700}{\degreeCelsius}, by the means of reactant gas pulses.
The heterogeneous catalysis reaction mechanism was found to be similar on both sample, a pre-exposition of the catalysts to oxygen facilitated the dissociation of \ce{NH_3}, without which the decomposition towards \ce{N_2} was not detected.
A high coverage of the catalyst by oxygen was linked to the formation of \ce{NO}, strongly bound oxygen-rich species favour the production of \ce{N_2}, whereas weakly bound were associated to \ce{NO} selectivity.
The importance of adsorbed oxygen species to prevent spontaneous \ce{NO} dissociation was confirmed.
In a second study \parencite{PerezRamirez2009}, they proved that increasing the \ce{O_2}/\ce{NH_3} ratio to 10 pushed \ce{NO} selectivity to almost \qty{100}{\percent}, \ce{N_2} and \ce{N2_O} production being both suppressed by favouring \ce{NO} desorption.

A similar study combining dynamical field theory (DFT) calculations and temporal analysis of products (TAP) by Baerns et al. \parencite*{Baerns2005} confirmed the importance of surface \ce{O} and \ce{OH} for the de-hydrogenation of \ce{NH_3} on the catalyst surface.
\ce{N_2O} could only be detected at ambient pressures, \ce{N_2} is favoured at lower temperature, whereas \ce{NO} is favoured at higher temperatures.
Importantly for the use of model catalysts, no difference in the temperature dependant production of \ce{N_2} and \ce{NO} between a Pt(533) single crystal, Pt foil, and knitted Pt gauzes could be observed.
Surface roughening of the catalyst at ambient pressure was linked to activation and selectivity change for a Pt foil.
Interestingly, the deactivation of Pt catalysts due to the adsorption of nitrogen species below \qty{115}{\degreeCelsius} was put into evidence, but with a reactivation above that temperature \parencite{Sobczyk2004}.

The first studies performed under working conditions brought forward two recurrent problems when carrying out low pressure studies.
First, the production of \ce{N_2O} was rarely discussed because undetected, van den Broek et al. \parencite*{vandenBroek1999} hypothesised that \ce{NO} was a precursor in the production of nitrous oxide.
\ce{N_2O} was then detected also by Pérez-Ramirez et al. \parencite*{PerezRamirez2004}, Baerns et al. \parencite*{Baerns2005} and Kondratenko et al. \parencite*{Kondratenko2007} which confirmed the precursor role of \ce{NO}, and that \ce{N_2} was produced by either dissociation of \ce{NO} or/and complete de-hydrogenation of \ce{NH_3}.

Secondly, the known roughening process occurring on the catalyst at working conditions could not be reproduced without long working times, or high pressures.
Kinetic studies on poly-crystalline Pt up to \qty{10}{\bar} but at temperatures below \qty{385}{\degreeCelsius} observed the roughening transition \parencite{Kraehnert2008}, which proved difficult to fit with develop kinetic models, attributed to local increase in temperature and surface area.

% The rate-limiting step in the ammonia decomposition has been identified to be the recombination of nitrogen atoms on Fe catalysts \parencite{Vilekar2012}

% DFT
\begin{figure}[!htb]
    \centering
    \includegraphics[width=\textwidth]{/home/david/Documents/PhD/Figures/ammonia/ReactionMechanism.png}
    \caption{
    Example of \ce{NH_3} stripping process by adsorbed \ce{O}, taken from Imbihl et al. \parencite*{Imbihl2007}.
    }
    \label{fig:ReactionMechanism}
\end{figure}

Novell-Leruth et al. \parencite*{NovellLeruth2005} confirmed the adsorption of \ce{NH_3} and \ce{NH_2} to occur respectively on top and bridge sites for both Pt (100) and Pt(111), but with a more favourable adsorption process on Pt (100).
Similar adsorption energies have been reported for \ce{NH} and \ce{N} that both adsorb on hollow sites \
Additional DFT studies of the reaction pathways and kinetics have confirmed a reaction mechanism following a step-by-step decomposition of ammonia on the catalyst surface, the stripping of adsorbed ammonia hydrogen atoms being facilitated by the presence of oxygen species \parencite{Offermans2006}.
A comparative study between the Pt (100), Pt (111) and Pt (211) (\textit{i.e.} stepped) surfaces did not reveal a strong structure sensitivity during the \ce{NH_3} stripping process \parencite{Offermans2007}.

Imbihl et al. \parencite*{Imbihl2007} have further improved the understanding of the production of \ce{N_2O}, which happens not only by the recombination of two adsorbed \ce{NO} species but also \textit{via} the reaction between adsorbed \ce{NO} and \ce{NH_x} species.

The first de-hydrogenation step was found to be the slowest, while the desorption of \ce{NO} is the rate limiting-step when aiming at the selective production of nitrous oxide \parencite{NovellLeruth2008}.
Interestingly, for the first time different oxygen species were reported to be responsible for the de-hydrogenation process depending on the surface structure, respectively \ce{O} for Pt(111) (fig. \ref{fig:ReactionMechanism}) and \ce{OH} for Pt(100).
A high energy barrier for \ce{NO} desorption and \ce{N_2O} formation explain the high temperature needed for their production in comparison with \ce{N_2}.

% conclude
Despite the large amount of work in the previous and current century, the mechanism of the oxidation of ammonia is still unclear, mostly due to the difficulties of reaching industrial conditions in \textit{operando} studies.
Indeed, the main drawback common to most of the literature studies is that they have either been performed \textit{ex-situ}, or at pressures and temperatures far from industrious conditions, even for some at ultra high vacuum.

This shows that the understanding of heterogeneous catalysis comes first from developing new methods working up to very high pressures and temperatures, especially when aiming at the study of a complex system with many competing reactions.
For example, Pottbacker et al. \parencite*{Pottbacker2022} have presented a new characterisation method to facilitate the study of the reaction kinetics at industrial conditions, by precisely measuring the temperature and compositional gradients present on the catalyst surface.

Replacing precious metals in industrial reactor can be thought as the natural step that will follow the comprehension of the reaction mechanism.
For example, the performance of new material has already been explored with \ce{V_2O_5} catalyst, reaching promising \ce{NO} yields at atmospheric pressure between \qtyrange{300}{650}{\degreeCelsius} \parencite{Ruan2022}.

%However, understanding the favourable conditions for the production of \ce{NO} can not be enough since the oxidation of ammonia can also serve at reducing toxic and pollutant nitrogen oxides, while preventing the slip of unreacted \ce{NH_3}, also harmful to the environment.

\section{Environmental impact}

\subsection{Greenhouse effect}

As illustrated in fig. \ref{fig:FertilizerWID}, nitrogen-based fertilisers have permitted an industrial development of agriculture.
Fertilisers can be either ammonium- or nitrate-based, when plants don't fully absorb all the nutrients, a series of microbe-mediated transformations occur which leads to the release of nitrogen back into the atmosphere, primarily as nitrogen gas (\ce{N_2}) and, to a lesser extent, as \ce{N_2O}.

For an equal amount of \ce{N_2O} and \ce{CO_2}, nitric oxide will trap 298 times more heat than the carbon dioxide over the next 100 years \parencite{MITCLIMATE}, responsible for \qty{6.2}{\percent} of the total U.S.A. greenhouse gases emissions in 2021 (fig. \ref{fig:PieGreenhouseNO2}).
Moreover, the production of the necessary ammonia which is in turn used for fertilisers often comes from natural gas, meaning that each ton of \ce{NH_3} produced is equivalent to the emission of 1.9 ton of \ce{CO_2} \parencite{Rafiqul2005, Chen2018}.

\begin{figure}[!htb]
    \centering
    \includegraphics[width=\textwidth]{/home/david/Documents/PhD/Presentations/Slides/PhdSlides/Figures/ammonia/NO2pie.pdf}
    \caption{
    Pie charts underlining the importance of different gas in the total US greenhouse gas emissions in 2021 (a) and the specific contribution of nitrogen-based fertilisers to the total \ce{N_2O} emissions in 2021.
    LULUCF means Land Use, Land-Use Change, and Forestry.
    Adapted from \cite{EPAGreenhouseGases}.
    }
    \label{fig:PieGreenhouseNO2}
\end{figure}

The presence of \ce{N_2O} in the atmosphere can be linked to development of agriculture towards a productivity model, helped by the means of nitrogen-based fertilisers.
Thus, the amount of nitric oxide, which has an atmospheric lifetime of 114 years, is expected to decrease in the future years in the northern hemisphere, but increase in the southern hemisphere following such transitions \parencite{Solomon2007, Davidson2009}.

In a review of the presence of \ce{N_2O} in the atmosphere linked to human activities, Pérez-Ramirez et al. \parencite*{PerezRamirez2003} have shown that the most important contribution to nitrous oxide in the atmosphere is from unused volumes of nutrients, but also from nitric acid manufacture.
Understanding and limiting the by-products of \ce{N_2O} by controlling the process selectivity is thus capital to limit the amount of nitric oxide released in the atmosphere.

\subsection{Pollution}

Nevertheless, it can also be interesting to be able to remove \ce{NH_3} from the atmosphere, a colourless gas with a pungent odour, that irritates the eyes, nose, throat, and respiratory tract if inhaled in small amounts due to its corrosive nature and is poisonous in large quantities.
Ammonia also pollutes and contributes to the eutrophication and acidification of terrestrial and aquatic ecosystems, and forms secondary particulate matter (PM2.5) when combined with other pollutants in the atmosphere (tab. \ref{tab:NitrogenGases}).
Moreover, nitrogen-based fertilisers and other human activities can lead to nitrogen runoff into water bodies, contributing to eutrophication (excessive growth of algae) and causing harm to aquatic ecosystems (tab. \ref{tab:NitrogenGases}).
Approximately half of the production of ammonia is lost to the environment \parencite{Erisman2007}.

To efficiently remove ammonia, the selectivity of the catalytic reaction must be tuned towards the production of \ce{N_2}, which is the only non pollutant and toxic gas.

Finally, the important effect of nitrogen oxides (\ce{NO_x}, \textit{i.e.} \ce{NO} and \ce{N_2O}) on the environment has brought forward the necessity to control their emissions, especially from the exhaust of diesel engines that are responsible for \qty{65}{\percent} of their emissions.
The selective catalytic reaction (SCR) using urea or ammonia as reductant (tab. \ref{tab:NitrogenGases}) has proven to be effective and to reach \qty{95}{\percent} efficiency \parencite{MitsubishiSCR}.
However, there can be a subsequent problem of unreacted ammonia \textit{slipping} from the reaction, which is also an important subject of study \parencite{Thermofischer}.

Recently, ammonia has been investigated as an energy vector for hydrogen fuel cells, which has reignited the interest in understanding the complex system drawn by the many simultaneous reactions \parencite{Afif2016, Georgina2021}.

Today, the ammonia oxidation is an essential catalytic reaction used in the production of nitrogen-based fertilisers and in environmental applications.
In both cases, particular focus is on two products of the reaction, namely, \ce{NO} and \ce{N_2}.
Depending on the application of the ammonia oxidation, the catalytic reaction must be tuned towards a specific product, this \textit{selectivity} is controlled by the reaction temperature, pressure, the  reactant ratio, and the type of catalyst.
To be able to drive the reaction, the impact of each parameter \textit{at relevant industrial conditions} on the product pressure must be studied.

\begin{landscape}
\begin{table}[!htb]
\centering
\resizebox{\columnwidth}{!}{%
    \begin{tabular}{@{}l|l|lll|l|l|ll@{}}
    \toprule
    Formula & \ammonia & \nitrogen & \nitrousoxide & \nitricoxide & \nitrogendioxide & \nitricacid & \ammoniumnitrate & \urea \\
    \midrule
    Name & Ammonia & Nitrogen & \begin{tabular}[c]{@{}l@{}}Nitrous oxide,\\ Laughing gas\end{tabular} & \begin{tabular}[c]{@{}l@{}}Nitrogen oxide,\\ Nitric oxide\\ Nitrogen monoxide\end{tabular} & Nitrogen dioxide & Nitric acid & \begin{tabular}[c]{@{}l@{}}Ammonium\\ nitrate\end{tabular} & Urea \\
    Origin & \begin{tabular}[c]{@{}l@{}}Haber-Bosch\\ process\end{tabular} & \begin{tabular}[c]{@{}l@{}}Naturally present\\ in the atmosphere,\\ Ammonia \\ oxidation,\\ Selective catalytic\\ reaction (SCR)\end{tabular} & \begin{tabular}[c]{@{}l@{}}Ammonia \\ oxidation,\\ Emissions from\\ nitrogen-based\\ fertilisers\end{tabular} & \begin{tabular}[c]{@{}l@{}}Ammonia oxidation,\\ Anthropogenic sources \\ (combustion process, \\ industry, agriculture, ...)\\ Naturally produced \\ from lightning or\\ volcanoes\end{tabular} & \begin{tabular}[c]{@{}l@{}}Ostwald process (step 1)\\ Anthropogenic sources \\ (combustion process, \\ industry, agriculture, ...)\\ Naturally produced \\ from lightning or \\ volcanoes\end{tabular} & \begin{tabular}[c]{@{}l@{}}Ostwald process\\ (step 2)\end{tabular} & \begin{tabular}[c]{@{}l@{}}Nitric acid\\ neutralisation \\ with ammonia\end{tabular} & Ammonia \\
    Major use & \begin{tabular}[c]{@{}l@{}}Ostwald process\\ (fertilisers),\\ Direct use in soil,\\ Fuel,\\ Hydrogen \\ carrier,\\ Cooling, SCR \end{tabular} & \begin{tabular}[c]{@{}l@{}}Ammonia \\ production\end{tabular} & \begin{tabular}[c]{@{}l@{}}Medicine,\\ Propellant\end{tabular} & Production of nitric acid & Production of nitric acid & \begin{tabular}[c]{@{}l@{}}Fertiliser production,\\ Nitration (explosives,\\ dyes, ...),\\ Propellant,\\ Etching\end{tabular} & \begin{tabular}[c]{@{}l@{}}Fertilising \\ agent,\\ Explosives\end{tabular} & \begin{tabular}[c]{@{}l@{}}Fertilising \\ agent,\\ SCR to reduce\\ NOx into N2\end{tabular} \\
    Toxicity & \begin{tabular}[c]{@{}l@{}}Dangerous for\\ the environment\\ Toxic, \\ Corrosive\end{tabular} & \begin{tabular}[c]{@{}l@{}}Asphyxiation by\\ displacing O2\end{tabular} & \begin{tabular}[c]{@{}l@{}}Anaesthetic, \\ euphoric\end{tabular} & \begin{tabular}[c]{@{}l@{}}Oxidising, Corrosive,\\ Toxic\end{tabular} & \begin{tabular}[c]{@{}l@{}}Oxidising, Corrosive,\\ Toxic, Health hazard\end{tabular} & Oxidising, Corrosive &  &  \\
    Pollution & PM formation &  & \begin{tabular}[c]{@{}l@{}}Ozone\\ depletion\end{tabular} & \begin{tabular}[c]{@{}l@{}}Smog, acid rains, \\ ozone depletion,\\ Precursor to \ce{NO_2} in\\ the atmosphere\end{tabular} & \begin{tabular}[c]{@{}l@{}}Smog, acid rains, \\ ozone formation\end{tabular} & \begin{tabular}[c]{@{}l@{}}Decomposes towards\\ \ce{NO_2}\end{tabular} & \begin{tabular}[c]{@{}l@{}}Decomposes \\ into \ce{NO_2} when\\ used as fertiliser,\\ Eutrophication\end{tabular} & \begin{tabular}[c]{@{}l@{}}Decomposes \\ into NH3 when\\ used as fertiliser,\\ Eutrophication\end{tabular} \\
    \begin{tabular}[c]{@{}l@{}}Greenhouse \\ effect\end{tabular} &  &  & \begin{tabular}[c]{@{}l@{}}Very important\\ (298 \ce{CO_2} eq.)\end{tabular} &  &  &  & \begin{tabular}[c]{@{}l@{}}Nitrogen-based\\ fertiliser release\\ N2O in the \\ atmosphere\end{tabular} & \begin{tabular}[c]{@{}l@{}}Nitrogen-based\\ fertiliser release\\ N2O in the \\ atmosphere\end{tabular} \\ \bottomrule
    \end{tabular}%
    }
    \caption{
        Nitrogen based species involved in the oxidation of ammonia, the Haber-Bosch process, the Ostwald process or nitrogen-based fertilisers.
        Information compiled from various sources: \cite{Thiemann2000, Harrison2001, Baerns2005, Imbihl2007, Hatscher2008, Davidson2009, Resta2020a, Borodin2021, Pottbacker2022}.
    }
    \label{tab:NitrogenGases}
\end{table}
\end{landscape}

\section{From industry to model catalysis}\label{sec:LiteratureAmmonia}

\epigraph{"A long standing conundrum in the catalysis community emerged at the interface between surface science and heterogeneous catalysis, better known as the pressure and materials gap."}{\textit{Nature Catalysis editorial, \cite*{NatureEditorial2018}.}}

\begin{table}[!htb]
    \centering
    \begin{tabular}{l|l|l|l}
    \toprule
                & Pressure    & Material                         &     Temperature \\
    \midrule
    Industry   & \qtyrange{1}{12}{\bar} & Knitted gauzes wires   & \qtyrange{750}{900}{\degreeCelsius} \\
               &              & (diameter \qty{\approx 80}{\um}) & \\
    \midrule
    Literature & UHV, mbar    & Single crystals                  & \qtyrange{25}{900}{\degreeCelsius} \\
    \midrule
    This study & Near ambient & Single crystals                  & \qtyrange{25}{600}{\degreeCelsius} \\
               & pressure (\qty{0.5}{\bar})  & and nanoparticles & \\
    \bottomrule
    \end{tabular}
    \caption{
        The difficulty in understanding the mechanisms at play during industrial reactions is highlighted with the example of the oxidation of ammonia using heterogeneous catalysis.
        Reproducing the same exact industrial reaction conditions and sample can be difficult in a laboratory due to the nature of the probe, the sensitivity of the technique, and the design of reactor cells for synchrotrons \parencite{Hatscher2008}.
        This is the so-called material and pressure gap in heterogeneous catalysis.
    }
    \label{tab:Gap}
\end{table}

% techniques
The pressure and material gap in heterogeneous catalysis has been shown to limit the understanding of the heterogeneous catalysis process.
The development of x-ray diffraction techniques, utilising a probe intrinsically well suited when working at high pressures thanks to its high penetration in gases has progressively reduced the gap.

Bridging the pressure and material gap by the development of new techniques has been the subject of several dissertations in recent years, Ackermann \parencite*{Ackermann2007} has for example pushed forward the use of \textit{operando} surface x-ray diffraction (SXRD) for the study of heterogeneous catalysis.
This effort has recently been followed towards spectroscopy techniques by Dann \parencite*{Dann2019} such as x-ray photoelectron spectroscopy (XPS).
This progress has often been preceded by the development of catalysis reactors compatible with synchrotrons beamlines, \textit{i.e.} a closed environment penetrable by x-rays, and accommodating a wide range of temperature, total pressure and types of gases \parencite{VanRijn2010}.

% single crystals
The samples typically used when combining SXRD and XPS are single crystals, much larger than industrial samples (\qty{\approx 1}{\cm} whereas the gauze diameter is of about \qty{70}{\um}), tending towards model catalysts since they only exhibit a single type of structure on their surface \parencite{Goodman1994}.
The reason behind the use of such sample is a large surface area exposed to the reacting gases and thus yielding an increased surface signal, most important to understand heterogeneous catalysis which is a surface process.
Nevertheless, they can show some limitations, for example large Pt (111) single-crystals also exhibit a large amount of steps on their surface also contributing to the catalytic activity \parencite{CalleVallejo2017}.

% nanoparticles
Only recently have supported platinum nanoparticles nanoparticles (\qty{\approx 1}{\nm} large) been used during the oxidation of ammonia, showing a remarkable selectivity towards \ce{NO} at high temperature and atmospheric pressure \parencite{Schaffer2013}.
Reducing Pt supported nanoparticles by \ce{H_2} showed to improve their catalytic activity below \qty{200}{\degreeCelsius} in a study aiming at the production of \ce{N_2} \parencite{Svintsitskiy2020}.
Similarly, supported palladium nanoparticles have shown high selectivity towards production of \ce{N_2} in the selective catalytic oxidation of \ce{NH_3} below \qty{200}{\degreeCelsius} \parencite{Dann2019}.
Nanoparticles allow to reduce the material gap in heterogeneous catalysis since they exhibit a various amount of active sites such as different facets (e.g. (111), (110), (100), ...), edges, corners and defects.
They are thus considered to be a good approximation of real catalysts \parencite{Somorjai2007, Molenbroek2009, Cuenya2010, Kwangjin2012, Schauermann2013}.
Studies have shown that nanoparticle surface strain can be controlled, opening up a new path to tune and optimise nanoparticle catalysts \parencite{Zhang2014, Sneed2015, Wang2016}.

% introduce bcdi
Bragg coherent diffraction imaging (BCDI) is a technique only recently applied to catalysis \parencite{Ulvestad2016}, that can only be applied to sub-micron samples due to instrumental limitations.
Indeed, the sample can not be larger than the coherence lengths of the instrument, about \qty{1}{\um} at $3^{rd}$ generation synchrotrons.
The x-ray beam, focused by some optical elements to micrometer size to respect this requirement, must fully illuminate the particle during the experimental process.
Therefore, by the means of BCDI, it is possible to reduce the material gap in heterogeneous catalysis.

However, samples below a certain size are not easily measured experimentally with other diffraction techniques for a simple reason which is that the outcoming photon flux is proportional to the sampled volume.
This limitation also exists in BCDI, despite the highly focused beam, which draws a limit between experimental samples (the smallest imaged nanoparticle is \qty{60}{\nm} large - \cite{Bjorling2019, Carnis2021}) and the few \unit{\nm} large nanoparticles used in dynamical field theory (DFT) or molecular dynamics (MD), theoretical approaches to molecular adsorption and particle relaxation only limited by the current computational power.
Moreover, the ability to resolve the surface signal with BCDI is still limited in comparison with SXRD, which is why both techniques must be used together to obtain a full understanding of the dynamics at play during the catalytic reaction.

\begin{figure}[!htb]
    \centering
    \includegraphics[height=3.75cm]{/home/david/Documents/PhD/Presentations/Slides/PhdSlides/Figures/sample/pt_gazes.png}
    \includegraphics[trim=140 100 0 75, clip, height=3.75cm]{/home/david/Documents/PhD/Presentations/Slides/PhdSlides/Figures/bcdi_data/B7/B7_facets.png}
    \includegraphics[height=3.75cm]{/home/david/Documents/PhD/Presentations/Slides/PhdSlides/Figures/sample/sxrd_sample.png}
    \caption{
    Platinum gauzes used in industry, diameter \qty{\approx 80}{\um} (left), image taken from industry website \parencite{PtRhGauze}.
    Reconstructed Pt particle, surface coloured by the displacement of surface layers from their equilibrium positions, diameter of about \qty{300}{\nm} (middle).
    The orientation of each facet on the particle surface is indicated.
    Pt $(111)$ single crystal used in SXRD and XPS experiments, diameter of about \qty{8}{\mm} (right).
    }
\end{figure}

\section{Aim and Scope}

The oxidation of ammonia is a catalytic reaction that has had an extremely high impact on the $20^{th}$ and $21^{st}$ century, being at the origin of dramatic changes in the world demography with the fertiliser industry, and being today part of numerous industrial process that not only contribute to the fast ongoing climate change, but also to the ever growing pollution of our ecosystems.

In this first chapter, the importance of the oxidation of ammonia has been underlined.
It was shown that despite being a major catalytic process in a multi-billion industry, the exact mechanisms of action are not yet understood, which is the first step to controlling the reaction selectivity, thereby reducing pollution from \ce{NH_3} and \ce{NO_x}, but also the greenhouse effect of \ce{NO_2} specifically.
In the future, the understanding of the reaction mechanism will help design novel catalysts moving away from expensive precious metals.

The focus of this thesis will be set on complementary goals.
First, a suitable environment for the study of \textit{operando} catalytic reactions with BCDI will be created at the SixS beamline in the synchrotron SOLEIL, already specialised in the study of surfaces by x-ray scattering techniques.
Extending the panel of available techniques will help bridging the material and pressure gap by bringing forth the possibility for synchrotron users to study catalytic reactions at the same exact conditions and environment, but with different techniques.
Optical elements necessary for the use of a focused coherent beam will be implemented and characterised.

The origin, advantages, drawbacks and workflows of each technique will be detailed, especially for Bragg coherent diffraction imaging, a technique that has yet to reach its full potential through the development of $4^{th}$-generation synchrotrons and powerful computing clusters.
An efficient and comprehensive workflow leading to reproducible results saved in the common NeXuS format \parencite{Konnecke2015} will be developed in \textit{Python} to facilitate the reduction and analysis of BCDI data, but also the approach of users outside the community to the technique by the means of a user friendly graphical user interface (GUI).

The reactor used to study catalytic reaction at SixS is already compatible with highly oxidative environments \parencite{VanRijn2010, Resta2020a}, and will permit the study of the oxidation of ammonia \textit{via} surface x-ray diffraction and Bragg coherent diffraction imaging.
Both the material and pressure gap will be partly bridged by operating at temperatures before and after the catalyst light-off, at almost industrial pressure.
Pt nanoparticles will be used during \textit{operando} SXRD and BCDI, while Pt (111) and Pt (100) single crystals will be used with SXRD.
The structure-selectivity relationship will be explored at ambient pressure and high temperature by the means of a vast array of \ce{O_2}/\ce{NH_3} ratios, favouring either the production of \ce{N_2} or \ce{NO}.
The presence of platinum oxides will be monitored in order to understand its importance in the reaction mechanism, as well as in potential catalyst reconstructions.
Thus, the difference in activity between different crystalline facets will be studied with different samples that together offer a good compromise between industrial catalysts and samples compatible with the experimental setup of synchrotron beamlines.

Complementary studies with x-ray photoelectron spectroscopy will be performed at the same ammonia to oxygen ratio and temperature.
Information about the presence or not of nitrogen and oxygen-rich species on the catalyst surface will improve the understanding of the reaction mechanism.
Finally, to explore the correlation between surface structure and catalytic activity, mass spectrometry data will be collected simultaneously to all measurements.

\section{Thesis outline}

The initial chapter will provide a concise explanation of heterogeneous catalysis and of the fundamental principles governing the interaction between x-rays and matter.
This will serve to highlight the origins, benefits, and limitations of each technique employed in this study and explore how catalytic reactions can be indirectly observed using x-rays, leveraging the unique signatures they leave on the materials involved.
The SixS beamline (synchrotron SOLEIL), which serves as the primary location for conducting the majority of experiments, will be presented.
Notably, the latest advancements in experimental techniques specific to this beamline will be covered.
Finally, the latest computer programs used for the analysis of the collected data and partly developed during this thesis will be presented.

In the second and third chapter, the results obtained during the \textit{operando}, near-ambient pressure study of the catalytic reaction by synchrotron techniques on Pt nanoparticles and single crystals respectively will be presented.
Sample preparation, the choice of experimental conditions, the reproducibility of the results, the quality of the final data as well as the different difficulties encountered during data collection are discussed.

Finally, in the last chapter of this thesis will be discussed quantitatively and qualitatively the relation between the result of each technique, their comparison to literature findings, as well as their reliability, representativity and validity, paving the way for any future study.
     
% Theory
    \chapter{Theory and methods}
    \section{Heterogeneous Catalysis}\label{sec:Catalysis}

\textit{
A catalyst is a substance that speeds up a chemical reaction, or lowers the temperature or pressure needed to start one, without itself being consumed during the reaction.
Catalysis is the process of adding a catalyst to facilitate a reaction.
}

Chemical reactions involve the breaking, rearranging, and rebuilding of bonds between atoms in molecules, resulting in the formation of new molecules.
Catalysts play a crucial role in enhancing the efficiency of these reactions by lowering the \textit{activation energy}, the energy barrier that must be overcome for the reaction to take place.
This process facilitates the breaking and formation of chemical bonds, leading to the creation of new combinations and substances \parencite{Schlogl2015}.

The use of catalysts has several advantages, including faster, selective, and more energy-efficient chemical reactions, enabling them to direct reactions towards producing higher amounts of the desired product while reducing unwanted byproducts \parencite{Schlogl2015}.
Over the years, scientists have developed specialized catalysts for various real-world applications, today 90\% of chemical processes involve catalysts in at least one of their steps \parencite{WEINER1998915, DeVries2012}.
Notable advancements in catalysis have led to the production of biodegradable plastics, novel pharmaceuticals, and eco-friendly fuels and fertilizers \parencite{FECHETE20122}.

Today, several challenges have emerged in the field of heterogeneous catalysis related to improving efficiency, reducing environmental impact, and developing sustainable processes.
First, environmental challenges concern minimizing and/or managing by-products, reducing contamination in effluents/wastewaters, and using sustainable sources of raw materials \parencite{LUDWIG2017313, Lange2021} and energy supplies.
Secondly, economical challenges which imply using cheaper, readily available raw materials, increased productivity, and decreased lag-time between discovery to commercialization \parencite{Keisuke2019, Gunay2021}.
For example, recent studies suggest that alternative, more economical catalysts, such as non-noble metals \parencite{Zhong2021} and other derived metal-based compounds, need to be tested as possible substitutes for the most frequently used noble metals, which are very efficient but expensive.

Catalysis also has a role to play to combat pollution and create cleaner energy with for example the development of efficient water-splitting technologies \parencite{AHMAD2015599}, and enhancing the use of biomass and other energy vectors such as ammonia \parencite{Fang2022}.
Finally, challenges also arise in automotive exhaust where catalyst participate in the reduction of the emissions of toxis gases and particles \parencite{WHOAirPollution, GANDHI2003433}.
Some of the major air pollutants such as nitrogen oxides, $NO_x$ ($NO$ and $NO_2$), and particulate matter (PM) are emitted by road traffic (65\% of $NO_x$, $\approx 35\%$ of PM), mainly by diesel vehicles, and directly inhaled by nearby major city inhanbitants.
To set a striking example, in Paris in 2018, 700 000 inhabitants were exposed to $NO_2$ concentrations exceeding the regulations (fig. \ref{fig:NO2Paris}), 60 000 inhabitants for PM$_{10}$, and all Parisians were concerned by exceeding the WHO recommendations for PM$_{2.5}$ \parencite{AirParis}.
Air quality is the main environmental concern of Ile-de-France residents (65\% of total mentions) ahead of climate change (63\%) and food (38\%) \parencite{AirParis}.

\begin{figure}[!htb]
    \centering
    \includegraphics[width=\textwidth]{/home/david/Documents/PhD/Figures/ammonia/ParisNO2.png}
    \caption{
        $NO_2$ levels in Paris are on average twice superior to the annual limit of $40 \, \mu g / m^3$ \parencite{AirParis}.
    }
    \label{fig:NO2Paris}
\end{figure}

\subsection{Mechanisms}

Heterogeneous catalysis is a type of catalytic process where the catalyst exists in a different phase (solid, liquid, or gas) from the reactants.
In other words, the catalyst and the reactants are present in distinct physical states.
Most commonly, the catalyst is in the solid phase, while the reactants are in either the gas or liquid phase.
This distinction sets heterogeneous catalysis apart from homogeneous catalysis, where the catalyst and reactants are in the same phase.

One of the key advantages of heterogeneous catalysis is the ease of catalyst separation and reuse.
Since the catalyst is in a different phase, it can be easily separated from the reaction mixture once the reaction is complete.
This makes the catalyst recyclable and economically attractive for industrial processes \parencite{FECHETE20122}.

The Sabatier principle, at the origin of the Nobel price of chemistry of 1912 \parencite{Che2013}, states that when a heterogeneous catalyst is used, there exists an optimum catalyst binding energy for the reactant on the catalyst surface.
If the catalyst binding energy is too weak, the reactant molecules do not adsorb strongly enough on the catalyst, leading to low reaction rates.
On the other hand, if the catalyst binding energy is too strong, the reactant molecules bind too tightly and do not easily react with the other reactants, resulting in low selectivity and efficiency.

In a founding study in 1922, Langmuir detailed three possible mechanism of actions for heterogeneous catalysis \parencite{Langmuir1922, Prins2018}.
In the first mechanism both reactants are adsorbed in adjacent spaces at the surface of the catalyst, react at the surface to form the product which is subsequently desorbed from the surface.
This is the Langmuir–Hinshelwood mechanism.

In the second mechanism, it is a molecule from the gas phase that comes and interact with the adsorbed reactant without being itself adsorbed.
This is called the Eley–Rideal mechanism for which the product is directly in the gas phase \parencite{rideal_1939, Weinberg1996}.

The first mechanism (Langmuir–Hinshelwood) appears to be generally preferred, the bonding of the molecule to the atoms weakening its bonds and preparing the molecule for reactionby reducing the activation energy required for the reaction to occur \parencite{Baxter2002, Prins2018}.

A heterogeneous catalytic reaction consists is a series of elementary steps such as reactant dissociation, adsorption, surface diffusion, surface chemical reactions, and desorption that are nowadays extensively studied via different theoretical computing methods such as density-functional theory (DFT - \cite{Reuter2004, Molenbroek2009, Yawei2015, Gaggioli2019, Chatelier2020}) as a function of the reaction parameters.

Thanks to the increase of interest in the field of catalysis, new experimental methods have also been developped to be able to observe \textit{in-situ} and \textit{operando} the reaction

Catalysts are usually complex systems in powder form (presenting different surface orientation, i.e. facets) coupled with promoters (chemicals that improve the catalytic activity).
It is therefore difficult to provide a molecular-level understanding of such processes.
Model catalysts can therefore be used to simplify the investigation.
A well-built theory has been proposed by Hammer and Nørskov [49] and lays down the basic rules behind catalysis.
Some key parameters are used to rationalize and describe a catalytic process and catalysts performances.
More recent approaches involving the use of machine learning can help predict the key descriptors for catalysis [50, 51].
Hammer and Nørskov theory is more of a model theoretical approach that could be experimentally questioned by model catalysts.

Catalysis is described by key parameters such as the stability (the propensity of the catalyst to stay unchanged after the reaction), the activity and the turn-over frequency (TOF, number of mole of reactant that can be converted per mole of catalyst over time), the selectivity (for example targeting the production of one particular isomer), the propensity to deactivation of the catalyst (for instance due to its oxidation).
Depending on the chemical reaction, one wishes to have an active catalyst that is very stable and very selective towards a unique product, and doing so for a long time.
However, it is difficult to meet all the requirements at once.
A high activity is unfortunately often linked to a poor selectivity


The turnover frequency TOF quantifies the specific activity of a catalytic centre for a special reaction under defined reaction conditions by the number of molecular reactions or catalytic cycles occurring at the centre per unit time.
For heterogeneous catalysts the number of active centres is derived usually from sorption methods.

Separate from the TOF, evaluating the catalytic activity, the turnover number (TON) value is an important parameter to evaluate the stability of the catalyst. In homogeneous and heterogeneous catalysis, the TON is a dimensionless number,24,25 which is defined as the number of the molecules produced per catalytic site before deactivation under given reaction conditions.
That is to say, the catalyst can achieve the total number of turnovers until it is totally dead, regardless of the reaction time. In this respect, an ideal catalyst should have an infinite TON.
Thus, the TON represents the maximum yield of products attained from an active catalytic site up to the decay of activity for a specific reaction.
The TON of a catalyst for water oxidation is calculated according to Eq. (5):

Create new phases in the catalyst

\subsection{Linking strain and reactivity}



Hammer and Noskrov blabla


Role of strain :

\cite{Kitchin2004}

\cite{Mavrikakis1998}

We define a system where a (metal) sample, consisting of a surface and a bulk is in contact with a gaseous environment.

Hammer and Nørskov [49] have compiled and provided a well-built theory of adsorbates
surface interactions for simple transition metals.
As shown in Fig. 1.2, the model predicts that as the d-band of the metal shifts up towards the Fermi level (the filling of the band is kept fixed so that as the center of the d-band is shifted up, the band width decreases), the electron density of states of the adsorbate is modified and antibonding states appear above the Fermi level.
herefore they are empty and the bonds become stronger as the number of empty antibonding states increases.
In short, the closer to the Fermi level and the narrower the d-band, the stronger the bonding i.e. the chemisorption.

This model seems to work fine for simple transition metals (3d, 4d and 5d) [38, 39] for chemisorption (e.g. oxygen adsorption [49]) and also for molecular dissociation (e.g. CO dissociation [53], NO dissociation on Ru(0001) [42]).
A clear linear correlation between adsorption energies and d-band position is determined both experimentally and theoretically.
This is similar to the Brønsted-Evans-Polanyi linear relation between the activation and reaction energies.
In the case of a monolayer of a transition metal over a substrate, a similar behavior is observed and the model still works fine (e.g. 5d metals on Pt(111) [54])

\subsection{Active sites in heterogeneous catalysis}

The size of the catalyst particles can affect the surface area, the number of active sites, and the diffusion of reactants and products.
In general, smaller particles have a higher surface area-to-volume ratio, which can increase the number of active sites available for catalysis.
However, smaller particles can also be more prone to agglomeration and deactivation. Therefore, controlling the particle size and morphology is an important aspect of catalyst design.

Alexandr Yu. Stakheev, ..., Valerii I. Bukhtiyarov, in Advanced Nanomaterials for Catalysis and Energy, 2019

Relationships between turnover frequency and the size of supported Pt clusters are discussed for oxidation of hydrocarbons, CO, and NO by molecular oxygen.
Analysis of the experimental data indicates that TOF tends to increase for bigger platinum particles.
This tendency is particularly pronounced for the nanoparticles smaller than 4–5 nm. According to the most realistic models, the observed tendency stems from the deactivation of edge, corner, and neighboring atoms by two processes: (1) strong oxygen adsorption on edge and corner atoms with high degree of coordinative unsaturation, and (2) oxidation of Pt to PtOx, which is facilitated over undercoordinated sites.
As the metal clusters grow in size, the fraction of undercoordinated edge and corner atoms decreases leading to the increase in experimentally observed TOF.

The increase of supported platinum particle size led also to considerable changes in selectivity in the ammonia oxidation over a Pt/Al2O3 catalyst [7,22,26].
Large crystallites of 15.5 nm, for which over 98\% of the surface atoms are plane atoms [28], exhibited low selectivity to nitrogen formation. Selectivity to nitrogen increased with decreasing platinum loading

It was also reported that the stoichiometry of oxygen chemisorption increases by a factor 2.7 with increasing platinum crystallite size. This could also lead to an increase of the reaction rate if the oxygen adsorption is the rate-determining step in this system.

    \section{X-ray interaction with matter} \label{sec:XRIntMatter}

% Field / amplitude / intensity consistent ?
% $r_0$ use consistent ? $2\pi$ ?
% hypothesis to explain: DWBA, dipole approximation, perfect crystals
% When detecting the scattered field at a point $\vec{R}$, we assume that $\|vec{R}|$ is far greater than the electronic density, this is called the dipole approximation. NOT GOOD Each atom can be considered as a ball of certain radius (see \ref{fcc_lattice}).

\subsection{Scattering from electrons and atoms}

The duality between wave and particles was first mentioned by Max Planck and Albert Einstein in the early $20^{th}$ century and generalised to all matter by Louis-Victor de Broglie in 1924 with the famous formula:

\begin{equation}
	\lambda = \frac{h}{p}
\end{equation}

Electromagnetic waves, \textit{i.e.} light or photons can be characterised by their energy $E$ in \unit{\eV} and wavelength $\lambda$ in \unit{\meter}. $p$ corresponds to the momentum transfer.
The conversion between is realised thanks to Planck's constant $h = \qty{6.626e-34}{\joule.\second}$ and the speed of light in vacuum $c = \qty{2.9979e8}{\meter.\per \second}$ (eq. \ref{eq:EnergyLambda}).

\begin{equation}
    \label{eq:EnergyLambda}
	E = \frac{hc}{\lambda}
\end{equation}

The properties of the photon and its use in our society depends on its energy and wavelength.
For example, visible light is situated between \qty{1.65}{\eV} and \qty{3.26}{\eV}, micro-waves used in our everyday life are situated between \qty{e-6}{\eV} and \qty{e-3}{\eV}.
On the other side of the electromagnetic spectrum, we have higher energy photons such as x-rays (\qtyrange{e2}{e5}{\eV}) and $\gamma$-rays (\qty{>e5}{\eV}).

X-rays have a wavelength of a few \unit{\angstrom} (\qty{e-10}{\m}) making them the perfect probe to study the structures of materials at the atomic scale thanks to different interactions with matter.

\subsubsection{Cross-sections}

When electromagnetic wave interacts with matter it will be attenuated by absorption, reflection or scattering.
Each process can be quantified depending on the atoms (and thus on the electronic cloud and nucleus) the beam interacts with and the energy on the incident photon, this is illustrated in figure \ref{fig:cross_sections}.
The cross-section for a particular process $p$ is defined as follows \parencite{Willmott}:

\begin{equation}
	\sigma_p = (\Lambda_p N_i)^{-1}
\end{equation}

$\Lambda_p$ is the attenuation length in \unit{\meter}, \textit{i.e.} the length after which the beam's intensity is reduced to $1/e$, $N_i$ is the atomic number-density in atoms/unit volume.

\begin{figure}[!htb]
    \centering
    \includegraphics[width=\textwidth]{Figures/introduction/cross_sections.pdf}
    \caption{
        Cross-sections for platinum (Z=78) corresponding to different processes occurring when photons interact with matter.
        Data taken from the NIST (National Institute of Standards and Technology) \parencite{NIST_cross_sections} website.
        The energy range of the SixS beamline at SOLEIL is highlighted in blue.
    }
    \label{fig:cross_sections}
\end{figure}

Compton scattering, also named inelastic scattering, is a process during which some of the incident electromagnetic wave energy is transferred to the atoms' electrons.
This results in a lower energy for the scattered photon (and therefore a higher wavelength) compared to the incident photon.
%This effect has a low cross-section compared to the two other processes and is therefore not taken account during the experiments.

In the frame of this thesis, the (elastic) Thomson scattering cross-sections is the most important, at the origin of x-ray diffraction.
This process is dominant for energies below \qty{200}{\keV}, in the x-ray regime,  together with photoelectric absorption.% for which the K, L and M edges are shown.

\subsubsection{Scattering from an electron}

We begin our discussion of x-ray scattering by considering scattering from a single free electron using classical electromagnetic theory.
During elastic scattering, the oscillating electric field of the incident x-ray wave exerts an electromagnetic force on the electron, causing it to accelerate and oscillate in the same direction as the incident electronic field.

\begin{figure}[!htb]
    \centering
    \includegraphics[trim=0 75 0 25, clip, width=\textwidth]{Figures/introduction/torus.pdf}
    \caption{
        Effect of relation between synchrotron x-ray polarisation $\hat{\epsilon}$ and scattered x-ray polarisation $\hat{\epsilon}'$ on the scattered field.
        The scattered field intensity is attenuated by a $\cos{(\theta)}$ factor, where $\theta$ is the angle between the plane perpendicular to the electric field and the direction of observation $\vec{R}$ (b)
    }
    \label{fig:polarization_effect}
\end{figure}

During an elastic scattering event, the oscillating electron emits a spherical electromagnetic wave with the same wavelength as the incident beam (Thomson scattering).
The scattered field $\vec{E}_{scatt}$ is then proportional to the incident electromagnetic field $\vec{E}_{in}$ as follows \parencite{NielsenMcMorrow}:

\begin{equation}
    \label{eq:scatt_field}
    \frac{\vec{E}_{scatt}(R, \, t)} {\vec{E}_{in}} = -r_0 \frac{e^{\vec{k}.\vec{R}}} {|\vec{R}|}| \hat{\epsilon}.\hat{\epsilon}'|
\end{equation}
$|\vec{R}|$ is the distance at which the scattering is detected at the time $t$, $\hat{\epsilon}$ is the synchrotron x-ray polarisation vector, $\hat{\epsilon}'$ the scattered x-ray polarisation vector, and $r_0$ is the Thomson scattering length or the classical radius of the electron defined as:

\begin{equation}
    \label{eq:scatt_thomson_scat_length}
    r_0 = \frac{e^2} {4\pi\epsilon_0 m_e c^2}
\end{equation}
with $m_e$ the mass of the electron and $\epsilon_0$ the permittivity of free space.

The minus sign illustrates a phase shift of $\pi$ between the incident and scattered wave, $\hat{\epsilon}$ and $\hat{\epsilon}'$ are respectively the polarisation vectors of the incident and scattered electromagnetic fields.

The differential cross-section for Thomson scattering measures the efficiency of the scattering in the volume occupied by a solid angle $d\Omega$ in the direction $\vec{R}$ \parencite{NielsenMcMorrow}. It is defined as follows:
\begin{equation}
    \label{eq:dif_cross_sec_thomson1}
    \frac{d\sigma_{ts}} {d \Omega} = \frac{ |\vec{E}_{scatt}(R, t)|^2 R^2} {|\vec{E}_{in}|^2}
\end{equation}

By substituting eq. \ref{eq:scatt_field} into eq. \ref{eq:dif_cross_sec_thomson1}, it becomes clear that the scattering is proportional to the Thomson scattering length and that the intensity is attenuated depending on the dot product between the two polarisation.
The polarisation factor $P$ for scattered beams is defined as $P =  | \hat{\epsilon}.\hat{\epsilon}'|^2$ and we can write the differential cross-section as:

\begin{equation}
    \frac{d\sigma_{ts}} {d \Omega} = r_0^2 | \hat{\epsilon}.\hat{\epsilon}'|^2 = r_0^2 P
\end{equation}

The effect of the polarisation of the incident beam is illustrated in figure \ref{fig:polarization_effect}.
At synchrotrons, working in the vertical plane ($\vec{x}, \vec{z}$) is preferable to maximise the intensity of the scattered field.
This has practical repercussions in surface x-ray diffraction for which it is preferable to work in a vertical geometry to scan large 2D areas of the reciprocal space.

The total cross-section for the scattering event by a single free electron can be computed by integrating the differential cross-section over all the possible scattering angles, \textit{i.e.} by averaging all possible polarisation directions \parencite{Willmott}.
This yields $\sigma_{ts} = 8 \pi r_0^2 /3 = \qty{0.665}{\barn}$, (\qty{1}{barn} is equal to \qty{e-28}{\m^2}) the total Thomson scattering cross-section is constant, independent of the incoming photon energy. This results holds for x-rays for which the scatterer \textit{i.e.} the electron can be considered as free \parencite{Willmott}.

\subsubsection{Scattering from a single atom}\label{sec:scattering}

As we have seen the main scatterer for Thomson scattering is the electron.
An atom can be described as a small volume $d^3\vec{r}$ in which the electrons are localised.
The electromagnetic field scattered from an atom is therefore proportional to the atomic electronic density $\rho_{atom}(\vec{r})$.
For a single atom of atomic number $Z$ we have :

\begin{equation}
    \int \rho_{atom} (\vec{r}) d^3\vec{r} = Z
\end{equation}

To describe the phase of the scattered field, which depends on the phase of each incident wave at each point in the scattering volume (fig. \ref{fig:q}), we introduce the scattering vector, $\vec{q}$ (eq. \ref{eq:Q}), to derive the amplitude of a scattering event in the direction defined by the scattering angle $2\theta$.

\begin{equation}
    \label{eq:Q}
    \vec{q}=\vec{k}_s-\vec{k}_i.
\end{equation}

\begin{figure}[!htb]
    \centering
    \includegraphics[scale=0.6]{Figures/introduction/q.pdf}
    \caption{
    Geometry of the scattering vector $\vec{q}$ in reciprocal space, $2\theta$ is the scattering angle.
    The magnitude of the scattering vector can be derived from the scattering angle $\theta$ that draws a line cutting $\vec{q}$ at $|\vec{q}|/2$.
    }
    \label{fig:q}
\end{figure}

Therefore, the phase difference between a wave scattered at a position $O$ and a wave scattered at a distance $\vec{r}$ from $O$ is equal to $(\vec{k}_i - \vec{k}_s).\vec{r} = \vec{q}.\vec{r}$ (fig. \ref{fig:q}).

From fig. \ref{fig:q} and the relation between wavevector and wavelength ($k = 2 \pi/\lambda$), the scattering vector can be written as a function of the wavelength of the incident beam $\lambda$, and of the direction of detection defined by the scattering angle $2\theta$ (eq. \ref{eq:QSinTheta}) between the wavevector of the incident photon $\vec{k}_i$ and the wavevector of the scattered photon $\vec{k}_s$.
We assume here that the scattering event is elastic ($|\vec{k}_i|=|\vec{k}_s|$) and that the waves are plane waves parallel to each other when in the small scattering volume $d^3\vec{r}$.

\begin{equation}
    \label{eq:QSinTheta}
    |\vec{q}| = \frac{4\pi}{\lambda} \sin{\theta}
\end{equation}

Each electron in the small volume $d^3\vec{r}$ will have a contribution proportional to the Thomson scattering length $r_0$ to the scattered field with a phase $e^{i\vec{q}.\vec{r}}$.
By integrating over the volume occupied by the atom we obtain the total contribution of an atom to the scattered field in the direction $2\theta$:

\begin{equation}
    \label{eq:AtomicFormFactor}
    -r_0 \int \rho_{atom} (\vec{r}) e^{i\vec{q}.\vec{r}} d\vec{r} = -r_0 f^0(\vec{q}) = -r_0 FT [\rho_{atom} (\vec{r})]
\end{equation}

The scattering amplitude as a function of $\vec{q}$ is described by the atomic from factor $f^0(\vec{q})$, which is defined as the Fourier transform of the atomic electronic density $\rho_{atom}(\vec{r})$ \parencite{Paganin}.
The values for the atomic from factor can be approximated by a sum of Gaussians using tabulated coefficients (eq. \ref{eq:AtomicFormFactorTab}) available online \parencite{InterTablesOfCryst}.
It decreases with $\vec{q}$ (or $\sin(\theta) / \lambda)$ as illustrated in figure \ref{fig:atomic_form_factor}.

\begin{equation}
    \label{eq:AtomicFormFactorTab}
    f^0(\vec{q}) = \sum_{i=1}^4 a_i \exp (-b_i (\frac{q} {4\pi})^2) + c
\end{equation}

The scattering intensity is equal to the square of the scattering amplitude.
For example, the scattering intensity of palladium atoms (Z=46) at $|\vec{q}| \qty{\approx 2.75}{\angstrom^{-1}}$ is only \qty{\approx 31}{\percent} of that of platinum atoms (Z=78).
In the case of oxygen (Z=8), the intensity falls down to \qty{\approx 6.7}{\percent}.
This difference in scattering intensity between elements becomes crucial when working with small objects that have a small scattering volume such as nanoparticles.

\begin{figure}[!htb]
    \centering
    \includegraphics[width=\textwidth]{Figures/introduction/atomic_form_factor.pdf}
    \caption{
    Atomic form factor calculated for platinum (Z=78) using tabulated values \parencite{InterTablesOfCryst} for equation \ref{eq:AtomicFormFactor}. The scattering intensity decreases with the scattering angle $\theta$ but increases with the incident wavelength $\lambda$. $i$ and $c$ respectively designate the Gaussian contribution and constant in eq. \ref{eq:AtomicFormFactorTab}.
    }
    \label{fig:atomic_form_factor}
\end{figure}

\subsection{Refraction, reflection and absorption effects}\label{sec:RefractionReflectionAbsorption}

For now we have considered the electron as free.
However, electrons in atoms or molecules exist in discrete bound states (characterised by a binding energy $E_b$), and their response to the incoming electromagnetic field becomes dependent on the incoming energy.
In a classical point of view, we can reconsider the bound electron as a damped harmonic oscillator with associated resonant frequencies $\omega_b$ corresponding to the electron's binding energy.
First, when approaching the binding energy, the amplitude of the electron's oscillation are reduced and so is the atomic form factor by a real energy-dependant factor $f'(\hbar\omega)$.
Secondly, when approaching the binding energy, the oscillation of the scattering electron experiences a phase lag.
This is taken into account by introducing an imaginary energy-dependent factor $f''(\hbar\omega)$.
The atomic form factor $f^0(\vec{q})$ must therefore be corrected by these two correction factors, we define the total atomic form factor $f(q, \hbar\omega)$ as \parencite{NielsenMcMorrow}:

\begin{equation}
    f(\vec{q}, \hbar\omega) = f^0(\vec{q}) + f'(\hbar\omega) + if''(\hbar\omega) = f_1(\vec{q}, \hbar\omega) + i f_2(\hbar\omega)
\end{equation}

The values of the correction factors $f'$ and $f''$, also known as the dispersion corrections to $f^0$, depend on the electronic structure of the atom and are maximum when approaching the binding energies of the electrons in the atom.
To understand the impact of the correction factors on the electromagnetic wave inside a material, we introduce the complex refraction index $n$, which for x-rays can be written as \parencite{NielsenMcMorrow}:

\begin{gather}
    \label{eq:delta}
    n = 1-\delta+i\beta = 1 - \frac{r_0 \lambda^2}{2\pi} \sum_i N_i f_i(\vec{q}=0)\\
    \delta = \frac{r_0 \lambda^2}{2\pi} \sum_i N_i f_{i, 1}(\vec{q}=0)\\
    \beta = \frac{r_0 \lambda^2}{2\pi} \sum_i N_i f_{i, 2}
\end{gather}
with $f_i(\vec{q}=0)$ the total atomic form factor of the atom $i$ in the forward direction ($\vec{q}=0$), and $N_i$ the number of atoms corresponding to the element $i$ per unit volume.
$\delta$ is the refractive index and $\beta$ the absorption index.

During simple experiments, when x-rays impinge on a plane surface between air ($n$=1) and another medium of refractive index $n$, x-rays are refracted, the wavelength changes to $\lambda/n$, and the direction of propagation becomes closer to the surface between both mediums.
Below a certain incident angle known as the \textit{critical angle} $\alpha_c$ between the direction of the incident x-rays and the plane interface, the x-rays will be totally reflected.
This effect is used in focusing mirrors for optics but also in techniques used to probe the surface of materials to obtain information about the surface density, the critical angle being approximately equal to $\sqrt{2\rho}$ (when working at the same energy) \parencite{Willmott}.
Moreover, during surface x-ray diffraction, the incoming angle perpendicular to the surface is kept to very low values approaching the critical angle to limit the volume probed by the x-rays to increase the contrast between the amplitude of the x-rays scattered from the latest surface layers in regard to the x-rays scattered from the bulk.
The penetration depth of the x-rays in the material is detailed is sec. \ref{sec:SXRD}.

Far from the binding energies of the electron in the system, the correction factors $f'$ in the expression of the scattering factor can be ignored.
% and the sum in eq. \ref{eq:delta} is then equal to the electronic density of the system.
\textcolor{Important}{Doucle check}

Finally, to highlight the impact of the complex part of the absorption index $\beta$, we express the electric field of a plane electromagnetic wave propagating in the $\vec{z}$ direction inside a material of refractive index $n$ as follows:

\begin{equation}
    E(\vec{z}, t) = E_0 e^{i(n\vec{k}.\vec{z} - wt)} = E_0 e^{-\beta \vec{k}.\vec{z}} e^{i(\delta\vec{k}.\vec{z} - wt)}
\end{equation}

The imaginary part of the form factor translates into an exponential decay of the intensity of the incoming wave as a function of the absorption index $\beta$, the intensity of the electromagnetic field decreases; \textit{i.e.} the wave is absorbed by the material.
The absorption of the photons by the material is at the origin of many different phenomena such as the promotion of electrons to either higher energetic bound states or directly to vacuum.
The kinetic energy of the electrons released from the material in the latter scenario is equal to the incident photon energy $\hbar\omega$, minus the energy required to release the electron from its bound state $E_b$, minus the energy required for the electron to escape from the electronic cloud $W$ (also known as the work function).

\begin{equation}
    E_{kin} = \hbar\omega - W - E_b
\end{equation}

Measuring the kinetic energy of the photoelectron that escaped the material yields direct information about the different binding energies present in the material, \textit{i.e.} about its electronic structure.
The study of these binding/kinetic energies is known as x-ray photoelectron spectroscopy (XPS), a technique used in this thesis to study the electronic structure of the atoms within the topmost atomic layers of the catalyst.
Electrons released from atoms in deep layers are ultimately absorbed by the surrounding atoms.
The cross-section for XPS is directly proportional to the absorption cross-section $\sigma_a$ (eq. \ref{eq:AbsorptionCrossSection}, fig. \ref{fig:cross_sections}) which can be derived from the imaginary atomic form factor:

\begin{equation}
    \label{eq:AbsorptionCrossSection}
    \sigma_a(\hbar\omega) = 2 r_0 \lambda f_2(\hbar\omega)
\end{equation}

\section{Scattering from crystals}\label{sec:ScatCrystal}

A crystal is a solid material composed of a regular, repeating arrangement of atoms, ions, or molecules (\textit{i.e.} a pattern), exhibiting a highly ordered structure with long-range periodicity in three dimensions.

A Bravais lattice refers to an infinite array of points (nodes) that represents the basic repeating unit of a crystal lattice, defining the translational symmetry of the crystal structure.
It is characterised by a set of three basis vectors and their linear combinations, which generate the entire lattice when translated in space.
The \textit{primitive} unit cell is the smallest cell with which you can describe the crystal.

The structure of the Bravais lattice (cubic, hexagonal, ...) combined with the position of the patterns in the lattice and the symmetry relations between them defines the crystal \textit{space group}.
In total there exist 230 space group in crystallography.
In the simplest case the pattern consists of a single atom, for example pure platinum ($Z=78$) crystallises at room temperature in a cubic Bravais lattice (fig. \ref{fig:fcc}).

We can define the scattering factor $F_{crystal}$ of the crystal, that describes the amplitude of the scattered waves, by the sum of the atomic scattering factor $f_j(\vec{q})$ of each atom present in the crystal, while the phases between the scattered waves depend on the positions of the atoms $\vec{R}_j$ in the direction that is perpendicular to the scattering vector $\vec{q}$ (eq. \ref{eq:FCrystal1}).
In other words, each atom can be described as a small volume $d^3\vec{r}$ in the electronic density $\rho(\vec{r})$, and the scattered field by the superposition of the contribution from the electronic cloud surrounding the atoms.
The atomic scattering factor is equal to the Fourier transform of the electronic density of a single atom, the structure factor of a crystal can be written as the Fourier transform of the crystal electronic density \parencite{Paganin}.
%Thus, the total scattering amplitude at $2\theta$ will be equal to the vector sum of the scattering amplitudes in this direction from all volume elements $d^3\vec{r}$ in $\rho(\vec{r})$ taking into account the phases between them.

\begin{equation}
    \label{eq:FCrystal1}
    F_{crystal}(\vec{q}) = \sum_j^{N_{atoms}} f_j(\vec{q}) e^{i\vec{q}.\vec{R}_j} = FT[\rho(\vec{r})]
\end{equation}

$f_j(\vec{q})$ is the atomic scattering factor of the $j^{th}$ atom at position $\vec{R}_j$ in a crystal made of $N_{atoms}$ atoms.
The Thomson scattering length is left apart for simplicity.

\begin{figure}[!htb]
    \centering
    \includegraphics[height=5cm]{/home/david/Documents/PhDScripts/Drawing/blender/FCC.png}
    \includegraphics[height=5cm]{/home/david/Documents/PhD/Figures/introduction/PtClean.pdf}
    \caption{
    Face centred cubic (FCC) lattice of Pt (space group 225).
    Atoms are represented as solid balls and situated on the corners and at the middle of the faces.
    A schematic representation of the unit cell is shown on the right to be able to visualise the positions of each atom.
    The lattice parameter at room temperature is $a = \qty{3.9242}{\angstrom}$.
    Close packed direction is achieved along the diagonal of the lateral faces, the distance between the atoms then becomes \qty{2.7748}{\angstrom}.
    }
    \label{fig:fcc}
\end{figure}

The position of any atom in the crystal $\vec{R}_j$ is equal to the sum of the position of the unit cell containing the atom $\vec{R}_{uc}$, plus its position within the unit cell $\vec{r}_j$.
For a crystals made of $N_{uc}$ unit cells each composed of $N_{atoms,uc}$ we have:

\begin{equation}
    F_{crystal}(\vec{q}) = \sum_j^{N_{atoms,uc}} f_j(\vec{q}) e^{i\vec{q}.\vec{r}_j} \sum_k^{N_{uc}} e^{i\vec{q}.\vec{R}_k}
    \label{eq:Fcrystal}
\end{equation}

The Bravais lattice is defined by three basis vectors $\vec{a},\ \vec{b},\ \vec{c}$ and three angles $\alpha$ [$\angle (\vec{b}, \vec{c})$] , $\beta$ [$\angle (\vec{c}, \vec{a})$] and $\gamma$ [$\angle (\vec{a}, \vec{b})$].
Any vector $\vec{R}_k$ describing the position of a node in the real space can be created by a linear combination of these three vectors:

\begin{equation}
    \label{eq:R_k}
    \vec{R}_k=n_1\vec{a} + n_2\vec{b} + n_3\vec{c}, \quad \ (n_1,n_2,n_3) \in \mathbb{Z}^3
\end{equation}

To understand the contribution of the second sum in eq. \ref{eq:Fcrystal} to the scattering amplitude, it is convenient to introduce the \textit{reciprocal space} which is the Fourier transform of the real space.
It is defined by three basis vectors $\vec{a}^*,\ \vec{b}^*,\ \vec{c}^*$.
Similarly, the nodes of the reciprocal space can be accessed from its origin by a linear combination $\vec{G}_{hkl}$ of these three vectors.

\begin{equation}
    \vec{a}^*=\frac{2\pi}{V}(\vec{b}\times \vec{c}), \qquad
    \vec{b}^*=\frac{2\pi}{V}(\vec{c}\times \vec{a}), \qquad
    \vec{c}^*=\frac{2\pi}{V}(\vec{a}\times \vec{b}), \qquad
    \vec{a}_i . \vec{a}_j^* = \delta_{i,j}
\end{equation}

\begin{equation}
    \label{eq:G}
    \vec{G}_{hkl}=h\vec{a}^* + k\vec{b}^* + l\vec{c}^*, \quad \ (hkl) \in \mathbb{Z}^3
\end{equation}

By combining eq. \ref{eq:R_k} and eq. \ref{eq:G} we find that the solution to the second sum in eq. \ref{eq:Fcrystal}, \textit{i.e.} to the equation $\vec{q}.\vec{R}_k = n \times 2\pi$ $(\ n \in \mathbb{Z})$ is to write $\vec{q}$ as a linear combination of the reciprocal space vectors, \textit{i.e.} as $\vec{G}_{hkl}$.

\begin{equation}
    \label{eq:LaueCond}
    \vec{G}_{hkl} . \vec{R}_k = hn_1 + kn_2 + ln_3 = n \quad  with \ n \in \mathbb{Z}
\end{equation}

This is also known as the Laue condition, that ensures that only certain scattering vectors, corresponding to the reciprocal lattice points, fulfil the condition for constructive interference between the scattered waves.
These specific scattering vectors $\vec{G}_{hkl}$ determine the scattering angles of the peaks observed in a diffraction pattern, also known as Bragg peaks.
% \begin{equation}
%     \label{eq:LaueCond2}
%     \vec{q} = \vec{G}_{hkl}  = h\vec{a}^* + k\vec{b}^* + l\vec{c}^*
% \end{equation}

\subsection{Bragg's law}\label{sec:BraggLaw}

The intensity scattered from a crystal as a function of the scattering vector $\vec{q}$ provides information about the arrangement and spacing of crystalline planes perpendicular to $\vec{q}$.
The Miller indices are a set of integers used to represent the planes' orientation and spacing.
A plane denoted by the ($h, k, l$) indices intercepts the axes $\vec{a}$, $\vec{b}$, $\vec{c}$ on the points $|\vec{a}|/h, |\vec{b}|/k, |\vec{c}|/l$.
The direction perpendicular to the $(hkl)$ plane is written as $[hkl]$, the distance between each plane is $d_{hkl}$.

\begin{figure}[!htb]
    \centering
    \includegraphics[scale=0.6]{Figures/introduction/BraggLaw.pdf}
    \caption{
    The difference in the path length (in red) between plane waves scattered at an angle $\theta$ is equal to $2d_{hkl} \sin{\theta}$.
    This distance must be an integer multiple of the wavelength for constructive interference to occur.
    }
    \label{fig:BraggLaw}
\end{figure}

A Bragg peak results from the constructive interference between scattered waves at discrete values of the incident angle $\theta$ on a specific set of crystalline planes.
From fig. \ref{fig:BraggLaw} we can retrieve the condition to have constructive interference, also known as Bragg's law and given by eq. \ref{eq:Bragglaw}.

\begin{equation}
    \label{eq:Bragglaw}
    \lambda = 2d_{hkl} \sin{\theta}%, \quad \ n \in \mathbb{Z}
\end{equation}

In the case of a cubic Bravais lattice, the distance between Miller planes can be written as:
\begin{equation}
    \label{eq:Interplanarspacing}
    d_{hkl}=\frac{2\pi}{|\vec{a}^*|\sqrt{h^2 + k^2 + l^2}}=\frac{|\vec{a}|}{\sqrt{h^2 + k^2 + l^2}}
\end{equation}

For a hexagonal Bravais lattice, we have:
\begin{equation}
    \label{eq:InterplanarspacingHex}
    \frac{1}{{d_{hkl}}^2}=\frac{4}{3a^2}(h^2 + hk + k^2) + \frac{l^2}{c^2}
\end{equation}

From eqs. \ref{eq:QSinTheta} and \ref{eq:Bragglaw}, we fall back on the Laue condition which is a generalisation of Bragg's law.
The diffraction order ($n$) in the Bragg's law equation can be omitted since it is implicitly determined by the Miller indices (hkl) representing the crystalline planes involved in the diffraction.
Since we are in the Laue condition, $|\vec{G}_{hkl}|$ is used rather than $|\vec{q}|$ in eq. \ref{eq:QandD3}.

\begin{align}
    \label{eq:QandD1}
    |\vec{q}| & = \frac{4\pi \sin(\theta)}{\lambda},\\
    \label{eq:QandD2}
    \sin(\theta) / \lambda & = \frac{1}{2d_{hkl}},\\
    \label{eq:QandD3}
    |\vec{G}_{hkl}| & = \frac{2\pi}{d_{hkl}}
\end{align}{}

When fulfilling the Laue (or Bragg) condition, the direction of the scattering vector is perpendicular to the crystalline planes represented by the ($h, k, l$) Miller indices, and its magnitude is inversely proportional to the distance between consecutive crystalline planes.

\subsection{Structure factor} \label{sec:StructureFactor}

On one hand, in real space, the crystal electronic density can be formed by repeating the unit cell at each lattice point, \textit{i.e.} it can be represented as a convolution of the lattice and unit cell functions.
This representation captures the periodicity and arrangement of the crystal.

On the other hand, in reciprocal space (Fourier space), the crystal scattering factor $F_{crystal}$ is the Fourier transform of the electronic density.
According to the convolution theorem, the Fourier transform of a convolution is equal to the product of the Fourier transforms of the individual functions \parencite{Mcalister2003}.
The scattering factor can therefore be expressed as the product of the Fourier transforms of the lattice and unit cell functions.
These Fourier transforms are known as the lattice factor $F_{lat}$ and the unit cell structure factor $F_{uc}$.

\begin{equation}
    F_{crystal} = FT[\rho(\vec{r})] = F_{uc} \times F_{lat}
    \label{eq:FcrystalFlatFuc}
\end{equation}

The lattice factor $F_{lat}$ represents the periodic arrangement of the crystal lattice in Fourier space, corresponding to the reciprocal lattice.
It determines the positions and intensities of the diffraction peaks in the scattering pattern.
The unit cell structure factor $F_{uc}$ describes the distribution of electron density within the unit cell of the crystal and modulates the scattered beam amplitude depending on what atoms are in the unit cell (amplitude) and their positions (phase).
By identifying the different parts of eq. \ref{eq:Fcrystal} and eq. \ref{eq:FcrystalFlatFuc} we can first isolate the unit cell structure factor:

\begin{equation}
    \label{eq:StrucFactor}
    F_{uc} = \sum_j^{N_{atoms}} f_j(\vec{q}) e^{i\vec{q}.\vec{r}_j}
\end{equation}

Each diffraction peak in the scattering pattern corresponds to a specific Fourier component, representing a sinusoidal wave of electron density with a particular frequency and direction determined by its scattering vector $\vec{q}$, \textit{i.e.} by its position in Fourier space.
By knowing the phase relationships between these Fourier components, which can be obtained from the measured diffraction pattern, the electron density within the unit cell can be reconstructed.
The superposition of these sinusoidal waves, representing the diffraction peaks or Fourier components, recreates the original electron density distribution of the crystal.

To summarise, the diffraction peaks in the scattering pattern represent the Fourier components of the crystal's electron density distribution, and their spatial frequencies and phase relationships carry information about the crystal structure.
%The first sum is the unit cell structure factor, that modulates the scattered beam amplitude depending on what atoms are in the unit cell (amplitude) and their positions (phase).
%The second sum is the lattice factor that takes the value of $N_{uc}$ if $\vec{q}.\vec{R}_k = n \times 2\pi$ $(\ n \in \mathbb{Z})$.

The structure factor $F_{uc}$ is the summation of the contribution of each atom $j$ at the position $\vec{r}_j$ of atomic form factor $f_j(\vec{q})$ in the unit cell for a given scattering vector $\vec{q}$.
The position of each atom is defined in the unit cell by eq. \ref{eq:AtomPos}, an example with the platinum atoms in its primitive unit cell is given in tab. \ref{tab:PtAtoms}.

\begin{equation}
    \label{eq:AtomPos}
    \vec{r}_j = x_j\vec{a} + y_j\vec{b} + z_j\vec{c}
\end{equation}

\begin{table}[!htb]
    \centering
    \begin{tabular}{@{}lllllllllllllll@{}}
    \toprule
    Atom & Pt & Pt  & Pt  & Pt  & | Pt & Pt & Pt & Pt  & Pt  & Pt & Pt & Pt & Pt & Pt  \\ \midrule
    x    & 0  & 0.5 & 0.5 & 0   & | 1  & 0  & 1  & 0.5 & 1   & 0  & 1  & 0  & 1  & 0.5 \\
    y    & 0  & 0.5 & 0   & 0.5 & | 0  & 1  & 1  & 1   & 0.5 & 0  & 0  & 1  & 1  & 0.5 \\
    z    & 0  & 0   & 0.5 & 0.5 & | 0  & 0  & 0  & 0.5 & 0.5 & 1  & 1  & 1  & 1  & 1   \\ \bottomrule
    \end{tabular}
    \caption{Position of platinum atoms in the face-centred cubic (FCC) unit cell. Illustrated in fig. \ref{fig:fcc}. The first four atomic positions  lead to the other positions by symmetry (space group 225).}
    \label{tab:PtAtoms}
\end{table}

Systematic extinctions ($F_{uc} = 0$) occur when certain diffraction peaks are forbidden due to the crystal's symmetry elements, such as the atomic positions.
For example in the FCC lattice of platinum, systematic extinctions occur when the value of Miller indices are not all even or all odds.
Otherwise, the intensity of a Bragg peak detected at the scattering vector $\vec{q}$ is equal to $4\times f_{Pt}(\vec{q})$ where $f_{Pt}(\vec{q})$ is the atomic scattering factor of platinum.

\subsection{Lattice factor} \label{sec:LatticeFactor}

If the scattering factor considers the composition of the unit cell, the lattice factor represents the contribution of the crystal lattice to the overall scattering amplitude.
It takes into account the arrangement of atoms or scatterers within the crystal lattice and their interaction with the incident wave.

By identifying the different parts of eq. \ref{eq:Fcrystal} and eq. \ref{eq:FcrystalFlatFuc} we can now isolate the lattice factor (eq \ref{eq:LatFactor}).
In the case of a simple cubic arrangement of unit cells in three directions perpendicular to each other, we can divide the sum over all the unit cells $N_{uc}$ in the crystal into three sums, each in one of the directions of the crystal lattice ($\vec{a}, \, \vec{b}, \, \vec{c}$) so that $N_{uc} = N_{uc, x} \times N_{uc, y} \times N_{uc, z}$.

\begin{gather}
    \label{eq:LatFactor}
    F_{lat} = \sum_k^{N_{uc}} e^{i\vec{q}.\vec{R}_k} = \sum_{n_x}^{N_{uc, x}} e^{i n_x\vec{q}.\vec{a}} \times \sum_{n_y}^{N_{uc, y}} e^{i n_y\vec{q}.\vec{b}} \times \sum_{n_y}^{N_{uc, z}} e^{i n_y\vec{q}.\vec{c}}\\
    F_{lat} = \prod_{j=\{x,y,z\}} \Bigg( \sum_{n_j}^{N_{uc, j}} e^{i n_j\vec{q}.\vec{a}_j} \Bigg)\\
    F_{lat} = \prod_{j=\{x,y,z\}} S_{N_{uc, j}}(\vec{q}.\vec{a}_j)
\end{gather}

\begin{gather}
    \label{eq:SFunction1}
    S_{N_{uc, j}}(\vec{q}.\vec{a}_j) = \sum_{n_j}^{N_{uc, j}} e^{in\vec{q}.\vec{a}_j} = \sum_{n_j}^{N_{uc, j}} e^{(i\vec{q}.\vec{a}_j)^n_j} = \frac{1-e^{(i N_{uc, j} \vec{q}.\vec{a}_j)}}{1-e^{(i\vec{q}.\vec{a}_j)}},\\
    \label{eq:SFunction2}
    |S_{N_{uc, j}}(\vec{q}.\vec{a}_j)| = \frac{\sin(N_{uc, j} \vec{q}.\vec{a}_j/2)}{\sin(\vec{q}.\vec{a}_j/2)},\\
    \label{eq:SFunction3}
    |S_{N_{uc, j}}(\vec{q}.\vec{a}_j)| \underset{N_{uc, j} \to \infty}{\longrightarrow} \approx \delta (\vec{q})
\end{gather}

Where $S_{N_{uc, j}}(\vec{q}.\vec{a})$ is a function whose modulus $|S_{N_{uc, j}}(\vec{q}.\vec{a})|$ tends towards a Dirac function when $N_{uc, j} \longrightarrow \infty$.
The condition to observe peaks in eq. \ref{eq:SFunction2} is for the denominator to be equal to zero: $\frac{\vec{q}.\vec{a}_j}{2} \equiv \pi$ which brings us back to the Laue condition: $\vec{q} = \vec{G}$ where $\vec{G}$ is a reciprocal space lattice vector.
Minor peaks appear when the numerator is equal to one: $N\frac{\vec{q}.\vec{a}_j}{2} \equiv \frac{\pi}{2}$, the distance between each minor peak is equal to $\frac{2\pi}{t_j}$ where $t_j$ is the thickness of the crystal in the direction $\vec{a_j}$.

To resume, the lattice factor $F_{lat}(\vec{q})$ defines the shape of Bragg peaks.
For a large crystal (\textit{i.e.} a large number of unit cells in three directions), ideal Bragg peaks will look like Dirac functions (in reality they are widened due to different experimental factors).
For small crystals such as nanoparticles, the fringes of the lattice factor will be further apart and their intensity more visible in contrast to the intensity of the main peak (fig. \ref{fig:3DDP}).
In a more general sense, the lattice factor is the Fourier transform of the crystal's lattice, the effect of the Fourier transform is that fringes will be visible in the reciprocal space in directions perpendicular to the crystal's surface.
The ability to effectively sample these fringes during experiments is at the root of the BCDI technique, to be able to perform the inverse Fourier transform of the Bragg peak to then retrieve the crystal shape (sec. \ref{sec:BCDI}).

\subsection{Coherence} \label{sec:Coherence}

Every x-ray source can be described by several parameters.
The brilliance $\mathcal{B}$ is a measure of the distribution of flux in both space and angular range, it indicates how the flux is spread over the source area and solid angle.

\begin{equation}
    \mathcal{B} = \frac{\unit{\photons \per \second}}{(\unit{\mm^2} \text{source area})(\unit{\milli\radian})^2(\qty{0.1}{\percent} \; \text{bandwidth})}
\end{equation}

The spectral flux, on the other hand, quantifies the number of photons passing through a defined area per unit bandwidth per second.
Synchrotron facilities aim to optimise both the photon flux and brilliance, a higher flux meaning faster experiments whereas a high brilliance is important for those that need a coherent beam.
In the last evolution from $3^{rd}$ to $4^{th}$ generation synchrotrons, the source size (convolution of the size of the photon source and the transverse size of the electron beam) has been reduced by an order of magnitude \parencite{Willmott}.

\begin{equation}
    \label{eq:Emittance}
    \epsilon = D \times \Delta\theta
\end{equation}
The emittance $\epsilon$ is defined as the product of the source size $D$ and the beam divergence $\Delta\theta$ (eq. \ref{eq:Emittance}), a low emittance is characteristic of a small source size and almost parallel emitted beams.

\begin{figure}[!htb]
    \centering
    \includegraphics[scale=0.8]{Figures/introduction/coherence_lengths.pdf}
    \caption{
    (a) Two waves become out of phase in the direction of the beam at a distance $L_L$ from being in phase after $N/2$ wavelengths for $\lambda$ and $(N+1)/2$ wavelengths for $\lambda - \Delta\lambda$. Since $N >> 1$, we can write $N\approx\lambda/\Delta\lambda$ and find eq. \ref{eq:LongitudinalCoL}.
    (b) A plane wave emitted from one extremity of the source is out of phase with another plane wave emitted from the other extremity with maximum angular divergence $\Delta\theta$ at a distance $L_T$ in the plane perpendicular to the direction of the beam, situated at a distance $R$ from the source.
    }
    \label{fig:CoherenceLengths}
\end{figure}

The longitudinal coherence length $L_L$ describes the distance in the direction of the beam after which two waves of slightly different wavelength become out of phase due to the small extent of the beam bandwidth $\Delta\lambda$ (fig. \ref{fig:CoherenceLengths} - a).
The bandwidth is linked to the quality of the monochromator used to obtain a monochromatic beam through the relation $\lambda/\Delta\lambda$ (eq. \ref{eq:LongitudinalCoL}).

\begin{equation}
    \label{eq:LongitudinalCoL}
    L_L = \frac{\lambda^2}{2\Delta\lambda}
\end{equation}

The transverse coherence length $L_T$ describes the distance in the plane perpendicular to the beam after which two waves become out of phase due to the finite source size $D$ and to the angular divergence of the beam $\Delta\theta$ (fig. \ref{fig:CoherenceLengths} - b).
By assuming that the source has a Gaussian profile, we can relate $D$ to the standard deviation of the beam $\sigma_{H, V}$ in the horizontal and vertical directions perpendicular to the beam (\cite{Willmott}, eq. \ref{eq:TransverseCoL}).

\begin{equation}
    \label{eq:TransverseCoL}
    L_{T,(H, V)} = \frac{\lambda}{2\Delta\theta_{(H, V)}} = \frac{\lambda R}{2 D_{(H, V)}} = \frac{\lambda R}{2\sqrt{\pi}\sigma_{H, V}}
\end{equation}

At synchrotrons, the vertical and horizontal source sizes are significantly different due to the shape of the electron beam, much more extended in the horizontal plane.
The horizontal transverse coherence length is usually smaller and the limiting factor.
By multiplying the two transverse coherence lengths $L_{T,H}$, $L_{T,V}$ and the longitudinal coherence length $L_L$ we obtain the coherent volume.
An example is given in sec. \ref{sec:SIXS} for the SixS beamline of the synchrotron SOLEIL.

To perform Bragg coherent diffraction imaging experiments for which the coherent volume must be larger than the sample, the beamline must have low vertical and horizontal emittances, high monochromaticity and a sample environment situated as far as possible from the source.
Therefore, the coherent flux $F_{coh}$ as a function of the wavelength is directly related to the brilliance and bandwidth of the source \parencite{Willmott}:

\begin{equation}
    \label{eq:CoherentFlux}
    F_{coh} = \mathcal{B} \lambda^2 \frac{\Delta \lambda}{\lambda}
\end{equation}

Indeed, to be able to resolve the interference fringes resulting from the lattice factor, a high brilliance is crucial since the intensity of the total scattered field, proportional to the scattering volume, is weak.
A detailed explanation of how to optimise BCDI experiments is given in sec. \ref{sec:BCDI}.

\textit{
So far we have limited ourselves to the kinematical approach of diffraction in the frame of this thesis which means that the scattering is considered weak enough so that multiple scattering can be ignored.
This allows us to assume that the sample is only a perturbation to the incident field and that the structure factor can be written as the Fourier transform of the electronic density (Born approximation).
We have also assumed that the incident wave is plane within the scattering volume and that the distance between the scatterer and the detector is large enough for the scattered waves to be considered planes (far-field or Fraunhofer regime).
}

    \section{SXRD}

\subsection{Crystal truncation rods}

Thus the diffraction intensity of the finite-sized crystal has diffuse streaks connecting all the Bragg points. The diffuse intensity far from the nodes is of order of magnitude $N^4$ compared with $N^6$ at the nodes.

Scattering that is sharp in two directions and diffuse in the third (referred to as a "rod" of scattering) must arise from a crystalline object that is localized in one dimension and extended in the other two.

We are then left with only the sixth component due to the sharply truncated surface. We will call these features "crystal truncation rods. '

We now wish to estimate the strength of the truncation rods in the Bragg geometry. We must first modify Eqs. (1) and (2) by including the x-ray coherence length, $m$ (measured in unit cells), of the experimental configuration. This broadens all the diffraction features to $\frac{1}{m}$ reciprocal units. The Bragg points then have intensity of order $N_1 N_2 N_3 m^3$, while the diffuse intensity is $N_1 N_2 m^2$

A typical penetration depth is $1 \mu m$ so $N_3 =10^3$ unit cells (perpendicular to the face). With $m \approx 100$ unit cells, this gives a relative intensity
$\frac{I(Bragg peak)}{I(truncation rod)} = N_3 m \approx 10^5$

More roughness means wider Bragg peaks and deeper valleys between the BP.

It is not clear that different detailed models of roughness could be distinguished at this level of accuracy. One central concept to all descriptions of crystal truncation rods, however, is the continuation of the crystal lattice into the roughened region (not defects, keeping symmetry).

\subsection{Reciprocal space mapping}


\subsection{Reflectivety}


\subsubsection{ROD}

    \section{BCDI}

\subsection{Coherent diffraction}

Bragg coherent diffractive imaging (BCDI) is a lensless x-ray imaging technique that uses computational algorithms in place of physical lenses to achieve high-resolution imaging. It can be used to visualize the Bragg electron density and atomic displacement fields of crystalline materials in three-dimensional (3D) detail and with nanometer resolution.

We focus only on elastic scattering as that is the main process exploited when studying the structure of materials (the x-ray photon is elastically scattered (energy is conserved) and both the incident and scattered photons have the same wavelength).

By adding up the scattered amplitudes of an arrangement of atoms, we can then get the diffracted amplitude for a crystal. Note that the following derivations only apply if the diffraction pattern is viewed at a distance far away from the diffracting object. This region is known as the far-field or Fraunhofer region.

\subsection{Scattering by strained crystals}


\subsection{Phase retrieval}

To understand the phase problem, it's important to know that when X-rays interact with a crystal, they are diffracted by the crystal lattice structure. The diffracted X-rays form a complex interference pattern of constructive and destructive interference, resulting in a series of spots known as diffraction peaks.

The intensities of these diffraction peaks can be measured experimentally, providing valuable information about the distribution of electrons within the crystal. However, the phases of the diffracted waves are not directly determined from the intensity data.

In order to obtain a complete and accurate picture of the electron density within the crystal, both the amplitudes and phases of the diffracted waves are needed. The problem lies in determining these phases, which is often challenging or even impossible to do directly from experimental measurements.

The phase problem arises because the intensities of the diffracted waves are obtained through measurement, but the phases are typically missing. Without the phase information, it is impossible to directly reconstruct the electron density distribution and obtain a detailed structural model of the crystal.

Basic algorithm is Error-Retrieval (ER), quick but can converge towards local minimum. 
The support corresponds to a shape when the object is included. The density of the object is thus equal to zero outside the support. (Fienup, 1978).

In practice, a slow convergence of the ER algorithm is often observed. The error-metric does not evolve and the algorithm is sort of stuck in a local minimum.

To overcome the problem of stagnation in local minima from ER, (FIenup, 1982) introduced the Hybrid-Input-Output algorithm, that differs in its application of real space constraints. Feedback parameter $\beta$, 
In practice, this adaptation is efficient and significantly enhances the convergence speed. It can be seen as a little perturbation that allows to leave a local minimum.


However, the HIO algorithm still fails sometimes, and this explains why the ER and HIO algorithms are generally used in combination.

\subsubsection{Oversampling}
Oversampling in Fourier transforms refers to the practice of sampling a signal at a higher sampling rate than the Nyquist rate, which is twice the highest frequency present in the signal. By increasing the sampling rate, more samples are taken per unit of time, resulting in a denser set of data points.

When it comes to Fourier transforms, oversampling can offer several advantages:

1. Increased frequency resolution: Oversampling provides finer frequency resolution in the frequency domain. The additional samples capture more detailed information about the signal's frequency content, allowing for better analysis and representation of high-frequency components.

2. Reduced spectral leakage: Spectral leakage occurs when the frequency components of a signal spread into neighboring frequency bins in the Fourier transform. By oversampling, the spectral leakage effect can be minimized since the frequency bins become narrower, making it easier to distinguish and analyze closely spaced frequencies.

3. Improved interpolation and reconstruction: Oversampling can enhance the accuracy of interpolation and signal reconstruction from the Fourier domain back to the time domain. With more densely spaced samples, the interpolation process can more faithfully reconstruct the original signal without introducing artifacts.

4. Mitigation of aliasing effects: Aliasing occurs when high-frequency components of a signal are mistakenly represented as lower frequencies due to insufficient sampling. Oversampling helps to reduce aliasing by capturing more data points, allowing for a more accurate representation of the original signal's frequency content.

5. Enhanced filtering and noise suppression: Oversampling can provide improved filtering capabilities and noise suppression. With more data points available, it becomes easier to design and apply filters in the frequency domain, effectively attenuating unwanted frequency components and reducing noise interference.

It's important to note that oversampling comes at the cost of increased computational requirements and data storage. More samples mean more data to process and store, which can impact the efficiency and memory requirements of Fourier transform operations.

In summary, oversampling in Fourier transforms involves sampling a signal at a higher rate than the Nyquist rate, offering benefits such as increased frequency resolution, reduced spectral leakage, improved interpolation and reconstruction, mitigation of aliasing effects, and enhanced filtering and noise suppression. It is a technique commonly used when higher accuracy and detailed frequency analysis are desired in applications involving Fourier transforms.

\subsubsection{Fast Fourier transform}

The Fast Fourier Transform (FFT) algorithm is a widely used computational technique for efficiently computing the discrete Fourier transform (DFT) and its inverse.
The DFT is a mathematical operation that transforms a time-domain signal into its frequency-domain representation, revealing the spectral content of the signal.

The FFT algorithm was developed by Cooley and Tukey in the 1960s and revolutionized the field of digital signal processing due to its significant speed improvement over the conventional DFT calculation.

Simplified explanation of the FFT algorithm:

1. Input: The algorithm takes as input a sequence of N complex numbers, representing a discrete-time signal.

2. Splitting: The sequence is divided into two halves, containing the even-indexed and odd-indexed elements. This splitting forms a butterfly pattern, where each element in the upper half interacts with a corresponding element in the lower half.

3. Recursive computation: The FFT algorithm recursively applies the DFT to each of the two halves. This process continues until the sequence size reduces to a base case, often a sequence of length 2.

4. Butterfly operations: At each recursion level, a series of butterfly operations are performed. In a butterfly operation, pairs of elements from the two halves are combined using twiddle factors (complex exponential factors) and summed. The twiddle factors rotate and scale the frequency components of the input signal.

5. Combining: As the recursion unwinds, the computed DFT values from the lower levels are combined to form the final DFT of the original input sequence.

The key concept behind the FFT algorithm is the exploitation of symmetry and periodicity properties of the complex exponential functions involved in the DFT computation. By using these properties and recursively dividing the sequence, the algorithm achieves a significant reduction in the number of operations required, resulting in a faster computation compared to the straightforward DFT calculation.

The FFT algorithm has a time complexity of $O(N log N)$, where N is the size of the input sequence. This is a significant improvement over the $O(N^2)$ complexity of the direct DFT calculation.

The FFT algorithm is extensively used in various applications, such as digital signal processing, image processing, audio processing, telecommunications, and many scientific and engineering fields that involve spectral analysis or frequency domain operations on discrete signals.

Overall, the FFT algorithm provides an efficient and powerful tool for transforming signals between the time and frequency domains, enabling a wide range of applications that require efficient spectral analysis and manipulation of digital signals.

\subsubsection{Support determination}
There are several techniques to estimate the support. In some cases, the shape and dimensions of the object have been already determined by other techniques (such as SEM or AFM for instance), and a support can be built from this knowledge. 

\subsubsection{Patterson function}
When the shape of the object is unknown, a rough estimate of the support can be obtained from the diffraction signal using the autocorrelation function (Marchesini 2003). It is based on the Patterson function which can be defined as the invert Fourier transform of the diffracted intensity. This function can be expressed as the convolution of the complex electron density.

The size of the crystal is overestimated by the Patterson function, since it provides its autocorrelation. In practice, a non uniform density leads to a non-trivial shape of the autocorrelation. The function needs to be threshold to start with a reasonable approximation. In most of the reconstructions in this manuscript, the threshold was set to 2\% of the maximum of the Patterson function. As discussed by Vaxelaire (2011), the method is not adapted to highly strained objects.

For large strain, the diffraction pattern has a large extent in the reciprocal space
(Beutier et al. 2013a). As a consequence, the Patterson function underestimates the size of the object, preventing any chance of success in the phase-retrieval procedure. In summary, if the shape and size of the object is unknown, it is not recommended to use the autocorrelation function as a first estimate of the support in the case of an highly strained system.

In combination of the ER and HIO algorithms, a third algorithm is routinely used for CDI. It is known as the shrink wrap (SW) algorithm and allows to update the support during the reconstruction. It was first introduced by Marchesini (2003) and has proven to greatly improve the convergence of the procedure.

In practice, the estimate is smoothed by convolution with a Gaussian. After convolution, a thresholding is applied to the smoothed image to a typical value of 10\% of the maximum value of the amplitude. Values above the threshold are set to 1 and values below are set to 0. The threshold is generally set to such low values to avoid to suppress too large parts of the support. Nevertheless, the convolution step allows to recover from a support that has been reduced

\subsection{Accessing strain and displacement}


\subsection{Computer programs}

\subsubsection{\textit{PyNX}}
Coherent X-ray imaging techniques developed during the last 20 years thanks to high brilliance in synchrotrons. Wide range of techniques:

\begin{itemize}
    \item Phase Contrast Imaging
    \item Coherent Diffraction Imaging, allowing to reconstruct single objects from their diffraction pattern alone, including strain imaging of crystalline nano-objects in the Bragg geometry.
    \item X-ray Ptychography, used in both near and far field regime, developed for imaging extended objects (larger than the incident beam), both in small angle and Bragg geometry, also usable in the Fourier regime by scanning the transmitted beam.
\end{itemize}
These techniques all provide high resolution 2D or 3D imaging, down to 5 to 15 nm resolution, depending on the instrumental setup.

Requires a coherent X-ray beam, readily available at synchrotrons facilities. Will benefit from upgrades of synchrotrons rings, which promises 2 orders of magnitude increase in the available coherent X-ray flux thanks to higher brilliance. This will enable faster dynamics and imaging experiments as well as reaching higher resolutions.
Higher energy coherent X-ray ($\>20 keV$) will enable data collection for thicker samples and allow to mitigate radiation damage with lower absorption.

PyNX has open-source coherent X-ray imaging modules \parencite{Favre-Nicolin2020}. All calculations for coherent imaging modules (\texttt{cdi, ptycho, wavefront}), respectively for Coherent Diffraction Imaging, Ptychography and coherent wavefront propagation (mostly used for simulation purposes) are executed on the GPU by using pyCUDA or pyOpenCL libraries. Language and GPU automatically selected based on tests.

Autocorrelation

The CDI technique consists of reconstructing an object from a far-field diffraction pattern alone, a technique which has been expanded to 3D reconstruction by collecting multiple ($>100$) projections around a rotation axis, either in the small angle, or in the Bragg geometries - the latter approach yielding information about strain in the reconstructed object \parencite{Li2020}.

In order to recover the object from non-redundant diffraction data, it is necessary to recover the lost phases of the measured amplitude (c.f. phase loss problem).

A variety of algorithms are available, all of which rely on alternating between a real-space estimate of the object and diffraction (Fourier) space, where an amplitude constraint can be applied from the measured intensity.


\subsubsection{\textit{bcdi}}

\subsubsection{Preprocessing}

Signal to noise ratio that influence FFT window size for cropping

Oversampling ratio

Poisson (Shot) noise 

    \newpage
\section{Synchrotron radiation for the study of materials} \label{sec:SIXS}

\subsection{Synchrotron radiation}

SOLEIL (Source Optimisée d’Énergie Intermédiaire du LURE (Laboratoire pour l’Utilisation du Rayonnement Électromagnétique)) is a 3rd generation synchrotron source facility build in 2006 and localized near Paris, France, which operates at an electron energy of $2.75 \, GeV$ with a storage ring diameter of $354 \, m$.

\begin{figure}[!htb]
    \centering
    \includegraphics[width=0.49\textwidth]{/home/david/Documents/PhD/Figures/sixs/RingSoleil.jpg}
    \includegraphics[width=0.49\textwidth]{/home/david/Documents/PhD/Figures/sixs/BeamlineSoleil.jpg}
    \caption{
    	The storage ring alternates between straight and curved sections (left), insertion devices are situated at curves sections and deliver the synchrotron radiation to the beamline (right).
    	Copyright by SOLEIL (\url{https://www.synchrotron-soleil.fr/en/research}).
    }
    \label{fig:SOLEIL}
\end{figure}

Synchrotron radiation is generated when electrons are accelerated to relativistic speeds and forced to travel in curved trajectories by strong magnetic fields.
As these high-energy electrons move along their circular paths, they emit horizontally polarized electromagnetic radiation, known as synchrotron radiation.
This radiation spans a broad spectrum, from infrared to X-rays, depending on the energy of the accelerated particles and the strength of the magnetic fields.

First, electrons are produced by an electron gun and accelerated in a linear accelerator (LINAC), then by a booster ring.
Secondly, once the electrons reach the desired energy, they are injected into a storage ring, which is a circular vacuum chamber surrounded by strong magnets.
The magnetic fields in the ring act as a guiding force, bending the electrons' trajectories into circular paths (fig. \ref{fig:SOLEIL} - left).

Due to their high energy, the electrons travel at speeds close to the speed of light, making them relativistic particles.
As they move through the curved paths, they experience centripetal acceleration, continuously changing direction, and emitting electromagnetic radiation tangentially to their trajectory as a result of their acceleration \parencite{Willmott, NielsenMcMorrow}.

By the use of insertion devices such as wiggler or undulators, synchrotron radiation is exceptionally bright and brilliant, with a high flux of photons and an excellent collimation, wigglers are best used when in the need of a broad energy range, whereas undulators produce beams with extremely high brilliance and narrow bandwidth.
Each beamline at SOLEIL begins with one of those insertion devices (fig. \ref{fig:SOLEIL} - right).

\subsection{The SixS beamline}

SixS (Surfaces Interfaces X-ray Scattering) is a wide-energy range ($[5, \, 20] keV$) beamline dedicated to the structural characterization of surfaces and interfaces (solid-solid or solid-liquid), as well as nano-objects in controlled environments by the means of surface-sensitive x-ray scattering techniques.

\begin{figure}[!htb]
    \centering
    \includegraphics[width=\textwidth]{/home/david/Documents/PhD/Figures/sixs/SixSBeamline.png}
    \caption{
		Schematic view of SixS, the monochromator is situated in the optical hutch, the focusing mirrors, the attenuators and the MED end-station are in the first experimental hutch, additional mirrors and the UHV end-station are in the second experimental hutch.
		Only the MED end-station was used in the frame of this thesis.
    }
    \label{fig:SixSBeamline}
\end{figure}

The beamline operates with a $U_{20}$ undulator situated before the front-end (shutter) which delivers a beam characterized by its horizontal and vertical full-width at half maximum (FWHM), respectively $\sigma_H = 900 \, \mu m$ and $\sigma_V = 20 \, \mu m$.

A diaphragm is placed just after the undulator to cut any parasitic signal, a double crystal monochromator (DCM) of silicium (111) allows the selection of a monochromatic beam of thin bandwidth $\Delta \lambda$ after receiving the white beam, the first crystal is cooled by liquid nitrogen due to the high intensity of the white beam.
The second crystal is slightly curved for the sagital focus of the beam whereas the tangential focus is handled by focusing mirrors (fig. \ref{fig:SixSBeamline}).
The shape and width of the beam can be controlled by the use of slits which act as a secondary source, primary slits are placed just before the DCM, and secondary slits just before the focusing optics when performing BCDI measurements.
The role of the primary slits is to select a homogeneous portion of the beam to work with to avoid fluctuations of the beam intensity.
The distance between each element is recapitulated in tab. \ref{tab:DistanceSixS}.
After the focusing optics are the attenuators that limit the intensity of the incident beam when working in the direct geometry, followed by the first experimental end-station, the multi-environment diffractometer (MED).

\begin{table}[!htb]
	\centering
	\begin{tabular}{@{}llllllll@{}}
	\toprule
	$U_{20}$ & Diaphragm & Ring & Primary  & DCM      & Secondary  & Focusing &  MED \\
	         &           & wall & slits    & Si (111) & slits      & optics   &  \\
 	\midrule
 	$0 \, m$ &$11.7 \, m$&$15 \, m$& $16.7 \, m$ & $18.8 \, m$ & $ 30 m$ &$\approx 30.5 \,  m$ & $31 \, m$
	\end{tabular}
	\caption{
		Distance between each main element that allow the focusing of a monochromatic beam on the sample stage at SixS.
	}
    \label{tab:DistanceSixS}
\end{table}

\subsection{Multi environment diffractometer}

The multi-environment diffractometer (MED) end-station at SixS can be used in either a vertical or horizontal configuration, and can accomodate a large variety of experimental chambers around the sample stage.

Situated at a distance $R = 31 \, m$ from the source, we can compute the transverse coherence length at the sample stage when working with an energy of $8.5 \, keV$ following eq. \ref{eq:TransverseCoL}.
For a Gaussian distribution, the FWHM is related to the standard deviation $\sigma$ by the relation $FWHM = 2\sqrt{2 ln (2) } \sigma$.
We obtain for the transverse coherent lengths $L_{T,H} = 3.34 \, \mu m$, and $L_{T,V} = 150.20 \, \mu m$.
The longitudinal coherence length can be computed from eq. \ref{eq:LongitudinalCoL}, the ratio $\lambda/\Delta\lambda$ approximately equal to $1\times10^{-4} \, keV$ at the energy of $8.5 \, keV$, which gives $L_L = 729 \, nm$.
These corresponds to perfect coherent lengths without taking into account the focalisation of the beam by the mirrors which has the effective effect of increasing the virtual source size and decreasing the transverse coherence lengths \parencite{vincentjacques2010}.

Secondary slits are placed just before the sample to remove any parasitic signal linked to the focalisation of the beam and to decrease the effective source size seen by the sample.
However, since they also have the counter effect of decreasing the total photon flux on the sample, the width of these slits is subject to a compromise to increase the coherent flux on the sample.

\begin{figure}[!htb]
    \centering
    \includegraphics[width=0.8\textwidth]{/home/david/Documents/PhD/Figures/sixs/OpticalSetup.png}
    \caption{
    	Optical setup used for Bragg coherent diffraction imaging at the MED end-station.
    }
    \label{fig:OpticalSetup}
\end{figure}

\begin{table}[!htb]
    \centering
	\begin{tabular}{l|l|l|l|l}
	     & Beam stop & FZP & OSA & Beam\\ \hfill
	    Diameter & $80 \, \mu m$ & $300 \, \mu m$ & $70 \, \mu m$ & $\approx 1 \, \mu m$\\
	\end{tabular}
	\caption{
	Diameter of the coherence focusing optics used to increase the coherent flux on the sample.
	}
    \label{tab:OpticsBCDI}
\end{table}

The use of additional focusing optics such as coherent refractive lenses (CRL), Kirckpatrick-Baez (KB) mirrors or Fresnel zone plates that preserve the coherent wavefront together with secondary slits that match the transverse coherent length of the beam has been proven to increase the coherent flux \parencite{Schroer2008, diaz_coherent_2009, Mastropietro2011}.
At SixS, Fresnel zone plates (FZP) are used together with an order-selecting aperture (OSA) and a beamstop that block the direct beam and the higher diffraction orders from the FZP (fig. \ref{fig:OpticalSetup}).

\subsection{Catalysis chamber}

A 3D view of the horizontal configuration with the high pressure surface diffraction reactor XCAT (X-ray CATalysis) mounted on the goniometer is illustrated in fig. \ref{fig:MEDDiffractometer}.

\begin{figure}[!htb]
    \centering
    \includegraphics[trim=100 225 100 275, clip, width=0.9\textwidth]{/home/david/Documents/PhD/Figures/sixs/Diffractometer.pdf}
    \caption{
    	Diffractometer used at SixS at the multi-environment diffractometer (MED) in the horizontal configuration, accomodating the XCAT reactor.
    	Copyright by Leiden Probe Microscopy (\url{https://leidenprobemicroscopy.com}).
    }
    \label{fig:MEDDiffractometer}
\end{figure}

This reactor is used for the study of heterogeneous catalysis with surface x-ray diffraction and includes an ion gun for the cleaning of the surface by sputtering.
Sputtering is a process in which atoms or ions from a solid target material are ejected or "sputtered" from the surface of the target due to the impact of high-energy particles.
The reactor also comes with a heater which can increase the temperature of the sample up to 800°C at ambient pressure.

Moreover, the pressure and atmosphere inside the volume of the reactor can be controlled by a mass flow controller, from a few $m\si{bar}$ to $1 \, \si{bar}$ (fig. \ref{fig:GasSupplySystem}).
The flow inside the reactor is set in standard cubic centimeters per minute (SCCM) for each gas, the standard bottles supported are argon ($Ar$), oxygen ($O_2$), nitrogen oxide ($NO$), carbon monoxide ($CO$) and hydrogen ($H_2$).
It is possible to change the type of gas used by computing the correct conversion factor for the mass flow controller, originally set to work with specific gases.
For example, the bottle of $H_2$ was replaced by a bottle of ammonia ($NH_3$) during the experiments involving the catalytic oxidation of ammonia.

Finally, the evolution of the product and reactant pressure during the catalytic reaction is probed by a residual gas analyser (RGA) which is connected to a leak from the reactor output.
The leak pressure is usually in the range of $1e^{-6} \, \si{bar}$ and the assumption is made that the partial pressure of each mass detected by the RGA after the leak is equal to the partial pressure in the reaction chamber multiplied by the same constant for all masses.

\begin{figure}[!htb]
    \centering
    \includegraphics[trim=75 50 100 50, clip, width=\textwidth]{/home/david/Documents/PhD/Figures/sixs/GasSupplySystem.pdf}
    \caption{
    	Gas supply system used at SixS together with the multi-environment diffractometer (MED) for the high pressure surface diffraction reactor.
    	The mixing of reacting gases is performed before the reaction chamber (MIX), and Argon is used as a carrier gas (MRS).
    	The leak to the RGA is represented as a side circuit from the reactor output.
    	Copyright by Leiden Probe Microscopy (\url{https://leidenprobemicroscopy.com}).
    }
    \label{fig:GasSupplySystem}
\end{figure}

When performing BCDI measurements, a smaller version of the XCAT is used which is light enough to be supported vertically by the goniometer.
The volume in the reacting chamber is the same.
Moreover, the same sample holder can be used which carries the graphite heater, the only setback being the lack of sputtering gun.

The measurement process for the collection of the scattered intensity is detailed in sec. \ref{sec:DataCollectionSXRD} for surface x-ray diffraction and in sec. \ref{sec:DataCollectionBCDI} for Bragg coherent diffraction imaging.
Different detectors can be accomodated on the diffractometer arm such as the Maxipix \parencite{ponchut_maxipix_2011} that, by combining a small pixel size ($55 \, \mu m$) with a large array $(515, 515)$ and a large dynamic counting range, is well situated for coherent experiments \parencite{Schavkan2013, Li2020}.

The XPAD detector \parencite{Basolo2005, Dawiec_2016} is used for surface x-ray diffraction experiments, its pixel size is equal to $75 \, \mu m$.
Different chips can be used that change the array size, respectively  $(515, 70)$ for the XPAD70 and $(515, 140)$ for the XPAD140.
The XPAD detector is usually set with its largest side along $\vec{c}$ to collect the highest range in $l$ during the measurements.
    \section{Computer programs}

With the upgrade of synchrotrons to more brilliant sources, BCDI beamlines (ID01 - ESRF, P10 - PETRA III, SixS and CRISTAL - SOLEIL, NanoMAX - MAX IV, 34-ID-C - APS) have received increasing attention from the scientific community.
Imaging experiments yield larger and more complex data that far exceed the basic diffraction pattern (a reconstructed BCDI measurement consists of a complex array in 3 dimensions).
In a general frame, the increase in flux at synchrotrons leads to quicker experiments, which in turn leads to an increased amount of data stored on the beamlines, often in different formats.

Data produced at synchrotron has reached a volume and complexity that can nowadays be considered as part of \textit{big data} \parencite{Alizada2017, Wang2018}, referring to vast volumes, generated at high velocities, and from various sources.
Big data becomes too complex or large to be processed and analysed using traditional tools or methods, originally written to simulate or fit reduced amounts of data.
In the specific case of Bragg coherent diffraction imaging, new tools have been developed during this thesis (sec. \ref{sec:Gwaihir}, \cite{Carnis2021c, Simonne2022}) to help create a fast and reproducible workflow.
Nevertheless, it is a more general way of working that must be adopted in synchrotrons to adapt and profit from the transition to big data \parencite{Wang2018}.
This does not only concern scientific analysis, but also the complexity of operating synchrotron beamlines, that could benefit from this transition to upgrade their performances \parencite{Diadem}.

First, synchrotron data must be stored in a comprehensive format (e.g. \textit{NeXuS} format, Könnecke et al \cite*{Konnecke2015}).
The stored files must not only contain detector images, but routinely include all of the important metadata (informational data) to provide a more comprehensive and detailed view of the experiments.

Secondly, synchrotrons must offer high performance computing (HPC) clusters \parencite{Wang2021}, and make the data available to their users through an interface compatible with the most common integrated development environment (IDE).

Finally, synchrotrons must either have an internal data analysis strategy, or trust their users to be able to upgrade and develop tools that, in a first stage, are able to handle large volumes of data.
In a second stage, those tools can take advantage from that volume to uncover valuable insights \parencite{Wang2016a, Khaleghi2019}.
These insights can lead to new discoveries and enhance our understanding of complex phenomena, driving scientific progress.

% The use of big data expands research capabilities by allowing studies to be conducted on a much larger scale whereas traditional scientific experiments often rely on a limited number of observations or samples, which may not capture the full complexity of real-world phenomena.

\subsubsection{Result reproducibility}

At synchrotrons, where experimental techniques become ever more complicated, the first step of result reproducibility consists in the capacity for scientists to replicate experiments while obtaining consistent data.
The data that stems from the experiment is also sometimes called \textit{raw data}, which is then \textit{reduced} (ordered and simplified, \textit{e.g.} by phase retrieval in BCDI or integration in SXRD) and \textit{analysed} to derive scientific \textit{results}.
The second step refers to the ability to replicate or re-run the data reduction and analysis processes from the raw data to obtain similar results.
A \textit{dataset} is then defined as a collection of data, \textit{e.g.} in BCDI a dataset is constituted by the raw data, the detector mask, the measurement metadata, the retrieved Bragg electronic density, the displacement and strain arrays, and the data reduction parameters.

Achieving reproducibility with big data can be challenging due to the massive volume, velocity, variety and complexity of the data involved.
However, linking reproducibility and big data is crucial for ensuring the credibility and reliability of the insights derived from large and complex datasets.

For example, it is possible to measure a single rocking curve in less than a minute at specialised BCDI beamlines, such as ID01 at the upgraded European synchrotron \parencite{Leake2019}.
The high complexity of the data reduction and analysis process in BCDI makes the replication of result complex for external scientists, that often do not have access to the same data reduction softwares, or to the value of the parameters used during the process.

Discussions about defining a set of rules that regulate research practice \parencite{Kretser2019}, and reduce the grey zone that includes scientific misconduct at all levels of academia \parencite{Kornfeld2016} are growing, raising awareness on reproducibility in the scientific community.

Staggering numbers \parencite{Baker2016} show that about \qty{65}{\percent} of scientists in the field of physics and engineering struggle to reproduce others' results, and about \qty{50}{\percent} fail to reproduce their own results.
These numbers can sometimes be linked to very precise environments and techniques, with experimental conditions and processes difficult to replicate between different laboratories, or to knowledge transfer from academia to industry \parencite{Sarwitz2015}.
However, according to the same study \parencite{Baker2016}, code availability, insufficient peer reviewing, and access to raw data contribute together to this issue.
Moreover, it is largely accepted that the publishing industry has its own role to play by facilitating peer-reviewing to promote reproducibility \parencite{Lee2017}.

Therefore, code-availability, access to raw data combined with metadata, and well-defined workflows are goals of utmost importance for experimental science \parencite{Munafo2017}.
Reproducible results lead to a global improvement of confidence in new techniques, such as BCDI, which could subsequently result in growth of interest and community.
Key problems are for example the difficulty to reconstruct highly strained objects, or the determination of a metric that would allow scientists to \textit{blindly} trust reconstructed data, permitting a fully automatised data reduction workflow.
Reproducibility is first permitted by the use of a common data reduction and analysis environment.

\subsubsection{\textit{Python} in the \textit{Jupyter} environment}

The \textit{Jupyter Notebook} environment \parencite{Perez2007, Kluyver2016} was chosen for the development of data reduction and analysis tools during this thesis for its versatile, user-friendly and browser-based interface.
Notebooks can be used to take notes during experiments, shared in the \textit{.ipynb} format or as \textit{.pdf} documents.
They allow the use of \textit{Python}, an accessible programming language that has gradually become one of the most popular, versatile \parencite{Perez2007, Behnel2011, Newville2016, Ronaghi2017}, and widely-taught \parencite{Ayer2014, Scopatz2015, McKinney2017, Boulle2019} programming languages in science.

Moreover, \textit{Jupyter Notebook} has proven to be an effective tool for the reduction and analysis of synchrotron data in terms of graphical user interface (GUI) \parencite{Martini2019a,Simonne2020}, but also in terms of supporting scientific communities looking for high-performance frameworks \parencite{jupyter_computing_4, jupyter_computing_1, jupyter_computing_3, jupyter_computing_2}.

Large scale facilities and institutions seek ways to provide remote-access to high performance computing services for their users, which combine existing solutions in an interactive and user-friendly environment.
To simplify the data reduction and analysis pipelines in fourth-generation synchrotrons, it is of critical importance to offer the possibility for external users to analyse raw data remotely, with access to computational environments.
\textit{Jupyter} (\url{https://jupyter.org/}) is particularly advantageous and was chosen by several institutions for such purposes, \textit{e.g}. Google (Google Colab), the EGI federation, or the European Synchrotron.

Remote access to high performance computational environments, interfaced with \textit{Jupyter Notebook} or \textit{JupyterLab}, is provided by \textit{JupyterHub}.
For example, specific hardware such as graphical processing units (GPUs), mandatory for accelerated phase retrieval with \textit{PyNX}, can be managed with \textit{JupyterHub}.
Researchers can create their own work-spaces, with direct access to tailored computational environments, while relying on system administrators that can efficiently manage complex environments accessible for all users.
Remotely accessing data avoids storage issues, which can quickly become problematic with current experiments.
Note that in the latest version \parencite{JupyterNotebook7}, real-time collaboration (RTC) will be supported, which also pushes \textit{Jupyter Notebook} forward as a tool for laboratory notebooks.

\subsection{\textit{Gwaihir}} \label{sec:Gwaihir}

BCDI relies on iterative algorithms to solve the phase lost during the measurement (\cite{Robinson2009}).
A 3D intensity distribution in the vicinity of a Bragg peak (stack of diffraction patterns forming a 3D reciprocal space map with the proper sampling) is collected from a sample illuminated with coherent light \parencite{Robinson2005}, and serves as input for phase retrieval.

\begin{figure}[!htb]
    \centering
    \includegraphics[width=0.66\textwidth]{/home/david/Documents/PhD/Figures/gwaihir/Packages.png}
    \caption{
    Flow chart illustrating the main steps in the BCDI data reduction workflow.
    \textit{Gwaihir} links the \textit{bcdi} and \textit{PyNX} packages \textit{via} its graphical user interface and command line scripts, resulting in a complete and understandable workflow.
    Opt. stands for optional.
    }
    \label{fig:Packages}
\end{figure}

There are three main steps to compute the strain, detailed in fig. \ref{fig:Packages}.
The raw data must first be pre-processed.
It can then be inverted \textit{via} phase retrieval, the phase containing structural information about the sample that are lost during the measurement.
Finally, it is possible to extract information on the physical state of the particle such as its shape or internal strain from the images after post-processing.
Several software packages were developed to solve these steps but none bring a comprehensive pipeline from start to finish.

For example, \textit{PyNX} \parencite{FavreNicolin2011} focuses on the phase retrieval step, \textit{bcdi} \parencite{Carnis2021c} on data pre-processing and post-processing, while \textit{Cohere} \parencite{Frosik2021} focuses on pre-processing and phase retrieval.
A few graphical user interfaces also exist, such as \textit{Cohere} \parencite{Frosik2021}, \textit{Phasor} \parencite{Dzhigaev2021}, and \textit{Bonsu} \parencite{Newton2012}.
Providing a workflow will reduce the time spent on data reduction for newcomers, and improve results reproducibility by facilitating sharing while keeping track of parameters and metadata.

\textit{Gwaihir} is a tool developed during this thesis, which brings together the \textit{PyNX} and \textit{bcdi} packages, binding them in a graphical user interface built for the \textit{Jupyter} framework \parencite{Kluyver2016}.
It provides an interface to the bleeding edge of data reduction in BCDI, and can be used both locally or remotely, offering an interactive and user-friendly interface with complex functionality satisfying both beginners and experts.

\textit{Gwaihir} works with \textit{Python} 3.9 and is licensed under the GNU General Public License v3.0.
The source code as well as the latest developments are available on GitHub, while each stable version is released on the \textit{Python} Package Index (PyPi), along with its documentation.

\subsubsection{Workflow for Bragg coherent diffraction imaging} \label{sec:Workflow}

\begin{figure}[!htb]
    \centering
    \includegraphics[width=0.7\textwidth]{/home/david/Documents/PhD/Figures/gwaihir/Workflow.png}
    \caption{Workflow steps taken in \textit{Gwaihir}; the circular workflow illustrates result reproducibility, a key concept, facilitated by using the \textit{cxi} architecture.}
    \label{fig:Workflow}
\end{figure}

\textit{Gwaihir} offers an interactive workflow meant to be reproducible, and resulting in the phase and amplitude of the probed object (fig. \ref{fig:Workflow}).
To illustrate this, the results of the following procedure on a dataset collected at the P10 beamline in at PETRA III are shown in fig. \ref{fig:GUI_file} (CXI dataset ID 195).

\begin{figure}[!htb]
    \centering
    \includegraphics[width=0.49\textwidth]{/home/david/Documents/PhD/Figures/gwaihir/CropDiffPatternRed.png}
    %\includegraphics[width=\textwidth]{/home/david/Documents/PhD/Figures/gwaihir/CropObjectAmplutideRed2.png}
    \includegraphics[width=0.49\textwidth]{/home/david/Documents/PhD/Figures/gwaihir/crop_phase_red.png}
    \caption{
    A 2D slice of a 3D coherent diffraction pattern is shown (left).
    The phase of the Bragg electronic density (\unit{\radian}) of the reconstructed object is shown (right), the facets are clearly visible.
    Image displayed \textit{via} \textit{JupyterHub}.
    The particle is \qty{300}{\nm} wide.
    }
    \label{fig:GUI_file}
\end{figure}

\subsubsection{Pre-processing} \label{sec:preprocess}

Data pre-processing aims at improving the quality of phase retrieval by optimising the size and content of the 3D array used as input \parencite{Ozturk2017}.
The minimal processing consists in loading the raw data and stacking it together as an input for the phase retrieval.
Intermediate optional steps can be added \textit{via} a YAML (YAML Ain't Markup Language) configuration file.

Once the raw data is collected, different pre-processing parameters can be modified to optimise the 3D diffraction intensity.
For example, the array must be centred for the Fourier transforms, the centre is either fixed manually as the centre of the Bragg peak if known, or determined as the centre of mass of the 3D array during the reduction process.
The array can also be cropped to reduce its size and decrease the computing time during phase retrieval by removing the points furthest from the Bragg peak where the signal-to-noise ratio is low (fig. \ref{fig:GUI_file}).

It is important to create a detector mask prior to the experiment to correct the raw data for hypothetical hot-pixels and uneven background.
Moreover, it is also possible to correct the raw data image by image for spurious data which taints the diffraction pattern.
For example, it is possible, while recording the 3D diffracted intensity of a given reflection, to have signal coming either from the substrate or from neighbouring objects that will be summed to the probed object's intensity.

Finally, it is possible to normalise the raw data by an intensity monitor, or compute the scattering vector $\vec{q}$ of the measurement, from the instrumental geometry and parameters.

Aside from the specific details of the experimental setup (diffractometer and setup geometry, detector type, file system), the majority of the data reduction process is beamline-independent \parencite{Yang2019a}.
Based on this observation, \textit{bcdi} leverages inheritance and transforms the raw data to a common data format used for phase-retrieval (fig. \ref{fig:Packages}).
It frees the user from having to learn or remember the technical details for each beamline and sets a common strategy across beamlines.

\subsubsection{Phase retrieval} \label{sec:phaseretrievalpynx}

To retrieve the phase from the diffracted intensity, \textit{PyNX} functions with different iterative algorithms \parencite{Gerchberg1972, Fienup1982, Fienup1978, Marchesini2007, FavreNicolin2020}.
\textit{PyNX}'s recent update includes mathematical operators \parencite{FavreNicolin2020} which represent most of the reconstruction operations traditionally used in phase retrieval, and is at the foundation of quick and interactive phase retrieval in \textit{Gwaihir}.
Indeed, state-of-the-art GPUs available onsite where data collection occurs enable almost real time data reduction.
This results in quick visualisation of the probed object's amplitude and phase, offering the possibility to optimise data acquisition during the experiment.
Such live feedback is critical to the success of an experiment, and provides broader opportunities in the type of experiments that can be carried on.

The different parameters such as e.g. the support threshold, the number of iterations for each algorithm, the object initialisation procedure (square, sphere, auto-correlation, ...) can be modified in the GUI (fig. \ref{fig:PRT}).
Initial guesses are given for each parameter but must be refined by the user to optimise the results.
The impact of each parameter can be viewed directly in the GUI after phase retrieval.
This allows to refine the phase retrieval input parameters before submitting a \textit{batch job}, that will spawn a sub-process on the computing cluster for phase retrieval.
This highly optimised procedure can yield dozens of solutions in a few minutes.

With well-tuned parameters and high quality datasets, phase retrieval converges towards the same solution, but with minor differences between each reconstructed object, related to the phase retrieval process.
The most important limiting factors for the convergence of the reconstruction algorithms are the quality of the measurement, and the amount of strain going through the particle.
In the case of highly strained object, part of the intensity will be scattered in a direction that does not fit the area collected by the detector, resulting in void areas inside the reconstructed electronic density.
Moreover, the support becomes very difficult to determine since the diffraction intensity does not correspond to the lattice factor anymore.

\begin{figure}[!htb]
    \centering
    %\includegraphics[width=\textwidth]{/home/david/Documents/PhD/Figures/gwaihir/phase_retrieval_tab.png}
    \includegraphics[width=\textwidth]{/home/david/Documents/PhD/Figures/gwaihir/PRTab.pdf}
    \caption{
    Phase retrieval tab in \textit{Gwaihir}.
    Parameters are separated into groups (files, support, point-spread function, algorithms, ...) and detailed in the \textit{Readme} tab.
    The object can be reconstructed through a batch job, submitted to the computing cluster in the backend, or with operators, that will plot the evolution of the reconstruction object in the Notebook.
    }
    \label{fig:PRT}
\end{figure}

\textit{Gwaihir} provides a wide range of selection criteria to find the best solution.
Each reconstructed object has a list of final attributes that can be used as a criterion for selection, such as the free log-likelihood \parencite{FavreNicolin2020a} or the standard deviation of the modulus of the reconstructed object.
In the case of crystallographic defects, specific metrics (Chi, Sharp, Max volume, ...) were derived that perform best depending on the type of defect in the object \parencite{Ulvestad2017}.
Following the selection criterion, it is possible to quickly identify the solutions of poor quality that must be ignored to create a set of best solutions.
\textit{PyNX} then offers a method that merges this set into a single solution by computing eigen-vectors for the selected solutions \parencite{FavreNicolin2020}.
An alternative approach also available in \textit{Gwaihir} is to take the average of the best solutions \parencite{Ulvestad2014}.

\subsubsection{Post-processing} \label{sec:postprocess}

Post-processing regroups methods applied to the complex output of the phase retrieval.
Once the solution with the best figure of merit is selected, it is possible to use the \textit{bcdi} scripts to process the object.
If the data is still in the detector frame (geometric transformation not applied during pre-processing), the data can be interpolated in the orthogonal laboratory frame ($\vec{z}$ downstream, $\vec{y}$ vertical up, $\vec{x}$ outboard) or in the sample frame ($\vec{z}$ out-of-plane, $\vec{y}$ perpendicular to the beam, $\vec{x}$ parallel to the beam) using a transformation matrix.
This allows an easier comparison between the object's evolution when probing different Bragg reflections \parencite{Lauraux2021}.
The correct beamline and instrumental parameters must be selected (\textit{e.g.} sample-detector distance, probing energy, detector pixel size, \textit{etc}...), usually constant throughout the experiment.
The geometric transformation can be realised using either the transformation matrix \parencite{Mark2005} or the \textit{xrayutilities} package \parencite{Kriegner2013}, depending on which reference basis is needed.

After phase unwrapping, a refraction and absorption correction are optionally applied, and possible phase ramp and phase offset are removed.
At this point the displacement and the strain component are calculated from the phase.
The phase origin can first have an impact when comparing the lattice displacement between different reconstructions \parencite{Atlan2023}.
In the case of weak strain, it can be sufficient to consider the centre of mass of the object as the origin of phase.
However, this can become quite complex in the case of defects or defaults in the object.
Special methods are defined to target this issue in \textit{bcdi}.
For example, Guizar-Sicairos et al. \parencite*{GuizarSicairos2011} and Hofmann et al. \parencite*{Hofmann2020} proposed a convenient method for the numerical calculation of phase gradients in the presence of phase jumps.

\subsubsection{Graphical interface}

\textit{Gwaihir} links in a unique user-friendly and interactive GUI the aforementioned packages, whilst offering 2D/3D browser-based data visualisation tools.
Therefore, the main purpose of \textit{Gwaihir}, is to offer a possibility for the user to work solely in the \textit{Jupyter} environment.
The graphical user interface is divided into 10 tabs, aiming to achieve a comprehensible but fluid workflow, whilst still separating each step.
A final \textit{Readme} tab contains information about the different methods and parameters used in the workflow, as well as a tutorial on the GUI.

\textit{Jupyter} natively offers multiple options for interactive data plotting.
Most of the figures displayed in the GUI are based on the \textit{matplotlib} package \parencite{Hunter2007}.
For example, it is possible to select a list of 3D complex data arrays, and to visualise 2D slices in each dimension of their amplitude or phase, with different colour-maps (diverging, sequential, cyclic) and scale (linear, logarithmic).

\begin{figure}[!htb]
    \centering
    %\includegraphics[width=\textwidth]{/home/david/Documents/PhD/Figures/gwaihir/3D_obj.png}
    \includegraphics[width=\textwidth]{/home/david/Documents/PhD/Figures/gwaihir/PlotTab.pdf}
    \caption{
    Visualisation tab in \textit{Gwaihir}.
    The selection of the data array, the colormap, and the contour of the resulting 3D object are performed through different widgets.
    }
    \label{fig:3D_object}
\end{figure}

Moreover, an interactive 3D visualisation tool is provided that relies on the \textit{ipyvolume} library \parencite{Breddeld2021}, itself built on top of \textit{ipywidgets}, and specifically designed to quickly render large 3D data arrays.
In the specific case of single volume rendering for the reconstructed object, the object's surface is defined by a threshold of its maximum density (fig. \ref{fig:3D_object}).
The object surface can be colour-mapped with the values of the displacement and strain retrieved during the data reduction.

More complex interaction with figures and images (\textit{e.g.} zoom, set the colorbar range) is implemented with \textit{Bokeh} \parencite{Bokeh}, a \textit{Python} library that transforms figures in interactive web-pages (fig. \ref{fig:BokehDetector}), that can also be displayed in \textit{Jupyter Notebook}.

\begin{figure}[!htb]
    \centering
    \includegraphics[width=0.66\textwidth]{/home/david/Documents/PhD/Figures/gwaihir/detector_slice_bokeh.png}
    \caption{
    Detector images can be viewed interactively with \textit{Bokeh}, to zoom on the data, and visualise, for example, the intensity collected on each pixel.
    }
    \label{fig:BokehDetector}
\end{figure}

\subsubsection{Command line scripts}

The complete workflow for data processing can also be launched from the command-line using \textit{Python} scripts, whereas the link between each package is based on \textit{BASH} scripts.
This approach is both fast and versatile, designed to quickly iterate on several datasets to test parameters values, but less intuitive compared to the interactive GUI.

A set of default parameters stored in a configuration file is used, that can also be overwritten by directly providing keywords directly in the command line.
For example, the same file can be used for different measurements by just changing the scan number in the command.
The configuration files are written in \textit{YAML} (fig. \ref{fig:YAML_file}).
% One may still manually mask the data, or manually select the best reconstruction before post-processing.
% This option allows a quick visualisation of the probed object.

\begin{figure}[!htb]
    \centering
    \includegraphics[width=\textwidth]{/home/david/Documents/PhD/Figures/gwaihir/yaml.png}
    \caption{
    Configuration file in YAML, a human-readable data-serialisation language with a minimalist syntax.
    A configuration file is generated for the pre-processing and post-processing scripts, as well as for the phase retrieval.
    }
    \label{fig:YAML_file}
\end{figure}

\subsubsection{The CXI database}

Raw and processed data can both be accessed for the BCDI technique through the \textit{.cxi} database (\url{https://cxidb.org/}, Maia \cite*{Maia2012}), which aims at creating a single data-storing architecture/format for coherent X-ray imaging experiments.
The data input/output follows Nexus definitions \parencite{Konnecke2015}.
Both clarity and consistency in data formatting encourages the reproducibility of the science produced (\url{https://www.panosc.eu/}) and guarantees the workflow.

\begin{figure}[!htb]
    \centering
    \includegraphics[width=0.7\textwidth]{/home/david/Documents/PhD/Figures/gwaihir/tree.png}
    \caption{
    Each parameter value used during the workflow is stored in the same \textit{.cxi} file, along with the results, in a nested architecture, displayed \textit{via} \textit{JupyterHub}.
    \text{instrument\_1} regroups parameters associated with the instrumental setup.
    \text{data\_1} is the collected diffraction pattern.
    \text{image\_2} regroups the parameters associated with phase retrieval.
    \text{data\_2} is the reconstructed Bragg electronic density chosen for post-processing.
    \text{image\_3} regroups parameters linked to data processing, as well as the processed amplitude, phase and resulting strain.
    \text{data\_3} is a link to the processed phase of the object.
    }
    \label{fig:TREE}
\end{figure}

Since not all beamlines provide self-explaining NeXuS datasets, it is the \textit{bcdi} package together with \textit{xrayutilities} \parencite{Kriegner2013} that allows the support of most of the coherent imaging beamlines: ID01 (ESRF), P10 (PETRA III), SixS and CRISTAL (SOLEIL), NanoMAX (MAX IV) and 34-ID-C (APS).
Data pre-processing will generate two files stored as NumPy arrays \parencite{VanDerWalt2011}, corresponding to the diffraction intensity and mask.
These two files are then used for phase retrieval, for which the final object is saved in a \textit{cxi} file, later used for data post-processing.
In the case of simulation, the simulated diffraction intensity can be stored as a NumPy array to start the workflow from phase retrieval.

In \textit{Gwaihir}, results sharing across teams and team members is facilitated by the creation of a single output file, respecting the \textit{CXI} \parencite{Maia2012}, and thus the NeXuS \parencite{Konnecke2015} architecture (fig. \ref{fig:TREE}), which pushes towards results reproducibility.

Key parameters are the transformation matrix used for the interpolation in the final frame, the voxel size of the resulting 3D array, the probed reciprocal space range after data pre-processing $(\delta q_x , \delta q_y , \delta q_z)$, the iso-surface threshold, \textit{etc}.
Comments or metadata, such as the horizontal and vertical coherence lengths, or beam size, if determined prior to the experiment, can also be saved.
The aim is first to have a comprehensible architecture, and secondly to be able to reproduce anyone result from a \textit{single file}.
On a small scale, results are easier to share between collaborators and more understandable, while on a larger scale peer-review is facilitated.

\subsection{BINoculars} \label{sec:BINoculars}

BINoculars \parencite{Roobol2015} is a data reduction software used to concatenate the scattering intensity from successive angular scans.
It can also be used to optionally change the data's reference frame (e.g. from angular space to $q$ space).
Those two steps can be performed simultaneously by assigning to each voxel of the 3D scattered intensity a position in the new frame.
From subsequent scans, it is possible that each position in the new frame was scanned multiple times, meaning that there were multiple contributions (possibly with different intensity attenuation correction) to the total scattered intensity at that position.
The correct intensity is therefore the division of the total scattering intensity in each voxel by the number of contribution to that voxel (eq. \ref{eq:BinocularsIntensity}).
If there are no contributions, the intensity is set to NaN (Not a Number).

\begin{equation}
    \label{eq:BinocularsIntensity}
    Intensity =
        \begin{cases}
            counts \, /  \,contributions  & \text{if contributions $\neq$ 0} \\
            NaN & \text{if contributions = 0}
        \end{cases}
\end{equation}

Different orthonormal frames can be of interest when analysing the data, the q-space ($q_x, q_y, q_z$) is useful to index Bragg peaks and to access the related inter-reticular spacing $d_{hkl}$.
Different lattices can be used to highlight the difference of geometry between the bulk structure and the surface structure (fig. \ref{fig:MapExampleBinoculars}).

\begin{figure}[!htb]
    \centering
    \includegraphics[width=0.66\textwidth]{/home/david/Documents/PhDScripts/SixS_2022_01_SXRD_Pt100/figures/Map_hkl_surf_or_2227-2283_patched.pdf}
    % \includegraphics[width=0.49\textwidth]{/home/david/Documents/PhD/Figures/sxrd_data/Pt100/maps/Map_hkl_hex_or_2227-2283_not_patched.png}
    \caption{
    Reciprocal space in-plane maps.
    The large high intensity peaks are Bragg peaks from the bulk structure, the low intensity peaks are from surface structures.
    The map is computed using a cubic unit cell, two hexagonal unit cells corresponding to the surface structures are drawn as diamonds.
    }
    \label{fig:MapExampleBinoculars}
\end{figure}

The reciprocal space in-plane maps collected during surface x-ray diffraction experiments are usually projected perpendicular to $l$ (or $q_z$), to show possible Bragg peaks from surface reconstruction (fig. \ref{fig:MapExampleBinoculars}).
They contain the sum of the scattering intensity in a small out-of-plane layer $\delta l$ (or $\delta q_z$).
It is of utmost importance to select a thin layer to not drown the surface signal with the background or bulk signal.

BINoculars also contains a sub-module (\textit{fitaid}) that allows the integration of the scattered intensity in the reciprocal space as a function of one axis.
Usually, to study the evolution of crystal truncation rods, the data is projected either in the ($h, k, l$) or ($q_x, q_y, q_z$) frame and then integrated in a square area around the CTR signal as a function of $l$ or $q_z$ (fig. \ref{fig:BinocularsBackground}).

The voxel size in each direction is set through a parameter, which raises a compromise between array size, resolution, and an interpolation process.
Moreover, to compare different maps, it is better to use the same resolution so that the intensities of the different signals are on the same scale.

When in the ($q_x, q_y, q_z$) space, it is possible to go as low as \qty{0.005}{\angstrom^{-1}} in each direction, the limit being set by instrumental parameters such as the detector pixel size, the amount of angular scans performed, the counting time, or the number of iteration per scans (and by the computer memory when computing or visualising the final space).

If the number of iteration is too low for the desired resolution, the process will yield an image with empty areas corresponding to voxels where no data could be recorded.
This is often the case during the collection of crystal truncation rods, at high values of $l$, since the region probed by the detector becomes thinner in $l$ with every iteration (sec. \ref{sec:DataCollectionSXRD} - \cite{Drnec2014}).
To avoid having to rely on an interpolation to fill these voxels, it is better to use a larger resolution when concatenating the images, and then to use the same step when integrating the data.

\begin{figure}[!htb]
    \includegraphics[width=\textwidth]{/home/david/Documents/PhD/Figures/sxrd_data/background_fitaid.png}
    \caption{
        Graphical user interface for the \text{fitaid} module of BINoculars.
        The CTR structure factor is computed using eq. \ref{eq:BinocSF} in a rectangular region of interest around the CTR signal.
        Four rectangular regions of interests around the signal are used to compute the background.
    }
    \label{fig:BinocularsBackground}
\end{figure}

To quantitatively compare the integrated intensity with simulated structure factors, it is mandatory to subtract the background intensity from the total intensity.
The background is calculated by taking the sum of all the values of selected background regions (fig. \ref{fig:BinocularsBackground}), around the CTR, corrected by the number of voxels in those area.
The average background value per voxel is then subtracted to each voxel in the CTR region of interest (ROI), the structure factor $F$ can then be computed as follows:

\begin{equation}
    |F|^2 \propto I_{roi} - \frac{N_{roi}}{N_{bkg}} \times I_{bkg}
    \label{eq:BinocSF}
\end{equation}

$I_{roi}$ is the integrated intensity in the ROI, $N_{roi}$ the number of voxels in the ROI, $I_{bkg}$ the integrated background intensity, and $N_{bkg}$ the number of voxels in the background.
% This becomes more complicated when there are additional signals such as powder rings that cross the CTR at different $l$ values.

A very fine resolution (e.g. \numproduct{0.005 x 0.005 x 0.05}) will result in extremely large arrays that can be difficult to manipulate on low resources computer.
For example, the maps shown in fig. \ref{fig:MapExampleBinoculars} have a three dimensional shape equal to (\numproduct{560 x 953 x 39}) which results in \num{2.1e7} voxels in the array.
Originally, this corresponds to the concatenation of 40 angular scans, each with 1010 steps, \textit{i.e.} containing 1010 2D detector images that have a shape equal to (\numproduct{240 x 560}).
This represents \num{5.5e9} pixels to process together to create such in-plane maps.
High performance computing clusters are therefore needed for the reduction of surface x-ray diffraction data with BINoculars, the data analysis is performed in a second step with programs such as \textit{ROD} \parencite{Vlieg2000}.

\subsection{\textit{ROD}} \label{sec:ROD}

Part of the aim of SXRD data analysis can be to comprehend the 3D structure of potential surface layers on top of the sample surface, as well as the magnitude of surface relaxations on its topmost layers.
Such information can be extracted by combining in-plane reciprocal space maps and crystal truncation rods, \textit{i.e.} by comparing the position and intensity of the scattered intensity (and structure factor) in reciprocal space with simulated data.

The primary challenge associated with surface x-ray diffraction lies in the complexity of data analysis, as researchers are limited to a small portion of the reciprocal space due to the extended data acquisition time \parencite{Gustafson2014}.
On one hand, a fine sampling of the reciprocal space will take hours which is hardly compatible with time-resolved experiments.
On the other hand, skipping areas of the reciprocal space could result in missing critical information if the symmetry of the surface structure is not known prior to the experiment.
For high symmetry structures, a small portion of the reciprocal space can sometimes be sufficient to have a final solution.
The typical approach for time-resolved experiments is to follow the intensity of a surface signal as a function of time.

\textit{ROD} is a computer program written by Vlieg et al. \parencite*{Vlieg2000}, that can be used qualitatively to simulate structure factors from a given structure, and quantitatively to refine the atomic positions within the structure by fitting the structure factors with the experimental data.
Since \textit{ROD} is written in \textit{C} and does not come with a graphical user interface, alternatives have been developed such as GenX \parencite{Bjorck2007, Glavic2022} written in \textit{Python} which mostly focuses on reflectivity.
In the frame of this thesis, \textit{ROD} was chosen for the analysis of SXRD data.

\begin{SCfigure}
    \centering
    \includegraphics[trim=0 1cm 0 1cm, clip, width=0.35\textwidth]{/home/david/Documents/PhD/Figures/introduction/Pt3O4.pdf}
    \caption{
        \ce{Pt_3O_4} bulk unit cell.
        Platinum atoms are situated on the faces on the cubic unit cell (e.g. $(0, 1/2, 1/4)$, $(0, 1/2, 3/4)$), while the eight oxygen atoms are inside the unit cell at the positions $(1/4, 1/4, z)$, $(1/4, 2/4, z)$, $(2/4, 1/4, z)$, $(2/4, 2/4, z)$ for $z=1/4$ and $z=3/4$.
    }
    \label{fig:Pt3O4_ROD}
\end{SCfigure}

The structure of our crystal must be understood as follows when working with surfaces, infinite in both in-plane dimensions, finite in the out-of-plane dimension.
The origin of the out-of-plane axis, commonly denoted $\vec{z}$, is situated at the surface of the crystal.
When $z$ is negative the \textit{bulk} structure of the crystal is described, whereas when $z$ is positive, the \textit{surface} structure of the crystal is described, usually only a few atomic layers thick where e.g. surface relaxation and reorganisation effects can be detected.

\begin{figure}[!htb]
    \includegraphics[width=\textwidth]{/home/david/Documents/PhDScripts/SixS_2022_01_SXRD_Pt100/simulations/rod_pt304/figures/rod_4_0_pt3o4.pdf}
    \caption{
        Simulation of out-of-plane structure factors computed with \textit{ROD}.
        Different layers within a single unit cell of \ce{Pt_3O_4} are used in (a), while a different amount of the same \ce{Pt_3O_4} unit cell is used in (b).
        Contribution from \ce{Pt3O4} are most important in the anti-Bragg region.
    }
    \label{fig:SimROD}
\end{figure}

Additional surface layers can be present on top of the main crystal structure, the example of a \ce{Pt_3O_4} platinum oxide is presented in fig. \ref{fig:SimROD}.
At least two layers in $\vec{c}$ are necessary to see a modulation of the out-of-plane signal, the Bragg peaks linked to the presence of a bulk oxide start to be clearly visible after 4 unit cells of \ce{Pt_3O_4} are present on the platinum surface.
The unit cell of \ce{Pt3O4} is presented in fig. \ref{fig:Pt3O4_ROD}.

In a second step, it is possible to refine the position of each atom in the unit cell to take into account possible strain along $\vec{c}$, that will have a significant effect in the shape of the crystal truncation rods as shown previously in fig. \ref{fig:CTRSimulation}.
The structure along $\vec{c}$ is refined by computing the shape of crystal truncation rods, and comparing them with experimental data.
The more rods measured, and the higher the extent of each rod in $L$, the more accurate the final structure.

    
% Results BCDI
    \chapter{Structural evolution of Pt nanoparticles during the oxidation of ammonia}
    \section{Introduction}

% Reaction
During the catalytic oxidation of ammonia, if Pt catalysts were first used when aiming at the production of \ce{NO}, Pt-Rh catalysts have then proven to be more effective since the 1920s when Rhodium was added to the catalyst \parencite{Handforth1934, Heck1982}.
Recently, the ammonia oxidation reaction over a \ce{Pt_{25}Rh_{75}}(001) single crystal was studied with a surface science approach by operando techniques such as near-ambient pressure X-ray photoemission spectroscopy (NAP-XPS) and surface x-ray diffraction (SXRD) combined with mass spectrometry \parencite{Resta2020a}.
It was shown that \ce{NO} production coincides with significant changes of the surface structure and the formation of a \ce{RhO_2} surface oxide.
Moreover, changes in the surface relaxation related to the catalyst selectivity as well as to the presence of an hysteresis cycle in the reaction were reported.

% Single crystal ok but
Using single crystals as models to understand \textit{in-situ} and \textit{operando} the structure-activity relationship offers some perks such as a large surface area which increases the surface signal (lowering the pressure gap with some techniques such as XPS), and the possibility to isolate specific facets to progressively build up the understanding of the role of the sample surface on the catalytic activity \parencite{Hejral2016, Resta2020a}.
However, they fail when aiming at closing the material gap since they only exhibit a single crystallographic orientation on their surface, and are much larger (almost a \unit{cm} large) than the width of Pt-Rh gauzes used for the oxidation of ammonia (few 10s of \unit{\micro\meter} large, \cite{Kaiser1909}).

% use nanoparticles
In order to push for the reduction of the material gap as well as the pressure gap in heterogeneous catalysis, Pt nanoparticles have been used in this study, approaching the size of individual grains on the industrial catalysts \cite{} while exhibiting many different types of facets on their surface.
Specifically, the catalytic structure-activity relationships of Pt nanoparticles will be investigated during the oxidation of ammonia, as a stepping stone to the study of Pt-Rh nanoparticles, recently studied as well during the oxidation of carbon monoxide \parencite{Kim2021}, in which the compositional strain must also be taken into account \parencite{Kawaguchi2019}.
Moreover, studies have shown that nanoparticle surface strain can be controlled, opening up a new path to tune and optimise nanoparticle catalysts \parencite{Zhang2014, Sneed2015, Wang2016}.

% sxrd
It is first intended to measure the total signal scattered from \ce{Al_2O_3}-supported platinum particles by taking advantage of the possibility to carry out grazing-incidence diffraction measurements at the SixS beamline of synchrotron SOLEIL.
At grazing incidences the beam footprint extends across the whole sample length, the scattered beam being then proportional to the ensemble behaviour of the nanoparticles \parencite{Nolte2008, Hejral2013, Hejral2016}.

As seen in sec. \ref{sec:SXRD}, truncated surfaces such as facets give rise to crystal truncation rods (CTR) in the reciprocal space, whose intensity as a function of the scattering vector and width in reciprocal space are related to the facet size, roughness and strain.

The average nanoparticle shape and structure will be probed by studying the intensity of crystal truncation rods in directions perpendicular to the expected facets on the nanoparticle surface.
Prior experiments with similar samples \parencite{Dupraz2017, Li2020, Lim2021, Dupraz2022} have shown that the particles exhibit a Winterbottom shape \parencite{Winterbottom1967, Boukouvala2021}, typical of nanoparticles epitaxied on a substrate, with mainly \{111\}, \{110\}, and \{100\} facets, and a [111] orientation perpendicular to the substrate.

By having the incident beam illuminating all of the particles, the CTR signal in e.g. the [111] direction will be the sum of the contribution from the [111] facet of every illuminated nanoparticle (as well as their [$\bar{1}\bar{1}\bar{1}$] facets).
Therefore, phenomena inducing structural change such as particle re-faceting/reshaping at a given condition are expected to be visible by an evolution of the different CTR.

% bcdi
To obtain a more detailed picture of the phenomena at play, surface x-ray diffraction which provides ensemble-averaged properties when studying nanoparticles will be followed by the use of another technique which will bring information on the structural behaviour of single nanoparticles.

With Bragg coherent diffraction imaging (BCDI), the individual three-dimensional (3D) structural response of isolated nanoparticles to changes in the gas mixtures will be measured under \textit{in situ} and \textit{operando} conditions, enabling us to probe in detail the structure evolution of single facets during heterogeneous catalytic reactions (sec. \ref{sec:FacetAnalysis}).

A careful in situ analysis of the properties of the nanoparticles (size, shape, strain, re-faceting, support interaction, oxide formation, etc.) in 3D during the oxidation of ammonia is of essential importance to gain more understanding of the behaviour of these nanocrystals during a catalytic reaction.

However, it must first be ensured that the nanoparticles do not move or change their epitaxial relationship with the substrate during the exposition to reacting conditions at high temperatures.
Therefore, the preceding SXRD experiment will also aim at resolving the evolution of the epitaxial relationship between the Pt nanoparticles and the \ce{Al_2O_3} substrate before measuring single nanoparticles with Bragg coherent diffraction imaging, during which the beam is reduced to micrometric size, and for which a movement of the nanoparticles prevents the measurements of rocking curves due to loss of alignment.
Both BCDI and SXRD can be applied to high gas pressures due to the x-rays’ high penetration in gasses.

\subsection{$\alpha-$\ce{Al_2O_3} supported platinum nanoparticles synthesis}\label{sec:PtParticles}

The platinum nanoparticles were synthesised thanks to a collaboration with the Israel Institute of Technology (Technion, collaboration with Dr. Eugene Rabkin), their use first showcased in the work by Dupraz et al. \parencite*{Dupraz2017}.
A 30 nm thick homogeneous layer of platinum is deposited at room temperature on a [001] oriented alumina ($\alpha-$\ce{Al_2O_3}) substrate, which crystallises in a hexagonal unit cell.
The Pt nanocrystals have their c-axis oriented along the [111] direction, normal to the (0001) plane of the $\alpha-$\ce{Al_2O_3} substrate.
A mask is then applied on the sample and a lithographic process route ensures that the platinum layer transforms to nanoparticles that are between 100 and 1000 nm large, epitaxied on the substrate surface.

Two different samples are used during the oxidation of ammonia.
On one hand, in order to obtain a more important scattered intensity with SXRD, the sample surface is homogeneously covered with Pt particles.
On the other hand, there is a need to obtain isolated nanoparticles for BCDI to avoid alien signal coming from neighbouring nanoparticles.
The application of a specific patterned mask during the lithographic process then yields a patterned sample with isolated nanoparticles in the middle of \qty{100}{\um} large squares, visible in fig. \ref{fig:Mask}.

The position of the squares is designated with Arabic numbers (row), letters (column) and roman numbers (large rectangle).
All of the position indicators are constituted of platinum nanoparticle as well, which allows the mapping of the sample's surface in Bragg condition to find the nanoparticles' positions (fig. \ref{fig:SampleMapping}).
The hole diameter in the mask changes in different areas of the sample and has a direct impact on the average nanoparticle size in those areas.
After dewetting and heating at \qty{1100}{\degreeCelsius} for \qty{30}{\minute}, the platinum nanoparticles exhibit a well-faceted Winterbottom shape observed in various BCDI measurements \parencite{Dupraz2017}.

\begin{figure}[!htb]
    \centering
    \includegraphics[width=0.49\textwidth]{/home/david/Documents/PhD/Figures/sample/mask.png}
    \includegraphics[width=0.49\textwidth]{/home/david/Documents/PhD/Figures/sample/litho1_brighter.png}
    \caption{
        Mask applied during sample preparation (left) and resulting pattern on the sample surface observed with an optical microscope (right).
    }
    \label{fig:Mask}
\end{figure}

The epitaxial relationship between platinum and $\alpha$-\ce{Al_2O_3} (also known as sapphire) has been shown to consist into the superposition of the Pt [111] plane on top of the $\alpha$-\ce{Al_2O_3} [0001] plane, \textit{i.e.} Pt[111]||Al$_2$O$_3$[0001] (fig. \ref{fig:Epitaxy} - \cite{Farrow1993}).
The \{110\} platinum planes are perpendicular to the Pt (111) plane, the second-neighbour Pt atoms are arranged in a hexagonal pattern.

\begin{minipage}{0.55\linewidth}
    \centering
    \includegraphics[width=\linewidth]{/home/david/Documents/PhD/Figures/introduction/Epitaxy.pdf}
    \captionof{figure}{
    Pt[111]||Al$_2$O$_3$[0001] epitaxy relationship.
    }
    \label{fig:Epitaxy}
\end{minipage}%
\hfill% Add horizontal space between minipages
\begin{minipage}{0.44\linewidth}
    \begin{equation}
        \epsilon = \big( a_{Pt} \frac{\sqrt{3}}{\sqrt{2}} - a_{Sapphire} \big) / a_{Sapphire}
        \label{eq:MisfitStrain}
    \end{equation}
\end{minipage}%

Lattice strain in diffraction is usually defined as the difference between the reference and experimental lattice parameter values, respectively $a_{ref}$ and $a$ (eq. \ref{eq:StrainDiffraction}).

\begin{equation}
    \epsilon = \frac{a - a_{ref}}{a_{ref}}
    \label{eq:StrainDiffraction}
\end{equation}

In this case, the misfit strain at the platinum-alumina interface can be computed using eq. \ref{eq:MisfitStrain}, the reference lattice parameter is equalled to the in-plane lattice parameter of sapphire, while the lattice parameter of the larger hexagonal arrangement of the platinum atoms on the (111) surface is equal to $\sqrt{3} a_{Pt} / \sqrt{2} $.
At room temperature, the misfit strain computed from literature values ($a_{Pt} = \qty{3.9242}{\angstrom}$, $a_{Sapphire} = \qty{4.7602}{\angstrom}$) is computed to be equal to $\epsilon = \qty{0.96}{\percent}$.

\subsection{Catalysis reactor calibration for near-ambient pressure \textit{operando} studies}

The catalytic activity of the platinum nanoparticles was first studied as a function of the temperature to (i) be certain that the sample is sufficiently catalytically active for the reaction products to be detected by the mass spectrometer, (ii) identify the catalyst light-off temperature and (iii) observe an evolution of the product selectivity as a function of the temperature.
Similar scans where also performed on the empty reactor/sample holder to ensure the absence of activity towards the oxidation of ammonia.

The heater temperature was calibrated by measuring the temperature in the reactor at different pressures by the means of a type K thermocouple, as a function of the current intensity (fig. \ref{fig:TempRamps} - a), \qty{10}{\minute} separate consecutive datapoints to ensure the heater stability.
The experimental data points were fit to be able to set the reactor to any temperature from \qtyrange{25}{600}{\degreeCelsius} when working under vacuum or at ambient pressure (\qty{0.3}{\bar} or \qty{0.3}{\bar} of \ce{Ar}).
The thermal conductivity of the gases involved in the oxidation of ammonia are in the similar order of magnitude (tab. \ref{tab:ThermalConductivity}).
Moreover, the same inert gas used to set the reactor pressure - Argon - is used as a carrier gas during the experiments, constituting at least \qty{80}{\percent} of the gas flow, and allowing us to assume that the temperature in the reaction chamber is well approximated.

\begin{table}[!htb]
\centering
    \begin{tabular}{@{}llllllll@{}}
    \toprule
     & \ce{Ar} & \ce{NH_3} & \ce{O_2} & \ce{NO} & \ce{N_2O} & \ce{N_2}& \ce{H_2O} \\
    \midrule
    \qty{300}{\kelvin} & \num{17.7} & \num{25.1} & \num{26.5} & \num{25.9} & \num{17.4} & \num{26.0} & \num{18.6} \\
    \qty{400}{\kelvin} & \num{22.4} & \num{37.2} & \num{34.0} & \num{33.1} & \num{26.0} & \num{32.8} & \num{26.1} \\
    \qty{500}{\kelvin} & \num{26.5} & \num{53.1} & \num{41.0} & \num{39.6} & \num{34.1} & \num{39.0} & \num{35.6} \\
    \qty{600}{\kelvin} & \num{30.3} & \num{68.6} & \num{47.7} & \num{46.2} & \num{41.8} & \num{44.8} & \num{46.2} \\
    \bottomrule
    \end{tabular}%
\caption{Thermal conductivity in \unit{\mW \per \meter \per \kelvin} of gases \parencite{ThermalConductivityOfGases}.}
\label{tab:ThermalConductivity}
\end{table}

A temperature ramp at a reactor pressure of \qty{0.3}{\bar} (fig. \ref{fig:TempRamps} - b) was carried out to probe the evolution of the reaction products as a function of the temperature under reacting conditions, \textit{i.e.} a constant gas flow of \qty{41}{\ml\per\min} of \ce{Ar}, \qty{8}{\ml\per\min} of \ce{O_2}, and \qty{1}{\ml\per\min} of \ce{NH_3}.
In the frame of this thesis, the same ratio between incoming gas flows in assumed to be respected between the different gas partial pressures, in this case the partial pressure of oxygen is then equal to \qty{80}{\milli\bar}, the partial pressure of ammonia to \qty{10}{\milli\bar}, and the partial pressure of argon to \qty{410}{\milli\bar}.

\begin{figure}[!htb]
    \centering
    \includegraphics[width=0.495\textwidth]{/home/david/Documents/PhDScripts/SixS_2022_01_SXRD_Pt100/gas_analysis/figures/ThermocoupleCalibration.pdf}
    \includegraphics[width=0.495\textwidth]{/home/david/Documents/PhDScripts/Test_Reactor_CO2_2021_01/Figures/TempRamp2.pdf}
    \caption{
        a) Temperature inside the reactor cell measured with a type C thermocouple under vacuum and different \ce{Ar} pressures.
        b) Partial pressures evolution under a constant gas flow (\qty{41}{\ml\per\min} of \ce{Ar}, \qty{8}{\ml\per\min} of \ce{O_2}, \qty{1}{\ml\per\min} of \ce{NH_3}) at a reactor pressure of \qty{0.3}{\bar} during increasing and decreasing (low transparency) temperature ramp to \qty{650}{\degreeCelsius} with 100 steps, each lasting \qty{10}{\second}.
        The partial pressure of oxygen is omitted for simplicity.
    }
    \label{fig:TempRamps}
\end{figure}

The catalyst light off temperatures for the production of \ce{N_2}, \ce{NO} and \ce{N_2O} can respectively be identified to be around \qty{300}{\degreeCelsius}, \qty{450}{\degreeCelsius}, and \qty{550}{\degreeCelsius}.
The excess of oxygen compared to ammonia increases the production of \ce{NO} at high temperatures, with a partial pressure similar to the partial pressure of nitrogen at \qty{650}{\degreeCelsius}.
No decrease of the nitrogen partial pressure can yet be detected as a function of the increasing temperature.
The production of nitrous oxide stays at low values during the temperature ramp, almost imperceptible from the background pressure.

According to these primary results, the study of the oxidation of ammonia was originally decided to be carried out at \qtylist{300;500;600}{\degreeCelsius}, temperatures before and after the catalyst light off (tab. \ref{tab:ConditionsNanoparticles}).

\begin{table}[!htb]
\centering
\resizebox{\textwidth}{!}{%
    \begin{tabular}{@{}lll|l|lll|l@{}}
    \toprule
    \multicolumn{3}{l|}{Gas Flow (\unit{\ml\per\min})} & Total pressure (\unit{\milli\bar}) & \multicolumn{3}{l|}{Partial pressures (\unit{\milli\bar})} & Targeted information \\
    \midrule
    \midrule
    \ce{Ar} & \ce{O_2} & \ce{NH_3} &  & \ce{Ar} & \ce{O_2} & \ce{NH_3} &  \\
    \midrule
    \midrule
    50 & 0 & 0 & 300 & 300 & 0 & 0 & Catalyst state without reactants (unactive) \\
    49 & 0 & 1 & 300 & 294 & 0 & 6 & Ammonia adsorption \\
    48.5 & 0.5 & 1 & 300 & 291 & 3 & 6 &  \multirow{4}{*}{\begin{tabular}[c]{@{}l@{}}Influence of (\ce{NH_3}/\ce{O_2}) ratio on the \\ catalyst selectivity and structure \end{tabular}}\\
    48 & 1 & 1 & 300 & 288 & 6 & 6 &  \\
    47 & 2 & 1 & 300 & 282 & 12 & 6 &  \\
    41 & 8 & 1 & 300 & 276 & 18 & 6 & \\
    50 & 0 & 0 & 300 & 300 & 0 & 0 & Returning to pristine state \\
    \bottomrule
    \end{tabular}%
}
\caption{Different atmospheres used to probe the ammonia oxidation on Pt nanoparticles with BCDI.}
\label{tab:ConditionsNanoparticles}
\end{table}
    \section{Collective behaviour of Pt nanoparticles: SXRD}

% Compare shape with wulff construction and other measurements
% Get particle average size from epitaxy
% Check if TEM in literature for reshaping
% Quantify amount of powder -> well epitaxied particles
% In plane strain from substrate peak and in-plane strain evolution from powder signal
% Double -110 facet, twin boundaries maybe ?

\subsection{Experimental setup for SXRD experiments in the vertical geometry}\label{sec:SXRDSetupV}

The MED diffractometer detailed in sec. \ref{sec:MED} was used in a vertical geometry (fig. \ref{fig:Diffractometer}), the incident angle $\mu$ was fixed to \ang{0.3}, so that the beam illuminates the entire sample surface, at an incoming energy of \qty{18.44}{\keV}, at which the photon flux is the highest at the SixS beamline.
% In-plane measurements were performed by rotating the in-plane sample angle $\omega$ together with the in-plane detector angle $\delta$.
% Out-of-plane measurements were performed by rotating the in-plane sample angle $\omega$ together with the in-plane and out-of-plane detector angles $\delta$ and $\gamma$, the incoming angle $\mu$ needing to stay at a constant low value to keep the grazing incidence of the beam.

In the vertical configuration, the sample must be exactly parallel to the plane drawn by the incoming beam and the vertical axis so that rotations of the sample do not change the orientation of the sample plane while measuring in-plane reciprocal space maps.
The alignment of the sample was performed in two consecutive steps, first the out-of-plane sample position was adjusted by using the direct beam, secondly any possible tilt of the sample surface was corrected by recording the intensity of the \textit{reflected} beam as a function of $u$ (when $\omega=\ang{0}$) or $v$ (when $\omega=\ang{90}$) angles of the hexapod (fig. \ref{fig:Diffractometer}).

\subsection{Epitaxial relationship under different atmospheres and temperature}

To ensure that the \ce{Al_2O_3} supported particles have the expected Pt[111]||Al$_2$O$_3$[0001] epitaxy relation with the substrate, the scattering intensity in the sample plane was measured.

\subsubsection{Room temperature study under inert atmosphere}

\begin{figure}[!htb]
    \centering
    \includegraphics[width=\textwidth]{/home/david/Documents/PhDScripts/SixS_2021_03_SXRD_NH3/figures/map/qxqyqz_38_40.pdf}
    \caption{
        In-plane reciprocal space map at room temperature.
        Starting from the centre of the map are 6 peaks corresponding to six (2$\bar{1}$0)-type Bragg peak from the substrate, thin circular powder signal for the $(111)$, $(200)$ and $(220)$ Bragg peaks, and 6 isolated $(220)$ Bragg peaks.
    }
    \label{fig:QxQyMap}
\end{figure}

The unit cell used to index the different Bragg peaks of platinum in this section is the FCC unit cell presented in sec. \ref{sec:ScatCrystal} to introduce the notions of crystals.
This first map of the reciprocal space was recorded under inert argon atmosphere at room temperature (fig. \ref{fig:QxQyMap}), and shows the six expected (220)-type in-plane peaks for [111] oriented nanoparticles.
The h, k, and l Miller indices must have the same parity for the Bragg reflection to be allowed for face-centred cubic crystals, leading to the extinctions of (110)-type Bragg peaks.
From the average position of these peaks in q-space is computed the average in-plane lattice parameter of the platinum nanoparticles $a_{Pt}=\qty{3.909}{\angstrom}$, from which the scattering vector $\vec{q}$ of the (200) and (111) Bragg peaks is computed (tab. \ref{tab:Reflections}).

Three arcs are drawn in fig. \ref{fig:QxQyMap} to underline the value of the (111), (200) and (220) scattering vectors.
If no (200) and (111) Bragg peaks are observed, powder signals are visible for each reflection, which shows that some of the nanoparticles have a random orientation.
Six (2$\bar{1}$0)-type Bragg peaks coming from the \ce{Al_2O_3} substrate are also observed at the lowest magnitude of the scattering vector ($q = \qty{2.637}{\per\angstrom}$, tab. \ref{tab:Reflections}).
From the average position of these peaks in q-space is also computed the in-plane lattice parameter of the substrate, $a_{Sapphire}=\qty{4.766}{\angstrom}$, and a first approximation of the in-plane misfit strain at the interface of the nanoparticles with the substrate, following eq. \ref{eq:MisfitStrain}: $\epsilon = \qty{0.45}{\percent}$, the values of the lattice parameter being averaged over the nanoparticles present on the sample and not representative of the platinum-alumina interface.

\begin{table}[htb!]
    \begin{minipage}{.475\linewidth}
        \centering
        \resizebox{\textwidth}{!}{%
            \begin{tabular}{@{}lllll@{}}
            \toprule
            (h k l) & $2\theta$ & q & Int & Int (\%) \\
            \midrule
            (1, 1, 1) & \ang{17.1255} & \qty{2.784}{\per\angstrom} & \num{6362.52}  & \num{100.0} \\
            (2, 0, 0) & \ang{19.7996} & \qty{3.215}{\per\angstrom} & \num{3251.30}  & \num{51.10} \\
            (2, 2, 0) & \ang{28.1439} & \qty{4.546}{\per\angstrom} & \num{2399.87}  & \num{37.72} \\
            (3, 1, 1) & \ang{33.1305} & \qty{5.331}{\per\angstrom} & \num{2913.37}  & \num{45.79} \\
            (2, 2, 2) & \ang{34.6492} & \qty{5.568}{\per\angstrom} & \num{843.07}   & \num{13.25} \\
            % (4, 0, 0) & \ang{40.2235} & \qty{6.430}{\per\angstrom} & \num{390.01}   & \num{6.13}  \\
            % (3, 3, 1) & \ang{44.0121} & \qty{7.007}{\per\angstrom} & \num{1155.08}  & \num{18.15} \\
            % (4, 2, 0) & \ang{45.2178} & \qty{7.189}{\per\angstrom} & \num{1054.07}  & \num{16.57} \\
            % (4, 2, 2) & \ang{49.8120} & \qty{7.875}{\per\angstrom} & \num{755.99}   & \num{11.88} \\
            % (5, 1, 1) & \ang{53.0614} & \qty{8.352}{\per\angstrom} & \num{808.25}   & \num{12.70} \\
            \bottomrule
            \end{tabular}%
        }
    \end{minipage}%
    \hfill
    \begin{minipage}{.475\linewidth}
        \centering
        \resizebox{\textwidth}{!}{%
            \begin{tabular}{@{}lllll@{}}
            \toprule
            (h k l) & $2\theta$ & q & Int & Int (\%) \\
            \midrule
            (1, -1, 2)   & \ang{11.0698} & \qty{1.804}{\per\angstrom} & \num{21.50} & \num{49.52} \\
            (1, -1, -4)  & \ang{15.1285} & \qty{2.461}{\per\angstrom} & \num{35.60} & \num{82.01} \\
            (2, -1, 0)   & \ang{16.2125} & \qty{2.637}{\per\angstrom} & \num{16.81} & \num{38.73} \\
            (0, 0, 6)    & \ang{17.8528} & \qty{2.901}{\per\angstrom} & \num{0.18}  & \num{0.42}  \\
            (2, -1, 3)   & \ang{18.5233} & \qty{3.009}{\per\angstrom} & \num{38.90} & \num{89.60} \\
            % (2, -2, -2)  & \ang{19.6734} & \qty{3.195}{\per\angstrom} & \num{0.59}  & \num{1.35}  \\
            % (2, -2, 4)   & \ang{22.2446} & \qty{3.607}{\per\angstrom} & \num{20.98} & \num{48.32} \\
            % (2, -1, -6)  & \ang{24.2056} & \qty{3.921}{\per\angstrom} & \num{43.41} & \num{100.0} \\
            % (3, -2, -1)  & \ang{25.0589} & \qty{4.057}{\per\angstrom} & \num{1.15}  & \num{2.65}  \\
            % (3, -1, -2)  & \ang{25.5964} & \qty{4.142}{\per\angstrom} & \num{1.47}  & \num{3.39}  \\
            \bottomrule
            \end{tabular}%
        }
    \end{minipage}%
    \caption{
        Scattering angle $\theta$, scattering vector magnitude $q$ and intensity of the scattered waves for different Bragg peaks as a function of the increasing scattering angle, computed for an energy of \qty{18.44}{\keV} using eq. \ref{eq:Bragglaw} and eq. \ref{eq:Fcrystal}.
    }
    \label{tab:Reflections}
\end{table}

\subsubsection{Study during ammonia oxidation before and after catalyst light off temperature}

In-plane maps were recorded on a smaller angular range (due to the long experimental time needed to collect large maps) at \qty{300}{\degreeCelsius}, a temperature slightly before the catalyst light off, and at \qty{500}{\degreeCelsius} and \qty{600}{\degreeCelsius}, both temperatures after the catalyst light off, under different atmospheres as detailed in tab. \ref{tab:ConditionsNanoparticles}.
The intensity was then integrated along a thin region around the scattering angle ($\delta/2$ in this geometry) around the value of the (111), (200) and (220) scattering angles.
These measurements are presented respectively in fig. \ref{fig:Epitaxy111}, fig. \ref{fig:Epitaxy200} and fig. \ref{fig:Epitaxy220}.

\begin{figure}[!htb]
    \centering
    \includegraphics[width=\textwidth]{/home/david/Documents/PhDScripts/SixS_2021_03_SXRD_NH3/figures/epitaxy/220.pdf}
    \caption{
        Integrated intensity in a \ang{1} range around the value of the (220) scattering angle, as a function of the in-plane sample angle $\omega$, presented for different atmospheres.
    }
    \label{fig:Epitaxy220}
\end{figure}

Two (220)-type Bragg peaks are measured in either condition, with a stable position and shape (fig. \ref{fig:Epitaxy220}), which shows that the crystals do not rotate around their [111] axis on the surface, and that they all share the same out-of-plane [111] orientation.
The intensity decreases as a function of the temperature, without an increase of the (220) powder signal, the peak shape is consistent during the experiment.
The loss of intensity could come from the loss of some nanoparticles during the change of conditions.

No (200) or (111)-type Bragg peaks have been seen to appear during the different measurements (figures visible in the appendix \ref{fig:Epitaxy200}, and \ref{fig:Epitaxy111}).
Moreover, there is no variation of the powder signal for the (200) Bragg peak, there is a slight decrease of the powder signal for the (111) Bragg peak which cannot be observed on the other powder signals.
% Additional powder diffractograms are visible in the appendix \ref{fig:PowderNanoparticles}, on which it is also clear that the powder and nanoparticles do not have the same in-plane lattice parameter (two peaks near the expected (220) Bragg peak), and that there is no variation of the in-plane lattice parameter of the nanoparticles as a function of the atmosphere.

Most importantly, these measurements confirm that the nanoparticles are still epitaxied on the substrate even at high temperature under highly aggressive environments (the ammonia oxidation reaction being strongly exothermic), proving a strong epitaxial relationship of the nanoparticles with the substrate

($\bar{1}\bar{1}\bar{1}$) and (111) facets are expected to be present on all nanoparticles, respectively at the interface with the substrate and at the top of the nanoparticles.
The measurement of crystal truncation rods resulting from those facets is the easiest to perform since the [111] direction is perpendicular to the sample.
For that reason, a [111]-oriented crystal truncation rod, identified in fig. \ref{fig:2DCTR111Particles}, was first studied to probe for the evolution of the (111) and ($\bar{1}\bar{1}\bar{1}$) facets.

(11$\bar{1}$) and (002) Bragg peaks can be identified respectively at $q_z \qty{\approx0.8}{\per\angstrom}$ and $q_z \qty{\approx1.8}{\per\angstrom}$.
Both peaks being visible along the same CTR shows that the nanoparticles do not have the same exact in plane orientation, due to different stacking of the (111) layers along the [111] axis, e.g. ABC or ACB stacking, rotated by \ang{180} \parencite{Jones2019}.
Low intensity CTRs in other directions due to other facets present on the nanoparticles can be seen around both Bragg peaks.
The peak at $q_z\approx1.58$ corresponds to the (2$\bar{1}$3) Bragg peak from the \ce{Al_2O_3} substrate ($q = \qty{3.014}{\per\angstrom}$, tab. \ref{tab:Reflections}), which is also at the origin of its own CTR in the [111] direction from the surface in contact with the nanoparticles, observed in the out-of-plane map in fig. \ref{fig:2DCTR111Particles}, parallel to the more intense Pt CTR.

\begin{figure}[!htb]
    \centering
    \includegraphics[height=7cm]{/home/david/Documents/PhDScripts/SixS_2021_03_SXRD_NH3/figures/ctr/CTR111_K_vmax100_edited.pdf}
    \includegraphics[height=7cm]{/home/david/Documents/PhDScripts/SixS_2021_03_SXRD_NH3/figures/ctr/CTR_300_edited.pdf}
    \caption{
        Left: Large out-of-plane reciprocal space maps in which a crystal truncation rod signal in the [111] direction can be identified, as well as weaker CTR in other directions passing through the Bragg peaks.
        The ($\vec{q}_x$, $\vec{q}_y$) plane was rotated around $\vec{q}_z$ to highlight the presence of signals from facets on the particles other than the (111) facet.
        Right: Crystal truncation rod signals in the $[111]$ direction as a function of $q_z$, presented for different atmospheres, at \qty{300}{\degreeCelsius}.
    }
    \label{fig:2DCTR111Particles}
\end{figure}

The background-subtracted CTR intensity was integrated in a square area as a function of $q_z$, presented in fig. \ref{fig:2DCTR111Particles} (right).
There is no visible evolution of the CTR shape, the anti-Bragg region where the relaxation effects are the most visible near $q_z=1.58$ being masked by the Bragg peak of the substrate.
Efforts were then focused on resolving the CTR signal around the Bragg peaks coming from different facets present on the nanoparticles surfaces.

\subsection{Particle reshaping during the oxidation of ammonia}

\begin{figure}[!htb]
    \centering
    \includegraphics[width=\textwidth]{/home/david/Documents/PhDScripts/SixS_2021_03_SXRD_NH3/figures/facets/facets_together.pdf}
    \caption{
        a) Projection perpendicular to $\vec{q}_z$ of a reciprocal space volume measured around the $(\bar{1}11)$ Bragg peak at \qty{300}{\degreeCelsius} under Argon.
        The three black dotted line are respectively a projection in the $(\vec{q}_x, \vec{q}_y)$ plane of the $[\bar{1}11]$, $[\bar{1}10]$ and $[010]$ directions.
        The red dotted line correspond to a parasite signal linked to the use of attenuators when the detector records the intensity of the Bragg peak.
        b-c) Slices in the $(\vec{q}_y, \vec{q}_z)$ plane after rotation of the $\vec{q}_x$ and $\vec{q}_y$ axis around the $\vec{q}_z$ axis by \ang{30} and \ang{60} to highlight the presence of crystal truncation rods in different directions.
        d) Slices in the $(\vec{q}_y, \vec{q}_z)$ plane to highlight the presence of a crystal truncation rods in the $[1\bar{1}1]$ direction.
        The interval delimited by the vertical white dotted lines is used for the qualitative analysis of the CTRs intensities.
    }
    \label{fig:FacetMaps}
\end{figure}

A volume of the reciprocal space was collected around the ($\bar{1}$11) Bragg peak to try to find the direction of the observed facet signals, and to quantify their evolution as a function of the different conditions.
Similar experiments have shown the oxygen-induced shape transformation of rhodium \parencite{Nolte2008} or platinum nanoparticles \parencite{Hejral2013}, which was also linked to the presence of surface oxides.
The data was computed in the q-space with different in-plane offsets, so that each rod observed in fig. \ref{fig:FacetMaps} (a) becomes parallel to the $\vec{q}_x$ direction for a certain in-plane offset, thereby facilitating the data analysis.

Besides the crystal truncation rods, there are two different parasitic signals going through the Bragg peak.
First, a curved signal almost parallel to $\vec{q}_y$ extends from $q_y = \qtyrange{-2.80}{-2.40}{\per\angstrom}$ in the ($\vec{q}_x, \vec{q}_y$) plane, coming from scattered x-rays in Bragg condition diffusing towards the detector when near the Bragg peak, and increasing the background signal.
Secondly, a very intense powder ring ($\approx \num{1e2}$) can be seen around the crystal truncation rod perpendicular to $\vec{q}_z$ (area with high intensity, $> \num{1e4}$).

\begin{figure}[!htb]
    \centering
    \includegraphics[width=\textwidth]{/home/david/Documents/PhD/Figures/introduction/stereographic_projection_top.pdf}
    \caption{
        Stereographic projection perpendicular to $[111]$ crystallographic orientation (for a face-centred cubic lattice).
        The circles describe the angle with the $[111]$ direction from \ang{0} (centre) to \ang{90} (outer-ring).
    }
    \label{fig:StereoTop}
\end{figure}

The identification of the crystallographic direction corresponding to the orientation of the different crystal truncation rods present around the Bragg peak was performed thanks to the stereographic projection perpendicular to the [111] direction presented in fig. \ref{fig:StereoTop}, which in the current experiment is parallel to $\vec{q}_z$.
Each dot represents a crystallographic direction, \textit{e.g.} [100], [110], etc.
The distance between the dots and the centre of the figure is a function of the angle with the [111] direction, whereas their angular distribution corresponds to the planar angle of the component perpendicular to the [111] direction.
Therefore, a signal that is for example at an angle of \ang{30} with the [111] direction and \ang{30} with the [$\bar{1}$01] direction can be identified to be in the [113] direction.

A CTR parallel to $\vec{q}_y$ is visible in fig. \ref{fig:FacetMaps} - (a), with an angle of \ang{72} with the [111] direction (fig. \ref{fig:FacetMaps} - (d)), thus corresponding to a signal in the [1$\bar{1}$1] direction.
After having first identified the [111] and [$1\bar{1}1$] directions, the identification of the direction of the other CTRs becomes straightforward.

We can identify two CTRs in the [$1\bar{1}3$] and [$0\bar{1}1$] directions, both have an angle of \ang{30} with the [$1\bar{1}1$] direction (visible in  fig. \ref{fig:FacetMaps} - a), and an angle of respectively \ang{30} and \ang{90} with the [$\bar{1}\bar{1}\bar{1}$] direction in the ($\vec{q}_x, \vec{q}_{y, 30}$) plane visible in fig. \ref{fig:FacetMaps} - (b).

A [001] oriented CTR visible at \ang{60} from the [$1\bar{1}1$] direction (fig. \ref{fig:FacetMaps} - a), has an angle of \ang{60} with the [$\bar{1}\bar{1}\bar{1}$] direction in the ($\vec{q}_x, \vec{q}_{y, 60}$) plane visible in fig. \ref{fig:FacetMaps} - (c).

Two [$11\bar{1}$] and [$\bar{1}11$]-oriented CTRs are expected at \ang{120} in the ($\vec{q}_x, \vec{q}_y$) plane with the [$1\bar{1}1$]-oriented CTR.
Each [$1\bar{1}1$]-oriented CTR is also linked to a [$\bar{1}1\bar{1}$]-oriented CTR in the opposite direction, which is why there is a [$1\bar{1}1$]-oriented rod at \ang{60} from the [$1\bar{1}1$] direction (fig. \ref{fig:FacetMaps} - a), with an angle of \ang{108} with the [$\bar{1}\bar{1}\bar{1}$] direction in the ($\vec{q}_x, \vec{q}_{y, 60}$) plane visible in fig. \ref{fig:FacetMaps} - (c).

\begin{figure}[!htb]
    \centering
    \includegraphics[width=\textwidth]{/home/david/Documents/PhDScripts/SixS_2021_03_SXRD_NH3/figures/facets/facet_signal_evolution_together.pdf}
    \caption{
    Evolution of the scattered intensity taken along a square area perpendicular to the [111] direction at three different positions in the ($\vec{q}_x, \vec{q}_z$) plane to probe the evolution of crystal truncation rods.
    }
    \label{fig:FacetSignal}
\end{figure}

Despite the high intensity parasitic signal, it was possible to determine the orientation of 5 crystal truncation rods.
From these results, the average shape of the particles on the sample is expected to be highly faceted, with not only small (\{100\}, \{110\}, \{111\}) but also high indices facets such as \{113\} present on their surface.

The intensity of the crystal truncation rods listed above was studied by measuring the scattered intensity as a function of $q_z$ in a square area in the ($\vec{q}_x, \vec{q}_y$) plane.
The same width was used for each study, the region of integration is detailed in fig. \ref{fig:FacetMaps} - (b-c-d) with white dotted lines.
The evolution of the integrated scattered intensity as a function of $q_z$ is presented if fig. \ref{fig:FacetSignal} for different atmospheres and temperatures.

The scattered intensity does not evolve at \qty{300}{\degreeCelsius} and \qty{500}{\degreeCelsius} as a function of the atmosphere in the reactor cell.
However, there is a transition between \qty{300}{\degreeCelsius} and \qty{500}{\degreeCelsius} when looking at the signal from the ($0\bar{1}1$) facets, that evolves from a double to a single peak when increasing the reactor temperature.
This can be due to defects present in the nanoparticles at lower temperature despite the high temperature annealing, that are removed when heating the temperature (sec. \ref{sec:TempRampBCDI}).
For example, \textcolor{Important}{Ask yves to detail again}

At \qty{600}{\degreeCelsius}, a progressive increase of the intensity is observed for the [$1\bar{1}1$], [$\bar{1}\bar{1}1$] and [$001$]-oriented rods, whereas a progressive decrease of the intensity is observed for the [$0\bar{1}1$], [$1\bar{1}3$]-oriented rods, starting after the introduction of oxygen in the reactor.
There is an increase of scattered intensity in fig. \ref{fig:FacetSignal} (i) at $q_z \qty{\approx 0.95}{\per\angstrom}$ which could be from (110) facets at \ang{35} with the [111] direction (fig. \ref{fig:StereoTop}).
The low signal-to-background ratio in that region makes it challenging to be certain of the existence of this CTR.

Moreover, the introduction of oxygen induces a global shift of the facet signals towards higher $q_z$ values which can be linked to a homogeneous compressive lattice strain in the [111] direction, shifting the position of the Bragg peak.
\textcolor{Important}{Distorted lattice means not cubic anymore but ?, check that}
The position of the [$0\bar{1}1$]-oriented CTR, whose direction is perpendicular to the [111] direction, follows that shift in $q_z$, but its intensity vanishes after having more than \qty{1}{\ml\per\min} of ammonia in the reactor cell.
A qualitative evolution of the strain in the [111] direction can be obtained by looking at the remaining CTR position in $q_z$ (fig. \ref{fig:FacetSignal} - g\&i).
The gradual increase of the oxygen partial pressure in the cell is accompanied by two additional changes of strain (i) from compressive to tensile strain when increasing the oxygen partial pressure from \qty{5}{\milli\bar} to \qty{10}{\milli\bar} in the reactor and (ii) from tensile to compressive strain when increasing the oxygen partial pressure from \qty{20}{\milli\bar} to \qty{80}{\milli\bar}, the reference position being taken under argon and ammonia atmosphere.

These result indicate a progressive reshaping of the nano-catalysts towards particles exhibiting mostly \{111\} and \{100\} facets, together with the decrease of the ammonia to oxygen ratio.
A progressive roughening of the particles could explain the loss of intensity from the [$0\bar{1}1$]-oriented CTR but not the increase in the intensity of the [$1\bar{1}1$] and [$\bar{1}\bar{1}1$]-oriented CTRs.

The catalytic activity of the particles was recorded \textit{via} the use of a mass spectrometer and shows the production of nitrogen, nitrous oxide and nitrogen oxide (fig. \ref{fig:RGASXRDNanoparticlesComparison}), proving the activity of the catalyst.
At \qty{600}{\degreeCelsius}, the production of \ce{N_2} is favoured at lower oxygen partial pressures in comparison with \ce{N_2O} and \ce{NO}.
The opposite effects takes place at high oxygen partial pressures, with a clear transition between the main products from \ce{N_2} towards \ce{NO} when the oxygen to ammonia ratio is equal to 8.
The original mass spectrometer signal as a function of time can be seen in the appendix \ref{fig:RGA300SXRDNanoparticles}, \ref{fig:RGA500SXRDNanoparticles}, and \ref{fig:RGA600SXRDNanoparticles}.

\begin{figure}[!htb]
    \centering
    \includegraphics[width=\textwidth]{/home/david/Documents/PhDScripts/SixS_2021_03_SXRD_NH3/figures/rga/product_comparison_carrier_pressure.pdf}
    \caption{
        Evolution of reaction product partial pressures during the SXRD experiment on the non-patterned sample containing Pt[111]||Al$_2$O$_3$[0001] nanoparticles, at \qty{300}{\degreeCelsius}, \qty{500}{\degreeCelsius}, and \qty{600}{\degreeCelsius}.
        Mean partial pressures during \qty{1}{\minute} at the end of each condition, recorded from a leak in the reactor output by a residual gas analyser (RGA) as detailed in sec. \ref{sec:XCAT}.
        The pressure under \qty{49}{\ml\per\min} of argon and \qty{1}{\ml\per\min} of ammonia has been subtracted.
    }
    \label{fig:RGASXRDNanoparticlesComparison}
\end{figure}

The same measurements at \qty{600}{\degreeCelsius} were not repeated at a fixed atmosphere which prevents us from knowing whether or not the observed phenomena would have occurred as a function of time under a fixed atmosphere.
Moreover, the reversibility of this effect and the formation of possible platinum oxides (\textit{via} large in-plane maps) could not be investigated from the lack of available experimental time.

Similar experiments on Pt nanoparticles (average size about \qty{50}{\nm}) have been carried at \qty{6.5e-6}{\bar} and \qty{0.5}{\bar} of oxygen by \cite{Hejral2013} which show the formation of \ce{Pt_3O_4} and $\alpha-$\ce{PtO_2} bulk oxides, the formation of high indices facets, a decrease of \{111\} facet signals and an increase of \{100\} facet signals.
Different effects have been observed in this study with an increase in intensity of \{111\} and \{100\} facets and the decrease in intensity of \{110\} facets during the oxidation of ammonia.

A study of the structure/activity relationship of each facet must be undertaken during the ammonia oxidation to better understand the role of the surface structure and of potential surface oxides in the catalytic reaction, which is the subject of chapter 4.
    \textcolor{red}{This Chapter should demonstrate that you have conducted a thorough and critical investigation of relevant sources.
Apart from a presentation of the sources of your data, this chapter allows you to critically discuss the data (whatever these data are, ‘quantitative’ or ‘qualitative’, primary or secondary), which is proof of good research. You can even do good research with poor data but you must demonstrate that you are aware of the data quality and accordingly are careful in your interpretations. Essentially, there are three aspects to consider:
\begin{enumerate}
\item	Reliability, which, for example, could depend on whether they are estimates or more direct evidence;
\item	Representativity, which is about how typical the data are; for example, you may have arguments why the very few cases are typical or you may carry out statistical tests;
\item Validity, which is about the relevance of the data for your case. Strictly speaking, sometimes no valid data are available but one may argue that there are other data which could be used as ‘proxies’.) 
\end{enumerate}
}

Bimetallic nanomaterials have raised more and more significant interest from worldwide researchers in recent years, their new physical and chemical properties deriving from synergistic effects between the two metals being highly desirable for catalytic applications \cite{}.
For example, bimetallic catalysts consisting either of two noble metals (Pt-Rh, Au-Pd) or a noble and a 3d/4d transition metal (e.g. Pt-Cu) can show tremendously higher catalytic activities and altered selectivity in both conventional thermal and electrochemical catalysis \parencite{Resta2020a, Carnis2021b}.

In the case of the oxidation of ammonia, if Pt catalysts were first used when aiming at the production of \nitricoxide, Pt-Rh catalysts have then proven to be more effective since the 1920s when Rhodium was first added to the catalyst \parencite{Handforth1934, Heck1982}.
Recently, the ammonia oxidation reaction over a Pt-Rh binary alloy was studied with a surface science approach by operando techniques such as near-ambient pressure X-ray photoemission spectroscopy (NAP-XPS) and surface x-ray diffraction (SXRD) combined with mass spectrometry \parencite{Resta2020a}.
It was shown that \nitricoxide production coincides with significant changes of the surface structure and the formation of a \ce{RhO_2} surface oxide.
Moreover, changes in the surface relaxation related to the catalyst selectivity as well as the presence of an hysteresis cycle in the reaction were reported.

Recent studies have shown that nanoparticle surface strain can be controlled, opening up a new path to tune and optimise nanoparticle catalysts \parencite{Zhang2014, Sneed2015, Wang2016}.
However, these more conventional techniques traditionally provide ensemble-averaged properties when studying nanoparticles.
With Bragg coherent diffraction imaging (BCDI), we intend to address under \textit{in situ} and \textit{operando} conditions the individual three-dimensional (3D) structural response of an isolated nanoparticle to changes in gas mixtures, enabling us to probe more deeply the details of catalytic phenomena \parencite{Fernandez2019, Passos2020, Dupraz2022}.

In that frame of mind, we aim at investigating in situ the catalytic structure-activity relationships of Pt nanoparticles during the oxidation of ammonia, as a stepping stone to the study of Pt-Rh nanoparticles.
A careful in situ analysis of the properties of the nanoparticles (size, shape, strain, refaceting, support, interaction, oxide formation, etc) in 3D is of essential importance to gain more understanding of the behaviour of these nanocrystals during a catalytic reaction.

\section{Experimental setup}\label{sec:BCDISetup}

The BCDI experiment was performed at the SixS (Surface Interface X-ray Scattering) beamline of synchrotron SOLEIL, France (sec. \ref{sec:SIXS}).
As detailed in sec. \ref{sec:Gwaihir}, one of the bottlenecks of the BCDI technique is its slow data reduction and analysis process.
Moreover, on $3^{rd}$ generation synchrotrons that offer a lower coherent flux (eq. \ref{eq:CoherentFlux}) than $4^{th}$ generation synchrotrons, the measurement time can also be very long.
At SixS, a rocking curve lasts between \qtyrange{20}{90}{\min} depending on the particle size, the quality of the alignment, the strain of the particle, \textit{etc.}
Once the raw data is obtained, the particle must be \textit{reconstructed} (sec. \ref{sec:PhaseRetrieval}), and - if of interest - the displacement and strain arrays can be retrieved.
The analysis workflow can take up to an hour, which totals to an average of two hours from the start of the measurement to the moment when the user has a good idea of the sample shape and structure.

SixS is a beamline that does not only carry out BCDI experiment, but also SXRD experiments, in the same experimental end-station (sec. \ref{sec:MED}).
When aiming at performing \textit{operando} catalysis experiments, switching from one setup to another can take up to a few days.
This leaves only a limited remaining amount of time to align the sample, find a suitable nanoparticle, and carry out the experimental plan.

Li \textit{et al.} \parencite*{Li2020}, who have first shown that the SixS beamline could be used to carry out BCDI experiment, also started to work on improving the BCDI measurement process.
By comparing continuous and step-by-step measurements, they have shown that continous scanning would result in the same data quality while decreasing the measurement dead-time by \qty{30}{\percent}, thereby paving the way for quicker BCDI measurements.

During this thesis, the measurement process was further improved by taking advantage of the new possibility to perform continuous \textit{on-the-fly} scans at SixS, while moving the hexapod holding the sample.
When using the coherence setup (fig. \ref{fig:OpticalSetup}) in the vertical geometry (fig. \ref{fig:Diffractometer}), the beam is focused on the sample and is about \qty{1}{\um} large vertically, the horizontal footprint depending on the incident angle between the beam and the sample.
By satisfying Bragg's law in a specular geometry, the incoming angle is set to the Bragg angle $\theta$ (eq. \ref{eq:Bragglaw}).
The scattered x-rays are then collected by setting the out-of-plane detector angle $\gamma$ at a position $2\theta$ (similar to fig. \ref{fig:EwaldSphereSpecular}).
Finally, by simultaneously moving the sample with the hexapod and recording the Bragg scattered intensity with the detector, it is possible to map the sample surface with a sub-micron resolution (fig. \ref{fig:SampleMapping}).
A nanoparticle with a width equal to \qty{300}{\nm} was identified with this technique, which is a good estimate of the spatial resolution that can be attained, limited by the hexapod resolution (\qty{\approx 500}{\nm}) and the beam size.

Both in-plane angles are kept to zero, this is the simplest possible measurement geometry since the nanoparticles on the sample have their c-axis oriented along the [111] direction, parallel to the normal of the sample holder ($\vec{z}$ direction in the laboratory frame).

\begin{figure}[!htb]
    \centering
    \includegraphics[height=5cm]{/home/david/Documents/PhD/Figures/sample/microscope_image.png}
    \includegraphics[height=5cm]{/home/david/Documents/PhD/Figures/sample/microscope_image_photon.png}
    \caption{
        Microscope image of the sample seen through the sapphire window of the PEEK dome (left).
        Map of the sample performed in Bragg condition (right), the high intensity (red) areas correspond to platinum nanoparticles.
        The letters, numbers and isolated nanoparticles in the centre of squares can be recognized on the sample.
        \textcolor{Important}{colorbar}
    }
    \label{fig:SampleMapping}
\end{figure}

On the other hand, \textit{Gwaihir} (sec. \ref{sec:Gwaihir}) was developped primarily for the SixS beamline to counter the long analysis process, which allowed a significant reduction in the analysis time from around an hour to a few minutes.
The following beamtimes profit from the new software by having a more \textit{solution}-driven experimental process.
Indeed, to be measured, a nanoparticle must be isolated, not too small (weak scattered intensity), not too big (loss of coherence, fringes not visible), and not too initially strained (difficult to obtain a good guess of the support).
These conditions are sometimes difficult to assert by simply looking at the diffraction pattern.
Therefore, quick inversion using \textit{Gwaihir} allowed a faster decision process regarding the continuation or not of the nanoparticles measurement.

Successfull measurements by \cite{Lim2021} have permitted the simulatenous use of BCDI measurements from SixS with measurements from other imaging beamlines (ID01 - ESRF, P10 - DESY), designing a robust method to identify defects in the real space with convolutional neural networks (CNN).

In the frame of this thesis, the required beam size was obtained with a Fresnel zone-plate (focal distance of \qty{20}{\cm}), which focused the beam down to \qty{1}{\um} (horizontally) $\times$ \qty{2}{\um} (vertically).
A coherent portion of the beam was selected with high precision slits by matching their horizontal and vertical gaps with the transverse coherence lengths of the beamline: \qty{20}{\um} (horizontally) and \qty{100}{\um} (vertically).
\textcolor{Important}{check consistency}
A circular beam-stop, and a circular order-sorting aperture, were used to block the transmitted beam, and higher diffraction orders, respectively (fig. \ref{fig:OpticalSetup}).
The sample was mounted in the dedicated XCAT reactor with the substrate surface oriented in the vertical plane on a hexapod mounted vertically on the MED diffractometer (fig. \ref{fig:MEDV}).

The BCDI experiment was performed in a vertical specular geometry at a beam energy of \qty{8.5}{\keV} (wavelength of \qty{1.46}{\angstrom}).
Three-dimensional (3D) diffraction data were collected with rocking curves of the rotation angle around the normal of the sample, the diffracted beam was recorded with a 2D MAXIPIX photon-counting detector (\numproduct{515 x 515} square pixels, \qtyproduct{55 x 55}{\um} wide) positioned on the detector arm at a distance of \qty{1.22}{\meter}.
The in-plane ($\omega$) and out-of-plane ($\mu$) angles of the sample were \ang{0} and \ang{18.6}, respectively at \qty{25}{\degreeCelsius}, when the in-plane ($\delta$) and out-of-plane ($\gamma$) angles of the detector were \ang{0} and \ang{37.2}.
The value of the scattering angle $\mu$ ($\theta$ in eq. \ref{eq:Bragglaw}) and the out-of-plane detector angle ($2\theta$) are expected to vary as a function of the temperature during the experiment due to the thermal expansion of the sample.

\subsection{Synthetizing platinum nanoparticle}

The platinum nanoparticles were synthetized thanks to a collaboration with the Israel Institute of Technology (Technion) \parencite{Dupraz2017}.
A 30 nm thick homogeneous layer of platinum is deposited at room temperature on a (100) oriented alumina ($\alpha-$\ce{Al_2O_3}) substrate.
The Pt nanocrystals have their c-axis oriented along the [111] direction normal to the (0001) sapphire substrate.
A mask is then applied on the sample and a lithographic process route ensures that the platinum layer transform to nanoparticles that are between 100 and 1000 nm large,  epitaxied on the substrate surface.
After dewetting and heating at \qty{1100}{\degreeCelsius} for \qty{30}{\minute}, the platinum nanoparticles exhibit a well-faceted shape as seen in various BCDI measurements \parencite{Dupraz2017}. \textcolor{Important}{Add more ref with these samples}
The room temperature lattice parameter of platinum (\qty{3.924}{\angstrom}) is close to that of (100) alumina (\qty{4.122}{\angstrom}) resulting in \qty{\approx 5}{\percent} in-plane lattice strain.\textcolor{Important}{Check}
The mask yields isolated nanoparticles in the middle of \qty{100}{\um} large squares (fig. \ref{fig:Mask}).
The position of the squares is designated with arabic numbers (row), letters (column) and roman numbers (large rectangle).
The hole diameter in the mask changes in different rectangle and has a direct impact on the nanoparticle size since more matter is deposited.

\begin{figure}[!htb]
    \centering
    \includegraphics[width=0.49\textwidth]{/home/david/Documents/PhD/Figures/sample/mask.png}
    \includegraphics[width=0.49\textwidth]{/home/david/Documents/PhD/Figures/sample/litho1.png}
    \caption{
    	Mask applied during sample preparation (left) and resulting pattern on the sample surface (right).
    }
    \label{fig:Mask}
\end{figure}

All of the position indicators are constituted of platinum nanoparticle as well, which allows the scanning of the sample's surface in Bragg condition to map an area and find the nanoparticles' positions (fig. \ref{fig:SampleMapping}).

\subsection{Catalysis reactor calibration}

The catalytic activity of the platinum nanoparticles as a function of the temperature was studied to make certain that the nanoparticles were sufficently catalytically active for the reaction products to be detected by the mass spectrometer.
The catalytic activity of the reactor without any sample was also monitored and proven to be nul (sec. \ref{sec:SXRD100}), to make certain that no reaction occurs without the sample.

The heater temperature was first calibrated by measuring the temperature in the reactor at different atmospheres, as a function of the current intensity (fig. \ref{fig:TempRamps} - a).
The experimental data points were then fit using a polynomial of degree four to set the reactor to any temperature from \qtyrange{0}{600}{\degreeCelsius} (fig. \ref{fig:TempRamps} - b) when working under vacuum or at ambient pressure (\qty{0.3}{\bar} or \qty{0.5}{\bar} of \argon).
The thermal conductivity of the gases involved in the oxidation of ammonia is in the similar order of magnitude (tab. \ref{tab:ThermalConductivity}).
Moreover, the same gas - Argon - is used as a carrier gas during the experiments, constituting at least \qty{80}{\percent} of the gas flow, and allowing us to assume that the temperature in the reaction chamber is well approximated.

\begin{table}[!htb]
\centering
    \begin{tabular}{@{}llllllll@{}}
    \toprule
     & \argon & \ammonia & \dioxygen & \nitricoxide & \nitrousoxide & \nitrogen & \water \\
    \midrule
    \qty{300}{\kelvin} & \num{17.7} & \num{25.1} & \num{26.5} & \num{25.9} & \num{17.4} & \num{26.0} & \num{18.6} \\
    \qty{400}{\kelvin} & \num{22.4} & \num{37.2} & \num{34.0} & \num{33.1} & \num{26.0} & \num{32.8} & \num{26.1} \\
    \qty{500}{\kelvin} & \num{26.5} & \num{53.1} & \num{41.0} & \num{39.6} & \num{34.1} & \num{39.0} & \num{35.6} \\
    \qty{600}{\kelvin} & \num{30.3} & \num{68.6} & \num{47.7} & \num{46.2} & \num{41.8} & \num{44.8} & \num{46.2} \\
    \bottomrule
    \end{tabular}%
\caption{Thermal conductivity in \unit{\mW \per \meter \per \kelvin} of gases \parencite{ThermalConductivityOfGases}.}
\label{tab:ThermalConductivity}
\end{table}

Two temperature ramps at a reactor pressure of \qty{0.3}{\bar} (fig. \ref{fig:TempRamps} - c, d) were carried out to probe the evolution of the reaction product as a function of the temperature under a constant gas flow (\qty{41}{\ml\per\min} of \argon, \qty{8}{\ml\per\min} of \dioxygen, \qty{1}{\ml\per\min} of \ammonia).
The excess of oxygen compared to ammonia is expected to favour the production of \nitricoxide at high temperatures (sec. \ref{sec:AmoOxiHC}).

\begin{figure}[!htb]
    \centering
    \includegraphics[width=0.49\textwidth]{/home/david/Documents/PhDScripts/SixS_2022_01_SXRD_Pt100/gas_analysis/figures/ThermocoupleCalibration.pdf}
    \includegraphics[width=0.49\textwidth]{/home/david/Documents/PhDScripts/SixS_2022_01_SXRD_Pt100/gas_analysis/figures/ThermocoupleFit03bar.pdf}
    \includegraphics[width=0.49\textwidth]{/home/david/Documents/PhDScripts/Test_Reactor_CO2_2021_01/Figures/TempRamp1.pdf}
    \includegraphics[width=0.49\textwidth]{/home/david/Documents/PhDScripts/Test_Reactor_CO2_2021_01/Figures/TempRamp2.pdf}
    \caption{
        a) Temperature inside the reactor cell measured with a type C thermocouple under vacuum and different \argon pressures.
        b) Polynomial fit of the temperature as a function of the heater current.
        Partial pressures evolution under a constant gas flow (\qty{41}{\ml\per\min} of \argon, \qty{8}{\ml\per\min} of \dioxygen, \qty{1}{\ml\per\min} of \ammonia) at a reactor pressure of \qty{0.3}{\bar} during increasing and decreasing (low transparency) temperature ramps to c) \qty{525}{\degreeCelsius} with 150 steps, each lasting \qty{10}{\second}, and d) to \qty{650}{\degreeCelsius} with 100 steps, each lasting \qty{10}{\second}.
        Oxygen is omitted for simplicity.
    }
    \label{fig:TempRamps}
\end{figure}

The first temperature ramp to \qty{525}{\degreeCelsius} shows that the air present in the cell was not well extracted before reaching a temperature of \qty{300}{\degreeCelsius} \qty{14}{\min} after starting heating, the pressure of \nitrogen and \water continuously decreasing until then.
Above \qty{300}{\degreeCelsius}, the pressure of \nitrogen and \nitricoxide starts to increase, with a sligtly lower activation temperature for \nitrogen.
No production of \nitrousoxide can be detected during this temperature ramp.

A second temperature ramp was carried out to \qty{650}{\degreeCelsius} to see if the the activation temperature for the production of \nitrousoxide could be achieved.
The partial pressure of each reaction product is lower at the beginning of the temperature ramp which shows that the remaining air was well extracted.
The activation temperature for the production of \nitrogen can be more easily identified to be around \qty{300}{\degreeCelsius} and near \qty{450}{\degreeCelsius} for \nitricoxide, which is in accord with the data from the first temperature ramp.

However, the activation temperature for \nitrousoxide in the second temperature ramp is at about \qty{450}{\degreeCelsius}, a temperature that was reached during the first temperature ramp as well, but without the detection of \nitrousoxide.
This could be linked to either an activation process of the catalyst that lowers the activation temperature for the production of nitrous oxide, or more probably to a very low signal to noise ratio that hides the evolution of the partial pressure due to the remaining air in the reactor, the partial pressure on \nitrousoxide being the lowest of all the detected gases.

According to these primary results, the study of the oxidation of ammonia with BCDI was originally decided to be carried out at \qtylist{300;500;600}{\degreeCelsius}, temperatures before and after the catalyst light off (tab. \ref{tab:Conditions}).

\begin{table}[!htb]
    \centering
    %\resizebox{\textwidth}{!}{%
    \begin{tabular}{@{}cclc@{}}
    \toprule
    \multicolumn{3}{c}{Gas flow (\unit{\ml\per\min})} & Targeted information \\
    \multicolumn{1}{l}{\argon} & \multicolumn{1}{l}{\ammonia} & \dioxygen & \multicolumn{1}{l}{} \\
    \midrule
    50 & 0 & 0 & Catalyst state without reactants (unactive) \\
    49 & 1 & 0 & \ammonia introduction influence \\
    48.5 & 1 & 0.5 & \multirow{4}{*}{\begin{tabular}[c]{@{}c@{}}Influence of \ammonia / \dioxygen ratio as a function\\ of the temperature and vice-versa\end{tabular}} \\
    48 & 1 & 1 &  \\
    47 & 1 & 2 &  \\
    41 & 1 & 8 &  \\
    \bottomrule
    \end{tabular}%
    %}
    \caption{}
    \label{tab:Conditions}
\end{table}

\section{Temperature ramp}\label{sec:TempRampBCDI}

% strain field energy: kinda useless
% peak width:
% rotate so z is good ...
% could add scans at 200, 250, 300 and 350 °C,
% 600°C actually ok under argon...
% need to put scan table in appendix

To de-corellate the effect of temperature from the effect of the catalytic reaction on the nanoparticle structure, the alumina-supported nanoparticles were gradually heated from \qtyrange{25}{600}{\degreeCelsius} under a constant \argon-based gas flow (\qty{50}{\ml\per\min}) and at a pressure of \qty{0.3}{\bar}.

The same nanoparticle (so-forth called nanoparticle Amaterasu) was tracked during the heating process, rocking-curves around the [111] Bragg peak were measured every \qty{50}{\degreeCelsius} to probe its structural evolution, with \qty{201}{steps}, each counting for \qty{5}{\second}, resulting in an angular step equal to \ang{0.005}.
The measurement of a rocking curve took approximately \qty{22}{\minute}, with \qty{25}{\percent} of dead time.
The use of iterative algorithm for phase retrieval is detailed in tab. \ref{tab:ReconstructionProcess}.

\begin{table}[!htb]
\centering
\resizebox{\textwidth}{!}{%
    \begin{tabular}{@{}llll@{}}
    \toprule
    Iteration & Algorithm & PSF & Description \\
    \midrule
    0-199 & HIO & False & Work on the gross identification of the support (typical support voxel \% = 20) \\
    200-599 & RAAR & False & Refining the support (typical support voxel \% = 20) \\
    600 & RAAR & True & \begin{tabular}[c]{@{}l@{}}Activate the use of a point-spread function (PSF) to take into account the partial coherence \\ and the response of the detector.\end{tabular} \\
    601-999 & RAAR & True & Refine the PSF shape, the support must be already well defined to avoid diverging. \\
    1000-1200 & ER & True & Further refine the support by reducing the algorithm flexibility. \\
    \bottomrule
    \end{tabular}%
}
\caption{Example of algorithm chain used in BCDI for the phase retrieval.}
\label{tab:ReconstructionProcess}
\end{table}

The surface of the Pt nanoparticle coloured by the surface voxel strain values is presented in fig. \ref{fig:Amaterasu}, only 4 temperatures are shown because the shape and strain of the nanoparticle does not evolve between \qty{25}{\degreeCelsius} and \qty{400}{\degreeCelsius} besides the removal of the defect.
A dislocation identified at the interface with the substrate at \qty{25}{\degreeCelsius} from its strain signature and missing pipe of electronic density \parencite{Dupraz2015} was removed by an instrumental mistake which resulted in a flash heating procedure, the heater going from \qtyrange{150}{800}{\degreeCelsius} for a few minutes before coming back to \qty{150}{\degreeCelsius}.

The displacement field could not be perfectly unwrapped around the dislocation, resulting in two positive strain regions (in red) in the surrounding low strain region.

\begin{figure}[!htb]
    \centering
    \includegraphics[width=0.49\textwidth]{/home/david/Documents/PhDScripts/SixS\_2021\_01/paraview/1414top.png}
    \includegraphics[width=0.49\textwidth]{/home/david/Documents/PhDScripts/SixS\_2021\_01/paraview/1414bottom.png}
    \includegraphics[width=0.49\textwidth]{/home/david/Documents/PhDScripts/SixS\_2021\_01/paraview/1483top.png}
    \includegraphics[width=0.49\textwidth]{/home/david/Documents/PhDScripts/SixS\_2021\_01/paraview/1483bottom.png}
    \includegraphics[width=0.49\textwidth]{/home/david/Documents/PhDScripts/SixS\_2021\_01/paraview/1588top.png}
    \includegraphics[width=0.49\textwidth]{/home/david/Documents/PhDScripts/SixS\_2021\_01/paraview/1588bottom.png}
    \includegraphics[width=0.49\textwidth]{/home/david/Documents/PhDScripts/SixS\_2021\_01/paraview/1675top.png}
    \includegraphics[width=0.49\textwidth]{/home/david/Documents/PhDScripts/SixS\_2021\_01/paraview/1675bottom.png}
    \caption{
        Surface of the reconstructed Amaterasu Pt nanoparticle at \qty{25}{\degreeCelsius} (before the start of the temperature ramp), at \qty{150}{\degreeCelsius} (after the flash annealing), at \qty{450}{\degreeCelsius} and at \qty{600}{\degreeCelsius} (\qty{60}{\minute} after the introduction of \ammonia).
        The surface is coloured by the values of the heterogeneous strain as described in eq. \ref{eq:StrainTensorzz} at each surface voxel, the limit of each facet is delimited by thick white lines.
    }
    \label{fig:Amaterasu}
\end{figure}

\subsection{Shape evolution}

The nanoparticle shape stayed roughly the same during the beginning of the temperature ramp and after the introduction of ammonia, the triangular [111] facet at the top of the particle can be recognized, with the same agency of the surrounding facets (fig. \ref{fig:Amaterasu} - [111] facet surrounded by six facets, [100], [010], [001], [1$\bar{1}$1], [11$\bar{1}$], [$\bar{1}$11]).
The bottom facet, in contact with the substrate, has a [$\bar{1}\bar{1}\bar{1}$] orientation.

% 25 -> 150, one more 113
The nanoparticle shape near the interface has become less round and more faceted from the disappearance of the defect, the surface strain between the interface and the particle has practically disappeared, while just one more [$\bar{1}\bar{1}3$] facet appeared on the side (fig. \ref{fig:AmaterasuFacetsEvolution} - a).
However, the total surface area occupied by the [$\bar{1}\bar{1}\bar{1}$], \{$\bar{1}10$\} and \{100\} facets has increased, the removal of the defect directly increasing the amount of voxels in the neighbouring facets (fig. \ref{fig:AmaterasuDefect}).

\begin{figure}[!htb]
    \centering
    \includegraphics[width=0.49\textwidth]{/home/david/Documents/PhDScripts/SixS_2021_01/paraview/WireframeStrain1414.png}
    \includegraphics[width=0.49\textwidth]{/home/david/Documents/PhDScripts/SixS_2021_01/paraview/WireframeStrain1483.png}
    \caption{
        The dislocation results in a large volume of the particle that is not visible, due to the large strains \textcolor{Important}{Better explain this}.
        Removing the dislocation increase the area of the particle that is then recognized as facets.
    }
    \label{fig:AmaterasuDefect}
\end{figure}

%150->450, loose all 113
If one expects a certain degree of symmetry from the equilibrium Winterbottom shape of a particle on a substrate \parencite{WINTERBOTTOM1967, Boukouvala2021}, this symmetry seems to only be reached at \qty{450}{\degreeCelsius} with three [$\bar{1}1\bar{1}$], [1$\bar{1}\bar{1}$] and [$\bar{1}\bar{1}$1] facets around the substrate for a total of 11 facets, which is also the only reconstruction without any \{113\} facets (fig. \ref{fig:AmaterasuFacetsEvolution}).
The surface occupied by the \{100\} facets was multipied by \qty{50}{\percent} between \qty{25}{\degreeCelsius} and \qty{450}{\degreeCelsius}, which is also the temperature at which most of the particle surface was recognized as faceted (fig. \ref{fig:AmaterasuFacetsEvolution}).

The facets around the bottom of the particle are the most subject to change during this temperature ramp while the facets at the top of particle are more stable (fig. \ref{fig:AmaterasuFacetsEvolution}).
Moreover, it is possible that due to the low imaging resolution of the experiment, it is impossible to distinguish the smallest facets present on the particle surface.
Indeed, fig. \ref{fig:AmaterasuFacetsEvolution} shows that the total surface area not recognized as facets from the algorithm is always near \qty{50}{\percent}.

\begin{figure}[!htb]
    \centering
    \includegraphics[width=\textwidth]{/home/david/Documents/PhDScripts/SixS_2021_01/FacetAnalyser/FacetSizeEvolution.pdf}
    \caption{
        a) Evolution of the number of facet from specific facet families on the particle.
        b) Evolution of the particle surface area occupied by specific facet families.
        c) Evolution of the particle surface area occupied by the $[111]$ top facet, the $[\bar{1}\bar{1}\bar{1}]$ bottom facet, the other $\{111\}$ facets ($\{\bar{1}1\bar{1}\}$ and $\{1\bar{1}1\}$).
        The surface not recognized as part of a facet by the algorithm (e.g. rough areas of the particle surface, edges and corners) is taken into account.
    }
    \label{fig:AmaterasuFacetsEvolution}
\end{figure}

% 450 -> 600, 110 and 113 appear
Interestingly, after the reshaping of the particle at \qty{450}{\degreeCelsius} that shows the disappearance of the \{113\} facets (fig. \ref{fig:AmaterasuFacetsEvolution}), the following rocking curves where impossible to reconstruct.
The disappearance of the \{113\} facets could have been the starting point of a thermally-induced mobility of the Pt atoms on the particle surface, only increasing with the temperature, and making it impossible for the algorithm to determine a fixed support for the reconstruction.

The introduction of \ammonia at \qty{600}{\degreeCelsius} seems to have stabilized the particle, which could again be succesfully reconstructed.
For the first time, a \{110\} facet appeared on the particle surface together with the return of a \{113\} facet, both seem to have formed at the junction of three [100], [$1\bar{1}1$] and [$\bar{1}1\bar{1}$] facets as seen in fig. \ref{fig:Amaterasu110}.
On the other side of the particle, a [$1\bar{1}1$] facet was replaced by a [$1\bar{1}0$] facet, without the appearance of a \{113\} facet.
The change is also visible in fig. \ref{fig:AmaterasuFacetsEvolution}, the relative surface occupied by the [111], [$\bar{1}\bar{1}\bar{1}$], \{110\} and \{113\} facets increasing, whereas the surface occupied by the \{$1\bar{1}1$\} and \{$\bar{1}1\bar{1}$\} decreases at \qty{600}{\degreeCelsius} under ammonia.

\begin{figure}[!htb]
    \centering
    \includegraphics[width=0.49\textwidth]{/home/david/Documents/PhDScripts/SixS\_2021\_01/paraview/110facet1588.png}
    \includegraphics[width=0.49\textwidth]{/home/david/Documents/PhDScripts/SixS\_2021\_01/paraview/110facet1675.png}
    \caption{
        Surface of the reconstructed Amaterasu Pt nanoparticle at \qty{450}{\degreeCelsius} under Argon and at \qty{600}{\degreeCelsius}, \qty{60}{\minute} after the introduction of \ammonia, highlighting the appearance of a [110] facet.
        The surface is coloured by the values of $\epsilon_{zz}$ at each surface voxel, the limit of each facet is delimited by white tubes.
    }
    \label{fig:Amaterasu110}
\end{figure}

When introducing \dioxygen at \qty{600}{\degreeCelsius} to study the oxidation of ammonia, the nanoparticle was definitely lost during the measurement, not to be found again, which underlines the difficulty to study a highly exothermic reaction with BCDI.
Indeed, according to \cite{}, the reaction can heat the catalyst to very high temperatures which in our case could have been the trigger for the loss of the particle during the measurement, especially since the particle had so far resisted to the beam during the temperature ramp to \qty{600}{\degreeCelsius}.
From this first set of results, the measurement of Pt nanoparticles with BCDI was decided to be carried out at \qty{300}{\degreeCelsius} and \qty{450}{\degreeCelsius}.

\subsection{Strain evolution} \label{sec:StrainTempRamp}

\subsubsection{Determination of strain}

Lattice strain in diffraction is usually defined as the difference between the  reference and experimental lattice parameter values, respectively $a_{ref}$ and $a$ (eq. \ref{eq:StrainDiffraction}).

\begin{equation}
    \epsilon = \frac{a - a_{ref}}{a_{ref}}
    \label{eq:StrainDiffraction}
\end{equation}

The values of the lattice parameter can be extracted from the position $\vec{G}$ of the Bragg peaks in reciprocal space \textit{via} eq. \ref{eq:QandD3} and eq. \ref{eq:Interplanarspacing}.
However, the information extracted from the retrieved phase in BCDI is more complex since, by measuring three non-coplanar Bragg peaks, one may retrieve the full strain tensor (eq. \ref{eq:StrainTensor}) from the reconstructed displacement field $\vec{u}_{\hat{q_x}, \hat{q_y}, \hat{q_z}}$ (eq. \ref{eq:DisplacementField}), $\hat{q_x}, \hat{q_y}, \hat{q_z}$ being three orthogonal basis vector in the reciprocal space \parencite{Karpov2019}.

\begin{equation}
    \vec{u}_{\hat{q_x}, \hat{q_y}, \hat{q_z}} =
     \begin{pmatrix}
        \vec{u}_{\hat{q_x}} \\
        \vec{u}_{\hat{q_y}} \\
        \vec{u}_{\hat{q_z}} \\
     \end{pmatrix}
     \label{eq:DisplacementField}
\end{equation}

\begin{multicols}{3}
    \begin{equation}
        \epsilon =
        \begin{bmatrix}
            \epsilon_{xx} & \epsilon_{yx} & \epsilon_{zx}\\
            \epsilon_{xy} & \epsilon_{yy} & \epsilon_{zy}\\
            \epsilon_{xz} & \epsilon_{yz} & \epsilon_{zz}
        \end{bmatrix}
        \label{eq:StrainTensor}
    \end{equation}
    \break
    \begin{equation}
      \epsilon_{ij} = \frac{1}{2}
        \Bigg(
        \frac{\partial \vec{u}_{\hat{q_i}}}{\partial \hat{q_j}}
        +
        \frac{\partial \vec{u}_{\hat{q_j}}}{\partial \hat{q_i}}
        \Bigg)
        \label{eq:StrainTensorIJ}
    \end{equation}
    \break
    \begin{equation}
      \epsilon_{zz} =
        \Bigg(
        \frac{\partial \vec{u}_{\hat{q_z}}}{\partial \hat{q_z}}
        \Bigg)
        \label{eq:StrainTensorzz}
    \end{equation}
\end{multicols}

The idea behind using the strain tensor is to identify shear components in the displacement field, \textit{i.e.} see if components in one direction depend on other directions.
This can also help to identify defects present in nanoparticles \parencite{Lauraux2021}.
Using a single Bragg peak, we can only retrieve one component of the displacement field, obtained from the division of the retrieved phase $\Phi$ of the scattered x-rays by the value of the scattering vector at the center of mass of the 3D Bragg peak (\textit{i.e.} $\vec{q} = \vec{G}$), following the assumptions for phase retrieval detailed in sec. \ref{sec:StrainBCDI}, eq. \ref{eq:FcrystalBCDI3} - \ref{eq:FcrystalBCDI7}.

In our case, the direction of the [111] scattering vector is perpendicular to the sample, along the $\vec{z}$ axis of the laboratory frame.
Therefore, the [111] scattering vector, $\vec{G}_{111}$, is so forth described as $\vec{q}_z$, of magnitude  $|\vec{q}_z|$, with the direction described by the unit vector  $\hat{q_z}$, to be consistent with the equations detailed above.
Our approach to the strain is considerably simplified since we can only correctly derive one component of the strain tensor, $\epsilon_{zz}$ (eq. \ref{eq:StrainTensorzz}) as detailed below in eq. \ref{eq:StrainFromPhase1} - \ref{eq:StrainFromPhase3}.

\begin{align}
    \label{eq:StrainFromPhase1}
    & \Phi =  \vec{q}_z.\vec{u} \\
    \label{eq:StrainFromPhase2}
    & \frac{\Phi}{|\vec{q}_z|} = \frac{\vec{q}_z}{|\vec{q}_z|}.\vec{u} = \hat{q_z}.\vec{u} = \vec{u}_{\hat{q_z}} \\
    \label{eq:StrainFromPhase3}
    & \vec{\nabla} \vec{u}_{\hat{q_z}} = \frac{\partial u_{\hat{q_z}}}{\partial z} = \epsilon_{zz}
\end{align}

\subsubsection{Homogeneous strain}

The deviation of the interplanar spacing from the room temperature value due to the thermal expansion of the crystal is expected to be homogeneous within the particle.
This isotropic \textit{homogeneous} strain ($\epsilon_{hmg}$) in the particle is removed by centering the Bragg peak before phase retrieval, otherwise resulting in a linear phase ramp \parencite{}.
It is of utmost important to study the evolution of the material first under an inert atmosphere to see if there is an evolution of the homogeneous strain not only due to the thermal expansion of the crystal, but also to global relaxation phenomena induced e.g. by the adsorption of molecules involved in the catalytic reaction.

\begin{figure}[!htb]
    \centering
    \includegraphics[width=\textwidth]{/home/david/Documents/PhDScripts/SixS\_2021\_01/Max.png}
    \caption{
        The maximum value of the 3D Bragg peak is used to center the intensity before phase retrieval and to compute the average interplanar spacing $d_{111}$.
    }
    \label{fig:MaxPeak}
\end{figure}

\begin{figure}[!htb]
    \centering
    \includegraphics[width=\textwidth]{/home/david/Documents/PhDScripts/SixS\_2021\_01/LatticeParameter.png}
    \caption{
        Evolution of the $d_{111}$ interplanar spacing and associated homogenous strain values by comparing with the measurement at \qty{25}{\degreeCelsius}.
    }
    \label{fig:HomoStrain}
\end{figure}

The average interplanar spacing was computed from the scattering vector $\vec{q}_z$, \textit{via} eq. \ref{eq:QandD3} (fig. \ref{fig:HomoStrain}).
The interplanar spacing follows approximately a linear increase from \qty{2.2517}{\angstrom} at \qty{25}{\degreeCelsius} to \qty{2.2627}{\angstrom} at \qty{600}{\degreeCelsius} as a function of the temperature.
\textcolor{Important}{Errorbars}

Interestingly, there is no change of the average interplanar spacing after removal of the defect at \qty{150}{\degreeCelsius}.
There is however a very strong increase of the interplanar spacing at \qty{600}{\degreeCelsius}, which decreases after the introduction of ammonia.

\subsubsection{Heterogeneous strain}

The remaining strain after phase retrieval is called the \textit{heterogeneous} strain ($\epsilon_{htg}$) \parencite{GREDIAC1996, FAVIER2007, Atlan2023}, the total strain observed during this experiment being equal to $\epsilon_{tot} = \epsilon_{hmg} + \epsilon_{zz, htg}$

In BCDI, it is the displacement of small unit blocks making up the crystal lattice that is observed.
Depending on the instrumental parameters, these small unit blocks, called \textit{voxels}, have a more or less large size.
In this study, they are approximately \qtyproduct{10x10x10}{\nm} large.
The strain is thus not directly related to the deviation between the interreticular planes, but to the displacement field of these unit blocks from their equilibrium position.
The outermost layers of the crystal only constitute a low percentage of the surface voxels (\qty{\approx 11}{\percent} if we consider 5 atomic layers separated by the value of the interplanar spacing $d_{111}$), which lowers the contribution of the surface strain to the total strain contained in the surface voxels, reducing the ability to properly resolve e.g. surface relaxation effects.
The use of padding during the reconstruction algorithm decreases the voxel size, but without relying on the sampling of high-frequency components of the scattering amplitude and therefore does not increase the strain resolution.

\begin{figure}[!htb]
    \centering
    \includegraphics[width=0.6\textwidth]{/home/david/Documents/PhDScripts/SixS\_2021\_01/FacetAnalyser/Angles.pdf}
    \caption{
        Angles between the $[111]$ direction (normal to the top facet and opposite to the bottom facet of the particle) and other crystallographic directions that describe the normals to the other facets of the particle.
    }
    \label{fig:Angles}
\end{figure}

The crystallographic orientation of a facet can be determined by computing the angle between the [111] direction and its normal, which is then related to a direction in real space (fig. \ref{fig:Angles}).
Facets with the same orientation (e.g. [100], [010], [001]) are grouped around the same value of the interplanar angle.
To be more precise, the angular value is computed as the average of the angle between the facet normal and the [111] direction plus the angle between the facet normal and the [$\bar{1}\bar{1}\bar{1}$] direction minus \ang{180}, \textit{i.e.} $(\angle (111, \vec{n}) + (\angle (\bar{1}\bar{1}\bar{1}, \vec{n}) -180))/2$.

If equivalent orientation translates into equal surface atomic structures (e.g. [111] and [$\bar{1}\bar{1}\bar{1}$]), the environment of equivalent facets is not always the same.
For example, it is important to differ between [1$\bar{1}$1] and [$\bar{1}$1$\bar{1}$] facets, the higher the value of the interplanar angle, the closer the facet is to the interface with the substrate, and thus the more its influence in prominent.
Moreover, if the [111] facet is at the top of the crystal and the furthest away from the substrate, the [$\bar{1}\bar{1}\bar{1}$] facet is expected to be fully in contact with the substrate.
Therefore, the mean value and standard deviation of the heterogeneous strain distribution on each facet is presented on fig. \ref{fig:AmaterasuStrain}, as a function of the angle between the facets normals and the [111] direction, to highlight this difference.

\begin{figure}[!htb]
    \centering
    \includegraphics[width=\textwidth]{/home/david/Documents/PhDScripts/SixS\_2021\_01/FacetAnalyser/FacetStrainEvolution.pdf}
    \caption{
        Mean value and standard deviation of the heterogeneous strain ($\epsilon_{zz, htg}$) distribution as a function of the angle between the normal of each facet on the particle surface and the $[111]$ direction.
        Upwards and downwards arrow are represented for respectively positive and negative strain.
    }
    \label{fig:AmaterasuStrain}
\end{figure}

% introduce in plane out of plane
It is important to realize that the displacement observed is \textit{only} in the [111] direction, which means that it translates deviations of the crystal structure perpendicular to the [111] and [$\bar{1}\bar{1}\bar{1}$] facets, but parallel to the [$\bar{1}$10] facets, themselves parallel to the [111] direction (fig. \ref{fig:Angles}).
For example, if there existed a displacement of the atoms on the \{100\} facets in the direction perpendicular to the [100] planes, its contribution to the displacement would not be visible.

Moreover, if the $\epsilon_{zz}$ component of the strain tensor on the [111] and [$\bar{1}\bar{1}\bar{1}$] facets is easily assimilated to variation of the interplanar spacing $d_{111}$ (positive strain is tensile strain, negative strain is compressive).
This physical meaning of the strain becomes more complex when observing facets that are neither parallel nor perpendicular to the [111] direction, such as the \{113\} or \{100\} facets.
For this reason, the following analysis of the heterogeneous strain is meant to be qualitative, a quantitative analysis could only be performed with the full strain tensor.
% The relation between in-plane and out-of-plane strain can be rationalized by the Poisson effect \parencite{Atlan2023}, which relates an in-plane tensile (compressive) strain to an out-of-plane compressive (tensile) strain (depending on the origin of the displacement field).

The values of the displacement are in general quite low (always less than \qty{0.2}{\percent}), which puts us far away from the BCDI limit discussed in sec. \ref{sec:StrainBCDI}.
Let's take the example of the particle at \qty{25}{\degreeCelsius} for which the absolute value of the heterogeneous strain is the highest.
We have $|\vec{G}| = \qty{}{\angstrom}$, the furthest value of the scattering vector probed during the measurement is $|\vec{q}| = \qty{}{\angstrom}$ which gives $(\vec{q}-\vec{G}).\vec{u}_k<<1$.
\textcolor{Important}{Get values from data for dq}

% Decribe removal of defect strain evolution 25°C -> 150 °C
The very large standard deviations seen in fig. \ref{fig:AmaterasuStrain} at \qty{25}{\degreeCelsius} for the [$\bar{1}\bar{1}\bar{1}$], [$\bar{1}1\bar{1}$], [$1\bar{1}1$] and [$11\bar{3}$] facets can be explained by the presence of the defect, around which the phase was not well unwrapped as seen in fig. \ref{fig:AmaterasuDefect}.
At \qty{150}{\degreeCelsius}, after removal of the defect, all of the facets show low strain values as well as low standard deviation.

% 150 -> 450
The strain values are still relatively low at \qty{450}{\degreeCelsius} but when comparing to the measurement at \qty{150}{\degreeCelsius}, we can see in fig. \ref{fig:Amaterasu} that the extremities of the [$\bar{1}\bar{1}\bar{1}$] facet are in tension, whereas the center is in compression.
The opposite effect seems to take place around the [111] facet, visible both in fig \ref{fig:Amaterasu} and \ref{fig:AmaterasuStrain}.
\textcolor{Important}{Why ?}

% Decribe strain change between 450°C under Argon and 600°C under ammonia
After the introduction of ammonia at \qty{600}{\degreeCelsius}, the particle could be reconstructed again.

\begin{figure}[!htb]
    \centering
    \includegraphics[width=0.32\textwidth]{/home/david/Documents/PhDScripts/SixS\_2021\_01/paraview/1675_strain_and_facets1.png}
    \includegraphics[width=0.32\textwidth]{/home/david/Documents/PhDScripts/SixS\_2021\_01/paraview/1675_strain_and_facets2.png}
    \includegraphics[width=0.32\textwidth]{/home/david/Documents/PhDScripts/SixS\_2021\_01/paraview/1675_strain_and_facets3.png}
    \caption{
        View of the Amaterasu particle at \qty{600}{\degreeCelsius} after the introduction of ammonia.
        The three \{100\} facets are surrounded by either 4 \{111\} facets and one \{110\} facets (left), by 3 \{111\} facets and one \{110\} facets (middle) or by 4 \{111\} facets (right).
    }
    \label{fig:AmaterasuStrain1675}
\end{figure}

The alternating highly positive and negative strain regions in the middle of the [$\bar{1}\bar{1}\bar{1}$] facet (fig. \ref{fig:Amaterasu}) is the signature of a dislocation network forming in the interface with the substrate \parencite{Dupraz2015}, also responsible for the high strain standard deviation of this facet (fig. \ref{fig:AmaterasuStrain}).
A higher resolution in a 3D displacement field together with simulations of the impact of defect at the interface are needed to fully characterized this network, which is outside the scope of this thesis.

% 111 different effect
% other facets all same
% dif in env of 113 and 111 ?

In comparison with the measurement at \qty{450}{\degreeCelsius} under Argon, the [111] facet goes into compression (negative strain) as seen on fig. \ref{fig:Amaterasu}, \ref{fig:AmaterasuStrain} and \ref{fig:AmaterasuStrain1675}.
Moreover, we can see that the same effect takes place on the neighbouring \{100\} facets, none of which have the same environment (fig. \ref{fig:AmaterasuStrain1675}), while the edges and corners around the [111] and \{100\} facet show a positive strain.
Both [$0\bar{1}1$] and [$01\bar{1}$] facets have a very low strain.

At first sight, all facets that have a similar orientation show a similar strain (e.g. \{1-10\}, \{100\}).
However, the two \{111\} facets that are close to the substrate do not show the same strain than the three \{111\} facets closer to the top of the particle.
The strain being in the [111] direction with the origin of the displacement field in the center of the particle, if all of these facet had experienced the same compressive (tensile) strain perpendicular to their surface \textit{via} the adsorption of ammonia, they would all show a negative (positive) strain (of lower amplitude due to the angle between these facets and the [111] direction).

This behaviour could be linked to a strong effect of the support which has the dual effect of first preventing the [$\bar{1}\bar{1}\bar{1}$] facet from being exposed and secondly forcing them to accomodate the strain linked to the substrate.
There are three possibilities, first ammonia is effectively adsorbed on these facets but the influence of the substrate hides a possible common signature in the strain.
Secondly ammonia is not adsorbed and the strain difference is due to the influence of the substrate on the particle structure.
Thirdly, ammonia is adsorbed but only on the three top facets or only on the 2 bottom facets due to the influence of the substrate that either limitates or facilitates the adsorption process.

% Globally, the top of the particle seems to be in compression along $\vec{z}$ (fig. \ref{fig:Amaterasu}, \ref{fig:AmaterasuStrainSlices}), whereas the bottom of the particle is in tension, with the exception of the one [$1\bar{1}3$] facet (fig. \ref{fig:AmaterasuStrainSlices}) which is in compression.

% \begin{figure}[!htb]
%     \centering
%     \includegraphics[width=0.49\textwidth]{/home/david/Documents/PhDScripts/SixS\_2021\_01/paraview/1675_clipx.png}
%     \includegraphics[width=0.49\textwidth]{/home/david/Documents/PhDScripts/SixS\_2021\_01/paraview/1675_clipy.png}
%     \caption{
%         Particle slices perpendicular to the $\vec{x}$ and $\vec{y}$ directions at center of mass, a white line delimitates the particle surface.
%     }
%     \label{fig:AmaterasuStrainSlices}
% \end{figure}

Overall, it is difficult to conclude on any potential effect of the absorption of ammonia on the particle due to the presence of the dislocation network at the interface, to the lack of comparative reconstruction at \qty{600}{\degreeCelsius} under argon, and more importantly to the fact that we only possess one component of the strain tensor, making it difficult to quantify any relaxation effect on the facets that are not perpendicular to the [111] direction.

% The shape of the particle in the ($\vec{x}, \vec{y}$) plane at \qty{600}{\degreeCelsius} can be roughly approximated as a hexagon with 3 sides occupied by the [$\bar{1}11$], [$1\bar{1}1$] and [$11\bar{1}$] facets.
% The three other sides are occupied by the [100], [010], [001] facets near the top of the particle, under which are situated either 1) one large [$\bar{1}10$] facet, 2) one small [$1\bar{1}0$] facet under which in turn are two smalls [113] and [$\bar{1}1\bar{1}$] facets, 3) a large [$1\bar{1}\bar{1}$] facet.

%%
% The negative strain of the [$1\bar{1}1$] facets and the positive strain on the [$\bar{1}1\bar{1}$] facets (fig. \ref{fig:AmaterasuFacetsEvolution}) follows the respective evolution of the [111] and [$\bar{1}\bar{1}\bar{1}$] facets with a lower amplitude.
% An increase of the negative strain can also be seen on all three [100] facets but, based on the angle between those facets and the [111] direction (\ang{54.7}), the interpretation of this value is difficult to understand.

    \section{Measuring the oxidation of ammonia}\label{sec:BCDIAmmoniaOxidation}

% Lattice parameter - homogeneous strain evolution: OK
% 3D plots: OK
% strain field energy: OK
% Peak width :kinda OK
% Field data tables : prob too many so no
% Facet strain evolution plots just like temp ramp: (no disp, useless unless seen in bulk)
% Strain histogram ? useless we want to see the evolution of specific facets, good id not too many facets, also too many scans
% need to do smtg with edges and corners, plot as a fct of the coordination nb ?
% explain how exactly a single voxel was found when getting the surface voxels in python

After having shown that the Pt particles epitaxied on alumina are stable under different atmospheres at \qty{300}{\degreeCelsius}, \qty{500}{\degreeCelsius} and \qty{600}{\degreeCelsius}, the oxidation of ammonia was measured using BCDI starting at \qty{300}{\degreeCelsius}.

The same sample, experimental setup, type of measurements and phase retrieval algorithms were used as detailed in the previous section for the temperature ramp.
Two well faceted nanoparticles were successully measured and reconstructed at room temperature (fig. \ref{fig:IsanagiSusanooFacets}).
Isanagi exhibits a round shape with many low index facets, \qty{40}{\percent} of its surface covered by \{113\} facets, \qty{40}{\percent} by \{111\} facets, \qty{40}{\percent} by \{110\} facets, and \qty{40}{\percent} by \{100\} facets.
The particle is at the largest \qty{300}{\nm} wide.
On the contraty, Susanoo exhibits a more rectangular shape without any \{113\} facets, \qty{40}{\percent} of its surface is covered by \{111\} facets, \qty{40}{\percent} by \{110\} facets, and \qty{40}{\percent} by \{100\} facets.
The particle is at the largest \qty{800}{\nm} wide.

Together with the Amaterasu nanoparticle, we have measured three nanoparticles that have a different size, shape and facets exhibited on the surface.
The smallest nanoparticle (Isanagi) is the one with the most \{113\} facets, whereas the largest (Susanoo) does not have any.
Amaterasu which is about \qty{600}{\nm} wide does only exhibits small \{113\} facets depending on the temperature.
Overall, the sample is covered with thousands of nanoparticles that all together contribute to the catalytic activity and that probably show an even greater variance of shape and size.
Therefore, this analysis using BCDI can only aim to reveal the structure variation during a catalytic reaction of a few nanoparticles without statistically representing the population on the sample, the measurement process being far too time extensive to probe more than a few nanoparticles.

\textcolor{Important}{Need to say that we did this in a experiment}
proven to be able to bring a more

Three scans were recorded under each condition (tab. \ref{tab:Conditions}) at \qty{300}{\degreeCelsius} and \qty{400}{\degreeCelsius} for each nanoparticle to probe for any evolution of the particles surface structure during a fixed atmosphere as a function of time, and to otherwise demonstrate the reproducibility of the measurements.

\begin{figure}[!htb]
    \centering
    \includegraphics[width=0.49\textwidth]{/home/david/Documents/PhD/Figures/bcdi_data/B7/B7_facets.png}
    \includegraphics[width=0.49\textwidth]{/home/david/Documents/PhD/Figures/bcdi_data/D6/D6_facets.png}
    \caption{
        3D view of the Isanagi (left) and Susanoo (right) particles measured at room temperature showing a highly faceted surface.
    }
    \label{fig:IsanagiSusanooFacets}
\end{figure}

The 3D diffraction patterns were orthogonalised from the laboratory frame ($\vec{z}$ downstream, $\vec{y}$ vertical up, $\vec{x}$ outboard) to the $\hat{q_x}, \hat{q_y}, \hat{q_z}$ following the cxi convention ($\vec{z}$ perpendicular to the sample holder, $\vec{y}$ parallel to the beam direction and $\vec{x}$ in the sample plane).

\begin{figure}[!htb]
    \centering
    \includegraphics[width=\textwidth]{/home/david/Documents/PhD/Figures/bcdi_data/D6/DP_not_ortho.png}
    \includegraphics[width=\textwidth]{/home/david/Documents/PhD/Figures/bcdi_data/D6/DP_ortho.png}
    \caption{
        Sum of the diffracted intensity in the $\vec{x}$, $\vec{y}$ and $\vec{z}$ directions of the laboratory frame before (top) and after orthogonalisation (bottom) in the $\hat{q_x}, \hat{q_y}, \hat{q_z}$ directions.
    }
    \label{fig:IsanagiOrtho}
\end{figure}

\begin{figure}[!htb]
    \centering
    \includegraphics[width=\textwidth]{/home/david/Documents/PhD/Figures/bcdi_data/B7/DP_not_ortho.png}
    \includegraphics[width=\textwidth]{/home/david/Documents/PhD/Figures/bcdi_data/B7/DP_ortho.png}
    \caption{
        Sum of the diffracted intensity in the $\vec{x}$, $\vec{y}$ and $\vec{z}$ directions of the laboratory frame before (top) and after orthogonalisation (bottom) in the $\hat{q_x}, \hat{q_y}, \hat{q_z}$ directions.
    }
    \label{fig:SusanooOrtho}
\end{figure}

A 3D view of the reconstructed nanoparticles is presented in \ref{sec:3DAmmoniaOxidation}.

After having successfully reconstructed the Pt nanoparticles, the facets at their surface were retrieved using the \textit{FacetAnalyzer} script written for \textit{Paraview} (sec. \ref{sec:FacetAnalysis}).
No surface

\subsection{Mass spectromety results}

\begin{figure}[!htb]
    \centering
    \includegraphics[width=0.49\textwidth]{/home/david/Documents/PhDScripts/SixS_2021_06_BCDI_NH3/figures/homo_strain_vs_condition.png}
    \includegraphics[width=0.49\textwidth]{/home/david/Documents/PhDScripts/SixS_2021_06_BCDI_NH3/figures/homo_strain_vs_condition_no_25.png}
    \caption{
        Evolution of the homogeneous strain computed from the 3D Bragg peak center of mass.
    }
    \label{fig:AmaterasuStrainSlices}
\end{figure}

\subsection{Strain field energy}

    \section{Discussion}

% Particle A, Temp ramp
The reshaping of a Pt nanoparticle (particle \textit{A}) during heating from \qty{125}{\degreeCelsius} to \qty{600}{\degreeCelsius} under an inert gas flow was put into evidence.
An important evolution in the type of facets present near the particle interface with the substrate was reported as a function of the sample temperature (fig. \ref{fig:AmaterasuFacetsEvolution}).
An interfacial dislocation was successfully removed by temperature treatment, yielding a defect free interface until the introduction of ammonia at \qty{600}{\degreeCelsius}, which induced the appearance of a dislocation network (fig. \ref{fig:AmaterasuA}, \ref{fig:AmaterasuB})

The importance of the metal-support interactions for catalysis was demonstrated for smaller nanoparticles (below \qty{4}{\nm}) in a general study by van Deelen et al. \parencite*{vanDeelen2019}.
In the current study, the importance of taking into account the presence of interfacial strain was also highlighted with particle \textit{A} (fig. \ref{fig:AmaterasuStrain}), similarly to other BCDI studies of catalytic reactions.
For example, Kim et al. \parencite*{Kim2021} have measured different strain evolution for \{111\} facets depending on their orientation with the \ce{SrTiO_3} substrate (\ce{CO} oxidation, Pt-Rh nanoparticles, \qty{100}{\nm} large).
Similar conclusions have been reached by Dupraz et al. \parencite*{Dupraz2022}, also during the oxidation of \ce{CO} with Pt nanoparticles (\qty{300}{\nm} large particle).
The relative facet size can also have an impact on the facet strain.
Both effects could not be de-correlated in this study since facets of same structure but different orientation, \textit{i.e.} (1$\bar{1}$1) and ($\bar{1}$1$\bar{1}$), have different sizes, whereas facets of same orientation have similar sizes (e.g. (1$\bar{1}$1)-type facets in fig. \ref{fig:AmaterasuStrain1675}).

Particle \textit{A} exhibited a symmetric shape only at \qty{450}{\degreeCelsius} (fig. \ref{fig:AmaterasuFacetsEvolution}), isolated \{110\} \{212\} and \{211\} facets were otherwise detected.
A reconstruction at a higher temperature (\qty{600}{\degreeCelsius}) showed a hole in the Bragg electronic density near the particle substrate (fig. \ref{fig:AmaterasuB}).
The evolution of the lattice parameter as a function of the temperature (fig. \ref{fig:AmaterasuHomoStrain}) shows that nanoparticles do not follow the same thermal relaxation curves when in the presence or not of interfacial defects, having an impact on the strain field.
 % as expected from literature values, which is probably due to an important surface/volume ratio, \textit{i.e.} surface energy minimisation processes \parencite{Winterbottom1967, Boukouvala2021}, having an impact on the strain field.

% Ammonia ox
The oxidation of ammonia has been extensively studied by the means of surface techniques in the previous and current century as detailed in sec. \ref{sec:LiteratureAmmonia}.
Since BCDI is still a recent technique first applied to catalysis less than 10 years ago \parencite{Ulvestad2016}, no comparative measurements could be found during the presence of ammonia or nitrogen based species.

% Particle A
The introduction of ammonia in the reactor at \qty{600}{\degreeCelsius} was linked to an inversion of the average facet strain for particle \textit{A} (fig. \ref{fig:AmaterasuFacetsEvolution}), together with the appearance of a dislocation network at the interface with the substrate (fig. \ref{fig:AmaterasuB}).
Facets of the same structure, size, and distance from the substrate showed similar strain state (fig. \ref{fig:AmaterasuStrain}).
It is possible that a strong adsorption of ammonia induced such a modification of the strain field that the appearance of a dislocation network was provoked.
Indeed, no other variation of particle shape, or appearance of dislocation was observed without a change of temperature.
Time-resolved measurements of the dynamics of the dislocation network as well as adsorption/desorption cycles could bring additional information supporting this hypothesis.

The influence of the introduction of oxygen at \qty{600}{\degreeCelsius} could not be probed due to the loss of particle \textit{A}, which detached from the substrate.
The high temperature reached by the nano catalyst during the exothermic oxidation of ammonia \parencite{Hatscher2008} could have resulted in a loss of quality of the epitaxial relationship with the substrate.

Two additional particles, namely \textit{B} and \textit{C}, were measured at both \qty{300}{\degreeCelsius} and \qty{400}{\degreeCelsius}, respectively before and after the activation temperature expected from the characterisation of the catalyst shown in fig. \ref{fig:TempRamps}.

% Particle C
The introduction of ammonia at \qty{300}{\degreeCelsius} can be linked to a global decrease of the heterogeneous strain in particle \textit{C}, whose surface consists exclusively of \{111\}, \{110\} and \{100\} facets (tab. \ref{tab:FacetCoverage}).
This effect is visible from the decrease of the strain field energy in fig. \ref{fig:D6SFE}, the return to a symmetrical diffraction pattern in fig. \ref{fig:D6Ortho}, and the decrease of the FWHM in all three directions in fig. \ref{fig:D6FWHM}.
The opposite (same) effect was observed after removing (introducing) ammonia from the reactor at \qty{300}{\degreeCelsius} (\qty{400}{\degreeCelsius}), supporting the hypothesis of a nitrogen-rich (\ce{NH_x}, \ce{N}) adsorption/desorption induced hysteresis cycle on the particle surface.
The mechanism of such a cycle could correspond to a decrease of the surface strain magnitude by adsorption, thus relaxing the particle bulk by removing nano defects present to accommodate the high surface strain.
Such a scenario implies changes in the surface strain when introducing ammonia, which are not visible since the reconstruction of the measurements performed under inert atmosphere do not converge towards a well defined support.
% The introduction of ammonia at \qty{600}{\degreeCelsius} when measuring particle \textit{A} has had the opposite effect, \textit{i.e.} the creation of a dislocation network.
% However, both particles have different shape and facet coverage, which can be linked to different behaviour as highlighted in this study.

The absence of change in the facet strain following the introduction of oxygen and during the oxidation of ammonia can be due to different phenomena.

First, the change in the facet strain is too small to be detected due to the low spatial resolution of the experiment, meaning that it is damped by being averaged with \textit{bulk} layers.
For example, oxygen adsorption induced strain on the Pt terminated \{100\} and \{111\} facets has been simulated to be null \qty{8}{\nm} away from the surface in the bulk \parencite{Kim2021}.

Secondly, there is no change of facet strain following the introduction of oxygen, which could be interpreted by nitrogen-rich species being so strongly adsorbed on the particle surface that they do no participate in the catalytic reaction.
If this was indeed the case, it would mean that all three types of facets are poisoned at a \ce{NH_3} pressure equal to \qty{6}{\milli\bar}, showing a remarkably strong adsorption on many different atomic sites.
A strong evolution of particle \textit{A} was also observed under the sole presence of ammonia at \qty{600}{\degreeCelsius}, with the main structural differences being the absence of \{110\} facets (fig. \ref{fig:AmaterasuFacetsEvolution}), and a dislocation network at the interface (fig. \ref{fig:AmaterasuB}).

Thirdly, the oxidation of ammonia at ambient pressure does not involve the adsorption of oxygen on the catalyst surface, functioning \textit{via} an Eley-Rideal mechanism \parencite{Rideal1939}, the production of \ce{NO} and \ce{N_2O} happening by the reaction of gas-phase oxygen with nitrogen adsorbed species.
The surface strain would then not be expected to evolve during change of atmospheres.
This finding would be in contradiction with literature finding that have proved the importance of prior atomic \ce{O} and \ce{OH} coverage for the de-hydrogenation process of \ce{NH_3}, during a largely accepted Langmuir-Hinshelwood reaction mechanism \parencite{Bradley1995, Mieher1995,vandenBroek1999, Kim2000}.
The pre-exposition of the particle to ammonia would prevent this mechanism and thus the production of \ce{N_2}, \ce{NO} and \ce{NO_2}.

Moreover, a transition was observed in the homogeneous strain for particle \textit{C} at \qty{300}{\degreeCelsius} when introducing oxygen in the reactor (fig. \ref{fig:D6Latpara}, up by \qty{0.07}{\percent}), which was not reproduced at \qty{400}{\degreeCelsius}.

% Particle B
Particle \textit{B}, which exhibits \{113\} facets on its surface at room temperature in addition to \{111\}, \{110\} and \{100\} facets (tab. \ref{tab:FacetCoverage}), follows a different structural evolution during the experiment.
The surface coverage of \{110\} facets is also four times higher than for particle \textit{C}.
Heating the sample to \qty{300}{\degreeCelsius} has decreased the visible presence of facets on the particle surface.
While particle \textit{C} observes a large decrease of heterogeneous strain during the introduction of ammonia, particle \textit{B} suffers both a large increase of homogeneous strain with respect to the room temperature lattice parameter (fig. \ref{fig:B7Latpara}, up by \qty{0.09}{\percent}), together with an increase of the in-plane heterogeneous strain (fig. \ref{fig:B7FWHM}).
This effect is not reversible when removing ammonia from the reactor at \qty{300}{\degreeCelsius}, nor it is observed again when introducing ammonia at \qty{400}{\degreeCelsius}.
It is possible to link this effect to the particle first exposition to reducing atmosphere after argon and air.
The absence of strain evolution in the same condition for particle \textit{C} (fig. \ref{fig:D6Latpara}) can be attributed to different surface states under argon from contaminants/oxides, linked to different facet coverage.

The introduction of oxygen slightly decreases the heterogeneous strain in the particle (fig. \ref{fig:B7FWHM}), the in-plane heterogeneous strain is stable during the oxidation of ammonia while the out-of-plane heterogeneous strain increases slightly.
Thus, small changes are already visible at \qty{300}{\degreeCelsius} for particle \textit{B} during the oxidation of ammonia.

The most impressive evolution occurs at \qty{400}{\degreeCelsius}, with the appearance of a defect in the particle, characterised by a second peak in the 3D scattered intensity, shifted in $\vec{q}_y$ with respect to the Bragg peak (fig. \ref{fig:B7Ortho}).
The influence of the defect in both the heterogeneous (fig. \ref{fig:B7FWHM}, fig. \ref{fig:B7SFE}) and homogeneous strain (fig. \ref{fig:B7Latpara}) is clear once the ratio between \ce{O_2}/\ce{NH_3} ratio is equal to one, an atmosphere under which a non reversible increase of homogeneous and heterogeneous strain is observed.
Indeed, if the defect appeared during the increase of temperature to \qty{400}{\degreeCelsius}, the particle shape and strain state seems to be stable until specific reaction conditions are achieved.
To safely link defect dynamics and catalytic activity, several oxidation cycles would have to be measured, as well as an exposition of the particle to progressive amount of oxygen.
Nevertheless, it can safely be said that the evolution of particle \textit{B} seems to follow the reported activation and reconstruction process of industrial catalyst by defects reported by Hannevold et al. \parencite*{Hannevold2005}.

% Compare oxidation with literature
The exposition of platinum nanoparticles epitaxied on glassy carbon to oxygen at \qty{200}{\degreeCelsius} has been probed by Fernandez et al. \parencite*{Fernandez2019}, shown to be linked with an \qty{0.09}{\percent} homogeneous tensile expansion of the particle in the [111] direction, linked to the formation of platinum surface oxides.
The nanoparticles were not exposed to oxygen without ammonia during this experiment.
Nevertheless, a similar expansion was observed on particle \textit{C} but during reacting conditions, and only at \qty{300}{\degreeCelsius} (fig. \ref{fig:D6Latpara}, up by \qty{0.07}{\percent}).
Further increasing the oxygen pressure at \qty{300}{\degreeCelsius} did not have a significant effect on the homogeneous strain.
The increase of strain on particle \textit{B} under reacting conditions in not reversible when removing the reagents, and not observed at \qty{400}{\degreeCelsius}.
This means that if oxides have grown under reacting conditions at \qty{300}{\degreeCelsius}, they are stable throughout the experiment, and not removed by the sole presence of ammonia at \qty{400}{\degreeCelsius}.
The questions on Pt nanoparticles remains, do platinum oxides grow on the sample surface, and are they stable under reacting conditions ?
Future experiments on single crystals will yield completing information to answer this question.
Moreover, a different behaviour is observed for particle \textit{B}, for which the homogeneous strain does not evolve with oxygen at \qty{300}{\degreeCelsius} (fig. \ref{fig:D6Latpara}), which would necessarily link the growth of surface oxides to the nanoparticle shape, facet coverage, and initial strain state.

Kim et al. \parencite*{Kim2018} have also shown that the introduction of oxygen was followed by its adsorption on edge sites, visible by magnitude changes in the neighbouring displacement field.
Their hypothesis could be at the origin of different nano catalyst reshaping phenomena measured during the oxidation of \ce{CO} \textit{via} SXRD \parencite{Nolte2008, Hejral2016}, TEM \parencite{Vendelbo2014} and also other BCDI studies \parencite{Abuin2019}.

No reshaping of the particles \textit{B} or \textit{C} was observed during the study of the ammonia oxidation with BCDI.
However, a global reshaping of the nanoparticles was revealed by SXRD at \qty{600}{\degreeCelsius} under reacting conditions favouring oxygen rich products, \{113\} and \{110\} facets being replaced by \{100\} and \{111\} facets for which the CTR intensity increased.
This reshaping was not detected at lower temperatures, no decrease of the [1$\bar{1}$3] oriented CTR could be identified at \qty{300}{\degreeCelsius} or \qty{500}{\degreeCelsius}, meaning that the \{113\} facets present at room temperature on particle \textit{B} may yet still exist at \qty{300}{\degreeCelsius} and \qty{400}{\degreeCelsius}.
% A more detailed analysis of the 3D scattered intensity around the (111) Bragg peak may reveal if the particle has effectively lost the \{113\} facets or not.

A particle with a similar shape to that of particle \textit{B} was measured by Dupraz et al. \parencite*{Dupraz2022}, \{113\} facets are still visible during the presence of oxygen in the cell.
The spatial resolution of the experiment is much higher than in the present study, due to the very high brilliance of Petra III in comparison with other $3^{rd}$ generation synchrotrons \parencite{Bilderback2005}.
%Interestingly, they have also shown a different behaviour when exposing the catalyst to oxygen for the third time, after some \ce{CO} oxidation cycles.
A relaxation of the out-of-plane heterogeneous strain $\epsilon_{zz}$ was observed on all facets (besides ($\bar{1}$10)-type facets) during the presence of oxygen in the cell.
No such relaxation was observed during this experiment, which supports that the adsorption/desorption of species linked to the catalytic reaction is measured, and not the effect of the oxidation of the particle surface.

% Defects
The presence of steps/surface defects on Pt(111) facets has been shown to increase the catalytic activity of Pt nanoparticles \parencite{Segner1984, Chen2012}.
Indeed, defects have an impact of the local strain field, which can in turn influence the adsorption properties of molecules taking part in the reaction.
During the oxidation of ammonia, defects could play a role in the step-by-step de-hydrogenation process of ammonia on the catalyst surface, e.g. favouring \ce{NO} desorption, and thus influencing the catalyst selectivity.

Similarly to our results on nanoparticle \textit{B}, Kim et al. \parencite*{Kim2019} have measured the defect dynamics on a Pt nanoparticle during the catalytic methane oxidation.
They have shown a reversible transition of the particle shape induced by the adsorption of oxygen at \qty{425}{\degreeCelsius} (no indication of the oxygen partial pressure), no such transition was observed at \qty{325}{\degreeCelsius}.
A second peak has been observed in the diffraction pattern in the $\vec{q}_z$ direction, \qty{10}{\minute} after the introduction of oxygen, which is accompanied by an increase of the strain field energy, and the presence of defects in the particle.

Two main differences are noted in the current study, first a defect seems to appear on particle \textit{B} when \textit{heating} the sample up to \qty{400}{\degreeCelsius}, and not during reacting conditions.
Secondly, the transformation of the particle shape is not reversible upon removal of the reactants.
The reaction product partial pressures can be seen to drop in the minutes following the return to inert atmosphere at \qty{400}{\degreeCelsius} (fig. \ref{fig:RGA400BCDINanoparticles}), which shows that the sample has stopped its catalytic activity.
From the evolution of the strain field energy, the lattice parameter, and the FWHM, it is likely that the magnitude of the defect appearing at \qty{400}{\degreeCelsius} under inert atmosphere has been increased from the reaction, which irreversibly transformed the particle shape.
Defect dynamics have also been reported during the catalytic oxidation of \ce{CO} on Pt nanoparticles \parencite{Carnis2021b} using BCDI.

These result tend to support the importance of defects in the catalytic activity of nanoparticles during the oxidation of ammonia.
The defect introduced by heating at \qty{400}{\degreeCelsius} has probably altered the adsorption properties of particle \textit{B}, possibly creating steps on top of nanoparticle surface.
Interestingly the change in the particle strain field is only visible after a certain \ce{O_2}/\ce{NH_3} ratio is reached (fig. \ref{fig:B7SFE}, \ref{fig:B7FWHM}), which can be linked to a transition to the adsorption of oxygen rich species (e.g. \ce{NO}, \ce{N_2O}) on the catalyst surface.
The reproducibility of the result at \qty{300} and \qty{400} for particle \textit{B} for the same exposition time allow us to safely rule out the possibility of beam induced structural changes in particle \textit{B}.

% RGA
Reaction products were detected at \qty{300}{\degreeCelsius} and \qty{400}{\degreeCelsius} (fig. \ref{fig:RGANanoparticlesBCDIComparison}), overall an oxygen favoured \ce{O_2}/\ce{NH_3} ratio resulted in a higher proportion of \ce{NO} and \ce{N_2O}, which enabled us to probe the relationship between strain and catalyst selectivity.
Interestingly, an \textit{activation} of the sample was observed at both \qty{300}{\degreeCelsius} (fig. \ref{fig:RGA300BCDINanoparticles}) and \qty{400}{\degreeCelsius} (fig. \ref{fig:RGA400BCDINanoparticles}) when \ce{O_2}/\ce{NH_3} = 2.
No such effect was observed at \qty{300}{\degreeCelsius} in the previous experiment when using the non-patterned sample during surface x-ray diffraction (fig. \ref{fig:RGASXRDNanoparticlesComparison}).
The only difference in the history of both samples being that the patterned sample was exposed to reacting conditions at \qty{600}{\degreeCelsius} for approximately \qty{1}{\hour} before this experiment.
This transition in the product partial pressures when \ce{O_2}/\ce{NH_3} = 2 does not correspond to the condition at which particle \textit{B} showed a transition in terms of strain field energy or FWHM at \qty{400}{\degreeCelsius}, which occurred under an equal amount of ammonia and oxygen in the reactor (\ce{O_2}/\ce{NH_3} = 1).
This structural transition could be a precursor to the \textit{activation} process, keeping in mind that, as underlined by this experiment, the large difference in shape of the nanoparticles on the sample implies different structural evolution under various atmospheres.

% conclude
To conclude, it is of utmost importance to probe different nanoparticles before drawing a conclusion regarding their global behaviour during a heterogeneous catalytic reaction.
The importance of the particle shape, \textit{i.e.} facet size, coverage as well as initial strain state was put into perspective by the study of the oxidation of ammonia at \qty{300}{\degreeCelsius} and \qty{400}{\degreeCelsius}, as a function of the ratio between \ce{O_2} and \ammonia.

The presence of defect in particle \textit{B} was linked to an evolution of the particle strain state and shape once an oxygen rich atmosphere was reached in the reactor, favouring the production of \ce{NO} and \ce{N_2O} over \ce{N_2}
It is possible that the transformation of particle \textit{B} corresponds to the onset of activation and reconstruction observed in industrial samples, driven by the appearance of defects.
Moreover, this behaviour could be connected to a more global behaviour of the nanoparticles present on the sample from the visible transition in terms on product pressure visible in the RGA.

Particle \textit{C} has a completely different behaviour, showing no strain evolution during the catalytic reaction, possibly from an initial strain state that lead to strongly adsorbed species not participating in the catalytic activity.

This experiment has confirmed the importance of initial strain states for heterogeneous catalysis and will push forward nanoparticle strain engineering.
The possibility of having different phenomena occurring under the same reacting conditions was revealed.

Finally, this experiment has both demonstrated the value and limits of Bragg coherent diffraction imaging when studying heterogeneous catalytic reactions.
The recent improvement of the instrumental setup (rocking curves performed in fly mode instead of step-by-step mode, sample scanning in Bragg conditions to find the nanoparticles, sapphire window in the dome) tend in the right direction.
The planned upgrade of SOLEIL to a $4^{th}$ generation synchrotron will also play a key role in the hierarchy of SixS in the very competitive list of coherent imaging beamlines, the current resolution of the experiment being too low to properly resolve the smallest facets present of the particles, to distinguish between facets with similar orientations,to observe the growth of surface oxides or the strain in the topmost atomic layers.

In order to have a better understanding of surface dependent effects on the catalyst selectivity, and to probe for the possible growth of surface oxide during similar oxygen pressures, specific facets have been measured by SXRD, presented in the next chapter.

% Results SXRD
    \chapter{Structural and chemical evolution of single Pt facets during the oxidation of ammonia}
    \textcolor{red}{This Chapter should demonstrate that you have conducted a thorough and critical investigation of relevant sources.
Apart from a presentation of the sources of your data, this chapter allows you to critically discuss the data (whatever these data are, ‘quantitative’ or ‘qualitative’, primary or secondary), which is proof of good research. You can even do good research with poor data but you must demonstrate that you are aware of the data quality and accordingly are careful in your interpretations. Essentially, there are three aspects to consider:
\begin{enumerate}
\item   Reliability, which, for example, could depend on whether they are estimates or more direct evidence;
\item   Representativity, which is about how typical the data are; for example, you may have arguments why the very few cases are typical or you may carry out statistical tests;
\item Validity, which is about the relevance of the data for your case. Strictly speaking, sometimes no valid data are available but one may argue that there are other data which could be used as ‘proxies’.)
\end{enumerate}
}

\section{Introduction}

Past ammonia oxidation experiments on \ce{Pt_{25}Rh_{75}} (100) \parencite{Resta2020a} single-crystals gave significant new insights on the relaxation dynamics when the system crosses the catalyst activation barrier.
The experiment focused on measuring before and after the catalyst light off temperature, it was found that the transition from a nitrogen-rich to oxygen-rich input mixture coincides with a relaxation change in the topmost metallic layer: moving from outward to inward, but only for temperatures above the light off.
Thanks to NAP-XPS experiments, nitrogen rich species (such as \nitrogen) were associated to positive relaxation of the topmost layer, expansion; while oxidative species (such as NO and oxygen) were linked to negative relaxation, contraction.
This change in relaxation has an impact on the band structure and therefore on the catalytic properties as proposed in the d-band model as detailed in sec. \ref{sec:Catalysis}.
It is therefore our intent to study if this modification in the relaxation is present only in the \ce{PtRh} alloy or if it has a more general value also on the constituent elements of the alloy to fully understand the impact of the Pt surface orientation on the catalyst selectivity.
Different relaxation state and transient structures are expected to exist as a function of the ammonia to oxygen ratio.

Moreover, we have recently probed the relaxation of single multi-faceted nanoparticles at SixS under the oxidation of ammonia using surface x-ray diffraction, which showed a global reshaping of the nanoparticles at \qty{600}{\degreeCelsius} under reacting conditions, favouring \{111\} and \{100\} facets over \{110\} and \{113\} facets.
The evolution of two single nanoparticles exhibiting different type of facets on their surface was probed with Bragg coherent diffraction imaging, highlighting the importance of the particle structure
The 3D displacement and strain evolution of two different facets (\{100\} and \{111\}) will be compared to our results obtained on Pt 100 and 111 crystal surfaces.

We first intend to measure the total signal scattered from \ce{Al_2O_3}-supported platinum particles by taking advantage of the possibility to carry out grazing-incidence diffraction measurements at the SixS beamline of synchrotron SOLEIL.
By having a low incidence angle, the incident beam recovers the whole sample surface, the scattered beam being then proportionnal to the ensemble behaviour of the nanoparticles \parencite{Nolte2008, Hejral2013}.

The goal of this experiment is two-fold.
First, the nanoparticles epitaxy must be ensured to be stable at different temperatures and atmospheres, before measuring a single nanoparticle with Bragg coherent diffraction imaging, where the beam is reduced to micrometric size.
Secondly, the average nanoparticle shape and structure will be probed by studying the intensity of crystal truncation rods (CTR) in directions perpendicular to the expected facets on the nanoparticle surface.
Prior experiments with similar samples \parencite{Dupraz2017, Li2020, Lim2021, Dupraz2022} have shown that the particles exhibit a Winterbottom shape \parencite{WINTERBOTTOM1967, Boukouvala2021}, typical of nanoparticles epitaxied on a substrate, with mainly \{111\}, \{110\}, and \{100\}-type facets, and a [111] orientation perpendicular to the substrate.

We have seen in sec. \ref{sec:SXRD} that truncated surfaces such as facets give rise to crystal truncation rods in the reciprocal space, the intensity of which is proportionnal to the size, roughness and strain of the related surface, as a function of the scattering vector.
By having the incident beam covering all of the particles, the CTR signal in e.g. the [111] direction will be the sum of the contribution from the [111] facet of every nanoparticle on the sample (as well as their [$\bar{1}\bar{1}\bar{1}$] facets).
Therefore, phenomena inducing structural change such as particle refaceting/reshaping at a given condition are expected to be visible by an evolution of the different CTR.

To have a more important scattered intensity, a slightly different sample was used from the sample used in BCDI, with the only difference being a surface homogenously covered with Pt particles.

\subsection{Experimental setup for SXRD experiments in the horizontal geometry}\label{sec:SXRDSetupH}

The diffractometer was used in a horizontal geometry (fig. \ref{fig:Diffractometer}), so that is it possible to clean the sample surface by sputtering and annealing.
There is no need for high incident angles when performing surface x-ray diffraction, the incident beam is set at a grazing angle to cover the entire sample surface.
No cleaning process was applied to this sample that has the same history as the sample with isolated nanoparticles used for BCDI experiments, \textit{i.e.} annealing at \qty{1100}{\degreeCelsius} for \qty{30}{\minute} before cooling to room temperature.
The temperature in the MED end-station is of about \qty{25}{\degreeCelsius}.

The incident beam was fixed to \ang{0.3} with the angle $\mu$, so that the beam recovers the entire sample surface.
In-plane measurements were perfomed by rotating the in-plane sample angle $\omega$ together with the in-plane detector angle $\delta$.
Out-of-plane measurements were performed by rotating the in-plane sample angle $\omega$ together with the in-plane and out-of-plane detector angles $\delta$ and $\gamma$, the incoming angle $\mu$ must stay at a low value to keep the grazing incidence of the beam as detailed in sec. \ref{sec:DataCollectionSXRD}.

The alignement of the beam was performed with the direct beam (all angles at \ang{0})
% to ensure that the sample surface is parallel to the direct beam at $\mu=0$ when $\omega = \ang{0}$ and when $\omega = \ang{90}$.
First, to place the sample in the beam, its position was gradually increased in the direction perpendicular to the sample plane while recording the intensity of the direct beam.
The sample was then moved to the position at which the intensity of the direct beam was equal to half of its intensity without the sample in the beam path, so that when incresing the incidence angle to \ang{0.3}, the beam covers the entire sample surface.
Secondly, any possible tilt of the sample surface was corrected by recording the intensity of the \textit{reflected} beam as a function of the $u$ (when $\omega=\ang{0}$) or $v$ (when $\omega=\ang{90}$) angles (fig. \ref{fig:Diffractometer}).

This experiment proved to be difficult to realise experimentally.
The graphite layer used to heat the sample is covered by a Boron Nitride solid surface with 4 holes (fig. \ref{fig:SampleHolder}), two holes are used with small screws to fix the sample on top of the heater, while the two others are used to fix the heater to the sample holder.
Despite an extra layer of Boron Nitride that was applied around these screws and holes, the high temperature and highly oxidating atmosphere managed twice to corrode the conducting screws which resulted in a contact loss with the heater, thus a change of heater and realignment of the sample surface.
However, most of the experimental plan was still carried out, lacking only in-plane measurements at \qty{600}{\degreeCelsius} under \qty{8}{\ml\per\min} of oxygen and \qty{1}{\ml\per\min} of ammonia (tab. \ref{tab:Conditions}), due to a lack of experimental time.


\subsection{Crystal structures}

Platinum crystallizes in a face-centered cubic structure with a lattice parameter $|\vec{a}|$ equal to \qty{3.9254}{\angstrom} at room temperature.
The structure of the [001] and [111] planes differ greatly as shown in fig. \ref{fig:Cubic100Hex111}.
The shortest distance between the atoms on the [001] and [111] surfaces is $a/\sqrt{2} = \qty{2.78}{\angstrom}$, which is the magnitude of the \textit{in-plane} vectors used to describe the two surfaces.
However, the atoms on the [001] plane follow a cubic structure, whereas the atoms on the [111] plane follow a hexagonal structure, therefore the angle $\gamma$ between the two in-plane vectors is equal to \ang{90} for the [100] surface and to \ang{120} for the [111] surface.
The \textit{out-of-plane} vector, which is in both case perpendicular to the two surfaces, changes magnitude since there are 3 [111] planes contained in a classic unit cell (fig. \ref{fig:Cubic100Hex111}).
The different structures are resumed in tab. \ref{tab:Structures}.

\begin{figure}[!htb]
    \centering
    \includegraphics[width=0.49\textwidth]{/home/david/Documents/PhD/Figures/introduction/100.pdf}
    \includegraphics[width=0.49\textwidth]{/home/david/Documents/PhD/Figures/introduction/111.pdf}
    \caption{
        Face-centered cubic unit cell of Pt with two [001] crystallographic planes drawn in green at z=0 and z=1 (left).
        The $\vec{a}$, $\vec{b}$, and $\vec{c}$ vectors are used classically to decribe the FCC cubic structure, the $\vec{a}_s$ and $\vec{b}_s$ in-plane vectors separated by \ang{90} and of magnitude \qty{2.78}{\angstrom} can also be used specifically to describe the structure of the [001] planes.
        Face-centered cubic unit cell of Pt with $[111]$ crystallographic plane drawn in green (right).
        The arrangement of the Pt atoms on the $[111]$ crystal planes is hexagonal, which leads to a new definition of the surface unit cell with the $\vec{a}$ and $\vec{b}$ in-plane vectors separated by \ang{120} of magnitude \qty{2.78}{\angstrom}, and $\vec{c}$ perpendicular to the $[111]$ plane of magnitude \qty{6.81}{\angstrom}.
        There are three $[111]$ planes spanned by the magnitude of $\vec{c}$ (blue, red and green on the figure).
    }
    \label{fig:Cubic100Hex111}
\end{figure}

\begin{table}[]
\centering
% \resizebox{\textwidth}{!}{%
\begin{tabular}{@{}llllll@{}}
\toprule
     & $\alpha$ & $\beta$ & $\gamma$ & $|\vec{a}|$ & $|\vec{c}|$ \\
\midrule
FCC & \ang{90} & \ang{90} & \ang{90} & \qty{3.9242}{\angstrom} & \qty{3.9242}{\angstrom} \\
$[100]$ & \ang{90} & \ang{90} & \ang{90} & \qty{2.78}{\angstrom} & \qty{3.9242}{\angstrom} \\
$[111]$ & \ang{90} & \ang{90} & \ang{120} & \qty{2.78}{\angstrom} & \qty{6.81}{\angstrom} \\
\bottomrule
\end{tabular}%
% }
\caption{
    Different structures used in the frame of this thesis.
    Surface unit cells are used to simplify the study of surface relaxations
    }
\label{tab:Structures}
\end{table}

\subsection{X-ray photoelectron spectroscopy measurements} \label{sec:XPS111}

On the Pt-Rh alloy an important body of work has been
already carried out, revealing two oxide structures present in the temperature and
pressure ranges we intend to study [6].
It is our goal to retrieve the surface moieties on PtRh nanoparticles as function of the reactive atmosphere’s composition.
The particles are lithography patterned on the sapphire surface, allowing to retrieve each particle position across different setups.
Through Bragg coherent diffraction imaging (BCDI) particles will be studied under reactive condition, then we plan to measure those very same particles at B07. The BCDI measurements give the possibility to follow the particle internal stresses as a function of the gas composition and the behavior of each facet at the same time as difference to the single crystal approach that can probe only one facet at the time.
The B07 setup is sensitive to the surface moieties on the particle across the different for each gas condition.
The colored insert in In Figure 1 is one particle from lithographed sample in operando condition: SixS beamline data (SOLEIL synchrotron).
The particles are sub-micron, like the grains in the metallic gauzes industrially used for this reaction. We plan to expose the particles to five ammonia to oxygen ratios, spanning from oxygen rich to oxygen poor conditions.

The goal of the experiment is to provide new insights on the PtRh catalysts during the NH3 oxidation.
Core level photoemission spectroscopy provides different signatures for many moieties adsorbed on the surface during reactions and therefore the possibilityto strengthen the links between structure, surface moieties and reaction products.
Our recently published results from both surface x-ray diffraction experiments (SXRD) and NAP-XPS [7] link surface relaxation to changes in surface moieties.
The intent of the experiment is to extent such concept to a particle and link internal stresses/surface-relaxations obtained from BCDI with the chemistry on its surfaces.

The B07 NAP-PXS end station can control the xyz positioning of the sample with a precision in the order of 10um, allowing to position the 50um beam[5] in the center of
any of the 100x100um squares of the sample.
At the center of each square is located one isolated particle.
This approach allows to study single particle behavior as function of the NAP-XPS atmosphere but also to retrieve the very same particle studied with BCDI on the SixS beamline.
As for the experiment in Ref. [7], we intent to apply the same temperatures and pressures for BCDI and NAP-XPS.
Temperatures and gas pressures will range between 450, and 650 K and from 0.5 to few tens of mBar respectively.
For each combination of temperature and ammonia to oxygen ratio, we will measure: O1s, N1s, Pt4d, Pt4f and Rh3d core levels that should carry information of the surface state.
Essential for the experiment is also to collect the gas phase composition for each condition to compare particles reactivity.

We expect to establish a connection between selectivity and surface moieties but also between surface moieties and particle internal stress. We also hope to be able to detect variations in the rhodium surface concentration related to the imposed gas conditions

B07 and its recent ambient pressure setup [5], that has now benefit from years of experience, push the high-pressure limits well into the mBar regime.
This new limit and the wide range of achievable temperatures/energies/pressures make it a unique tool for modern surface science and catalysis.
Timewise we expect to be able to characterize the nanoparticle(s) of interest as soon as they entered the analysis chamber from atmosphere and take reference scans after few oxidation/reduction cycles.
Reference scans will be taken on “as inserted” samples and on freshly oxidized/reduced particles with and without adsorbates reactants/products.
Similar scan will also be collected for the bare sapphire substrate. It is estimated that the first 2/3 days will suffice to collect references.
The operando experiments will be performed during the remaining 3/4 days of the beamtime.

\subsubsection{Experimental setup}

Since the XPS measurements were performed at different total pressures, the raw data had first to be reduced in order to analyse the different spectra in a systematic way.
The adopted workflow for the analysis of XPS data is to first align the recorded spectra on the Fermi edge that corresponds to the kinetic energy of the first electron that escapes the sample.
By doing so, one can be confident that any shift in the peak positions is due to chemical changes, such as the oxidation state of the sample, and not to charging effects of the sample (cite).

Secondly, to be able to quantify and compare the evolution of the peak intensity, one must normalize the intensity of the detected electron beam since the electron mean free path depends on the pressure in the reaction chamber \parencite{Willmott}.
The range of kinetic energy just before the absorption edge of Pt 4f was chosen since it had the best signal to noise ratio and does not depend on any experimental parameter besides the pressure.

Finally, for the peaks that showed a good signal to noise ratio, the fitting of the peak shape was realised thanks to the \textit{lmfit} \parencite{Newville2016} package by the means of the Doniach-equation which is the best approximation of the asymmetric peak shape resulting from the convolution of the analyser function and the photoelectron process in metals \parencite{Doniach_1970}.

\subsection{Experimental plan}\label{sec:SXRDPlan}

The oxidation of ammonia was measured at \qty{450}{\degreeCelsius} with less conditions than in sec. \ref{sec:BCDIAmmoniaOxidation} due to long measuring time necessary to collect a large volume of the reciprocal space.
To make certain that any possible surface relaxation effect was related to the presence of both ammonia and oxygen in the reactor, the sample was first exposed to \qty{80}{\milli\bar} of oxygen (i), followed by the introduction of \qty{10}{\milli\bar} ammonia (ii).
The pressure of oxygen in the reactor was then reduced to \qty{5}{\milli\bar} (ii), and to \qty{0}{\milli\bar} of oxygen (iv), keeping only \qty{10}{\milli\bar} of ammonia in the reactor (v).
The sample was characterized under \qty{500}{\milli\bar} of argon before and after the oxidation cycle, the total pressure in the reactor was kept to \qty{500}{\milli\bar} by adjusting the amount of argon.
    \section{Pt(111) single crystal studied at \qty{450}{\degreeCelsius}} \label{sec:SXRD111}

Platinum crystallises in a face-centred cubic structure with a lattice parameter $a_{Pt}$ equal to \qty{3.92}{\angstrom} at room temperature \parencite{Waseda1975}.
Its structure was first presented in sec. \ref{sec:ScatCrystal} to introduce the notions of crystals.

The arrangement of the Pt atoms on the (111) surface is hexagonal, which leads to the definition of the surface unit cell shown in fig. \ref{fig:SurfaceUnitCellPt111} to be able to better represent the surface arrangement of the Pt atoms.
The in-plane vectors $\vec{a}_{(111)}$ and $\vec{b}_{(111)}$ are of equal magnitude ($a_{Pt} / \sqrt{2} = \qty{2.775}{\angstrom})$, separated by \ang{120}.
The out-of-plane vector $\vec{c}_{(111)}$ is perpendicular to the (111) plane, and of magnitude $3 a_{Pt} / \sqrt{3} = \qty{6.797}{\angstrom}$.

\begin{SCfigure}
    \centering
    \includegraphics[trim=0 2cm 0 2cm, clip, width=0.70\textwidth]{/home/david/Documents/PhD/Figures/introduction/111.pdf}
    \caption{
        Face-centred cubic unit cell of Pt with (111) crystallographic plane drawn in green.
        $\vec{a}_{(111)}$, $\vec{b}_{(111)}$ and $\vec{c}_{(111)}$ are the $(111)$ surface unit cell vectors.
        There are three \{111\} planes spanned by the magnitude of $\vec{c}_{(111)}$ (blue, red and green on the figure).
    }
    \label{fig:SurfaceUnitCellPt111}
\end{SCfigure}

\subsection{Oxide growth under \qty{80}{\milli\bar} of oxygen}

To identify the presence of surface reconstructions and/or surface oxides, in-plane reciprocal space maps were collected at the atmospheres detailed in tab. \ref{tab:ConditionsSXRD} by rotating the in-plane sample and detector angles ($\omega$ and $\gamma$) from \ang{0} to \ang{120} to collect a third of the reciprocal space in the ($\vec{q}_x$, $\vec{q}_y$) plane, considering a hexagonal symmetry in the position of the Bragg peaks.

The reciprocal space in-plane maps were computed in both $q$-space (to obtain the interplanar spacing related to the observed signals) and ($hkl$)-space (fig. \ref{fig:MapsPt111A} and \ref{fig:MapsPt111B}) to visualise the arrangement of surface structures or surface relaxations in comparison with the hexagonal structure of the Pt atoms on the (111) surface, the $H$ and $K$ values being computed using the hexagonal surface unit cell described above.

\begin{figure}[!htb]
    \centering
    \includegraphics[width=0.495\textwidth]{/home/david/Documents/PhDScripts/SixS_2023_04_SXRD_Pt111/figures/map_hkl_76-115.pdf}
    \includegraphics[width=0.495\textwidth]{/home/david/Documents/PhDScripts/SixS_2023_04_SXRD_Pt111/figures/map_hkl_285-320.pdf}
    \includegraphics[width=0.495\textwidth]{/home/david/Documents/PhDScripts/SixS_2023_04_SXRD_Pt111/figures/map_hkl_481-516_patched.pdf}
    \includegraphics[width=0.495\textwidth]{/home/david/Documents/PhDScripts/SixS_2023_04_SXRD_Pt111/figures/map_hkl_681-689_patched.pdf}
    \caption{
        Reciprocal space in-plane maps collected under different atmospheres measured at \qty{25}{\degreeCelsius} under UHV and at \qty{450}{\degreeCelsius} under different atmospheres, with a total pressure equal to \qty{500}{\milli\bar}, computed using the hexagonal lattice of Pt(111).
    }
    \label{fig:MapsPt111A}
\end{figure}

The first map was collected at low pressure (\qty{<1e-5}{\milli\bar}) after the cleaning of the sample by sputtering and annealing (fig. \ref{fig:MapsPt111A} - a).
The ($\bar{1}$$\bar{1}$0) and ($\bar{2}$10) Bragg peaks can be observed, together with the bottom of [111]-oriented crystal truncation rods going through the [0, $\bar{2}$], [0, $\bar{1}$], [$\bar{1}$, 0], [$\bar{1}$, 1], and [$\bar{2}$, 0] positions in the (H, K) plane.
No difference can be observed when introducing \qty{500}{\milli\bar} of argon in the reactor (fig. \ref{fig:MapsPt111A} - b).

New peaks can first be observed under \qty{80}{\milli\bar} of oxygen, identified by different colour circles in fig. \ref{fig:MapsPt111A} (c), the corresponding interplanar spacings are given in tab. \ref{tab:InterplanarSpacingsPt111Oxygen}.
The angle between the peaks circled in grey and white and the peaks circled in red and purple is equal to \ang{60}, which could be the signature of the existence of two hexagonal surface structures, each rotated by $\pm \ang{6}$ with respect to the Pt(111) surface unit cell and with a larger in-plane lattice parameter (angles measured in q-space, fig. \ref{fig:481QSpace}).
No apparent second order peak can be linked to those rotated hexagonal structures.
Summing the two vectors going from the centre of the reciprocal space to the white and purple circled peaks yields the position of the black circled peak, the angle between both vectors is then equal to \ang{42.65} (angle measured in q-space, fig. \ref{fig:481QSpace}).

The measurement of the entire reciprocal space in-plane map took about \qty{3}{\hour}\qty{35}{\minute}, the entire map being a concatenation of multiple $\omega$ scans during which the plane of the sample is rotated, the first scan starting near the centre of the reciprocal space.
The contribution of each scan during the measurement is visible from their different background.
Therefore, there is approximately \qty{1}{\hour} between the measurement of the peaks circled in red, grey, purple and white, and the peak circled in black.

A second map was measured in a smaller region of the reciprocal space \qty{9}{\hour}\qty{30}{\minute} after the introduction of oxygen in the cell to see if the intensity and position of the newly found peaks changed (fig. \ref{fig:MapsPt111A} - d).
The measurement of this map took about \qty{1}{\hour}\qty{15}{\minute}.
Three additional peaks (circled in black in fig. \ref{fig:MapsPt111A} - d) appeared on a (8x8) coincidence with comparison to the hexagonal lattice of Pt(111), describing a hexagonal lattice of lattice parameter equal to $|\vec{a_{hex}}| = \qty{3.118}{\angstrom}$ (drawn in black dotted lines in fig. \ref{fig:MapsPt111A} - d).

The grey circled peak seems to have disappeared from the reciprocal map (fig. \ref{fig:MapsPt111A}, c, d), but an additional peak circled in red is revealed at \ang{120} from the other peak circled in red, and \qty{60} from the peak circled in grey, at the same distance from the centre (fig. \ref{fig:MapsPt111A} - d).
From these first measurement, a first hypothesis to explain the position of the observed peaks is given.
First, two rotated hexagonal structures appear with a larger in-plane lattice parameter in comparison to Pt(111) (fig. \ref{fig:MapsPt111A} - c).
Secondly, a (8x8) hexagonal structure appears, detected while measuring the large reciprocal space in-plane map, with a larger in-plane lattice parameter, but in the same direction as the in-plane unit cell vectors of Pt(111) (fig. \ref{fig:MapsPt111A} - d), explaining the position of the black circled peak in the first map.
It is possible that the disappearance of the grey circle peak is linked to some alignment problems, visible from the large low intensity region in fig. \ref{fig:MapsPt111A} (d).

Out-of-plane measurements were performed perpendicular to peaks belonging to the (8x8) hexagonal structure, [H, K] = [-0.89, 0] (fig. \ref{fig:LScans80} - a), and rotated hexagonal structures, [H, K] = [-0.15, -0.74] (fig. \ref{fig:LScans80} - b), to probe their 3D structures.
A second out-of-plane measurement was performed at [H, K] = [-1.78, 0] to probe the intensity of a potential second-order peak for the (8x8) hexagonal structure (fig. \ref{fig:LScans80} - c), \qty{10}{\hour} after the measurement perpendicular to [H, K] = [-0.89, 0].
A peak is detected, which supports the existence of a (8x8) structure.
The background-subtracted scattered intensity was integrated as a function of $L$ using the \textit{fitaid} module of \textit{binoculars} (fig. \ref{fig:LScans80}).

\begin{figure}[!htb]
    \centering
    \includegraphics[width=\textwidth]{/home/david/Documents/PhDScripts/SixS_2023_04_SXRD_Pt111/figures/l_scans_high_oxygen.pdf}
    \caption{
        2D detector slices at $L=0$ during out-of plane measurements for three different [H, K] positions under \qty{80}{\milli\bar} of \ce{O_2} (a-b-c).
        The integrated intensity is presented in (d) as a function of $L$.
    }
    \label{fig:LScans80}
\end{figure}

The integrated intensity perpendicular to [H, K] = [-0.15, -0.74] shows a small peak at $L\approx 0.2$ followed by a constant intensity, characteristic of non-periodic out-of-plane structures such as monolayers (fig. \ref{fig:SimROD}, Robinson et al. \cite*{Robinson1991}).

Four different bulk platinum oxides have been reported to exist in literature, $\alpha$-\ce{PtO_2}, $\beta$-\ce{PtO_2}, \ce{Pt_3O_4} and \ce{PtO}.
The corresponding experimental structures are described in tab. \ref{tab:PtOxides} together with the expected equilibrium structure found by Seriani et al \parencite*{Seriani2006, Seriani2008} from first-principles atomistic thermodynamics calculations and molecular dynamics simulations based on density functional theory.

\begin{table}[!htb]
\centering
% \resizebox{\textwidth}{!}{%
    \begin{tabular}{@{}lllll@{}}
    \toprule
     & \ce{PtO} & \ce{Pt_3O_4} & $\beta$-\ce{PtO_2} & $\alpha$-\ce{PtO_2} \\ \midrule
    Bravais lattice & Tetragonal & Cubic & Orthorhombic & Hexagonal \\
    a (\unit{\angstrom}) & 3.10 (3.08) & 5.65 (5.59) & 4.49 (4.48) & 3.14 (3.10) \\
    b (\unit{\angstrom}) & 3.10 (3.08) & 5.65 (5.59) & 4.71 (4.52) & 3.14 (3.10) \\
    c (\unit{\angstrom}) & 5.41 (5.34) & 5.65 (5.59) & 3.14 (3.14) & 5.81 (4.28-4.40) \\
    $\alpha$ & \ang{90} & \ang{90} & \ang{90} & \ang{90} \\
    $\beta$ & \ang{90} & \ang{90} & \ang{90} & \ang{90} \\
    $\gamma$ & \ang{90} & \ang{90} & \ang{90} & \ang{120}\\
    \bottomrule
    \end{tabular}%
    %}
    \caption{
    Experimental values taken from \cite{McBride1991} for \ce{PtO}, from \cite{Muller1968} for \ce{Pt_3O_4} and $\alpha$-\ce{PtO_2}, and from \cite{McBride1991} for $\beta$-\ce{PtO_2}.
    Theoretical values and table adapted from Seriani et al. \parencite*{Seriani2006}.
    Similar values have been reported by Nomiyama et al. \parencite*{Nomiyama2011} from DFT calculations.
    }
\label{tab:PtOxides}
\end{table}

The lack of peaks at $L=0$ in fig. \ref{fig:LScans80} for the (8x8) hexagonal structure shows that there is no bulk structure on the catalyst surface.
However, Seriani et al. \parencite*{Seriani2006} have also found that the most stable surface oxide on the Pt(111) surface is expected to be hexagonal $\alpha$-\ce{PtO_2}.
A high uncertainty regarding the lattice parameter in the $\vec{c}$ direction is reported in experimental and theoretically studies due to a poor crystallisation of the bulk oxide \parencite{Muller1968}, and to underestimation of weak interlayer Van der Waals forces in theoretical studies \parencite{Li2005}.

The formation of hexagonal on hexagonal surface $\alpha$-\ce{PtO_2} on Pt(111) is expected to induce a high amount of in-plane compressive strain in the oxide layer from the large difference between the Pt-Pt distance on the (111) surface (\qty{2.775}{\angstrom}) and the in-plane lattice parameter of $\alpha$-\ce{PtO_2} (\qty{3.14}{\angstrom}).
A rotation of \ang{30} is expected to reduce the in-plane strain, resulting in a (2X2) arrangement.
The presence of $\alpha$-\ce{PtO_2} on the Pt(111) surface was also shown to provide favourable special sites that could contribute to the catalytic activity \parencite{Li2005}.
The 3D structure of bulk $\alpha$-PtO$_2$ is presented in fig. \ref{fig:AlphaPtO2}.

\begin{SCfigure}
    \centering
    \includegraphics[trim=0 2.5cm 0 2.5cm, clip, width=0.5\textwidth]{/home/david/Documents/PhD/Figures/introduction/AlphaPtO2.pdf}
    \caption{
        $\alpha$-\ce{PtO_2} bulk unit cell.
        Platinum atoms are situated on the unit cell corners while the two oxygen atoms are at the positions $(1/3, 2/3, 1/4)$ and $(2/3, 1/3, 3/4)$.
    }
    \label{fig:AlphaPtO2}
\end{SCfigure}

Ellinger et al. \parencite*{Ellinger2008} reported the existence of surface $\alpha$-\ce{PtO_2} on Pt(111) in an experimental study using SXRD at similar conditions (\qty{500}{\milli\bar} of oxygen, at a temperature between \qty{245}{\degreeCelsius} and \qty{635}{\degreeCelsius}).
The in-plane lattice parameter was found equal to \qty{3.15}{\angstrom}, close to the reported bulk value of $\alpha$-PtO$_2$ (\qty{3.14}{\angstrom}).
Similar results were found by Ackermann \parencite*{Ackermann2007}.
Ellinger et al. \parencite*{Ellinger2008} proposed a surface model for surface $\alpha$-\ce{PtO_2} consisting of a full unit cell, with an out-of-plane lattice parameter reduced by \qty{15}{\percent} in comparison with the bulk structure (\qty{3.62}{\angstrom}), and with a lateral displacement of \qty{\approx33}{\percent} in the [110] direction of the Platinum atoms on the top layers with respect to the Pt atoms of the bottom layer, in contact with the Pt(111) surface.
The question of the epitaxial relation between both structures remains, since the proposed structure does come with a large misfit strain (\qty{13.5}{\percent}) when in a hexagonal on hexagonal arrangement.
% Ellinger et al. \parencite*{Ellinger2008} proposed
% As our fit is not compatible with domain formation, we are forced to conclude that oxide formation is initiated at step edges.
Similarly to the current experiment, an (8x8) coincidence is reported between the Pt(111) hexagonal surface and surface $\alpha$-\ce{PtO_2}.
The out-of-plane measurements also performed by Ellinger et al. \parencite*{Ellinger2008} perpendicular to [H, K] = [-0.89, 0] presents a peak at $L=0.65$ ($L$ computed using the distorted $\alpha$-\ce{PtO_2} unit cell).

The two $L$-scans performed in the current study were fitted using two Gaussian peaks of same full width at half maximum (FWHM) and a constant background (fig. \ref{fig:LScans80Fit}), the second peak near $L=1$ in the $L$-scan perpendicular to [H, K] = [-0.89, 0] comes from a Pt powder signal.

\begin{figure}[!htb]
    \centering
    \includegraphics[width=\textwidth]{/home/david/Documents/PhDScripts/SixS_2023_04_SXRD_Pt111/figures/ctr_reconstructions_fitting_result.pdf}
    \caption{
        Out-of-plane measurements and fit result under \qty{80}{\milli\bar} of \ce{O_2}.
    }
    \label{fig:LScans80Fit}
\end{figure}

A peak at [-0.89, 0, 0.65] is reported, but using the Pt(111) surface unit cell, with an out-of-plane lattice parameter of \qty{6.797}{\angstrom}, incompatible with the results presented in Ellinger et al \parencite*{Ellinger2008}.
It seems that a different out-of-plane structure is present in the current work.

Additional simulations are needed to fully understand the structures in this study, with different surface models including the \textit{spoked-wheel} superstructure identified \textit{via} scanning tunnelling microscopy (STM) by \cite{VanSpronsen2017, Boden2022} which also exhibits a (8x8) coincidence with the Pt(111) lattice.
Different surface oxides have also been identified by Miller et al \parencite*{Miller2011, Miller2014}, Fantauzzi et al \parencite*{Fantauzzi2017}, as well as Farkas et al. \parencite*{Farkas2017}, respectively under \qty{6.66}{\milli\bar} of \ce{O_2} at \qty{450}{\degreeCelsius} and under \qty{1}{\milli\bar} of oxygen between \qty{160}{\degreeCelsius} and \qty{410}{\degreeCelsius}.
\textcolor{Important}{which one..., mention stripe stuff}

\begin{figure}[!htb]
    \centering
    \includegraphics[width=\textwidth]{/home/david/Documents/PhDScripts/SixS_2023_04_SXRD_Pt111/figures/reflecto_80.pdf}
    \caption{
    	X-ray reflectivity curves measured in a specular geometry (full lines) under \qty{80}{\milli\bar} of oxygen.
    	Curves fitted using \text{GenX} are shown an order of magnitude below as dotted lines.
    }
    \label{fig:Reflecto80}
\end{figure}

Reflectivity curves measured in a specular geometry and fitted using \textit{GenX} \parencite{Bjorck2007, Glavic2022} are presented in fig. \ref{fig:Reflecto80}.
%The roughness is adjusted by the means of the surface root mean square roughness $\sigma$ in \unit{\angstrom}, as defined by Bennet et al. \parencite*{Bennett1961}.
The specular X-ray reflectivity simulations are conducted using the Parratt algorithm \parencite{Parratt1954}.
The classical way of implementing roughness is based on a Gaussian distribution, introducing corrective factors to the electric field amplitudes at the interfaces in accordance with the Nevot-Croce model \parencite{Nevot1980}.
For the computation of material refractive indices, the Henke tables \parencite{Henke1993} are employed.

A simple model was used consisting of a semi-infinite slab of platinum, on top of which is present a homogeneous layer of platinum oxide.
The density of the layer was set free during the minimisation process between zero and \qty{0.021172}{FU\per\cubic\angstrom} (formula unit per cubic Angström, computed from unit cell scattering length and volume), the value for $\alpha$-PtO$_2$.
The fitted density values is divided by two between the first (\qty{0.00727}{FU\per\cubic\angstrom}) and second (\qty{0.00358}{FU\per\cubic\angstrom}) measurement, and is always an order of magnitude below the expected value for a homogeneous layer of $\alpha$-PtO$_2$.
The layer was found to be \qty{13.89}{\angstrom} and \qty{14.185}{\angstrom} thick in the first and second reflectivity curves, with a very low roughness (\qty{<1e-7}{\angstrom}) in comparison with the roughness of the Pt surface.

By combining the information obtained from in-plane, out-of-plane and reflectivity measurements, it is probable that the platinum surface is first covered by two types of domains corresponding to two rotated hexagonal structures.
Secondly, a $\alpha$-\ce{PtO_2} surface oxide grows on the surface, with an average thickness of \qty{\approx 14}{\angstrom}, \textit{i.e.}, a few monolayers thick, confirmed by the presence of peaks on the L-scans, in contrast with the rotated structures.

% \begin{figure}[!htb]
%     \centering
%     \includegraphics[trim=0 1cm 0 1cm, clip, width=\textwidth]{/home/david/Documents/PhD/Figures/introduction/EpitaxyPtO2Pt111.pdf}
%     \caption{
%     	Epitaxy relationship between Pt(111) surface unit cell (in red) and $\alpha$-\ce{PtO_2} basal unit cell.
%     }
%     \label{fig:PtO2OnPt111}
% \end{figure}

\subsection{Near ambient pressure ammonia oxidation cycle}

Ammonia was subsequently introduced in the reactor cell to investigate the evolution of the platinum oxide structures, as well as the presence of additional surface structures or surface reconstructions during the oxidation of ammonia on the platinum surface (fig. \ref{fig:MapsPt111B} a-b).
The presence of $\alpha$-PtO$_2$ on Pt(111) has been shown to not hinder the oxidation of \ce{CO} by Ackermann et al. \parencite*{Ackermann2007}, that would then occur \textit{via} a Mars-Van Krevelen mechanism \parencite{Mars1954}, inducing a progressive roughening of the platinum surface.

\begin{figure}[!htb]
    \centering
    \includegraphics[width=0.495\textwidth]{/home/david/Documents/PhDScripts/SixS_2023_04_SXRD_Pt111/figures/map_hkl_830-865.pdf}
    \includegraphics[width=0.495\textwidth]{/home/david/Documents/PhDScripts/SixS_2023_04_SXRD_Pt111/figures/map_hkl_1041-1076.pdf}
    \includegraphics[width=0.495\textwidth]{/home/david/Documents/PhDScripts/SixS_2023_04_SXRD_Pt111/figures/map_hkl_1413-1448.pdf}
    \includegraphics[width=0.495\textwidth]{/home/david/Documents/PhDScripts/SixS_2023_04_SXRD_Pt111/figures/map_hkl_1458-1493.pdf}
    \caption{
        Reciprocal space in-plane maps collected under different atmospheres measured at \qty{450}{\degreeCelsius}, computed using the hexagonal lattice of Pt(111).
    }
    \label{fig:MapsPt111B}
\end{figure}

During the oxidation of ammonia, none of the previously observed peaks could not be discerned, while no additional peaks emerged.
Even when the oxygen pressure within the cell was reduced to \qty{5}{\milli\bar} (favouring alternate products) or eliminated at \qty{0}{\milli\bar}, there was no induction of surface reconstructions or surface structures.
This result shows that the reaction removes the oxide layer present of the surface until the structure disappears.

No transition between the reaction products is observed in the first hours following the introduction of ammonia in the reactor (fig. \ref{fig:RGA450Pt111_2}, the presence of the surface oxide at the beginning of the condition seemingly not influencing the product selectivity.
All reciprocal space in-plane maps following the introduction of ammonia to a final atmosphere of \qty{500}{\milli\bar} of Argon showed a pristine surface structure (fig. \ref{fig:MapsPt111B} - c, d), proving that the oxidation cycle is reproducible, and that ammonia effectively removed the different surface oxides that grew under \qty{80}{\milli\bar} of oxygen.
The low roughness of the platinum surface seems to be in contradiction with a possible Mars-Van Krevelen mechanism during the oxidation of ammonia at ambient pressure.

\begin{figure}[!htb]
    \centering
    \includegraphics[width=\textwidth]{/home/david/Documents/PhDScripts/SixS_2023_04_SXRD_Pt111/figures/reflecto_cycle.pdf}
    \caption{
    	X-ray reflectivity curves measured in a specular geometry (full lines) under different atmospheres during the oxidation of ammonia.
    	Curves fitted using \text{GenX} are shown an order of magnitude below as dotted lines.
    }
    \label{fig:ReflectoCycle}
\end{figure}

Reflectivity curves were also measured and fitted with \textit{GenX} under pure Argon atmosphere (before and after the oxidation cycle), under different reacting conditions, and with only ammonia and argon in the reactor, presented in fig. \ref{fig:ReflectoCycle}.
In accordance with the results of the in-plane reciprocal space maps in fig. \ref{fig:MapsPt111B}, no oxide layer could be detected on the platinum surface.
The roughness of the [111]-oriented platinum crystal decreases following the introduction of ammonia after having increased to \qty{\approx 2.61}{\angstrom} when being exposed to \qty{80}{\milli\bar} of oxygen for more than \qty{20}{\hour}, almost reaching its value before the start of the oxidation cycle.
The largest decrease in roughness is seen during the reacting conditions, underlining the importance of the mobility of adsorbed species on the catalyst surface in the roughness evolution.

\subsection{Oxide growth under \qty{5}{\milli\bar} of oxygen}

Following the complete ammonia oxidation cycle, the sample was put under \qty{5}{\milli\bar} of oxygen to see if any of the detected surface structures would come to grow again, and to de-correlate the effect of the simultaneous presence of ammonia and oxygen in the cell on the surface when favouring the production of nitrogen at lower oxygen pressures.
The sample was cleaned with two sputtering and annealing cycles before the introduction of oxygen.
Large and small reciprocal space in-plane maps were measured to have the best compromise between a higher temporal resolution of the oxide growth, and the possibility to detect other peaks from corresponding surface unit cells.
The larger reciprocal space in-plane maps are shown in fig. \ref{fig:LargeMapsPt111LowOxygen}.

\begin{figure}[!htb]
    \centering
    \includegraphics[width=\textwidth]{/home/david/Documents/PhDScripts/SixS_2023_04_SXRD_Pt111/figures/large_maps_05O2.pdf}
    \caption{
        Reciprocal space in-plane maps collected under \qty{495}{\milli\bar} of argon and \qty{5}{\milli\bar} of oxygen at \qty{450}{\degreeCelsius}, for different exposure times.
    }
    \label{fig:LargeMapsPt111LowOxygen}
\end{figure}

In the first map, that started directly after the stabilisation of the oxygen pressure at \qty{5}{\milli\bar} in the reactor cell, three low intensity peaks can be detected (fig. \ref{fig:LargeMapsPt111LowOxygen} - a).
Those peaks correspond to the same peaks detected previously shortly after having a pressure of \qty{80}{\milli\bar} in the cell (fig. \ref{fig:MapsPt111A} - c).
Only the grey circled peak cannot be detected until \qty{4}{\hour} of measurement, which is also when some peaks start to split in a similar pattern, exhibiting three distinct peaks around a more diffuse region (fig. \ref{fig:LargeMapsPt111LowOxygen} - c).

This splitting of the peaks could be the signature of a Pt(111) surface covered by different domains exhibiting similar hexagonal structures.
These domains are rotated by $\pm \ang{6}$ with respect to the Pt(111) surface unit cell (angles measured in q-space, fig. \ref{fig:2064QSpace}), with a higher magnitude of the in-plane lattice parameter, similarly to what has been observed under an oxygen pressure of \qty{80}{\milli\bar} (fig. \ref{fig:MapsPt111A} - c).

A peak can be observed at [-0.89, -0.89] in the only large reciprocal space in-plane map collected between \qty{41}{\minute} and \qty{4}{\hour}\qty{3}{\minute} after the start of the condition (fig. \ref{fig:LargeMapsPt111LowOxygen} - b).
The peak position is the same as in the first large reciprocal space in-plane map measured under \qty{80}{\milli\bar} of oxygen (fig. \ref{fig:MapsPt111A} - d).

The peak position corresponds to the ($\bar{1}\bar{1}0$)$_{\alpha-PtO_2}$ Bragg peak of bulk $\alpha-PtO_2$, but no peaks at [-0.89, 0, 0] and [0, -0.89, 0] can be detected (respectively ($\bar{1}00$)$_{\alpha-PtO_2}$ and ($0\bar{1}0$)$_{\alpha-PtO_2}$).
The nature of the peak situated at [-0.89, -0.89] is not easily determined since it appears before the [-0.89, 0] and [0, -0.89] peaks, all three positions verify the (8x8) hexagonal structure.
As mentioned before, it is possible to draw a surface unit cell including this peak together with the white and purple circled peak, but with an in-plane angle $\gamma^*$ equal in this case to \ang{42.13} (angle measured in q-space, fig. \ref{fig:2064QSpace}).

The area sampled during the measurement was extended in the two last maps which allows us to observe more signals around the ($1\bar{1}0$) region (fig. \ref{fig:LargeMapsPt111LowOxygen} - e, f)
For each map, two peaks circled by the same colour draw two vectors of the same magnitude going through the centre of the reciprocal space and separated by \ang{120}, in the last two maps there are three doublet of peaks separated by \ang{120}, in red, grey and purple ($q$-map available in appendix \ref{fig:2064QSpace}).
Moreover, the purple and red circled peaks are separated by \ang{60}, likewise for the grey and white circled peaks.
During this set of measurement, the grey circled peak did not disappear on the contrary to the measurements carried out under \qty{80}{\milli\bar} of oxygen, which furthermore supports the existence of at least two rotated hexagonal structures.

The ($\bar{1}0$0)$_{\alpha-PtO_2}$, ($0\bar{1}$0)$_{\alpha-PtO_2}$ and ($\bar{1}\bar{1}$0)$_{\alpha-PtO_2}$ peaks corresponding to surface ${\alpha-PtO_2}$ are detected in the large maps after \qty{\approx 24}{\hour} of measurements, circled in black.
Smaller maps taken with a shorter time interval are presented in fig. \ref{fig:SmallMapsPt111LowOxygen} in which the ($\bar{1}0$0)$_{\alpha-PtO_2}$ peak is detected after \qty{\approx 23}{\hour} of measurements, when it was already detected at the very least after \qty{10}{hour} under \qty{80}{\milli\bar} of oxygen (fig. \ref{fig:MapsPt111A} - d), highlighting the importance of the oxygen pressure in the surface oxidation.

\begin{figure}[!htb]
    \centering
    \includegraphics[width=\textwidth]{/home/david/Documents/PhDScripts/SixS_2023_04_SXRD_Pt111/figures/small_maps_05O2.pdf}
    \caption{
        Reciprocal space in-plane maps collected under \qty{495}{\milli\bar} of argon and \qty{5}{\milli\bar} of oxygen at \qty{450}{\degreeCelsius}, for different exposure times.
    }
    \label{fig:SmallMapsPt111LowOxygen}
\end{figure}

A more quantitative analysis of the different structures appearing during the exposition to oxygen was performed by integrating the scattered intensity around the white, red and black circled peak, present in the ($\bar{1}$10) region.
The average background was subtracted to each reciprocal space voxel before integration.
%, while the intensity of the crystal truncation rod ($\bar{1}$10) Bragg peak was used for normalisation to correct possible miss-alignement effects on the integrated intensity.
The starting time of each reciprocal space in-plane map is used as estimate for the time since the introduction of oxygen, the evolution of each peak is presented in fig. \ref{fig:HexBraggPeaks}.
The growth of the ${\alpha-PtO_2}$ surface oxide peak at [-0.89, -0.89] seems to be exponential after \qty{23}{\hour} of exposition.

\begin{figure}[!htb]
    \centering
    \includegraphics[width=\textwidth]{/home/david/Documents/PhDScripts/SixS_2023_04_SXRD_Pt111/figures/intensity_comparison_hex_reconstructions.pdf}
    \caption{
        Intensity evolution for peaks corresponding to the hexagonal surface structure and to the surface oxide.
    }
    \label{fig:HexBraggPeaks}
\end{figure}

The white peak is visible from the start of the exposition to oxygen (fig. \ref{fig:SmallMapsPt111LowOxygen}), whereas the red peak is only observed \qty{7}{hour} later with a lower intensity.
The intensity of both peaks (related to the other rotated hexagonal structures) quickly increases in the first hours and then plateaus until \qty{\approx 23}{\hour}, at which it increases together with the appearance of the $\alpha$-\ce{PtO_2} surface oxide.
Both peak intensity then starts to decreases after \qty{\approx 25}{\hour} of exposition, a few hours after the increase of the $\alpha$-\ce{PtO_2} peak intensity.

Several out-of-plane measurements have been carried out perpendicular to the white and red circled peaks, presented in appendix \ref{fig:LScans05}.
The peak intensity is constant as a function of $L$, similarly to the out-of-plane measurements presented in fig. \ref{fig:LScans80}, which shows that the rotated hexagonal structures are not three-dimensional.
No $L$-scan was measured perpendicularly to the ($0\bar{1}$0)$_{\alpha-PtO_2}$ peak.

Lowering the oxygen partial pressure in the reactor cell from \qtyrange{80}{5}{\milli\bar} has demonstrated that similar in-plane structures appear, but with a slower growth rate, which were both absent in the simultaneous presence of oxygen and ammonia in the reactor.
Additional studies are needed to understand if whether or not the rotated hexagonal structures act only as precursors to the appearance of surface $\alpha$-\ce{PtO_2} oxide, or if those peaks are still present after an extended exposure to oxygen.

% Finally, reflectivity curves were measured before the introduction of oxygen, as well as \qty{14}{\hour}\qty{30}{\minute} and \qty{23}{\hour}\qty{30}{\minute} after the introduction of oxygen, shown in fig. \ref{fig:reflecto}.
% The curves were fitted with \textit{GenX} following the method described earlier.
% The hexagonal structure was not yet detected when measuring the second reflectivity curve, a very low density oxide layer can be
% There is a significant drop of intensity in the second reflectivity curve,

% \begin{figure}[!htb]
%     \centering
%     \includegraphics[width=\textwidth]{/home/david/Documents/PhDScripts/SixS_2023_04_SXRD_Pt111/figures/reflecto_05.pdf}
%     \caption{
%     	X-ray reflectivity curves measured in a specular geometry (full lines) under different atmospheres during the oxidation of ammonia.
%     	Curves fitted using \text{GenX} are shown an order of magnitude below as dotted lines.
%     }
%     \label{fig:reflecto_5}
% \end{figure}

\subsection{Surface roughness and surface relaxation effects}

\begin{figure}[!htb]
    \centering
    \includegraphics[width=\textwidth]{/home/david/Documents/PhDScripts/SixS_2023_04_SXRD_Pt111/figures/ctr_a.pdf}
    \includegraphics[width=\textwidth]{/home/david/Documents/PhDScripts/SixS_2023_04_SXRD_Pt111/figures/ctr_b.pdf}
    \caption{
        Evolution of [111] crystal truncation rods measured perpendicular to three different Bragg peaks under different atmospheres.
    }
    \label{fig:CTRPt111}
\end{figure}

Crystal truncation rods have been measured perpendicularly to the ($\bar{2}10$), ($\bar{1}00$) and ($\bar{1}\bar{1}0$) positions \qty{6}{\hour} after the start of each condition, each measurement lasted for \qty{2}{\hour}.
The background-subtracted intensity of the CTR was integrated using the \textit{fitaid} module of \textit{binoculars} as a function of $L$ with the same integration range.

The CTR intensity is presented in fig. \ref{fig:CTRPt111}, no additional peak could be detected at any condition, besides the Pt Bragg peaks at higher $L$ values.
At first sight, the roughness seems to evolve during the exposition to different atmospheres, visible from the increase and decrease of the  intensity minimum near $L=1.5$ (or $L=2.5$ for the CTR recorded perpendicular to the ($\bar{1}$00) Bragg peak).
The intensity of the CTR under \qty{500}{\milli\bar} of argon before the oxidation cycle was progressively lost during the measurements due to problems with the sample heater, also visible in the reciprocal space in-plane map under the same condition in fig. \ref{fig:MapsPt111A}, and is therefore not shown in the data besides for the ($\bar{1}\bar{1}$0) Bragg peak.

\begin{figure}[!htb]
    \centering
    \includegraphics[trim=0 5cm 0 4.5cm, clip, width=0.45\textwidth]{/home/david/Documents/PhD/Figures/introduction/Pt111HexA.pdf}
    \includegraphics[trim=0 5cm 0 4.5cm, clip, width=0.45\textwidth]{/home/david/Documents/PhD/Figures/introduction/Pt111HexB.pdf}
    \caption{
        View from above (a) and from the side (b) of the Pt(111) surface with the atoms belonging to the A, B and C layers respectively coloured in green, red and blue.
        The size of the Pt atoms has been tuned from (a) to (b) to be able to visualise the arrangement of the A, B and C layers.
    }
    \label{fig:Pt111StructureSideAndTop}
\end{figure}

To investigate potential surface relaxation effects, the three CTR were fitted together to increase the number of data points at each condition using \textit{ROD}, with a simple model consisting of three ABC layers of [111] oriented platinum, C being the topmost layer (fig. \ref{fig:Pt111StructureSideAndTop}).
These three layers are on top of the rest of the crystal, so forth denominated as the \textit{bulk}, separating it from the \textit{surface} of the crystal in which relaxation effects can be detected.

Atoms on the same layer always share the same out-of-plane position and atomic displacement, introduced to see if surface relaxations effect could be detected at different atmospheres.
Four different models have been tested to fit the CTR intensity as a function of $L$.
In the first model, only the Pt atoms in the first topmost layer have a common out-of-plane displacement parameter.
In the second and third models, the two and three topmost layers have a unique out-of-plane atomic displacement, while in the fourth model the two topmost layers share the same out-of-plane atomic displacement.

During the fitting process, in-plane atomic displacements were excluded as one of the parameters to be adjusted because the presence of too many parameters posed challenges in achieving a successful convergence of the fitting routine.
The roughness parameter $\beta$ was set free between 0 and 0.5.% (unitless parameter, see $\beta$ roughness model detailed in sec. \ref{sec:CTR}).

\begin{figure}[!htb]
    \centering
    \includegraphics[width=\textwidth]{/home/david/Documents/PhDScripts/SixS_2023_04_SXRD_Pt111/figures/fit_comparison_1_last_layer_free.pdf}
    \caption{
        Fitting results for roughness parameter $\beta$ (a) and out-of-plane strain $\sigma_z$ (b) as a function of the experimental conditions.
    }
    \label{fig:CTRFit111}
\end{figure}

The best fit was found for the first model in which only the C layer was allowed to have a common out-of-plane atomic displacement $\delta_z$ between \qty{-0.05}{\angstrom} and \qty{0.05}{\angstrom}.
The position of the Pt atoms on the A and B layers were fixed following the position of the atoms in the bulk.

The strain with respect to the bulk was computed following eq. \ref{eq:StrainDiffraction}, the reference was set to the magnitude of the Pt(111) out-of-plane vector, \textit{i.e.} $|\vec{c}_{(111)}| = \qty{6.797}{\angstrom}$.
The evolution of the CTR roughness and of the strain of each layer is shown in fig. \ref{fig:CTRFit111}.

The roughness of the Pt(111) surface (fig. \ref{fig:CTRFit111} - a) increases with the introduction of Argon in the cell at \qty{450}{\degreeCelsius}, which is probably due to the presence of impurities in the gas flow.
The introduction of oxygen in the cell further increases the surface roughness, as expected from the formation of the different surface oxides visible in the in-plane reciprocal space maps.
Adding ammonia in the reactor cell, which was seen to remove the different surface oxides, has also the effect of decreasing the surface roughness.

The surface roughness increases slightly again when oxygen is removed, but reaches a value of 0 when both gases are removed from the reactor, falling back to an inert atmosphere, with a lower surface roughness than at the beginning of the measurement (visible also in fig. \ref{fig:CTRPt111} and consistent with the reflectivity results inf fig. \ref{fig:ReflectoCycle}).
It seems that the ammonia oxidation cycle has effectively \textit{cleaned} the surface from the presence of impurities or surface oxides.
Finally, the re-introduction of \qty{5}{\milli\bar} of oxygen increases the surface roughness again, in accordance with the formation of surface oxides detected during in-plane reciprocal space maps.

Overall, a very low amount of strain is detected on the surface, almost imperceptible when observing position of the CTR minimum in fig. \ref{fig:CTRPt111}.
From the fitting results, the topmost layer is already under tensile strain at UHV, further increased by the presence of Argon and possible impurities from the opening of the gas valves.

The largest evolution in the strain values comes from the high oxygen atmosphere, which has the effect of decreasing the surface strain, with an out-of-plane lattice parameter almost equal to the bulk value.
The formation of surface oxides observed under this atmosphere does not seem to have a very important effect on the surface relaxation state.
The introduction of ammonia increases again the surface strain, higher ammonia to oxygen ratio coincides with higher tensile strain.

The reacting conditions and the sole presence of ammonia have had the effect of removing the different surface oxides present on the platinum surface.
The roughness quantified \textit{via} reflectivity measurements and the evolution of the $\beta$ parameter in fig. \ref{fig:CTRFit111} (a) also decreases.
It is possible that the tensile strain in the last layer under Argon after the oxidation cycle corresponds to the equilibrium state of a clean Pt(111) surface.

Finally, the re-introduction of \qty{5}{\milli\bar} of oxygen decreases slightly the tensile strain on the topmost layer, a weaker but similar effect to the presence of \qty{80}{\milli\bar} of oxygen.
To conclude, no changes from tensile to compressive strain are observed during the reaction, the presence of oxygen alone in the reactor cell during which the growth of surface oxides has been monitored has the effect of lowering the surface strain, while the presence of ammonia has the opposite effect.
Different reacting conditions are related to the same direction of displacement but with a lower magnitude when lowering the partial pressure of oxygen.

\subsection{Surface species presence}

In order to link surface structure, surface moieties and reaction products, the Pt 4f, N 1s and O 1s XPS spectra were recorded at near ambient pressure at the B07 beamline (Diamond synchrotron), at \qty{450}{\degreeCelsius}.
The same order in the ammonia oxidation cycle was repeated as during the SXRD experiment, with the same ratio between reaction products.
No carrier gas is used to keep the total pressure constant, the reactant pressure is lowered to \qty{11}{\percent} of the pressure during the SXRD experiment as a compromise between high pressure and surface photoelectron detection.
The conditions have been resumed in tab. \ref{tab:ConditionsXPS}.
The mass spectrometer available at the B07 beamline allows us to monitor the presence of the reactants and products close to the sample surface.
The pressure of gases going through the same aperture of the electron analyser is measured, shown in fig. \ref{fig:XPS111RGA}.

\begin{figure}[!htb]
    \centering
    \includegraphics[width=\textwidth]{/home/david/Documents/PhDScripts/B07_2022_04_XPS/Figures/pt111_time.pdf}
    \caption{
        Evolution of reaction product partial pressures as a function of time during the XPS experiment on the Pt(111) single crystals at \qty{450}{\degreeCelsius}.
        Transition between conditions are indicated with dashed vertical lines.
    }
    \label{fig:XPS111RGA}
\end{figure}

% Describe rga
A high \ce{O_2}/\ce{NH_3} ratio equal to \num{8} favours the production of \ce{NO} as expected, accompanied by a high amount of water, small amounts of \ce{N_2} and \ce{N_2O} can also be detected.
Approximately half of the pressure of ammonia is still detected, which means that the oxygen cannot be considered to be in excess, and that the complete oxidation of ammonia is probably limited by the availability of active sites.

Lowering the amount of oxygen by reducing the \ce{O_2}/\ce{NH_3} ratio to \num{0.5} has the remarkable effect of shifting the reaction selectivity entirely towards \ce{N_2}, water is also detected.
\ce{H_2} coming from the simultaneous dissociation of \ce{NH_3} can be measured, not observed under a higher pressure of oxygen, which means that this reaction is not favoured when oxygen is present in the reactor.
Oxygen being undetected by the mass spectrometer when the \ce{O_2}/\ce{NH_3} ratio is equal to \num{0.5}, all of the introduced oxygen dissociates on the catalyst surface and participates in the production of \ce{N_2} and \ce{H_2O} via the oxidation reaction.
Ammonia can be thus considered to be in excess, and partly decomposing towards \ce{N_2}.
The surface sites are probably occupied mainly by nitrogen-rich species that cannot find a nearby oxygen or OH to react with, eventually decomposing towards nitrogen.

The removal of oxygen shows that more ammonia is consumed but without producing water.
Only the dissociation of ammonia happens on the catalyst surface, the production of nitrogen decreases even though more ammonia is consumed.
It is not clear why more nitrogen is produced under the presence of oxygen, when more ammonia is used for less production of \ce{N_2} after the removal of oxygen.
Both the oxidation and dissociation of ammonia yield 0.5 nitrogen for each molecule of ammonia.
The transition around \qty{28.5}{\hour} is due to a problem in the monitoring of the reactor total pressure which was not correctly set to \qty{1.1}{\milli\bar}.

\subsubsection{N 1s and O 1s levels}

N 1s and O 1s levels were recorded to probe for the existence of specific surface species, allowing us to obtain more information about the reaction mechanism, as well as the link between surface state and selectivity.
The evolution of the N 1s and O 1s XPS spectra for different atmospheres is presented in fig. \ref{fig:O1sN1sPt111}.
Binding energy are given with reference to the Fermi level, all the reported peaks and corresponding species are detailed in tab. \ref{tab:XPSPt111}.

% high ox
The oxidation cycle is started by introducing \qty{8.8}{\milli\bar} of oxygen in the reactor.
The presence of gas phase oxygen (\ce{O_{2,g}}) can logically be confirmed in the O 1s spectra by a characteristic peak doublet around \qty{538}{\eV}.
The positions are shifted in energy with respect to literature (\qty{539.3}{\eV}, \qty{540.4}{\eV}, Avval et al. \cite{Avval2022}).
This signal is also present when the \ce{O_2}/\ce{NH_3} ratio is equal to 8, but not when equal to 0.5, confirming that all the oxygen is used in the reactor.
\ce{O_{2,g}} is also detected when only \qty{0.55}{\milli\bar} of oxygen is in the cell.
A very broad signal extending from \qtyrange{528}{532}{\eV} is probably hiding a various amount of peaks of similar intensity, linked to the presence of oxygen-rich species.

Fisher et al \parencite*{Fisher1980} report adsorbed oxygen (\ce{O_{a}}) to give a peak at \qty{529.8}{\eV} on Pt(111), with adsorbed hydroxyl groups (\ce{OH_{a}}) at \qty{531}{\eV} when exposing the surface to water.
Peuckert et al. \parencite*{Peuckert1984} have studied various oxidised Pt surfaces and indexed a peak at \qty{530.2}{\eV} for \ce{O_{a}} on Pt(111), and \ce{OH_{a}} at \qty{531.5}{\eV} for polycristalline Pt.
Derry et al. \parencite*{Derry1984} report a peak at \qty{530.8}{\eV} for \ce{O_{a}} on Pt(111) during its exposition to oxygen, while Zhu et al. \parencite*{Zhu2003} report \ce{O_{a}} at \qty{529.9}{\eV} when probing the dissociation of \ce{NO} on the Pt(111) surface.
Fantauzzi et al. \parencite*{Fantauzzi2017} report oxygen surface species at \qty{529.7}{\eV} during the oxidation of Pt(111) at \qty{225}{\degreeCelsius}, similarly to Miller et al. \parencite*{Miller2014}.

During a recent study of the oxidation of ammonia at different pressures and \ce{O_2}/\ce{NH_3} ratio on Pt(111), (2x2) chemisorbed oxygen and hydroxyl groups were reported respectively at \qty{529.7}{\eV} and \qty{531.4}{\eV} in \qty{1}{\milli\bar} of oxygen at \qty{325}{\degreeCelsius} \parencite{Ivashenko2021}.

In the current XPS study, the signal to noise ratio is too low under the presence of \qty{8.8}{\milli\bar} of oxygen to characterise the exact surface state, it is probable however than hydroxyl groups as well as atomic oxygen are adsorbed on the surface around \qty{529.7}{\eV} and \qty{531.4}{\eV} respectively.
Similar O 1s peaks can be seen when the oxygen pressure is equal to \qty{0.55}{\milli\bar} and without ammonia, but with a higher apparent amount of atomic oxygen species in comparison to hydroxyl groups.

\begin{figure}[!htb]
    \centering
    \includegraphics[width=\textwidth]{/home/david/Documents/PhDScripts/B07_2022_04_XPS/Figures/Pt111/O1sN1s_700.pdf}
    \caption{
        Spectra collected at the O 1s (a) and N1 s (b) levels under different atmospheres at \qty{450}{\degreeCelsius} with an incoming photon energy of \qty{700}{\eV}.
        The spectra are normalised and shifted in intensity to highlight the presence of different peaks.
    }
    \label{fig:O1sN1sPt111}
\end{figure}
\begin{table}[!htb]
\centering
\resizebox{\textwidth}{!}{%
    \begin{tabular}{@{}ll|lllllll@{}}
    \toprule
    \multirow{3}{*}{Partial pressures (mbar)} & \ce{Ar}   & 1 & 0   & 0   & 0    & 0   & 1 & 0    \\
                                              & \ce{NH_3} & 0 & 0   & 1.1 & 1.1  & 1.1 & 0 & 0    \\
                                              & \ce{O_2}  & 0 & 8.8 & 8.8 & 0.55 & 0   & 0 & 0.55 \\
    \midrule
    Gas presence (decreasing & & Ar & \ce{O_2} & \ce{O_2}, \ce{H_2O}, \ce{NO}   & \ce{H_2O}, \ce{NH_3} & \ce{H_2}, \ce{NH_3} & Ar & \ce{O_2} \\
    pressure order)          & &    &          & \ce{NH_3}, \ce{N_2}, \ce{N_2O} & \ce{N_2}, \ce{H_2}   & \ce{N_2}            &    &          \\
    \midrule
    \multicolumn{2}{l|}{N 1s: peak positions}
        & No data          & No peak          & \qty{402.6}{\eV} & \qty{404.1}{\eV} & \qty{405.3}{\eV} & \qty{400.4}{\eV} & No peak          \\
     &  &                  &                  &                  & \qty{399.8}{\eV} & \qty{400.8}{\eV} & \qty{398.4}{\eV} &                  \\
     &  &                  &                  &                  & \qty{397.5}{\eV} & \qty{398.4}{\eV} &                  &                  \\
    \multicolumn{2}{l|}{Attributed surface species}
        &                  &                  & Not assigned     & \ce{N_{2,g}}     & \ce{N_{2,g}}     & \ce{NH_{3,a}}    &                  \\
     &  &                  &                  &                  & \ce{NH_{3,a}}    & \ce{NH_{3,g}}    & \ce{NH_{x,a}}    &                  \\
     &  &                  &                  &                  & \ce{N_a}         & \ce{NH_{x,a}}    &                  &                  \\
    \midrule
    \multicolumn{2}{l|}{O 1s: peak positions}
        & \qty{532.4}{\eV} & \qty{538.2}{\eV} & \qty{538.5}{\eV} & \qty{534.0}{\eV} & \qty{532.4}{\eV} & \qty{534.0}{\eV} & \qty{538.3}{\eV} \\
     &  &                  & \qty{537.1}{\eV} & \qty{537.5}{\eV} & \qty{532.0}{\eV} &                  & \qty{532.4}{\eV} & \qty{537.2}{\eV} \\
     &  &                  & \qty{531.4}{\eV} & \qty{534.0}{\eV} &                  &                  &                  & \qty{531.4}{\eV} \\
     &  &                  & \qty{529.7}{\eV} &                  &                  &                  &                  & \qty{529.7}{\eV} \\
    \multicolumn{2}{l|}{Attributed surface species}
        & \ce{H_2O_a}      & \ce{O_{2,g}}     & \ce{O_{2,g}}     & \ce{H_2O_g}      & \ce{H_2O_a}      & \ce{H_2O_g}      & \ce{O_{2,g}}     \\
     &  &                  & \ce{O_{2,g}}     & \ce{O_{2,g}}     & \ce{H_2O_a}      &                  & \ce{H_2O_a}      & \ce{O_{2,g}}     \\
     &  &                  & \ce{OH_a}        & \ce{H_2O_g}      &                  &                  &                  & \ce{OH_a}        \\
     &  &                  & \ce{O_a}         &                  &                  &                  &                  & \ce{O_a}         \\
    \bottomrule
    \end{tabular}%
    }
    \caption{Indexing of peaks measured during the oxidation of ammonia of the Pt(111) surface.}
\label{tab:XPSPt111}
\end{table}

% ratio 8
No clear nitrogen species can be detected in the N 1s spectra when introducing \qty{1.1}{\milli\bar} of ammonia in the reactor while keeping the pressure of oxygen equal to \qty{8.8}{\milli\bar}.
The total pressure in the cell is probably too high to resolve the signals corresponding to different adsorbed nitrogen species.
The presence of gas phase nitric oxide (\ce{NO_{g}}), ammonia (\ce{NH_{3,g}}), nitrogen (\ce{N_{2,g}}), and nitrous oxide (\ce{NO_{2,g}}) is also expected from the mass spectrometer data but could not be detected.
The main product, \ce{NO_{g}}, is expected between \qty{404.5}{\eV} and \qty{406.7}{\eV}, observed during reacting conditions with an equal amount of oxygen and ammonia, at a total pressure of \qty{1}{\milli\bar}, and temperature of \qty{325}{\degreeCelsius} \parencite{Ivashenko2021}.
The same study reports \ce{N_{2,g}} between \qty{403.9}{\eV} and \qty{404.8}{\eV} and \ce{NH_{3,g}} at \qty{400.4}{\eV}, while \ce{NH_{3,g}} was also detected at \qty{400.7}{\eV} under \qty{1}{\milli\bar} of ammonia.
The peak at \qty{402.6}{\eV} in this study could not be indexed.

Likewise, no clear adsorbed nitrogen species could be detected in the O 1s level.
For example, molecular \ce{NO} is expected to yield peaks between \qty{530}{\eV} and \qty{532}{\eV} \parencite{Kiskinova1984, Zhu2003, Gunther2008}.
Nevertheless, the presence of those adsorbed species cannot be ruled out due to the high pressure in the chamber lowering the signal to noise ratio.
Gas phase water (\ce{H_2O_{g}}) is visible in the O 1s spectra by a peak at \qty{534}{\eV}, as reported during the oxidation of ammonia by Weststrate et al. \parencite*{Weststrate2006}, proving the catalytic activity.
The energy difference between the low energy peak of \ce{O_{2,g}} and \ce{H_2O_{g}}, \qty{3.5}{\eV}, is close to the difference reported in literature for pure gas phases, equal to \qty{3.3}{\eV} \parencite{Linford2019, Avval2022}.

% It is possible that the two peaks of gas phase \ce{O_2} are hiding the signal of gas phase \ce{NO}, expected to be near \qty{538}{\eV}.
% Indeed, the ratio between both \ce{O_{g}} peaks goes from to \num{2.02} to \num{2.54}.

% ratio 0.5
When lowering the pressure of oxygen to \qty{0.55}{\milli\bar}, a condition under which nitrogen-rich products are favoured, three peaks can be detected in the N 1s level.
The peak at \qty{397.5}{\eV} is characteristic of adsorbed atomic nitrogen (\ce{N_{a}}) on Pt(111) \parencite{vandenBroek1999, Zhu2003}.
The peaks at \qty{399.8}{\eV} and \qty{404.1}{\eV} are attributed to adsorbed ammonia (\ce{NH_{3,a}}) and \ce{N_{2,g}}.
The reason why we attribute the peak at \qty{399.8}{\eV} to \ce{NH_{3,a}} is because a peak at \qty{400.8}{\eV} is detected when only ammonia is in the cell, which corresponds to the aforementioned \ce{NH_{3,g}}.
The energy difference between \ce{N_{a}}, \ce{NH_{a}}, and \ce{NH_{2,a}} is reported to be approximately \qty{0.95}{\eV}, \qty{1.9}{\eV} in total on Pt(111) \parencite{Ivashenko2021}, which is too few to link the peak at \qty{399.8}{\eV} to \ce{NH_{2,a}} with respect to the \ce{N_{a}} peak.
% Moreover, \ce{NH_{3,a}} is reported at \qty{400.0}{\eV} on Pt(410) by Günther et al. \parencite*{Gunther2008}.

At this condition, no oxygen is anymore detected in the reactor, as observed with the mass spectrometer.
No \ce{OH_{a}} and \ce{O_{a}} peaks can be detected even though the total pressure was divided by \num{6}, increasing the detection of photo-electrons.
Adsorbed water groups (\ce{H_2O_{a}}) are reported between \qty{532.2} and \qty{532.9} eV on Pt(111) depending on the surface coverage \parencite{Fisher1980, Kiskinova1985}, to which we tentatively attribute the peak at \qty{532}{\eV}.

The presence of adsorbed water supports a Langmuir-Hinshelwood mechanism with quick stripping of hydrogen from \ce{NH_{x,a}} species by \ce{OH_a} and \ce{O_a}, eventually forming adsorbed water.
The de-hydrogenation process must be limiting the catalytic activity due to the lack of available adsorbed oxygen species near adsorbed ammonia, which could be why we observe \ce{NH_{3,a}} but not \ce{NH_{x,a}}.
Even though we cannot measure the presence of \ce{OH_a} and \ce{O_a}, an Elley-Rideal mechanism in which the \ce{O_2} dissociation happens without adsorbing on the catalyst surface is highly unlikely.
Moreover, it is clear that oxygen adsorbs on the surface from the peaks visible without ammonia present in the cell.
The presence of adsorbed nitrogen could be due to a long residual time on the catalyst before recombination and desorption of \ce{N_2}.
This hypothesis is in accordance with the commonly accepted reaction mechanism detailed in sec. \ref{sec:Mechanism}.

% ammonia
Once we remove oxygen, gas phase water disappears from the O 1s level as the oxidation reaction can no longer happen.
Adsorbed water is still visible, possibly from long desorption time before producing water, or from contaminants.
The energy level differs by \qty{0.5}{\eV} from adsorbed water during reacting conditions, signifying different electronic environments.

As observed in the mass spectrometer, the dissociation of ammonia towards nitrogen still occurs, a slightly shifted \ce{N_{2,g}} peak is reported at \qty{405.3}{\eV}, explained by a change in the work function for gas species exclusively \parencite{Starr2021}.
The large peak linked to atomic nitrogen has disappeared, a peak linked to \ce{NH_{3,g}} is reported as well as a large and weak peak probably linked to \ce{NH_x} groups.
The dissociation of ammonia without oxygen is reported to be slow without the help of oxygen species on Pt(111) \parencite{Offermans2006,Offermans2007, Imbihl2007, NovellLeruth2008}.
This could explain why such a large peak is observed and why \ce{NH_{3,g}} is observed rather than \ce{NH_{3,a}}, since most of the adsorption sites are probably occupied by \ce{NH_x} species.
The difficulty to fully dissociate ammonia is also correlated to the production of hydrogen, it is possible that the combination of two hydrogen removed from ammonia to produce \ce{H_2} is slow, and thus occupies part of the adsorption sites.

% argon
The introduction of argon and removal of ammonia increases the \ce{H_2O_{g}} signal, possibly from contaminants.
Some nitrogen rich species are still visible, possibly from long desorption time.
Removing argon and introducing \qty{0.55}{\milli\bar} of oxygen removes all the N 1s peaks, probably by the oxidation of the leftover \ce{NH_x} species.
The same peaks are observed in the O 1s level as under \qty{8.8}{\milli\bar} of oxygen, linked to hydroxyl groups, \ce{O_{2,g}}, and different atomic oxygen states.

\subsubsection{Pt 4f level}

\begin{figure}[!htb]
    \centering
    \includegraphics[width=\textwidth]{/home/david/Documents/PhDScripts/B07_2022_04_XPS/Figures/Pt111/Pt4f_550_no_fit_merged.pdf}
    \caption{
        Spectra collected at the Pt 4f level under different atmospheres at \qty{450}{\degreeCelsius} with an incoming photon energy of \qty{550}{\eV}.
        A Shirley-type background has been subtracted from all XPS spectra.
        Normalisation performed first by the background intensity and secondly by the maximum intensity to allow a qualitative comparison between different total pressures.
        Spectra before normalisation are shown on the top left.
    }
    \label{fig:Pt4fPt111}
\end{figure}

% For the peaks that showed a good signal to noise ratio, the fitting of the peak shape was realised thanks to the \textit{lmfit} \parencite{Newville2016} package by the means of the Doniach-equation which is the best approximation of the asymmetric peak shape resulting from the convolution of the analyser function and the photoelectron process in metals \parencite{Doniach1970}.

The Pt 4f level was also measured to report possible differences in the electronic configuration of surface platinum atoms.
From the previous SXRD measurements, it was seen that a surface $\alpha$-\ce{PtO_2} oxide grows under the presence of \qty{5}{\milli\bar} of oxygen, but only after at least \qty{22}{\hour}.
Another structure that was determined to be a precursor of surface $\alpha$-\ce{PtO_2} oxide was measured in the minutes following the introduction of oxygen.
This structure could correspond to the oxide stripe hypothesised to be a precursor for surface $\alpha$-\ce{PtO_2} oxide on Pt(111) by Hawkins et al. \parencite*{Hawkins2009}.
Therefore, if surface $\alpha$-\ce{PtO_2} oxide is not expected here since the duration of each condition is below \qty{5}{\hour}, the precursor may be associated to peaks in the O 1s level.

Miller et al. \parencite*{Miller2011} have measured two peaks at \qty{72.1}{\eV} and \qty{73.5}{\eV} under \qty{6.66}{\milli\bar} of oxygen at \qty{450}{\degreeCelsius}.
The \qty{72.1}{\eV} peak is assigned to (i) oxygen between the metallic surface and surface $\alpha$-\ce{PtO_2} oxide as reported by Ellinger et al. \parencite*{Ellinger2008}.
The \qty{73.5}{\eV} peak is attributed to (ii) Pt atoms within the trilayer oxide structure.

However, both peaks are absent under a pressure of \qty{0.66}{\milli\bar} at \qty{350}{\degreeCelsius}, for which the presence of (iii) p(2x2) chemisorbed oxygen and (iv) "4O" oxide surface stripes are linked to two other peaks, respectively \qty{71.1}{\eV} and \qty{71.6}{\eV}-\qty{71.7}{\eV}.

In a following study, slightly shifted high intensity peaks at \qty{72.2}{\eV} and \qty{73.6}{\eV} are linked to the presence of surface $\alpha$-\ce{PtO_2} oxide fully covering a Pt(111) crystal \parencite{Miller2014}.
The photon energy for both experiments is equal to \qty{275}{\eV}, whereas the photon energy is here equal to \qty{550}{\eV}.
Thus, both studies by the Miller group are more sensitive to the surface structure since the photons do not penetrate as deep in the catalyst.

Interestingly the preparation of the crystal surface to grow surface $\alpha$-\ce{PtO_2} oxide is not too far from this study.
The Pt(111) crystal was exposed to \qty{13.33}{\milli\bar} of oxygen for \qty{10}{\minute} while cycling the temperature from \qtyrange{25}{525}{\degreeCelsius} four times.

A small peak can be identified near \qty{71.6}{\eV} in the current experiment under \qty{8.88}{\milli\bar} of oxygen that could correspond to the oxide stripe structure.
Its presence is not certain since the intensity is very low, corresponding potential peaks in the O 1s level cannot be resolved either.

No clear peak could be detected near \qty{73.5}{\eV} in the Pt 4f$_{5/2}$ level, but a very small peak can be seen at \qty{72.1}{\eV}.
Since the presence of surface $\alpha$-\ce{PtO_2} oxide was linked to very high intensity peaks by Miller et al. \parencite{Miller2014}, we can safely assume that this structure is not present in this study.
It seems that the temperature cycling is crucial to grow $\alpha$-\ce{PtO_2} on Pt(111).

The peak position is shifted by \qty{0.09}{\eV} after the introduction of \qty{8.88}{\milli\bar} of oxygen in the cell compared to the presence of \qty{1}{\milli\bar} of argon.
Overall, the lack of clear peak corresponding the "4O" oxide stripe and $\alpha$-\ce{PtO_2} phases shows that the oxygen on the Pt(111) surface is mostly chemisorbed, which could also be why the peak is slightly shifted towards \qty{71.1}{\eV}.

When introducing \qty{1}{\milli\bar} of ammonia in the reactor, the Pt 4f$_{5/2}$ peak is shifted back to the position under inert atmosphere, while a new component is measured at \qty{70.5}{\eV}.
Reducing the pressure of oxygen to \qty{0.55}{\milli\bar} further increases the intensity of this component compared to the maximum peak intensity.
Removing oxygen but keeping ammonia in the reactor removes this peak.
Since the intensity of this peak increases when the \ce{O_2}/\ce{NH_3} ratio decreases, \textit{i.e.} when all the oxygen is consumed, it is probably linked to the presence of adsorbed nitrogen species on the platinum surface.
The absence of this peak under the presence of ammonia in the cell, for which adsorbed nitrogen can not be detected, supports a link with \ce{N_a}.
Multiple peaks may also be present.

Only a smaller shift is repeated when \qty{0.55}{\milli\bar} of oxygen are introduced after \qty{1}{\milli\bar} or argon, approximately equal to \qty{0.2}{\eV} after the ammonia oxidation cycle, both spectra are very similar, some nitrogen species possibly left on the surface are probably removed by oxygen which explains the lower intensity near \qty{70.5}{\eV}.

% Parkinson et al. \parencite*{Parkinson2003} measured a peak at \qty{76.8}{\eV} after oxygen adsorption at room temperature, which disappeared after annealing at \qty{500}{\degreeCelsius}.
% This peak was linked to a peak between \qty{530.2}{\eV} and \qty{530.8}{\eV} in the O 1s level, and asigned to a sub-surface oxide.
% Weaver et al. \parencite*{Weaver2005} have reported at peak at \qty{76.9}{\eV} at \qty{175}{\degreeCelsius} after deposition of oxygen on the surface.
    \newpage
\section{Surface x-ray diffraction on a Pt (100) single crystal} \label{sec:SXRD100}

% check values for in plane lattice parameter from orientation matrix
% time since oxidation
% keep the same notation for [H, K]

A similar experiment was carried out on a different surface of platinum, namely the Pt (100) surface.
The arrangement of the Pt atoms on the (100) surface is square, the distance between in-plane neighbouring Pt atoms is smaller than between out-of-plane atoms.
A surface unit cell must be derived, shown in fig. \ref{fig:SurfaceUnitCellPt100}, to be able to better represent the surface arrangement of the Pt atoms.
The in-plane vectors $\vec{a}_{(100)}$ and $\vec{b}_{(100)}$ are of equal magnitude ($a_{Pt} / \sqrt{2} = \qty{2.775}{\angstrom})$, separated by \ang{90}.
The out-of-plane vector $\vec{c}_{(100)}$ is perpendicular to the (100) plane, and of magnitude $a_{Pt} = \qty{3.9242}{\angstrom}$.

\begin{SCfigure}
    \centering
    \includegraphics[trim=0 2cm 0 2cm, clip, width=0.70\textwidth]{/home/david/Documents/PhD/Figures/introduction/100.pdf}
    \caption{
        Face-entered cubic unit cell of Pt with $(100)$ crystallographic plane drawn in green.
        $\vec{a}_{(100)}$, $\vec{b}_{(100)}$ and $\vec{c}_{(100)}$ are the $(100)$ surface unit cell vectors.
    }
    \label{fig:SurfaceUnitCellPt100}
\end{SCfigure}

\subsection{Ammonia oxidation cycle}

Reciprocal space in-plane maps were collected using the same experimental setup and at the same atmospheres detailed in tab. \ref{tab:ConditionsSXRD}, to probe the structural evolution of the sample during the oxidation of ammonia.
The total pressure is always kept to \qty{500}{\milli\bar}.
Considering the square symmetry in the position of the Bragg peaks, the in-plane reciprocal space maps were collected by rotating the in-plane sample and detector angles ($\omega$ and $\gamma$) from \ang{0} to \ang{90} to collect a quarter of the reciprocal space in the ($\vec{q}_x$, $\vec{q}_y$) plane

The reciprocal space in-plane map were computed in both $q$-space (to obtain the interplanar spacing related to the observed signals) and ($hkl$)-space to visualise the arrangement of surface structures or surface relaxations in comparison with the structure of the Pt atoms on the (100) surface, the $h$ and $k$ values being computed using the square surface unit cell of the Pt (100) surface.

\begin{figure}[!htb]
    \centering
    \includegraphics[width=0.495\textwidth]{/home/david/Documents/PhDScripts/SixS_2022_01_SXRD_Pt100/figures/Map_hkl_surf_or_1335-1375.pdf}
    \includegraphics[width=0.495\textwidth]{/home/david/Documents/PhDScripts/SixS_2022_01_SXRD_Pt100/figures/Map_hkl_surf_or_1596-1635_patched.pdf}
    \caption{
        Reciprocal space in-plane maps collected under different atmospheres measured at \qty{450}{\degreeCelsius}, computed using the surface lattice of Pt (100).
    }
    \label{fig:MapsPt100A}
\end{figure}

The first map was collected under \qty{500}{\milli\bar} of Ar, after the cleaning of the sample by sputtering and annealing.
The (200), (1$\bar{1}$0) and (0$\bar{2}$0) Bragg peaks can be observed, together with the extremity of [100]-oriented crystal truncation rods going through the [0, 1, 0] and [0, $\bar{1}$, 0] positions in reciprocal space.

Two different types of peaks can be observed under \qty{80}{\milli\bar} of oxygen, identified by red and green circles in fig. \ref{fig:MapsPt100A}.
The red peaks are situated at intermediate positions in reciprocal space in comparison with the Pt (100) lattice [H, K] = [1, $\bar{0.5}$], [H, K] = [1, $\bar{1.5}$], [H, K] = [0.5, $\bar{1}$] and [H, K] = [0.5, $\bar{1.5}$].
The green peaks are slightly shifted from the red peaks by the same amount $\delta = 0.07$ in either H or K.

Four consecutive out-of-plane measurements were performed perpendicular to four peaks, two of each type, up to $L=3.5$ to probe the related out-of-plane structure.
The background-subtracted intensity was integrated as a function of $L$ using the \textit{fitaid} module of \textit{binoculars} (fig. \ref{fig:LScansHighOxygenPt100}).

\begin{figure}[!htb]
    \centering
    \includegraphics[width=\textwidth]{/home/david/Documents/PhDScripts/SixS_2022_01_SXRD_Pt100/figures/l_scans_high_oxygen_no_map.pdf}
    \caption{
        Out-of plane measurements for four different positions under \qty{80}{\milli\bar} of \ce{O_2}, and \qty{420}{\milli\bar} of \ce{Ar}.
    }
    \label{fig:LScansHighOxygenPt100}
\end{figure}

The intensity as a function of $L$ for the shifted peaks was found to quickly decrease down to zero, which shows that they do not correspond to 3D structures but are more characteristic of monolayers with no out-of-plane periodicity (fig. \ref{fig:SimROD}, Robinson et al. \cite*{Robinson1991}).
No corresponding surface unit cell could be derived from their in-plane positions.

However, four peaks at different $L$ values are visible on both $L$-scans perpendicular to the red circled peak, including at $L=0$, which is characteristic of a bulk structure, \textit{i.e.} more than a few unit cells thick \parencite{Robinson1991}.
The peaks were fitted using a model with four Gaussian peaks and a constant background, the same full width at half maxima was used for all four peaks (\ref{fig:FitPt100LScans}).

Corresponding interplanar spacings were computed from the position of the peaks, which were found to coincide with a slightly distorted cubic unit cell of in-plane lattice parameter equal to \qty{5.60}{\angstrom} and out-of-plane lattice parameter equal to \qty{5.64}{\angstrom}, used to index each Bragg peak in the figure.

\begin{figure}[!htb]
    \centering
    \includegraphics[width=\textwidth]{/home/david/Documents/PhDScripts/SixS_2022_01_SXRD_Pt100/figures/ctr_reconstructions_fitting_result}
    \caption{
        Fit result for the two $L$-scans performed perpendicular to [H, K] = [0.5, $\bar{1}$] and [H, K] = [1, $\bar{1.5}$].
        The different peaks were indexed using a distorted cubic structure, corresponding to bulk \ce{Pt_3O_4}.
    }
    \label{fig:FitPt100LScans}
\end{figure}

Bulk \ce{Pt_3O_4} was reported to crystallise in a simple cubic structure with a lattice parameter equal to \qty{5.65}{\angstrom} \parencite{Galloni1941, Galloni1952, Muller1968}, more recent studies have proposed a theoretical value of \qty{5.59}{\angstrom} \parencite{Seriani2006}, already presented in tab. \ref{tab:PtOxides}, similarly to the parameters reported in this study.
The 3D structure of bulk \ce{Pt_3O_4} is presented in fig. \ref{fig:Pt3O4}.
The (212) and (232) Bragg peaks are not allowed reflections.

\begin{SCfigure}
    \centering
    \includegraphics[trim=0 2.5cm 0 2.5cm, clip, width=0.35\textwidth]{/home/david/Documents/PhD/Figures/introduction/Pt3O4.pdf}
    \caption{
        \ce{Pt_3O_4} bulk unit cell.
        Platinum atoms are situated on the faces on the cubic unit cell (e.g. $(0, 1/2, 1/4)$, $(0, 1/2, 3/4)$), while the eight oxygen atoms are inside the unit cell at the positions $(1/4, 1/4, z)$, $(1/4, 2/4, z)$, $(2/4, 1/4, z)$, $(2/4, 2/4, z)$ for $z=1/4$ and $z=3/4$.
    }
    \label{fig:Pt3O4}
\end{SCfigure}

The epitaxial relationship between both is \ce{Pt_3O_4}[001]||Pt[001], the in-plane lattice parameter of \ce{Pt_3O_4} being approximately twice that of the Pt (100) surface lattice parameter, which allows the formation of a cubic on cubic coherent interface.
%In the current experiment, the misfit strain computed with eq. \ref{eq:StrainDiffraction} is equal to \qty{1.7}{\percent}, which can be the reason for the distorted structure.

A first guess of the height of the oxide layer can be obtained by measuring the FWHM $\sigma$ of each peak in the $\vec{z}$ direction and thereby computing the layer thickness using the equation $2\pi/\sigma$ \parencite{Warren1990}.
A thickness of \qty{62.65}{\angstrom} and \qty{60.18}{\angstrom} is found for the first and second measurement.
For bulk \ce{Pt_3O_4} this would correspond to approximately 11 unit cells.

\begin{figure}[!htb]
    \centering
    \includegraphics[width=0.495\textwidth]{/home/david/Documents/PhDScripts/SixS_2022_01_SXRD_Pt100/figures/Map_hkl_surf_or_1880-1902_patched.pdf}
    \includegraphics[width=0.495\textwidth]{/home/david/Documents/PhDScripts/SixS_2022_01_SXRD_Pt100/figures/Map_hkl_surf_or_1930-1936_patched.pdf}
    \caption{
        Reciprocal space in-plane maps collected under different atmospheres measured at \qty{450}{\degreeCelsius}, computed using the surface lattice of Pt (100).
    }
    \label{fig:MapsPt100B}
\end{figure}

Introducing ammonia in the reactor has resulted in the loss of contact with the sample heater, from the corrosion of the screws responsible for the contact on the sample holder.
Nevertheless, half of the large reciprocal space in-plane maps could be measured prior to the loss of alignment (fig. \ref{fig:MapsPt100B}), in which the red circled peaks corresponding to \ce{Pt_3O_4} are still visible, the two other peaks that could have been measured at (0.93, -0.5) and (0.5, -0.93) have disappeared.
The sample was removed and cleaned to fix the sample heater, and then introduced in the reactor.
In order to probe for the reproducibility of the \ce{Pt_3O_4} growth on the surface, the partial pressure of oxygen was set to \qty{80}{\milli\bar} again, while measuring a small area of the in-plane reciprocal space to detect the same peaks as in fig. \ref{fig:MapsPt100A}.
The same peaks could indeed be detected as shown in fig. \ref{fig:MapsPt100B} but with a shift between the green and red circled peaks equal to $0.09$, more important than the previously measured value of $0.07$.

Ammonia was again introduced in the cell to be able to probe the relation between surface structure and selectivity during the oxidation of ammonia.
Interestingly, a different behaviour was measured during the following reciprocal space in-plane map (fig. \ref{fig:MapsAndLScansPt100HighOxAmmonia}).

\begin{figure}[!htb]
    \centering
    \includegraphics[width=0.53\textwidth]{/home/david/Documents/PhDScripts/SixS_2022_01_SXRD_Pt100/figures/Map_hkl_surf_or_1953-1981_patched.pdf}
    \includegraphics[width=0.46\textwidth]{/home/david/Documents/PhDScripts/SixS_2022_01_SXRD_Pt100/figures/l_scans_high_oxygen_ammonia_no_map.pdf}
    \caption{
        Reciprocal space in-plane maps collected under different atmospheres measured at \qty{450}{\degreeCelsius}, computed using the surface lattice of Pt (100).
        Out-of plane measurements for four different positions under \qty{80}{\milli\bar} of \ce{O_2}, \qty{10}{\milli\bar} of \ce{NH_3}, and \qty{410}{\milli\bar} of \ce{Ar}.
    }
    \label{fig:MapsAndLScansPt100HighOxAmmonia}
\end{figure}

Peaks separated by $0.1$ in H or in K were observed around the platinum Bragg peaks in a square arrangement, similarly to a (10x10) surface reconstruction, but with some extinctions.
The only row and columns in which the reconstructions are also seen that do not got through a Bragg peak are also shifted by the same amount equal to $0.09$ as the green circled signals observed while only oxygen is present in the cell.
Out-of-plane measurements were performed perpendicular to four peaks, also presented in fig. \ref{fig:MapsAndLScansPt100HighOxAmmonia}.
The peaks measured earlier perpendicular to [H, K] = [0.5, $\bar{1}$] at $L=0.7$, $L=1.4$, and $L=2.1$ corresponding to bulk \ce{Pt_3O_4} are no longer visible.

It is not certain that bulk \ce{Pt_3O_4} was present on the catalyst surface at the beginning of the reacting conditions.
Indeed, the sample was exposed to \qty{80}{\milli\bar} of oxygen for \qty{12}{\hour} when the bulk oxide could be measured.
The second exposition only lasted for \qty{1}{\hour}, during which the same in-plane could effectively be measured, but without out-of-plane information.
Therefore it is not clear whether the lack of bulk oxide is due to its removal from the reacting conditions or from the shorter exposition to a high oxygen atmosphere.
However, it is clear that the duration of the high oxygen condition, probably linked to the thickness of the \ce{Pt_3O_4} layer, has an effect on the catalyst surface during reaction condition since the in-plane signals measured after the introduction of ammonia are very different.
Regarding the $L$-scan at [H, K] = [1.9, 0], the two peaks near $L=0.1$ and $L=2.1$ are coming from the nearby Bragg peaks.
Otherwise, no structures signal in $L$ can be observed.

\ce{Pt_3O_4} has been proven to be a source of oxygen atoms sustaining the catalytic oxygenation of \ce{CO} \textit{via} a Mars Van Krevelen mechanism \parencite{Seriani2006, Seriani2008}.
In the current experiment, the heater problem prevented us from recording the evolution of the reaction products during the first exposition of the catalyst to the reacting conditions.
Additional measurements with first different exposition times to a pure oxygen atmosphere (while monitoring the thickness of the \ce{Pt_3O_4} layer), and secondly introducing ammonia in the reactor while comparing the product partial pressure could bring an answer to the role of \ce{Pt_3O_4} during the oxidation of ammonia.
It is possible that the (10x10)-type reconstructions observed in the current experiment are linked to the adsorption of nitrogen species on the catalyst surface.

\begin{figure}[!htb]
    \centering
    \includegraphics[width=0.53\textwidth]{/home/david/Documents/PhDScripts/SixS_2022_01_SXRD_Pt100/figures/Map_hkl_surf_or_2227-2283_patched.pdf}
    \includegraphics[width=0.46\textwidth]{/home/david/Documents/PhDScripts/SixS_2022_01_SXRD_Pt100/figures/l_scans_low_oxygen_ammonia.pdf}
    \caption{
        Reciprocal space in-plane maps collected under different atmospheres measured at \qty{450}{\degreeCelsius}, computed using the surface lattice of Pt (100).
        Out-of plane measurements for four different positions under \qty{5}{\milli\bar} of \ce{O_2}, \qty{10}{\milli\bar} of \ce{NH_3}, and \qty{485}{\milli\bar} of \ce{Ar}.
    }
    \label{fig:MapsAndLScansPt100LowOxAmmonia}
\end{figure}

Lowering the partial pressure of oxygen from \qty{80}{\milli\bar} to \qty{5}{\milli\bar} in the reactor has completely removed the square (10x10) reconstruction phenomena and revealed the existence of two hexagonal arrangements on top on the Pt (100) surface (fig. \ref{fig:MapsAndLScansPt100LowOxAmmonia}), with an in-plane lattice parameter equal to \qty{2.685}{\angstrom}, \qty{\approx 3.36}{\percent} lower than the distance between neighbouring Pt atoms on the Pt (100) surface.
Second order peaks can also be seen at the edge of the reciprocal space in-plane map, each domain has one axis parallel to either $\vec{a}_{(100)}$ or $\vec{b}_{(100)}$, \textit{i.e.} respectively in the [110] and [1$\bar{1}$0] directions.

Different hexagonal surface reconstructions on the Pt (100) surface have been reported also in the [110] direction at UHV conditions, summarised in \cite{Hammer2016}, based on an important body of work \parencite{Heilmann1979, Vanhove1981, Heinz1982, Mase1992, Kuhnke1992, Borg1994, VanBeurden2004, Havu2010}, evolving to rotated hexagonal reconstructions with angles between \ang{0.77} and \ang{0.94} depending on the sample temperature and on the previous temperature treatment.
The unit cell describing those reconstructions with respect to the Pt (100) surface varies, if first contained to (5XN) where N = 20–30, the latest study reports a commensurate c(26X118) superstructure.
Exposition of the rotated hexagonal structure to oxygen at \qty{450}{\degreeCelsius} studied by low energy electron diffraction (LEED) has been found to remove the hexagonal structure and precipitate the growth of surface oxides \textit{via} different phases \parencite{BradleyShumbera2007, BradleyShumbera2007a}, a similar conclusion was reached by \cite{Deskins2005} by DFT studies.
Exposition to \ce{NO} has been reported to stabilise the clean Pt (100) phase \parencite{Heinz1982}, while exposition to \ce{CO} removes the hexagonal reconstruction, an oscillatory behaviour between a clean (1X1) surface and the rotated hexagonal reconstruction was reported by Cox et al. \parencite*{Cox1983}.

A (2X2) reconstruction of the Pt (100) surface has been reported at an oxygen pressure of \qty{1e-3}{mbar} by the use of environmental TEM \parencite{Li2016}.

Subsurface oxygen has also been predicted to exist on Pt (100) \parencite{Gu2007}, reported at a pressure of \qty{0.133}{\milli\bar} \parencite{McMillan2005}, and participating in the catalytic oxidation of \ce{CO}.
Subsurface oxygen was identified as a precursor to a stable surface oxide during the catalytic oxidation of \ce{CO} by Dicke et al. \parencite*{Dicke2000}, at an oxygen pressure of \qty{0.09}{\milli\bar}, its appearance linked to the lifting of surface reconstructions on the clean Pt (100) surface from the adsorption of \ce{CO}, which then allowed oxygen atoms to penetrate under the topmost layer of platinum \parencite{Rotermund1993, Lauterbach1994}.

Overall, few works at ambient pressure have been found to exist, even less during the catalytic oxidation of ammonia.
Out-of-plane measurements were performed perpendicular to two peaks, also presented in fig. \ref{fig:MapsAndLScansPt100LowOxAmmonia}, no bulk structure could be detected.

\begin{figure}[!htb]
    \centering
    \includegraphics[width=0.495\textwidth]{/home/david/Documents/PhDScripts/SixS_2022_01_SXRD_Pt100/figures/Map_hkl_surf_or_2520-2570_patched.pdf}
    \includegraphics[width=0.495\textwidth]{/home/david/Documents/PhDScripts/SixS_2022_01_SXRD_Pt100/figures/Map_hkl_surf_or_2719-2767.pdf}
    \includegraphics[width=0.495\textwidth]{/home/david/Documents/PhDScripts/SixS_2022_01_SXRD_Pt100/figures/Map_hkl_surf_or_2905-2953_patched.pdf}
    \includegraphics[width=0.495\textwidth]{/home/david/Documents/PhDScripts/SixS_2022_01_SXRD_Pt100/figures/Map_hkl_surf_or_3154-3169.pdf}
    \caption{
        Reciprocal space in-plane maps collected under different atmospheres measured at \qty{450}{\degreeCelsius}, computed using the surface lattice of Pt (100).
    }
    \label{fig:MapsPt100C}
\end{figure}

After removing oxygen from the reactor, only the first order reflections corresponding to the hexagonal structures could be detected during the measurement of the reciprocal space in-plane map (fig. \ref{fig:MapsPt100C}), the following measurement under Argon showed a clean Pt (100) surface.
To make sure that the hexagonal structure was related to the reaction conditions and not only to a lower pressure of oxygen in the reactor, a partial pressure of \qty{5}{\milli\bar} of oxygen was set.
The following measurement showed the presence of many peaks that appeared in the first hour of the measurement.
A second map was measured directly after the end of the first map, in which none of the newly detected peaks could be detected.
These two measurements showed first that the hexagonal structure observed under reacting conditions with \qty{5}{\milli\bar} of oxygen is linked to the simultaneous presence of ammonia and oxygen in the reactor, and that the duration of some of the observed phenomena are too short to be effectively measured with the current time-resolution of in-plane reciprocal space maps.

\subsection{Surface roughness and surface relaxation effects}

\begin{figure}[!htb]
    \centering
    \includegraphics[width=\textwidth]{/home/david/Documents/PhDScripts/SixS_2022_01_SXRD_Pt100/figures/ctr_a.pdf}
    \includegraphics[width=\textwidth]{/home/david/Documents/PhDScripts/SixS_2022_01_SXRD_Pt100/figures/ctr_b.pdf}
    \caption{
        Evolution of [100] crystal truncation rods measured perpendicular to three different Bragg peaks under different atmospheres.
    }
    \label{fig:CTRPt100}
\end{figure}

Crystal truncation rods have been measured perpendicularly to the ($\bar{1}\bar{1}0$), ($\bar{1}00$) and ($\bar{2}10$) positions \qty{6}{\hour} after the start of each condition, each measurement lasted for \qty{2}{\hour}.
The background-subtracted intensity of the CTR was integrated using the \textit{fitaid} module of \textit{binoculars} as a function of $L$ with the same integration range.
The CTR intensity is presented in fig. \ref{fig:CTRPt100}, additional peaks could be detected under exposition to \qty{80}{\milli\bar} of \ce{O_2}, which shows that bulk \ce{Pt_3O_4} is epitaxied on the Pt (100) surface.

The presence of those peaks prevent us from resolving the minimal position of the CTR intensity, the absence of oscillations in the evolution of the CTR intensity as a function of $L$ also shows that there is no homogeneous layer of \ce{Pt_3O_4}, but rather many islands of different height covering the substrate.

The catalyst surface can be divided in three different components as a function of $\vec{z}$.
First, there is bulk Pt (100) in which the structure is homogeneous.
Secondly, there is the interface between bulk platinum and the different islands of \ce{Pt_3O_4}, in which the platinum atoms are expected to be displaced from their equilibrium positions due to the interaction with the oxide layer, and where there could also be the presence of some subsurface oxygen atoms.
Thirdly, there are the \ce{Pt_3O_4} islands, on top of the previous layers, with different heights.
The deconvolution of the signals originating from the different layers is complicated due to the very large amount of parameters needed to properly simulate the CTR intensity, and the impossibility to simulate more than two different surface areas with \textit{ROD}.
A first approach to understanding the catalyst surface is presented below.

\begin{figure}[!htb]
    \centering
    \includegraphics[width=\textwidth]{/home/david/Documents/PhDScripts/SixS_2022_01_SXRD_Pt100/figures/fit_8o2.pdf}
    \caption{
        Fitting results for crystal truncation rods collected under a \qty{80}{\milli\bar} of \ce{O_2}.
    }
    \label{fig:CTRFitHighOxygen}
\end{figure}

The structure factors $F_i$ resulting from the presence of one to nine unit cells on the Pt (100) surface of \ce{Pt_3O_4} was simulated with \textit{ROD}, the same interface with the substrate is used in each simulation, a $\beta$ roughness parameter of 0.6 was used.
The fitting routine consists in minimising the square root difference between the CTR structure factors $F_{obs}$ and the square root of the coherent sum of the squared simulated structure factors by adjusting the weight $W_i$ of each signal in the total signal $F_{calc}$ (eq. \ref{eq:Fcalc}).

\begin{equation}
    F_{calc} = \sqrt{\sum_{i=1}^{9} W_i F_i^2}
    \label{eq:Fcalc}
\end{equation}

The relation between each weight was adjusted by following a Gaussian distribution, expecting the different islands to not differ too much in height.
The result of the fitting routine are presented in fig. \ref{fig:CTRFitHighOxygen}, and show a Gaussian distribution centred around $3.6$ unit cells, with a standard deviation $\sigma$ equal to $1.3$ unit cells.
The position and width of the \ce{Pt_3O_4} peaks are well adjusted, the simulation struggles however to reproduce the intensity when below a certain threshold.
The ratio of intensty between the \ce{Pt_3O_4} peaks is also not perfect and could be further adjusted by changing the type of oxide layer present at the interface as well as the distance between the oxide and surface.
There is a large difference when comparing the results with the average thickness of \qty{\approx 60}{\angstrom} given by fitting the $L$-scans in fig. \ref{fig:LScansHighOxygenPt100}, since each unit cell is expected to be \qty{\approx5.64}{\angstrom} thick.
Overall, the existence of bulk \ce{Pt_3O_4} islands is confirmed to remove the coherent fringes visible when a homogeneous layer is present of the sample surface (fig. \ref{fig:SimROD}).

When observing the CTR intensity under other atmospheres, a clear evolution in the position of the minimum intensity between both reacting conditions (in red and green in fig. \ref{fig:CTRPt100}) is visible.
The CTR recorded after exposition to reacting conditions under ammonia or under argon have similar intensities and are almost indistinguishable.

All three CTR were fitted together using \textit{ROD} at each atmospheres besides the oxygen rich atmosphere, for which a surface model taking into account the existence of \ce{Pt_3O_4} islands as well as the relaxation of the topmosts platinum layers compatible with \textit{ROD} could not be derived.
Different models were tested, adding the possibility of in-plane lattice displacement did not show any improvement of the fit quality, and a simple surface model was kept consisting of two Pt (100) layers, each sharing an out-of-plane lattice displacement parameter, on top of bulk Pt (100).
The fitting results are shown in fig. \ref{fig:CTRFit100}.

\begin{figure}[!htb]
    \centering
    \includegraphics[width=\textwidth]{/home/david/Documents/PhDScripts/SixS_2022_01_SXRD_Pt100/figures/fit_comparison.pdf}
    \caption{
        Fitting results for roughness parameter $\beta$ (a) and out-of-plane strain $\sigma_z$ (b) as a function of the experimental conditions.
    }
    \label{fig:CTRFit100}
\end{figure}

The surface roughness evaluated \textit{via} the $\beta$ parameter in fig. \ref{fig:CTRFit100} (a) is consistent with the evolution of the CTR intensity in fig. \ref{fig:CTRPt100}.
A high roughness is already present after the exposition of the surface to Argon.
The average roughness after introducing a partial pressure of \qty{80}{\milli\bar} of oxygen in the cell could not be retrieved but seems to be at least equal to the roughness under Argon from observing the CTR signals.
The roughness under reacting conditions is higher than under Argon, these crystal truncation rods having been measured after the second introduction of ammonia in the cell, \textit{i.e.} after cleaning of the sample and a short exposition to a high oxygen atmosphere.
Lowering the amount of oxygen in the cell decreases the surface roughness, which stays at high values.
The lowest value is reached when oxygen is not present in the reactor anymore.
The presence of \qty{5}{\milli\bar} of oxygen after the ammonia oxidation cycle increases the surface roughness to higher values, almost equal to the maximum value under reacting conditions after the presence of \qty{80}{\milli\bar} of oxygen.

The strain of the second topmost layer, (B in fig. \ref{fig:CTRFit100} - b) is always very close to \qty{0}{\percent} besides interestingly under \qty{5}{\milli\bar} of oxygen.
The strain of the topmost layer (A in fig. \ref{fig:CTRFit100} - b) is always compressive and seen to increase during reacting conditions in comparison to under argon atmosphere.
A slight increase is reported when lowering the partial pressure of oxygen in the cell, whereas its removal leads to the lowest recorded strain values.

The ammonia oxidation cycle has shown to overall clean the sample surface with a clear difference in the sample roughness and out-of-plane strain before and after the reaction.
The lack of results under \qty{80}{\milli\bar} of oxygen without ammonia and the presence of compressive strain before the oxidation cycle under argon make it difficult to correlate reacting conditions with compressive strain, a second cycle after observing the clean surface state would lift this unknown since the strain and roughness are minimal after the oxidation cycle.

\subsection{Surface species presence}

\begin{table}[!htb]
\centering
\resizebox{\textwidth}{!}{%
	\begin{tabular}{@{}lllllllll@{}}
	\toprule
	\multirow{3}{*}{Partial pressures (mbar)} & Ar & 1 & 0 & 0 & 0 & 0 & 1 & 0 \\
	 & NH3 & 0 & 0 & 1.1 & 1.1 & 1.1 & 0 & 0 \\
	 & O2 & 0 & 8.8 & 8.8 & 0.55 & 0 & 0 & 0.55 \\
	\midrule
	\multicolumn{2}{l}{\begin{tabular}[c]{@{}l@{}}Peak position in N1s spectra (eV)\\ $E_{photon}$ = \qty{700}{\eV}\end{tabular}} & 403.56 & No peak & 404.32 & 404.32, 400.00 & 404.85, 400.60 & No peak & No peak \\
	\multicolumn{2}{l}{Corresponding surface moieties} &  &  & N, NH, NH2, NO ? & N, NH, NH2, NO ? & N, NH, NH2 ? &  &  \\
	\midrule
	\multicolumn{2}{l}{\begin{tabular}[c]{@{}l@{}}Peak position in O1s spectra (eV)\\ $E_{photon}$ = \qty{700}{\eV} \end{tabular}} & No peak & 529.64 & 533.72, 529.64 & 534.28, 532.04 & 532.38 & 531.27, 529.62 &  \\
	\multicolumn{2}{l}{Corresponding surface moieties} &  &  &  &  &  &  &  \\
    \bottomrule
	\end{tabular}%
}
\caption{}
\label{tab:XPSPt100}
\end{table}

\begin{figure}[!htb]
    \centering
    \includegraphics[width=\textwidth]{/home/david/Documents/PhDScripts/B07_2022_04_XPS/Figures/Pt100/O1sN1s_700.pdf}
    \caption{
        Spectra collected around the N1s edge (tabulated binding energy equal to \qty{409.9}{\eV}) under different atmospheres at \qty{450}{\degreeCelsius} with an incoming photon energy of \qty{700}{\eV}.
        The spectra are normalised and shifted in intensity to highlight the presence of different peaks.
    }
    \label{fig:O1sN1sPt100}
\end{figure}

\begin{figure}[!htb]
    \centering
    \includegraphics[width=\textwidth]{/home/david/Documents/PhDScripts/B07_2022_04_XPS/Figures/Pt100/Pt4f_550_no_fit.pdf}
    \caption{
    	Spectra collected around the Pt4f edge doublet (tabulated binding energy equal to \qty{74.5}{\eV} and \qty{71.2}{\eV}) under different atmospheres at \qty{450}{\degreeCelsius} with an incoming photon energy of \qty{550}{\eV}.
    }
    \label{fig:Pt4fPt100}
\end{figure}


    \section{Discussion}

%%%%%%%%%%%%%%%% high oxygen%%%%%%%%%%%%%%%%
% oxides Pt(111)
In this chapter was presented the structural and chemical evolution of Pt(111) and Pt(100) single crystals during ammonia oxidation.
The pre-oxidation of the platinum surfaces under \qty{80}{\milli\bar} of oxygen has allowed the identification of different surface structures and reconstructions.
For Pt(111), a Pt(111)-($8\times8$) superstructure was measured (fig. \ref{fig:MapsPt111A} - d), preceded by two Pt(111)-($6\times6$)-R\ang{\pm 8.8} superstructures (no visible second order peak, fig. \ref{fig:MapsPt111A} - c).
% Some of the peaks belonging to the rotated structures can also be described with the following matrix notation: Pt(111)-p$\begin{pmatrix} 1.08 & -0.21 \\ -0.21 & 1.08 \end{pmatrix}$, effectively describing a unit cell with a second order peak (fig. \ref{fig:MapsPt111A} - c).
Out-of-plane measurements have revealed a multi-layer thick structure (fig. \ref{fig:LScans80}), possibly corresponding to a $\alpha$-\ce{PtO_2} surface oxide.
The Pt(111)-($6\times6$)-R\ang{\pm 8.8} structures have been linked to monolayers (fig. \ref{fig:LScans80} - app. \ref{fig:LScans05}).
The complete understanding of the out-of-plane structures will be the subject of additional work.
A precursor relation was hypothesised between the Pt(111)-($6\times6$)-R\ang{\pm 8.8} and Pt(111)-($8\times8$) structures by time-resolved diffraction studies, under a lowered oxygen atmosphere (fig. \ref{fig:HexBraggPeaks}).
The importance of the partial pressure of oxygen in the growth kinetics has also been highlighted.

% oxides Pt(100)
On Pt(100), a bulk \ce{Pt_3O_4} oxide was identified during exposure to \qty{80}{\milli\bar} of oxygen (fig. \ref{fig:MapsPt100A} - b, fig. \ref{fig:FitPt100LScans}), but not under \qty{5}{\milli\bar} (fig. \ref{fig:MapsPt100D}), which shows the importance of the total pressure on the growth of \ce{Pt_3O_4}.
\ce{Pt_3O_4} follows a Pt(100)-($2\times2$) epitaxial relationship with the Pt(100) surface (fig. \ref{fig:Pt3O4onPt100}).
A second family of peaks was measured under high oxygen pressure (fig. \ref{fig:MapsPt100A} - b), slightly shifted in H or K with respect to the \ce{Pt_3O_4} peaks measured at $L=0$.
The shift was determined to depend on the elapsed time under oxygen atmosphere, two values were measured, $\delta_{H,K}=0.07$ and $\delta_{H,K}=0.09$, respectively after \qty{3}{\hour} (fig. \ref{fig:MapsPt100A} - b) and \qty{1}{\hour} (fig. \ref{fig:MapsPt100B} - b) of measurement.
No clear unit cell could be associated to the shifted peaks, the related out-of-plane signal is compatible with the existence of a monolayer on the Pt(100) surface (fig. \ref{fig:LScansHighOxygenPt100}).
Transient structures were observed at low oxygen pressure, which disappeared in the second measurement after \qty{5}{\hour} of elapsed time (fig. \ref{fig:MapsPt100D}).

% CTR
The measurement of crystal truncation rods revealed more important intensity changes on the Pt(100) surface (fig. \ref{fig:CTRPt100}) in comparison with the Pt(111) surface during the oxidation cycle (fig. \ref{fig:CTRPt111}).
For both surfaces (fig. \ref{fig:CTRFit111} - a, \ref{fig:CTRFit100} - a), the maximal roughness value is reached during exposition to high oxygen atmosphere, consistent with the formation of oxides under oxygen pressure.
Out-of-plane surface strain on Pt(111) was estimated to be contained on the topmost atomic layer, the presence of surface oxide structures was associated with compressive strain with respect to the initial values under argon atmosphere (fig. \ref{fig:CTRFit111} - b).
The influence of bulk \ce{Pt_3O_4} grown on Pt(100) under high oxygen atmosphere on the surface strain was not yet estimated, a lower oxygen atmosphere was related to compressive out-of-plane strain.
\ce{Pt_3O_4} was shown to not grow in a homogeneous layer, but rather in terms of islands with different thickness (fig. \ref{fig:CTRFitHighOxygen}).

% XPS
The XPS experiment was conducted at similar oxygen to ammonia partial pressure ratio, but with a decreased total pressure.
The highest oxygen pressure reached during the XPS experiment is \qty{8.8}{\milli\bar}, in the same order of magnitude as the low oxygen pressure condition for SXRD (\qty{5}{\milli\bar}).
Adsorbed oxygen was identified by XPS on both surfaces during this condition (fig. \ref{fig:O1sN1sPt111}, \ref{fig:O1sN1sPt100}), as well as under reduced oxygen pressure (\qty{0.55}{\milli\bar}).
The normalised intensity of the related peaks is approximately three times higher on the Pt(100) surface compared to the Pt(111) surface, which shows that the Pt(100) surface is more easily oxidised than the Pt(111) surface.
The duration of the high oxygen condition of Pt(111) was not long enough to enable the growth of the Pt(111)-($8\times8$) structure if dynamics at \qty{8.8}{\milli\bar} can be translated to \qty{5}{\milli\bar} (\qty{23}{\hour} vs. \qty{5}{\hour}).
Therefore, only the Pt(111)-($6\times6$)-R\ang{\pm 8.8} structure is expected to yield an additional peak in the O 1s and Pt 4f levels.
A signal was effectively identified in the Pt 4f level at \qty{71.6}{\eV}, but with low intensity (fig. \ref{fig:Pt4fPt111}).
Interestingly, two more peaks are identified in the O 1s level on Pt(100) at low oxygen pressure (\qty{0.55}{\milli\bar}) in addition to the \ce{O_a} signal, which could be linked to the transient structures measured with SXRD under \qty{5}{\milli\bar} of oxygen (fig. \ref{fig:MapsPt100D}).
% Two peaks was also measured in the Pt 4f level at \qty{71.8}{\eV} (fig. \ref{fig:Pt4fPt100}).

%%%%%%%%%%%%%%%%%%%% o2/nh3=8 %%%%%%%%%%%%%%%%%%%%
% Pt(111)
None of the aforementioned structures were measured under reacting conditions on Pt(111), for both \ce{O_2}/\ce{NH_3} ratios (\num{8} and \num{0.5}, fig. \ref{fig:MapsPt111B} - a \& b), which underlines that platinum oxides are not stable during the catalytic oxidation of ammonia on Pt(111) under those conditions.
Adsorbed nitrogen species (\ce{N_a}, \ce{NH_3}) were observed in the Pt4f (fig. \ref{fig:Pt4fPt111}) and N 1s (fig. \ref{fig:O1sN1sPt111}) levels, but without related surface reconstructions observable by SXRD on Pt(111) (fig. \ref{fig:MapsPt111B} - c).
Only adsorbed water was measured during the oxidation of ammonia in the O 1s level.

% Pt(100)
In contrast with Pt(111), reacting conditions were found to be linked with the appearance of surface reconstructions on Pt(100), depending on the oxygen exposure time.
After oxygen exposure for approximately \qty{17}{\hour}, the introduction of ammonia was linked to the removal of the shifted peaks near \ce{Pt_3O_4} signals (fig. \ref{fig:MapsPt100B} - a).
Both \ce{Pt_3O_4} and shifted signals could again be measured during a second exposure of the cleaned sample to oxygen for approximately \qty{1}{\hour} (fig. \ref{fig:MapsPt100B} - b).
The following introduction of ammonia was seen to induce signals compatible with Pt(100)-($10\times10$) surface reconstructions (fig. \ref{fig:MapsAndLScansPt100HighOxAmmonia} - a).
Some periodicity can be observed in the related out-of-plane signals at [H, K] = [0, -1.2] and [H, K] = [0.5, -1], high intensity peaks are observed at $L=1$ and $L=2$ for the measurement at [H, K] = [1.9, 0] (fig. \ref{fig:MapsAndLScansPt100HighOxAmmonia} - b), which can link the Pt(100)-($10\times10$) surface reconstructions with the presence of multilayers.
Additional in-plane signals are observed in fig. \ref{fig:MapsAndLScansPt100HighOxAmmonia} (a), shifted by the same amount $\delta_{H, K}=0.09$ as the signals present under high oxygen atmosphere (fig. \ref{fig:MapsPt100B} - b).
The intensity of this signal as a function of $L$ was measured at [H, K] = [0.5, -0.91] (fig. \ref{fig:MapsAndLScansPt100HighOxAmmonia} - b), decreasing with $L$, and thus showing no out-of-plane periodicity.

%%%%%%%%%%%%%%%%%%%% o2/nh3=0.5 %%%%%%%%%%%%%%%%%%%%

Lowering the \ce{O_2}/\ce{NH_3} ratio to \num{0.5} has removed the Pt(100)-($10\times10$) surface reconstructions, and induced the appearance of a hexagonal Pt(100)-Hex superstructure, with an out-of-plane signal consistent with that of monolayers (fig. \ref{fig:MapsAndLScansPt100LowOxAmmonia}).
Removing oxygen from the reactor resulted in the progressive removal of the hexagonal structures (fig. \ref{fig:MapsPt100C}), clearly linked to the simultaneous presence of both reagents.
Another key difference with Pt(111) is reported in the XPS spectra for Pt(100), with the visible presence of oxygen species in the O 1s level during reacting condition when the \ce{O_2}/\ce{NH_3} ratio is equal to \num{8} (fig. \ref{fig:O1sN1sPt100}), but not when this ratio is reduced to \num{0.5} by lowering the oxygen pressure.
No adsorbed nitrogen specie was identified when the \ce{O_2}/\ce{NH_3} ratio is equal to \num{8} in the N 1s level  (fig. \ref{fig:O1sN1sPt100}).
Therefore, the Pt(100)-($10\times10$) surface reconstruction can be linked to the presence of a surface oxide on the Pt(100) surface, which is only permitted by the simultaneous presence of ammonia, \textit{i.e.} linked to the reaction mechanism.
Additionally, the selectivity towards \ce{NO} in comparison with \ce{N_2} is three times more important than for Pt(111) (fig. \ref{fig:RGA450Pt111AndPt100}), which highlights the importance of oxygen adsorbed on the catalyst surface to facilitate the production of \ce{NO}.

\begin{figure}[!htb]
    \centering
    \includegraphics[width=\textwidth]{/home/david/Documents/PhDScripts/SixS_2023_04_SXRD_Pt111/figures/product_comparison_single_crystals.pdf}
    \includegraphics[width=\textwidth]{/home/david/Documents/PhDScripts/B07_2022_04_XPS/Figures/product_comparison_single_crystals.pdf}
    \caption{
        Evolution of reaction product partial pressures during the SXRD (top) and XPS (bottom) experiment on the Pt(111) and Pt(100) single crystals at \qty{450}{\degreeCelsius}.
        Mean partial pressures during \qty{1}{\minute} at the end of each condition, recorded from a leak in the reactor output by a residual gas analyser (RGA).
        The partial pressures have been normalised by the partial pressure of nitrogen.
    }
    \label{fig:RGA450Pt111AndPt100}
\end{figure}

This difference in selectivity is not observed during the SXRD experiments, this can be related to the position of the gas outlet far away from the sample (fig. \ref{fig:SampleHolder}), in contrast with the XPS experiment where the analyser nose is situated a few millimetres away from the sample surface (fig. \ref{fig:SampleSXRD} - c).
The contribution of the sample edges (fig. \ref{fig:SampleSXRD} - b), as well as the presence of steps on the single crystal surface may lower the specific contribution of the sample facet (\textit{i.e} (111) or (100)).

% roughness
The persisting presence of oxygen on the Pt(100) surface can also be linked to different evolution of the catalysts surface roughness.
The roughness on the Pt(111) surface decreases during ammonia oxidation (fig. \ref{fig:CTRFit111} - a), consistent with the removal of surface oxides following the introduction of ammonia.
Indeed, low roughness is already achieved when the \ce{O_2}/\ce{NH_3} ratio is equal to \num{0.5} (fig. \ref{fig:CTRFit111} - a).
For Pt(100), reacting conditions can also be associated to a decreasing roughness following pure oxygen atmosphere (fig. \ref{fig:CTRFit100} - a).
However, the minimal roughness is only reached once oxygen is completely removed from the reactor, \textit{i.e.} after the oxidation cycle.
The  Pt(100)-($10\times10$) surface reconstructions and Pt(100)-Hex hexagonal structures present on the Pt(100) surface during reacting conditions (fig. \ref{fig:MapsAndLScansPt100HighOxAmmonia} - a, \ref{fig:MapsAndLScansPt100LowOxAmmonia} - a), can be linked to the remaining high roughness, whereas Pt(111) exhibits a bulk terminated surface (fig. \ref{fig:MapsPt111B} - a \& b).

% surface strain
A similar behaviour is observed for the out-of-plane lattice strain.
The introduction of ammonia on Pt(111) was accompanied by a progressive reversal of the strain state, returning to the initial values measured under inert atmosphere when the \ce{O_2}/\ce{NH_3} ratio is equal to \num{0.5} (fig. \ref{fig:CTRFit111} - b).
For Pt(100), both reacting conditions show a strain state similar to the initial state under inert atmosphere (fig. \ref{fig:CTRFit100} - b).
The difference in out-of-plane strain compared to \ce{O2}/\ce{NH3} = 8 is in the same order of magnitude for both surfaces, but of different nature, \qty{\approx 0.06}{\percent} tensile / compressive strain on Pt(100) /  Pt(111).
Overall, the strain measured in this experiment is more important on Pt(100) than Pt(111).

%%%%%%%%%%%%%%%%%%%% Pure nh3 %%%%%%%%%%%%%%%%%%%%

Stopping the oxidation reaction by removing oxygen has finally resulted in the removal of surface structure on the Pt(100) surface (fig. \ref{fig:MapsPt100C} - a).
If the oxygen pressure is seen to decrease directly after the change of condition in the RGA signal (fig. \ref{fig:RGA450Pt100Cycle}), the hexagonal structures are still measured during the first hours of measurement.
It is possible that this progressive removal of the hexagonal surface structure is due to a slow transition between adsorbed nitrogen (\ce{N_a}) measured during the reaction, and adsorbed de-hydrogenated species observed only in the absence of oxygen (fig. \ref{fig:O1sN1sPt100} \& \ref{fig:Pt4fPt100}).
The largest transition in terms of surface strain is also during this change of condition for Pt(100), as the topmost layer goes from important compressive strain to almost bulk atomic positions for the Pt atoms (fig. \ref{fig:CTRFit100} - b, difference of \qty{\approx 0.33}{\percent}).
Atomic nitrogen is also observed in the N 1s level of Pt(111), but together with other nitrogen species (fig. \ref{fig:O1sN1sPt111} - \ref{fig:Pt4fPt111}), possibly impinging on its long-range ordering.
No important change is measured in strain while removing oxygen on Pt(111) (fig. \ref{fig:CTRFit111} - b), supporting the link between the Pt(111)-($8\times8$) surface structure, and changes in the out-of-plane strain.

%%%%%%%%% Argon %%%%%%%

The return to inert atmosphere after the oxidation cycle has shown that the final catalyst surface is more smooth from the roughness evolution (fig. \ref{fig:CTRFit111} - b, \ref{fig:CTRFit100} - b), without any adsorbed nitrogen or oxygen species (fig. \ref{fig:O1sN1sPt111}, \ref{fig:O1sN1sPt100}), and overall weaker strain values than during the reacting conditions.
For both surfaces, the surface roughness and strain are lower than measured at first under argon, before the reaction cycle.

% Discuss role of bulk oxides
\begin{table}[!htb]
\centering
\resizebox{\textwidth}{!}{%
\begin{tabular}{@{}llll@{}}
   \toprule
   \ce{O2} (\unit{\milli\bar}) & \ce{NH3} (\unit{\milli\bar}) & Pt(111) & Pt(100) \\
   \midrule
   0  & 0 & No oxide layer / no reconstruction                            & No oxide layer / no reconstruction\\
   80 & 0 & Two hexagonal rotated Pt(111)-($6\times6$)-R\ang{\pm8.8}            & Epitaxial Pt(100)-($2\times2$) \\
    &    & superstructures (monolayer)                                    & structure (bulk \ce{Pt3O4}),      \\
    &    & After \qty{9}{\hour}\qty{30}{\minute}: Pt(111)-($8\times8$) & and signals shifted in H or K     \\
    &    & superstructure (multilayer)                                    & (monolayer)                       \\
   \midrule
   80 & 10 & No oxide layer / no reconstruction                           & Pt(100)-($10\times10$) reconstruction\\
    &    & Weak signals in O 1s                                           & (multilayer)                      \\
    &    &                                                                & and signals shifted in H or K     \\
    &    &                                                                & (monolayer)                       \\
    &    &                                                                & Higher \ce{NO} selectivity, and   \\
    &    &                                                                & High signal in O 1s in            \\
    &    &                                                                & comparison with Pt(111)           \\
   10 & 10 & No oxide layer / no reconstruction                           & Pt(100)-Hex structure (monolayer) \\
    &    & No signal in O 1s level                                        & No signal in O 1s level           \\
    &    & \ce{N_a} and \ce{NH_{3,a}} in N 1s                             & Only \ce{N_a} in N 1s             \\
   \midrule
   0  & 10 & No oxide layer / no reconstruction                           & Progressive removal of Pt(001)-Hex structure \\
   0  & 0  & No oxide layer / no reconstruction                           & No oxide layer / no reconstruction\\
   \midrule
   5  &  0 & Back to the same structures as at \qty{80}{\milli\bar}       & Transient structures, different from signals \\
      &    & of \ce{O2}, but with decreased kinetics                      & observed at \qty{80}{\milli\bar} of \ce{O2} \\
   \bottomrule
\end{tabular}%
}
\caption{Brief summary of surface structures identified with SXRD, and relevant changes in XPS or RGA signals.}
\label{tab:RecapSXRD}
\end{table}

The different structures observed during this experiment are resumed in tab. \ref{tab:RecapSXRD}.
There is no apparent role of surface and bulk oxides on the Pt(111) surface, as the structures growing under high oxygen pressure have been removed directly during the reaction.
Pt(111)-($8\times8$) and Pt(111)-($6\times6$)-R\ang{\pm 8.8} cannot be linked to an increase or decrease of the catalytic activity, and thus do not seem to take part in the reaction mechanism.

\ce{Pt_3O_4}, observed here in the Pt(100)-($2\times2$) arrangement, has proven to be a source of oxygen atoms sustaining the catalytic oxygenation of \ce{CO} \textit{via} a Mars Van Krevelen mechanism \parencite{Seriani2006, Seriani2008}.
In the current experiment, a second structure possibly related to \ce{Pt_3O_4} was measured.
When under reacting conditions favouring \ce{NO}, which is the desired product for the Ostwald process, the in-plane peaks related to both structures were still measured, but being part of a Pt(100)-($10\times10$) reconstruction.
The presence of bulk \ce{Pt_3O_4} is clearly ruled out.
Nevertheless, it is possible that the reconstruction is due to the interaction of surface \ce{Pt_3O_4} with ammonia, resulting in an increased selectivity towards \ce{NO}.
Reducing the oxygen to ammonia ratio is linked to the reconstruction removal, the stopping of \ce{NO} production, as well as the removal of the adsorbed oxygen peak in the O 1s level.

The formation of \ce{RhO2} was observed on a Pt-Rh(100) single crystal by Resta et al. \parencite*{Resta2020a} at similar reagent partial pressures (total pressure of \qty{300}{\milli\bar}, \qty{3.5}{\milli\bar} of ammonia, and from \qtyrange{0}{20}{\milli\bar} of oxygen).
The rhodium surface oxide signal is only clearly measured when \ce{O2}/\ce{NH3} $>2$, and escalates when the temperature is increased from \qtyrange{175}{375}{\degreeCelsius}.
Both change of conditions are also linked to higher selectivity towards \ce{NO}.
The presence of this rhodium surface oxide is possibly impinging on the formation of \ce{Pt3O4} on Pt(100), observed in the current experiment, but at an oxygen partial pressure \num{4} times higher.
This would explain the role of rhodium in stabilising the catalyst surface \parencite{Fierro1990, Fierro1992, Bergene1996}, since platinum oxides have been reported to be more volatile at higher temperature \parencite{Alcock1960}.
\ce{Rh2O3} has been observed at elevated oxygen pressure on the same Pt-Rh(100) model crystal \parencite{Westerstrom2008}.
It is of key importance to increase the reagent partial pressure in future studies to be able to understand the role of platinum and rhodium oxides in the ammonia oxidation, facilitated now that in-plane signals have been detected.
This can help in the comprehension of the mechanisms driving the selectivity towards \ce{NO}.

% Conclusion
%     \chapter{Conclusion}
%     \section{Research aim}


\section{Perspectives}

Thus, it is not possible to determine the potential influence of the oxide thickness on the catalyst activity.
Nevertheless, since no bulk oxide can be measured under reacting conditions,

Additional measurements with first different exposition times to a pure oxygen atmosphere (while monitoring the thickness of the \ce{Pt_3O_4} layer), and secondly introducing ammonia in the reactor while comparing the product partial pressure could bring an answer to the role of \ce{Pt_3O_4} during ammonia oxidation.

The main drawback of this error-metric is that each voxel is given the same weight, if this assumption is not a problem when working near the Bragg peak, $4^{th}$ generation synchrotron that possess a higher flux make it possible to sample the signal in region further away from the Bragg peak.
A detailed study of the impact of trying to improve the image resolution by increasing the distance from the centre of the Bragg peak by combining simulations and experimental reconstructions would be of great interest to the community to understand the impact of the $(\vec{q}-\vec{G}).\vec{u}_k<<1$ approximation.

Know the oxide at reacting conditions ? metastable, or depending on the path

relaxation sous ammoniac, ok avec meilleure reconstruction de particule D6

also mention the facet evolution probed by sxrd

311 type facets

in situ and operando here, very cool

Mention bragg ptychography to probe several nanoparticles together

    
% % Bibliography
    \phantomsection
    \printbibliography

% Appendices
    \appendix

    \chapter{Diffraction vocabulary}
    \begin{table}[!htb]
\centering
\resizebox{\textwidth}{!}{%
    \begin{tabular}{@{}ll@{}}
    \toprule
    Symbol & Description \\
    \midrule
    a, b, c & Lengths of real space unit cell edges.\\
    $\vec{a},\vec{b},\vec{c}$ & Real space unit cell vectors. \\
    $\alpha, \beta, \gamma$ & Real space unit cell angles, respectively [$\angle (\vec{b}, \vec{c})$],  [$\angle (\vec{c}, \vec{a})$], [$\angle (\vec{a}, \vec{b})$]. \\
    $a*, b*, c*$ & Lengths of reciprocal space unit cell edges. \\
    $\vec{a}^*,\vec{b}^*,\vec{c}^*$ & Reciprocal space unit cell vectors. \\
    $\alpha^*, \beta^*, \gamma^*$ & Reciprocal space unit cell angles, respectively [$\angle (\vec{b}^*, \vec{c}^*)$],  [$\angle (\vec{c}^*, \vec{a}^*)$], [$\angle (\vec{a}^*, \vec{b}^*)$] \\
    (x, y, z) & Coordinates of any point within the unit cell, expressed in terms of a, b and c units.\\
    & z is negative when situated below the surface. \\
    {[}u v w{]} & Specific crystal axis, normal to a crystal plane. \\
    \textless{}u v w\textgreater{} & Sets of equivalent crystal axes, due to lattice symmetry. \\
    (hkl) & Miller indices of a specific set of crystal planes. \\
    \{hkl\} & Equivalent planes. \\
    hkl & Indices of the reflection from a set of parallel interplanar rows.\\
    & Coordinates of a rod in the reciprocal lattice as measured on any plane normal to the rods. \\
    $d_{hkl}$ & Interplanar spacing between (hkl) crystal planes.
    \end{tabular}%
}
\caption{
    Symbols used in crystallography for the description of real and reciprocal space structures.
    Subscript $_s$ used to distinguish surface from underlying structure  \parencite{Wood1964, Willmott}.
    }
\label{tab:Vocab}
\end{table}

    \chapter{Nanoparticles study}
    \section{Temperature ramp}

The particle Amaterasu was studied during a temperature ramp, 3D views of the particle can be found in fig. \ref{fig:Amaterasu} in sec. \ref{sec:TempRampBCDI}.

\begin{table}[!htb]
        \centering
        \resizebox{\textwidth}{!}{%
        \begin{tabular}{lSSSSSSSS}
        \toprule
        {} & {Facet} & {$<\epsilon_{zz}>$} & {$\sigma_{\epsilon_{zz}}$} & {$\vec{u}_{\hat{q_z}}$ (\unit{\angstrom})} & {$\sigma_{\vec{u}_{\hat{q_z}}}$ (\unit{\angstrom})} & {Angle with [111] direction (\unit{\degree})} & {Absolute facet size (in voxels)} & {Relative facet size} \\
        \midrule
        0 & 0 & -0.000126 & 0.000819 & -0.935006 & 0.937775 & 0.000000 & nan & nan \\
        1 & 1 & 0.000148 & 0.000140 & -0.755531 & 0.167763 & 0.000000 & 44397.663467 & 0.044881 \\
        2 & 2 & 0.000026 & 0.000112 & -1.017636 & 0.179658 & 54.735610 & 28554.191678 & 0.028865 \\
        3 & 3 & 0.000147 & 0.000111 & -1.258746 & 0.319739 & 58.372940 & 42252.889700 & 0.042712 \\
        4 & 4 & -0.000067 & 0.000063 & -0.457895 & 0.248747 & 61.113265 & 53275.838905 & 0.053855 \\
        5 & 5 & 0.000092 & 0.000095 & -1.231178 & 0.270344 & 72.272200 & 37115.885233 & 0.037520 \\
        6 & 6 & -0.000143 & 0.001033 & -1.548569 & 0.943487 & 71.266574 & 50106.894497 & 0.050652 \\
        7 & 7 & -0.000042 & 0.000041 & -0.553660 & 0.409718 & 79.225781 & 29784.140334 & 0.030108 \\
        8 & 8 & 0.000124 & 0.000038 & -1.661702 & 0.216682 & 96.642581 & 13676.155558 & 0.013825 \\
        9 & 9 & -0.000056 & 0.000558 & -0.366485 & 0.571493 & 100.218987 & 7184.353885 & 0.007262 \\
        10 & 10 & -0.000478 & 0.001634 & -1.908002 & 0.723495 & 103.737939 & 21640.156224 & 0.021876 \\
        11 & 11 & -0.000356 & 0.000223 & -0.213895 & 0.753181 & 117.452457 & 17727.454403 & 0.017920 \\
        12 & 12 & -0.000212 & 0.000820 & -0.361291 & 1.091395 & 178.494144 & 148119.668614 & 0.149731 \\
        \bottomrule
        \end{tabular}
        }
        \caption{
        Output of the \textit{FacetAnalyzer} plugin used in \textit{Paraview} when extracting the facets on different particle surfaces for the scan 1414 at \qty{25}{\degreeCelsius} under Argon atmosphere.
        }
        \label{tab:FieldData1414}
\end{table}

\begin{figure}[!htb]
    \centering
    \includegraphics[width=\textwidth]{/home/david/Documents/PhDScripts/SixS\_2021\_01/FacetAnalyser/FacetDispEvolution.pdf}
    \caption{
        Mean value and standard deviation of the displacement ($\vec{u}_{\hat{q_z}}$) distribution as a function of the angle between the normal of each facet on the particle surface and the [111] direction.
        Upwards and downwards arrow are represented for respectively positive and negative displacement.
    }
    \label{fig:AmaterasuDisplacement}
\end{figure}

\section{SXRD on Pt nanoparticles}

\begin{figure}[!htb]
    \centering
    \includegraphics[width=0.8\textwidth]{/home/david/Documents/PhDScripts/SixS_2021_03_SXRD_NH3/figures/epitaxy/200.pdf}
    \caption{
        Integrated intensity in a \ang{1} range around the value of the (200) scattering angle, as a function of the in-plane sample angle $\omega$.
    }
    \label{fig:Epitaxy200}
\end{figure}

\begin{figure}[!htb]
    \centering
    \includegraphics[width=0.8\textwidth]{/home/david/Documents/PhDScripts/SixS_2021_03_SXRD_NH3/figures/epitaxy/111.pdf}
    \caption{
        Integrated intensity in a \ang{1} range around the value of the (111) scattering angle, as a function of the in-plane sample angle $\omega$.
    }
    \label{fig:Epitaxy111}
\end{figure}

\begin{figure}[!htb]
    \centering
    \includegraphics[width=\textwidth]{/home/david/Documents/PhDScripts/SixS_2021_03_SXRD_NH3/figures/epitaxy/powder_300.pdf}
    \includegraphics[width=\textwidth]{/home/david/Documents/PhDScripts/SixS_2021_03_SXRD_NH3/figures/epitaxy/powder_500.pdf}
    \includegraphics[width=\textwidth]{/home/david/Documents/PhDScripts/SixS_2021_03_SXRD_NH3/figures/epitaxy/powder_600.pdf}
    \caption{
        Scattered intensity as a function of $\delta$ showing the Bragg peak originating from the substrate near \ang{16}, the
        (220) powder ring near \ang{27.9} and the (220) Bragg peak from the [111] oriented nanoparticles near \ang{28.1}.
    }
    \label{fig:PowderNanoparticles}
\end{figure}

\section{Mass spectrometer data during ammonia oxidation on Pt nanoparticles}

\subsection{Non patterned sample} \label{sec:RGANanoparticlesNonPatterned}

\begin{figure}[!htb]
    \centering
    \includegraphics[width=\textwidth]{/home/david/Documents/PhDScripts/SixS_2021_03_SXRD_NH3/figures/rga/rga_300.png}
    \caption{
        Time dependent partial pressures recorded from a leak in the reactor output as detailed in sec. \ref{sec:XCAT}, at \qty{300}{\degreeCelsius}.
    }
    \label{fig:RGA300SXRDNanoparticles}
\end{figure}

\begin{figure}[!htb]
    \centering
    \includegraphics[width=\textwidth]{/home/david/Documents/PhDScripts/SixS_2021_03_SXRD_NH3/figures/rga/rga_500.png}
    \caption{
        Time dependent partial pressures recorded from a leak in the reactor output as detailed in sec. \ref{sec:XCAT}, at \qty{500}{\degreeCelsius}.
    }
    \label{fig:RGA500SXRDNanoparticles}
\end{figure}

\begin{figure}[!htb]
    \centering
    \includegraphics[width=\textwidth]{/home/david/Documents/PhDScripts/SixS_2021_03_SXRD_NH3/figures/rga/rga_600.png}
    \caption{
        Time dependent partial pressures recorded from a leak in the reactor output as detailed in sec. \ref{sec:XCAT}, at \qty{600}{\degreeCelsius}.
    }
    \label{fig:RGA600SXRDNanoparticles}
\end{figure}

\subsection{Non patterned sample} \label{sec:RGANanoparticlesPatterned}

\section{3D views of reconstructed Pt nanoparticles during the oxidation of ammonia}\label{sec:3DAmmoniaOxidation}

The following images represent the surface of the D-6 and B-7 nanoparticles measured during two ammonia oxidation cycles at 300°C and 400°C.
The surface is created by using the Marching-Cubes algorithm \parencite{Lorensen1987} in \textit{Paraview} \parencite{Ahrens2001}, which works by first assigning a scalar value to each voxel of the data (in our case the amplitude of the retrieved complex Bragg electronic density), and secondly selecting an isovalue which acts as a threshold, separating the regions of the volume that have values above it from those with values below it.
Finally, by "marching" over the volume of the array, the algorithm derives a surface representation by assigning a configuration to each point in the array that depends on whether or not the 8 neighbouring cubes have values above or below the isovalue (fig. \ref{fig:MarchingCubes}).

As mentionned in sec. \ref{sec:BCDI}, the isovalue must be carefully chosen since it depends on the amplitude of the electronic density, \textit{i.e.} theoretically on the structure factor of each voxel in real space.
Therefore, there should be a clean drop of amplitude when a voxel is not in the reconstructed object, in the case of large strain or bad measurements this is not as clear as seen in the surface of particle B-7.

The view perpendicular to the three axes of the laboratory frame is then represented, the particle is tilted due to the incoming angle $\theta$ for the measurement of the [111] Bragg peak.
Both the retrieved displacement and strain fields are represented with the same respective colorbar.
The ammonia to oxygen ration is represented on the right part of the figure.

\begin{figure}[!htb]
    \centering
    \includegraphics[width=0.66\textwidth]{/home/david/Documents/PhD/Presentations/Slides/PhdSlides/Figures/bcdi_data/MarchingCubes.png}
    \caption{
    The Marching-cubes algorithm generates triangles connecting the vertices to form the mesh surface.
    Each selected configuration has a specific arrangement of vertices that dictate how to form triangles to create a smooth surface.
    Image taken from \url{https://fr.wikipedia.org/wiki/Marching_cubes}
    }
    \label{fig:MarchingCubes}
\end{figure}

\includepdf[pages=-]{/home/david/Documents/PhD/Presentations/Slides/PhdSlides/Figures/bcdi_data/3D_B7.pdf}\label{ref:AppendixB7}
\includepdf[pages=-]{/home/david/Documents/PhD/Presentations/Slides/PhdSlides/Figures/bcdi_data/3D_D6.pdf}\label{ref:AppendixD6}

    \chapter{Single crystals study}
    
\begin{figure}[!htb]
    \centering
    \includegraphics[width=0.49\textwidth]{/home/david/Documents/PhDScripts/SixS_2022_01_SXRD_Pt100/figures/hex_reconstruction.pdf}
    \caption{
    Sketch of Pt(100) surface and Pt(100)-p(33x25)-R\ang{0} structure unit cell vectors, in real and reciprocal space (a).
    }
    \label{fig:Pt100UnitCellsReconstruction}
\end{figure}

\begin{figure}[!htb]
    \centering
    \includegraphics[width=\textwidth]{/home/david/Documents/PhDScripts/SixS_2022_01_SXRD_Pt100/gas_analysis/figures/exp_before_heater_change_no_norm.pdf}
    \caption{
        Time dependent partial pressures recorded from a leak in the reactor output by a residual gas analyser (RGA) during the SXRD experiment on the Pt(100) single crystal at \qty{450}{\degreeCelsius}, before the change of heater.
        Vertical dotted lines indicate transitions between two conditions for which the \ce{NH_3} and \ce{O_2} flow is indicated in the legend.
    }
    \label{fig:RGA450Pt100BeforeHeaterChange}
\end{figure}

\begin{figure}[!htb]
    \centering
    \includegraphics[width=\textwidth]{/home/david/Documents/PhDScripts/SixS_2022_01_SXRD_Pt100/gas_analysis/figures/exp_cycle_no_norm.pdf}
    \caption{
        Time dependent partial pressures recorded from a leak in the reactor output by a residual gas analyser (RGA) during the SXRD experiment on the Pt(100) single crystal at \qty{450}{\degreeCelsius}, after the change of heater.
        Vertical dotted lines indicate transitions between two conditions for which the \ce{NH_3} and \ce{O_2} flow is indicated in the legend.
    }
    \label{fig:RGA450Pt100Cycle}
\end{figure}

\begin{figure}[!htb]
    \centering
    \includegraphics[width=\textwidth]{/home/david/Documents/PhDScripts/SixS_2022_01_SXRD_Pt100/gas_analysis/figures/exp_cond_g_no_norm.pdf}
    \caption{
        Time dependent partial pressures recorded from a leak in the reactor output by a residual gas analyser (RGA) during the SXRD experiment on the Pt(100) single crystal at \qty{450}{\degreeCelsius}, after ammonia oxidation cycle.
        Vertical dotted lines indicate transitions between two conditions for which the \ce{NH_3} and \ce{O_2} flow is indicated in the legend.
    }
    \label{fig:RGA450Pt100CondG}
\end{figure}
    % \begin{figure}[!htb]
%     \centering
%     \includegraphics[width=0.8\textwidth]{/home/david/Documents/PhDScripts/SixS_2023_04_SXRD_Pt111/figures/hscan_fit_768.pdf}
%     \caption{
%     }
%     \label{fig:HScan}
% \end{figure}

% \begin{figure}[!htb]
%     \centering
%     \includegraphics[width=\textwidth]{/home/david/Documents/PhDScripts/SixS_2023_04_SXRD_Pt111/figures/map_q_481-516_patched.pdf}
%     \caption{
%         Reciprocal space maps collected under \qty{420}{\milli\bar} of argon and \qty{80}{\milli\bar} of oxygen at \qty{450}{\degreeCelsius}after the introduction of oxygen.
%         The angle between the gray vectors is equal to \ang{42.65}.
%     }
%     \label{fig:481QSpace}
% \end{figure}

\begin{figure}[!htb]
    \centering
    \includegraphics[width=0.49\textwidth]{/home/david/Documents/PhDScripts/SixS_2023_04_SXRD_Pt111/figures/rotated_reconstruction_cells.pdf}
    \includegraphics[width=0.49\textwidth]{/home/david/Documents/PhDScripts/SixS_2023_04_SXRD_Pt111/figures/reconstruction_42punto65.pdf}
    \caption{
    Sketch of Pt(111) surface and Pt(111)-p(6x6)-R\ang{0} structure unit cell vectors, in real and reciprocal space (a).
    Pt(111)-p$\begin{pmatrix} 1.08 & -0.21 \\ -0.21 & 1.08 \end{pmatrix}$-R\ang{0} structure (b).
    }
    \label{fig:Pt11180O2Structures}
\end{figure}

\begin{figure}[!htb]
    \centering
    \includegraphics[width=0.49\textwidth]{/home/david/Documents/PhDScripts/SixS_2023_04_SXRD_Pt111/figures/alphapto2_reconstruction_cell.pdf}
    \caption{
    Sketch of Pt(111) surface and Pt(111)-p(26x26)-R\ang{0} structure unit cell vectors, in real and reciprocal space (a).
    }
    \label{fig:Pt111AlphaPtO2}
\end{figure}

% \begin{figure}[!htb]
%     \centering
%     \includegraphics[width=\textwidth]{/home/david/Documents/PhDScripts/SixS_2023_04_SXRD_Pt111/figures/map_q_2064-2072_patched_hex.pdf}
%     \caption{
%         Reciprocal space maps collected under \qty{495}{\milli\bar} of argon and \qty{5}{\milli\bar} of oxygen at \qty{450}{\degreeCelsius}, from \qty{24}{\hour}\qty{22}{\minute} to \qty{25}{\hour}\qty{43}{\minute} after the introduction of oxygen.
%         The angle between vectors of the same color is \ang{120}.
%         The angle between vectors going from the center to neighbouring peaks at the same magnitude of the scattering vector (e.g. purple and gray peaks) is equal to \ang{12}.
%     }
%     \label{fig:2064QSpace}
% \end{figure}

\begin{figure}[!htb]
    \centering
    \includegraphics[width=\textwidth]{/home/david/Documents/PhDScripts/SixS_2023_04_SXRD_Pt111/figures/l_scans_low_oxygen.pdf}
    \caption{
        Out-of plane measurements perpendicular to peaks observed in in-plane maps under \qty{5}{\milli\bar} of \dioxygen.
    }
    \label{fig:LScans05}
\end{figure}

\begin{table}[!htb]
    \centering
    \resizebox{\textwidth}{!}{%
    \begin{tabular}{@{}|l|l|lllllllllll|@{}}
        \toprule
        Structure & Interplanar & \multicolumn{11}{c|}{Oxygen pressure} \\
        \midrule
          & spacing (\unit{\angstrom}) & \multicolumn{2}{l|}{80 mbar} & \multicolumn{9}{l|}{5 mbar} \\
        \midrule
         & & \multicolumn{11}{c|}{Time since gas introduction (end of measurement)} \\
        \midrule
                                     &                  & 03h23 & 09h53 & \multicolumn{1}{|l}{00h34} & 04h03 & 08h00 & 15h57 & 22h56 & 24h08 & 25h43 & 26h36 & 27h28 \\
        \midrule % first struc split low O2
        Pt(111)-p(6x6)-R\ang{\pm 8.8} & $3.01 \pm 0.01$ & \yes  & \yes  & \multicolumn{1}{|l}{\yes}  & \yes  & \yes  & \yes  & \yes  & \yes  &                              \yes & \yes & \yes \\
        \midrule % first struc not split high O2
        Pt(111)-p(6x6)-R\ang{\pm 8.8} & $2.92 \pm 0.03$ & \yes  & \yes  & \multicolumn{1}{|l}{\no}   & \no   & \no   & \no   & \no   & \no   &                              \yes & \no & \no \\
        \midrule % first struc not split low O2
        Pt(111)-p(6x6)-R\ang{\pm 8.8} & $2.87 \pm 0.02$ & \no   & \no   & \multicolumn{1}{|l}{\yes}  & \yes  & \yes  & \yes  & \yes  & \yes  &                              \yes & \yes & \yes \\
        \midrule % first struc split low O2
        Pt(111)-p(6x6)-R\ang{\pm 8.8} & $2.79 \pm 0.07$ & \no   & \no   & \multicolumn{1}{|l}{\no}   & \no   & \yes  & \yes  & \yes  & \yes  &                              \yes & \yes & \yes \\
        \midrule % hex struct
        Pt(111)-p(8x8)-R\ang{0}       & $2.69 \pm 0.02$ & \no   & \yes  & \multicolumn{1}{|l}{\no}   & \no   & \no   & \no   & \no   & \yes  &                              \yes & \yes & \yes \\
        % \midrule % surf oxide
        % Pt(111)-p(6x6)-R\ang{\pm 8.8} & $2.31 \pm 0.04$ & \no & \yes & \multicolumn{1}{|l}{\no} & \no & \no & \yes & \yes & \yes & \yes & \yes & \yes \\
        \midrule % first struc not split high O2
        % NRHS & $1.550 \pm 0.000$ & \yes & nv & \multicolumn{1}{|l}{nv} & \no & nv & nv & nv & nv & nv & nv & nv \\
        % \midrule % first struc not split low O2
        Pt(111)-p(8x8)-R\ang{0}       & $1.53 \pm 0.00$ & \yes & nv     & \multicolumn{1}{|l}{nv}    & \yes  & nv    & nv    & nv & nv & nv & nv & nv \\
        % NRHS & $1.35 \pm 0.000$ & \no & nv & \multicolumn{1}{|l}{nv} & \yes & nv & nv & nv & nv & nv & nv & nv \\
        \bottomrule
    \end{tabular}
    }
    \caption{
        Interplanar spacings computed from signals observed during in-plane reciprocal maps, for different oxygen pressure and exposition times for Pt(111).
        The three markers (\yes, \no, \text{nv}) correspond to observed, non-observed, and non visible signals (\textit{i.e.} not in the area spanned by the map).
        The errors on the interplanar spacings are computed by considering the positions of similar peaks in q-space.
    }
    \label{tab:InterplanarSpacingsPt111Oxygen}
\end{table}

\begin{figure}[!htb]
    \centering
    \includegraphics[width=\textwidth]{/home/david/Documents/PhDScripts/SixS_2023_04_SXRD_Pt111/figures/T450_1.pdf}
    \caption{
        Time dependent partial pressures recorded from a leak in the reactor output by a residual gas analyser (RGA) during the SXRD experiment on the Pt(111) single crystal at \qty{450}{\degreeCelsius}.
        Vertical dotted lines indicate transitions between two conditions for which the \ce{NH_3} and \ce{O_2} flow is indicated in the legend.
        The RGA electron multiplier is off for all the masses.
    }
    \label{fig:RGA450Pt111_1}
\end{figure}

\begin{figure}[!htb]
    \centering
    \includegraphics[width=\textwidth]{/home/david/Documents/PhDScripts/SixS_2023_04_SXRD_Pt111/figures/T450_2.pdf}
    \caption{
        Time dependent partial pressures recorded from a leak in the reactor output by a residual gas analyser (RGA) during the SXRD experiment on the Pt(111) single crystal at \qty{450}{\degreeCelsius}.
        Vertical dotted lines indicate transitions between two conditions for which the \ce{NH_3} and \ce{O_2} flow is indicated in the legend.
        The RGA electron multiplier is on for all the masses besides oxygen and argon.
    }
    \label{fig:RGA450Pt111_2}
\end{figure}

\begin{figure}[!htb]
    \centering
    \includegraphics[width=\textwidth]{/home/david/Documents/PhDScripts/SixS_2023_04_SXRD_Pt111/figures/T450_3.pdf}
    \caption{
        Time dependent partial pressures recorded from a leak in the reactor output by a residual gas analyser (RGA) during the SXRD experiment on the Pt(111) single crystal at \qty{450}{\degreeCelsius}.
        Vertical dotted lines indicate transitions between two conditions for which the \ce{NH_3} and \ce{O_2} flow is indicated in the legend.
        The RGA electron multiplier is on for all the masses besides argon.
    }
    \label{fig:RGA450Pt111_3}
\end{figure}

\begin{figure}[!htb]
    \centering
    \includegraphics[width=\textwidth]{/home/david/Documents/PhDScripts/SixS_2023_04_SXRD_Pt111/figures/T450_4.pdf}
    \caption{
        Time dependent partial pressures recorded from a leak in the reactor output by a residual gas analyser (RGA) during the SXRD experiment on the Pt(111) single crystal at \qty{450}{\degreeCelsius}.
        Vertical dotted lines indicate transitions between two conditions for which the \ce{NH_3} and \ce{O_2} flow is indicated in the legend.
        The RGA electron multiplier is on for all the masses besides argon.
    }
    \label{fig:RGA450Pt111_4}
\end{figure}

% 4ème de couverture
%%%%%%%%%%%%%%%%%%%%%%%%%%%%%%%%%%%%%%%%%%%%%%%%%%%%%%%%%%%%%%%
\Ifthispageodd{\newpage\thispagestyle{empty}\null\newpage}{}
\thispagestyle{empty}
\newgeometry{top=1.5cm, bottom=1.25cm, left=2cm, right=2cm}
\fontfamily{rm}\selectfont

\lhead{}
\rhead{}
\rfoot{}
\cfoot{}
\lfoot{}

\noindent 
%*****************************************************
%***** LOGO DE L'ED À CHANGER IMPÉRATIVEMENT *********
%*****************************************************
\includegraphics[height=2.4cm]{logo/logo_edpif_4eme.png}
\vspace{0.5cm}
%*****************************************************
\fontfamily{cmss}\fontseries{m}\selectfont

\small

\begin{mdframed}[linecolor=Prune,linewidth=1]

\textbf{Titre:} Propriétés catalytiques à l’échelle nanométrique sondées par diffraction des rayons X de surface et imagerie de diffraction cohérente

\noindent \textbf{Mots clés:} Diffraction, Catalyse, Surface, Structure, Déformation

\vspace{-.5cm}
\begin{multicols}{2}
\noindent \textbf{Résumé:}
Le principal objectif de ce travail est d'étudier des catalyseurs hétérogènes \textit{in situ} et \textit{operando} pendant l'oxydation de l'ammoniac en se rapprochant des valeurs de température et pression industrielles.
Actuellement, ce processus catalytique et les changements structurels associés sont mal compris, et nous proposons d'utiliser différents échantillons en platine, nanoparticules et monocristaux afin de réduire l'écart entre les études scientifiques sur échantillons modèles et les catalyseurs utilisés en agro-industrie.
L'activité catalytique des différents échantillons est mesurée pour lier structure et sélectivité durant la réaction, qui peut être focalisée vers la production d'azote (\ce{N_2}) ou d'oxyde nitrique (\ce{NO}).
La production de protoxyde d'azote (\ce{N_2O}) doit être évitée de part son importante contribution à l'effet de serre.
Le développement d'une catalyse hétérogène avec une sélectivité ciblant les \qty{100}{\percent} est un défi constant, ainsi que la compréhension de la durabilité, du vieillissement et de la désactivation du catalyseur lui-même.
Mesurer la structure de nanoparticules à l'échelle nanométrique permet de révéler les effets de volume, de tension et de compression de surface et d'interface, ainsi que l'existence de différents types de défauts.
En complément des études d'imagerie par diffraction cohérente de rayons X en condition de Bragg sur des nanoparticules individuelles, l'étude d'un ensemble de nanoparticules sera effectuée \textit{via} la diffraction des rayons X à incidence rasante.
La diffraction cohérente de rayons X en condition de Bragg étant une nouvelle technique, une organisation typique de la réduction et analyse des données est proposée.
De plus, chaque type de surface présente sur les nanoparticules (e.g. (111), (100)) est ensuite étudiée à l'aide de monocristaux par diffraction des rayons X en surface et spectroscopie photoélectronique par rayons X.
De ce fait, la structure de surface ainsi que la présence d'espèces adsorbées peut être reliée à l'activité catalytique mesurée, permettant une meilleure compréhension du mécanisme de réaction.
Finalement, l'évolution de la structure d'ensemble et de monocristaux est comparée à celle des nanoparticules uniques pour confirmer/infirmer le lien entre structure moyenne et structure individuelle.

\end{multicols}

\end{mdframed}

\begin{mdframed}[linecolor=Prune,linewidth=1]

\textbf{Title:} Catalytic properties at the nanoscale probed by surface X-ray diffraction and coherent diffraction imaging

\noindent \textbf{Keywords:} Diffraction, Catalysis, Surface, Structure, Strain

\begin{multicols}{2}
\noindent \textbf{Abstract:}
The main objective of this work is to study heterogeneous catalysts \textit{in situ} and \textit{operando} during the oxidation of ammonia by approaching industrial temperature and pressure values.
Currently, this catalytic process and the associated structural changes are poorly understood.
We propose to use different samples in platinum, nanoparticles and single crystals in order to reduce the gap between scientific studies on model samples and catalysts used in the fertiliser industry.
The catalytic activity of the different samples is measured to link structure and selectivity during the reaction, which can be focused towards the production of nitrogen (\ce{N_2}) or nitric oxide (\ce{NO}).
The production of nitrous oxide (\ce{N_2O}) must be avoided due to its significant contribution to the greenhouse effect.
Developing heterogeneous catalysis with selectivity targeting \qty{100}{\percent} is an ongoing challenge, as is understanding the durability, ageing, and deactivation of the catalyst itself.
Measuring the structure of nanoparticles at the nanoscale makes it possible to reveal the effects of volume, surface and interface tension and compression, as well as the existence of different types of defects.
In addition to imaging studies by Bragg coherent X-ray diffraction imaging on individual nanoparticles, the study of a set of nanoparticles will be carried out \textit{via} grazing incidence X-ray diffraction.
Bragg coherent X-ray diffraction imaging being a new technique, a typical workflow for data reduction and analysis is proposed.
In addition, each type of surface present on the nanoparticles (e.g. (111), (100)) is then studied using single crystals by surface X-ray diffraction and X-ray photoelectron spectroscopy.
Therefore, the surface structure as well as the presence of adsorbed species can be linked to the measured catalytic activity, allowing a better understanding of the reaction mechanism.
Finally, the evolution of the nanoparticle ensemble structure and of single crystals is compared to that of single nanoparticles to confirm/refute the link between average structure and individual structure.
\end{multicols}
\end{mdframed}

%************************************
\vspace{\fill} % ALIGNER EN BAS DE PAGE
%************************************

\noindent
\color{Prune} \footnotesize Maison du doctorat, Université Paris-Saclay\\
2$^{\mathrm{e}}$ étage, aile ouest, École normale supérieure Paris-Saclay\\
4 avenue des Sciencs\\
91190 Gif-sur-Yvette, France
\end{document}

%%%%%%%%%%%%%%%%%%%%%%%%%%%%%%%%%%%%%%%%%%%%%%%%%%%%%%%%%%%%%%

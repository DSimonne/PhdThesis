\section{Conclusion and perspectives}

\textcolor{red}{This chapter covers three areas: analysis of the data; discussion of the results of the analysis; and how your findings relate to the literature. The analysis of the data can be discussed here but the details of any analysis, such as statistical calculations, should be shown in the appendices. You should present any discussion clearly and logically and it should be relevant to your research questions/hypotheses or aims and objectives. Insert any tables or figures that you decide are important in a relevant part of the text not in the appendices, and discuss them fully. Make sure that you relate the findings of your primary research to your literature review. You can do this by comparison: discussing similarities and particularly differences. If you think your findings have confirmed some literature findings, say so and say why. If you think your findings are at variance with the literature, say so and say why.}

This experiment has both demonstrated the value and limits of Bragg coherent diffraction imaging when studying heterogeneous catalytic reactions.
The instrumental procedure is very time-consuming, especially at the SixS beamline for which two weeks of beamtime are in general needed to perform an experiment, the first week being used to install the BCDI setup and to align the sample in the focal plane of the beam.
The recent improvement of the instrumental setup (rocking curves performed in Fly mode instead of step-by-step mode, sample scanning in Bragg conditions to find the nanoparticles, sapphire window in the dome) tend in the right direction since, as was demonstrated in this study, BCDI is of little use if one does not have the time to measure multiple reflections during an experiment.
The planned upgrade of SOLEIL to a $4^{th}$ generation synchrotron will also play a key role in the hierarchy of SixS in the very competitive list of coherent imaging beamlines, the current resolution of the experiment being too low to properly resolve the smallest facets present of the particles, to distinguish between facets with similar orientations,to observe the growth of surface oxides or the strain in the topmost atomic layers.

Nevertheless, we have successfully measured the evolution of Pt nanoparticles during a temperature ramp from \qty{25}{\degreeCelsius} to \qty{600}{\degreeCelsius} in which the reshaping of the particles as a function of temperature was put into evidence, with the appearance of dislocation loops and facets, together with the importance of the presence of a substrate on the displacement field.
The importance of probing different nanoparticles before drawing a conclusion to their global behaviour was also put into perspective by the study of the oxidation of ammonia at \qty{300}{\degreeCelsius} and \qty{400}{\degreeCelsius}, as a function of the ratio between \ce{Ar} and \ammonia.
The role of different facets in the stability of the particle was shown.

An upgraded beamline could allow future BCDI studies of the oxidation of ammonia, revealing the 3D structural dynamics of the hysteresis cycle identified on single crystal by \cite{Resta2020a}, or give additional detail regarding the presence of surface oxides on specific facets, as seen in the next chapter.

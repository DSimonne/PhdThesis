\section{Measuring the average and ensemble behaviour of Pt nanoparticles}

% Compare shape with wulff construction and other measurements
% Get particle average size from epitaxy
% Check if TEM in literature for reshaping
% Quantify amount of powder -> well epitaxied particles
% In plane strain from substrate peak and in-plane strain evolution from powder signal
% double -110 facet, twin boundaries maybe ?

We intend to measure the total signal scattered from \ce{Al_2O_3}-supported platinum particles by taking advantage of the possibility to carry out grazing-incidence diffraction measurements at the SixS beamline of synchrotron SOLEIL.
By having a low incidence angle, the incident beam recovers the whole sample surface, the scattered beam being then proportionnal to the ensemble behaviour of the nanoparticles \parencite{Nolte2008, Hejral2013}.

The goal of this experiment is two-fold.
First, the stability of the nanoparticles epitaxy must be ensured to be stable at different temperatures and atmospheres, before measuring a single nanoparticle with Bragg coherent diffraction imaging, where the beam is reduced to micrometric size.
Secondly, the average nanoparticle shape and structure will be probed by studying the intensity of crystal truncation rods (CTR) in directions perpendicular to the expected facets on the nanoparticle surface.
Prior experiment with similar samples \parencite{Dupraz2017, Li2020, Lim2021, Dupraz2022} have shown that the particles exhibit a Winterbottom shape \parencite{WINTERBOTTOM1967, Boukouvala2021}, typical of nanoparticles epitaxied on a substrate, with mainly (111), (110), and (100)-type facets, and a [111] orientation perpendicular to the substrate.

We have seen in sec. \ref{sec:SXRD} that truncated surfaces such as facets give rise to crystal truncation rods in the reciprocal space, the intensity of which is proportionnal to the size, roughness and strain of the related surface, as a function of the scattering vector.
By having the incident beam covering all of the particles, the CTR signal in e.g. the [111] direction will be the sum of the contribution from the [111] facet of every nanoparticle on the sample (as well as their [$\bar{1}\bar{1}\bar{1}$] facets).
Therefore, phenomena inducing structural change such as particle refaceting/reshaping at a given condition are expected to be visible by an evolution of the different CTR.

To have a more important scattered intensity, a slightly different sample was used from the sample used in BCDI, with the only difference being a surface homogenously covered with Pt particles.

\subsection{Experimental setup for SXRD experiments in the vertical geometry}\label{sec:SXRDSetupV}

The diffractometer was used in a vertical geometry (fig. \ref{fig:Diffractometer}), similarly to BCDI experiments but without the coherence focusing optics.
The horizontal geometry setup at SixS is used when the sample surface must be cleaned by sputtering and annealing, and when there is no need to have high incident angles.
No cleaning process was applied to this sample that has the same history as the sample with isolated nanoparticles used for BCDI experiments, \textit{i.e.} annealing at \qty{1100}{\degreeCelsius} for \qty{30}{\minute} before cooling to room temperature.
The temperature in the MED end-station is of about \qty{25}{\degreeCelsius}.

The incident beam was fixed to \ang{0.5} with the angle $\mu$, the footprint of the beam on the sample is of about \qty{1}{\mm} horizontally and \qty{1}{\mm} vertically covering approximately \qty{60}{\percent} of the sample surface.
\textcolor{Important}{these are random numbers, calculate good ones}
In-plane measurements were perfomed by rotating the in-plane sample angle $\omega$ together with the in-plane detector angle $\delta$.
Out-of-plane measurements were performed by rotating the in-plane sample angle $\omega$ together with the in-plane and out-of-plane detector angles $\delta$ and $\gamma$, the incoming angle $\mu$ must stay at a low value to keep the grazing incidence of the beam as detailed in sec. \ref{sec:DataCollectionSXRD}.
The alignement of the sample was performed with the same procedure as detailed in sec. \ref{sec:BCDISetup}.

This experiment proved to be difficult to realise experimentally.
The graphite layer used to heat the sample is covered by a Boron Nitride solid surface with 4 holes (fig. \ref{fig:SampleHolder}), two holes are used with small screws to fix the sample on top of the heater, while the two others are used to fix the heater to the sample holder.
Despite an extra layer of Boron Nitride that was applied around these screws and holes, the high temperature and highly oxidating atmosphere managed twice to corrode the conducting screws which resulted in a contact loss with the heater, thus a change of heater and realignment of the sample surface.
However, most of the experimental plan was still carried out, lacking only in-plane measurements at \qty{600}{\degreeCelsius} under \qty{8}{\ml\per\min} of oxygen and \qty{1}{\ml\per\min} of ammonia (tab. \ref{tab:Conditions}), due to a lack of experimental time.

\subsection{Crystal truncation rods}

The intensity of the crystal truncation rod perpendicular to the [111] direction has been recorded as a function of $l$ ($\vec{c}$ being in the [111] direction) for different atmospheres and conditions (fig. \ref{fig:2DCTR111Particles}).
Very weak CTR in other directions due to other facets present on the nanoparticles can be seen around the ($\bar{1}$11) and ($\bar{1}$12) Bragg peaks.
Large powder rings are visible through the Bragg peaks, that reduce the contrast with these signals.
The peak at $l\approx1.55$ corresponds to the (2$\bar{1}$3) Bragg peak from the \ce{Al_2O_3} substrate ($q = \qty{3.014}{\angstrom}$, tab. \ref{tab:Reflections}), which is also at the origin of its own CTR in the [111] direction since its surface in contact with the nanoparticles is [001] oriented, as observed in fig. \ref{fig:2DCTR111Particles} parallel to the more intense Pt [111] CTR.

\begin{figure}[!htb]
    \centering
    \includegraphics[width=\textwidth]{/home/david/Documents/PhDScripts/SixS_2021_03_SXRD_NH3/figures/ctr/CTR111_together.pdf}
    \caption{
        Crystal truncation rod signal in the $[111]$ direction, collected perpendicular to the $(\bar{1}10)$ Bragg peak and integrated in a thin slice around $k=1$, represented with different intensity scales.
    }
    \label{fig:2DCTR111Particles}
\end{figure}

The CTR signal in the [111] direction was integrated in a square area ([$\Delta H$, $\Delta K$] = [0.03, 0.03]) around the ($\bar{1}$10) Bragg peak as a function of $l$ (fig. \ref{fig:CTR111Particles}).
The background, computed as the average scattered signal in four neighbouring regions, was subtracted to the CTR as detailed in sec. \ref{sec:BINoculars}.
The CTR measured at \qty{600}{\degreeCelsius} under \qty{8}{\ml\per\min} of oxygen and \qty{1}{\ml\per\min} of ammonia stops at $l=1.3$ due to a lack of experimental time to complete the measurements.

\begin{figure}[!htb]
    \centering
    \includegraphics[width=\textwidth]{/home/david/Documents/PhDScripts/SixS_2021_03_SXRD_NH3/figures/ctr/CTR_together.pdf}
    \caption{
        Crystal truncation rod in the $[111]$ direction, collected perpendicular to the $(\bar{1}10)$ Bragg peak, for different atmospheres, at \qty{300}{\degreeCelsius} (a), \qty{500}{\degreeCelsius} (b) and \qty{600}{\degreeCelsius} (c).
        The peak at $l\approx1.55$ is from the \ce{Al_2O_3} substrate.
    }
    \label{fig:CTR111Particles}
\end{figure}

The drop of intensity near $l=0.75$ originates from an problem with the attenuators during the experiment.
\textcolor{Important}{Sure?}
There are no visible evolution of the CTR shape, the anti-Bragg region where the relaxations effect are the most visible (fig. \ref{fig:CTRSimulation}) near $l=1.5$ being masked by the Bragg peak of the substrate.
%However, it is difficult to de-convolute the signal coming from the [111] Pt surface from the signal coming from the [001] \ce{Al_2O_3} surface.
%The shape of the CTR seems to be sligtly skewed with a minimum situated above $l=1.5$ which could indicate a compressive strain on the Pt [111] surface.
Moreover, evolution of the Bragg peak coming from the substrate can be seen at \qty{500}{\degreeCelsius} and \qty{600}{\degreeCelsius} which probably indicate a loss of alignment during the experiment.

\subsection{Facet signal}

The same kind of measurements could not be reproduced for CTR in other directions due to a very low signal to noise ratio, and to the presence of powder rings in multiple directions that impinge on the rods signals as seen in fig. \ref{fig:2DCTR111Particles}.
Nevertheless, a volume of the reciprocal space was collected around the ($\bar{1}11$) Bragg peak to try to find the direction of these signals, and to quantify their evolution as a function of the different conditions.
The data was computed in the q-space with different in-plane offsets, so that the rods observed in fig. \ref{fig:FacetMaps} (a) become parallel to the $\vec{q}_x$ direction, thereby facilitating the data analysis.

Besides the crystal truncation rods, there are two different parasitic signals going through the ($\bar{1}11$) Bragg peak.
First, a curved signal almost parallel to $\vec{q}_y$ extends from $q_y = \qtyrange{-2.80}{-2.40}{\angstrom^{-1}}$ in the ($\vec{q}_x, \vec{q}_y$) plane.
This signal is due to the use of attenuators when the detector records a very high intensity of the Bragg peak, the correction factor is not enough to correct the signal far away from the Bragg peak where the noise is more important.
Secondly, a very intense powder ring ($\approx \num{1e2}$) can be seen around the crystal truncation rod perpendicular to $\vec{q}_z$ (area with high intensity, $> \num{1e4}$).
This powder signal occupies a large area in the $\vec{q}_x$ direction, is almost parallel to $\vec{q}_z$ as seen in fig. \ref{fig:FacetMaps} - (b-c-d), and prevents us from resolving the signal of possible rods in the $(\vec{q}_x, \vec{q}_z)$ plane of fig. \ref{fig:FacetMaps} - (a).

\begin{figure}[!htb]
    \centering
    \includegraphics[width=\textwidth]{/home/david/Documents/PhDScripts/SixS_2021_03_SXRD_NH3/figures/facets/facets_together.pdf}
    \caption{
        a) Projection perpendicular to $\vec{q}_z$ of a reciprocal space volume measured around the $(\bar{1}11)$ Bragg peak.
        The three black dotted line are respectively a projection in the $(\vec{q}_x, \vec{q}_y)$ plane of the $[\bar{1}11]$, $[\bar{1}10]$ and $[010]$ directions.
        The red dotted line correspond to a parasite signal linked to the use of attenuators when the detector records the intensity of the Bragg peak.
        b-c-d) Slices in the $(\vec{q}_y, \vec{q}_z)$ plane after rotation in the $(\vec{q}_x, \vec{q}_y)$ plane to highlight the presence of crystal truncation rods in different directions.
        The interval delimited by the vertical white dottel lines is then used for the qualitative analysis of the intensity of those rods.
    }
    \label{fig:FacetMaps}
\end{figure}

The identification of the crystallographic direction corresponding to the different rods present around the $(\bar{1}11)$ Bragg peak was performed thanks to the stereographic projection perpendicular to the $[111]$ crystallographic orientation presented in fig. \ref{fig:StereoTop}.
Each dot represents a crystallographic direction, \textit{e.g.} [100], [110], etc.
The distance between the dots and the center is a function of the angle with the [111] direction, whereas their angular distribution corresponds to the planar angle of the component perpendicular to the [111] direction.

\begin{figure}[!htb]
    \centering
    \includegraphics[width=\textwidth]{/home/david/Documents/PhD/Figures/introduction/stereographic_projection_top.pdf}
    \caption{
        Stereographic projection perpendicular to $[111]$ crystallographic orientation.
    }
    \label{fig:StereoTop}
\end{figure}

Despite the high intensity parasitic signal, it is possible to resolve the signal of 5 types of crystal truncation rods.
The [$1\bar{1}1$] rod parallel to $\vec{q}_y$ is visible in fig. \ref{fig:FacetMaps} - (a), with an angle of \ang{72} with the [111] direction (fig. \ref{fig:FacetMaps} - (d)).
The two other [$1\bar{1}1$]-type rods are at \ang{120} in the ($\vec{q}_x, \vec{q}_y$) plane, each [$1\bar{1}1$]-type rod in also linked to a [$\bar{1}1\bar{1}$]-type rod in the opposite direction, which is why there is a [$\bar{1}\bar{1}1$] rod at \ang{60} from the [$1\bar{1}1$] rod in the ($\vec{q}_x, \vec{q}_y$) plane visible in fig. \ref{fig:FacetMaps} - (c).

After having first identifies the [111] and [$1\bar{1}1$] directions, the identification of the other rods becomes easier when using the stereographic projection presented in fig. \ref{fig:StereoTop}.
We can identify rods in the [$1\bar{1}3$] and [$0\bar{1}1$] directions at \ang{30} from the [$1\bar{1}1$] rod in the ($\vec{q}_x, \vec{q}_y$) plane visible in fig. \ref{fig:FacetMaps} - (b).
There is also a [001] rod visible at \ang{60} from the [$1\bar{1}1$] rod in the ($\vec{q}_x, \vec{q}_y$) plane visible in fig. \ref{fig:FacetMaps} - (c).
From these results, the average shape of the particles on the sample is expected to be round with not only small indices facets but also high indices facets such as [113] present on their surface.

The intensity of the crystal truncation rods listed above was studied by measuring the scattered intensity as a function of $q_z$ in a square area in the ($\vec{q}_x, \vec{q}_y$) plane.
The same width ($\Delta q_x$, $\Delta q_y$] = [0.006 , 0.004]) was used for each study, the region of integration is detailed in fig. \ref{fig:FacetMaps} - (b-c-d).
The evolution of the integrated scattered intensity as a function of $q_z$ is presented if fig. \ref{fig:FacetSignal}.

\begin{figure}[!htb]
    \centering
    \includegraphics[width=\textwidth]{/home/david/Documents/PhDScripts/SixS_2021_03_SXRD_NH3/figures/facets/facet_signal_evolution_together.pdf}
    \caption{
    Evolution of the scattered intensity taken along a square area perpendicular to the [111] direction at three different positions in the ($\vec{q}_x, \vec{q}_z$) plane to probe the evolution of crystal truncation rods.
    }
    \label{fig:FacetSignal}
\end{figure}

The alignment of the sample was progressively lost at \qty{300}{\degreeCelsius} under \qty{8}{\ml\per\min} of oxygen and \qty{1}{\ml\per\min} of ammonia from a contact loss with the heater, the intensity around the ($1\bar{1}1$) Bragg peak could not be recorded properly at that condition.
The scattered intensity does not evolve at \qty{300}{\degreeCelsius} and \qty{500}{\degreeCelsius} as a function of the atmosphere in the reactor cell.
However, there is a transition between \qty{300}{\degreeCelsius} and \qty{500}{\degreeCelsius} when looking at the signal from the [$0\bar{1}1$] facets, that evolves from a double to a single peak when increasing the reactor temperature.
This can be due

At \qty{600}{\degreeCelsius}, a progressive increase of the intensity is observed for the [$1\bar{1}1$], [$\bar{1}1\bar{1}$] and [$001$]-type rods, whereas a progressive decrease of the intensity is observed for the [$0\bar{1}1$], [$1\bar{1}3$]-type rods, starting after the introduction of oxygen in the reactor.
There is an increase of a signal in fig. \ref{fig:FacetSignal} (i) at $q_z = \qty{\approx 0.95}{\angstrom}$ which could be from [110] facets at \ang{35} with the [111] direction (fig. \ref{fig:StereoTop}).
The low signal-to-background ratio in that region makes it challenging to be certain of the existence of this CTR.

Moreover, the introduction of oxygen induces a global shift of the facet signals towards higher $q_z$ values which can be linked to a homogeneous compressive lattice strain in the [111] direction, shifting the position of the Bragg peak.
The [$0\bar{1}1$] CTR is in the ($\vec{q}_x, \vec{q}_y$) plane and its position follows the position of the Bragg peak in $q_z$, but its intensity vanishes after having more than \qty{1}{\ml\per\min} in the reactor cell.
A qualitative evolution of the strain in the [111] direction can be obtained by looking at the other CTR position in fig. \ref{fig:FacetSignal} (g-i).
The gradual increase of the oxygen partial pressure in the cell is accompanied by two transitions from compressive to tensile strain and from tensile to compressive strain relative to the positions under argon and ammonia atmosphere.

This result could be explained by a surface catalytic process route involving the reshaping of the particles with a transition towards particles exhibiting mostly (111)-type and (100)-type facets.
A progressive roughening of the particles could explain the loss of intensity from the \{110\} rods but not the increase of intensity of the [$1\bar{1}1$] rod.
The catalytic activity of the particles was recorded \textit{via} the use of a mass spectrometer and shows the production of nitrogen, nitrous oxide and nitrogen oxide (fig. \ref{fig:RGASXRDNanoparticlesComparison}), proving the activity of the catalyst.
At \qty{600}{\degreeCelsius}, the production of \nitrogen is favoured at lower oxygen partial pressures in comparison with \nitrogendioxide and \nitricoxide.
The original mass spectrometer signal as a function of time can be seen in the appendix \ref{sec:RGANanoparticlesNonPatterned}.

\begin{figure}[!htb]
    \centering
    \includegraphics[width=\textwidth]{/home/david/Documents/PhDScripts/SixS_2021_03_SXRD_NH3/figures/rga/product_comparison_carrier_pressure.pdf}
    \caption{
        Evolution of reaction product partial pressures  recorded from a leak in the reactor output as detailed in sec. \ref{sec:XCAT}.
        The product pressure shown for each condition were computed as the mean product pressure during \qty{1}{\minute} at the end of each condition.
        The product pressure under \qty{49}{\ml\per\min} of argon and \qty{1}{\ml\per\min} of ammonia has been subtracted.
        Higher oxygen pressure and temperature favour the production of \nitrousoxide and \nitricoxide over \nitrogen.
    }
    \label{fig:RGASXRDNanoparticlesComparison}
\end{figure}

The same measurements were not repeated at a fixed condition which prevents us from knowing if the observed phenomena would have occured as a function of time under a fixed atmosphere.
Moreover, the reversibility of this effect and the formation of possible platinum oxides (\textit{via} large in-plane maps) could not be investigated from the lack of available experimental time.

To be certain of the combined effect of ammonia and oxygen, the same experiment could be carried without ammonia at the same temperature and oxygen pressure.
Similar experiments on Pt nanoparticles (average size about \qty{50}{\nm}) have been carried at \qty{6.5e-6}{\bar} and \qty{0.5}{\bar} of oxygen by \cite{Hejral2013} which show the formation of \ce{Pt_3O_4} and $\alpha-$\ce{PtO_2} bulk oxides, the formation of high indices facets, a decrease of (111)-type facet signals and an increase of (100)-type facet signals, neither contrary not similarly to what has been observed during this study.

\subsection{Particle stability}

To ensure that the \ce{Al_2O_3} supported particles have a stable epitaxy relation with the substrate as a function of the different atmospheres, the scattering intensity in the sample plane, perpendicular to the [111] direction, was measured.
The signal of six (220)-type Bragg peaks is expected to be measured, corresponding to the six [$\bar{1}10$]-type planes perpendicular to the [111] direction (fig. \ref{fig:StereoTop}).
The h, k, and l Miller indices must have the same parity for the Bragg reflection to be allowed for face-centered cubic crystals.

The scattered intensity was measured by rotating the sample in-plane angle ($\omega$ - fig. \ref{fig:Diffractometer}) from \ang{0} to \ang{360}, while the in-plane detector angle ($\delta$) was kept to a fixed angular value.
Multiple $\omega$ scans were performed while changing the value of $\delta$ from \ang{15} to \ang{30} so as to map an area of the reciprocal space.

This first map of the reciprocal space was recorded under inert argon atmosphere (fig. \ref{fig:QxQyMap}) at room temperature, and shows the six expected (220)-type in-plane peaks for [111] oriented nanoparticles.
From the position of these peaks is computed the in-plane lattice parameter for platinum $a_{Pt}=\qty{3.94}{\angstrom}$ (\qty{0.440}{\percent} in-plane strain from literature value of \qty{3.9242}{\angstrom}), from which the scattering angle of the (200)-type and (111)-type Bragg peaks is computed (tab. \ref{tab:Reflections}).

\begin{figure}[!htb]
    \centering
    \includegraphics[width=\textwidth]{/home/david/Documents/PhDScripts/SixS_2021_03_SXRD_NH3/figures/map/qxqyqz_38_40.pdf}
    \caption{
        In-plane reciprocal space map at room temperature.
        Starting from the center of the map are 6 peaks corresponding the bottom of crystal truncation rods going through $\{111\}$ peaks, a thin powder signal for the $[111]$ reflection.
        The three lines correspond the the scattering vector
    }
    \label{fig:QxQyMap}
\end{figure}

Three arcs are drawn in fig. \ref{fig:QxQyMap} to underline the (111), (200) and (220) scattering angles.
If no (200) and (111) Bragg peaks are observed, powder signals are visible for each reflection, which shows that some of the nanoparticles have a random orientation.
Six (1$\bar{2}$0)-type Bragg peaks coming from the \ce{Al_2O_3} substrate are also observed at the lowest magnitude of the scattering vector ($q = \qty{2.637}{\angstrom}$, tab. \ref{tab:Reflections}).
From these Bragg peaks is computed the in-plane lattice parameter of the substrate, $a_{\ce{Al_2O_3}}=\qty{3.94}{\angstrom}$ and the average in-plane strain at the interface of the nanoparticles ([$\bar{1}\bar{1}\bar{1}$] facet) with the substrate.
The nearest distance between the Pt atoms in the [111] plane with a lattice parameter $a_{Pt}=\qty{3.94}{\angstrom}$ is

\begin{table}[htb!]
    \begin{minipage}{.45\linewidth}
        \centering
        \begin{tabular}{@{}lllll@{}}
        \toprule
        (h k l) & $2\theta$ & q & Int & Int (\%) \\
        \midrule
        (1, 1, 1) & 17.0122 & 2.773 & 6275.71 & 100.00 \\
        (2, 0, 0) & 19.6417 & 3.202 & 3208.26 & 51.12 \\
        (2, 2, 0) & 27.9170 & 4.529 & 2371.06 & 37.78 \\
        (3, 1, 1) & 33.0049 & 5.310 & 2880.36 & 45.90 \\
        (2, 2, 2) & 34.5175 & 5.547 & 833.68 & 13.28 \\
        \bottomrule
        \end{tabular}%
    \end{minipage}%
    \hfill
    \begin{minipage}{.45\linewidth}
        \centering
        \begin{tabular}{@{}lllll@{}}
        \toprule
        (h k l) & $2\theta$ & q & Int & Int (\%) \\
        \midrule
        (1, -1, 2) & 11.0793 & 1.805 & 21.55 & 49.50 \\
        (0, 1, -4) & 15.1355 & 2.463 & 35.73 & 82.06 \\
        (1, -2, 0) & 16.2321 & 2.640 & 16.85 & 38.70 \\
        (0, 0, 6)  & 17.8528 & 2.901 & 0.19  & 0.43  \\
        (2, -1, 3) & 18.5405 & 3.012 & 38.96 & 89.47 \\
        \bottomrule
        \end{tabular}%
    \end{minipage}%
    \label{tab:Reflections}
    \caption{
        Scattering angle $\theta$, scattering vector magnitude $q$ and intensity of the scattered waves for different Bragg peaks as a function of the increasing scattering angle (up to $2\theta = \ang{30}$), computed for an energy of \qty{18.45}{\keV} using eq. \ref{eq:Bragglaw} and eq. \ref{eq:Fcrystal}.
    }
\end{table}

The same in-plane maps were recorded on a smaller angular range (\ang{110} are sufficient to see the Bragg peaks, fig. \ref{fig:QxQyMap}) at \qty{300}{\degreeCelsius}, \qty{500}{\degreeCelsius} and \qty{600}{\degreeCelsius} under different atmospheres as detailed in tab. \ref{tab:Conditions}.
The intensity was then integrated along a thin region in $\delta$ around the value of the (111), (200) and (220) scattering angles.
These measurements are presented in fig. \ref{fig:Epitaxy111}, fig. \ref{fig:Epitaxy200} and fig. \ref{fig:Epitaxy220}.

\begin{figure}[!htb]
    \centering
    \includegraphics[width=\textwidth]{/home/david/Documents/PhDScripts/SixS_2021_03_SXRD_NH3/figures/epitaxy/220.pdf}
    \caption{
        Integrated intensity in a \ang{1} range around the value of the (220) scattering angle, as a function of the in-plane sample angle $\omega$.
    }
    \label{fig:Epitaxy220}
\end{figure}

Three (220)-type Bragg peaks are measured in either condition, with a stable position and shape (fig. \ref{fig:Epitaxy220}), which shows that the crystals do not rotate around their [111] axis on the surface, and that they all share not only the same out-of-plane [111] orientation but also the exact same in-plane orientation.
The intensity decreases as a function of the temperature which could be due to the Debye-Waller factor, the thermally induced movements of the atoms around their equilibrium position resulting in a lower intensity of Bragg peaks at higher temperatures \parencite{Willmott}.
There is no increase of the powder signal corresponding to the (220) Bragg peak during the measurements.

No (200) or (111)-type Bragg peaks have been seen to appear during the different measurements (figures visible in the appendix \ref{fig:Epitaxy200}, and \ref{fig:Epitaxy111}).
Moreover, there is no variation of the powder signal for the (200) Bragg peak, there is a slight decrease of the powder signal for the (111) Bragg peak which cannot be observed on the other powder signals which could be linked to instrumental effects.
Additionnal powder diffractograms are visible in the appendix \ref{fig:PowderNanoparticles}, on which it is also clear that the powder and nanoparticles do not have the same in-plane lattice parameter, and that there is not variation of the in-plane lattice parameter of the nanoparticles as a function of the atmosphere.

Most importantly, these measurements confirm that the particles are stable on the substrate as different temperatures and atmospheres and prove that surface x-ray diffraction measurements with grazing incidence can yield information on the average nanoparticle behaviour.
Bragg peaks originating from the \ce{Al_2O_3} substrate together with very important powder signals prevented us from a more precise analysis of the facet signal.

The size distribution of the particles could have been measured before the experiment with other techniques such as TEM \parencite{Hejral2013}, to simulate an average diffraction pattern and compare its intensity with the experimental data.

Reflectivity measurements have been carried out as during the experiment to extract additionnal quantitative information such as the percentage of the surface covered by the nanoparticles from the values of the fitted electronic density, or the average height of the platinum nanoparticles.
However, repeated loss of contact with the heater resulted in experimental difficulties to keep the sample surface perfectly aligned and prevented us from measuring high quality reflectivity curves.
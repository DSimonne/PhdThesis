\section{Temperature ramp}\label{sec:TempRampBCDI}

% strain field energy: kinda useless
% peak width:
% could add scans at 200, 250, 300 and 350 °C,
% need to put scan table in appendix

To de-corellate the effect of temperature from the effect of the catalytic reaction on the nanoparticle structure, the alumina-supported nanoparticles were gradually heated from \qtyrange{25}{600}{\degreeCelsius} under a constant \argon-based gas flow (\qty{50}{\ml\per\min}) and at a pressure of \qty{0.3}{\bar}.

The same nanoparticle (so-forth called nanoparticle Amaterasu) was tracked during the heating process, rocking-curves around the [111] Bragg peak were measured every \qty{50}{\degreeCelsius} to probe its structural evolution, with \qty{201}{steps}, each counting for \qty{5}{\second}, resulting in an angular step equal to \ang{0.005}.
The measurement of a rocking curve took approximately \qty{22}{\minute}, with \qty{25}{\percent} of dead time.

\textit{Gwaihir} was used to process the data using the workflow detailed in sec. \ref{sec:Gwaihir}, each dataset was saved as a \textit{cxi} file containing all of the data and metadate related to the data reduction process, available on request.
The use of iterative algorithm for phase retrieval is detailed in tab. \ref{tab:ReconstructionProcess}.

\begin{table}[!htb]
\centering
\resizebox{\textwidth}{!}{%
    \begin{tabular}{@{}llll@{}}
    \toprule
    Iteration & Algorithm & PSF & Description \\
    \midrule
    0-199 & HIO & False & Work on the gross identification of the support (typical support voxel \% = 20) \\
    200-599 & RAAR & False & Refining the support (typical support voxel \% = 20) \\
    600 & RAAR & True & \begin{tabular}[c]{@{}l@{}}Activate the use of a point-spread function (PSF) to take into account the partial coherence \\ and the response of the detector.\end{tabular} \\
    601-999 & RAAR & True & Refine the PSF shape, the support must be already well defined to avoid diverging. \\
    1000-1200 & ER & True & Further refine the support by reducing the algorithm flexibility. \\
    \bottomrule
    \end{tabular}%
}
\caption{Example of algorithm chain used in BCDI for the phase retrieval.}
\label{tab:ReconstructionProcess}
\end{table}

The surface of the Pt nanoparticle coloured by the surface voxel strain values is presented in fig. \ref{fig:Amaterasu}, only 4 temperatures are shown because the shape and strain of the nanoparticle does not evolve between \qty{150}{\degreeCelsius} and \qty{450}{\degreeCelsius} besides the removal of the defect.
A dislocation identified at the interface with the substrate at \qty{150}{\degreeCelsius} from its strain signature and missing pipe of electronic density \parencite{Dupraz2015} was removed by an instrumental mistake which resulted in a flash heating procedure, the heater going from \qtyrange{150}{800}{\degreeCelsius} for a few minutes before coming back to \qty{150}{\degreeCelsius}.

The displacement field could not be perfectly unwrapped around the dislocation, resulting in two positive strain regions (in red) in the surrounding low strain region.

\begin{figure}[!htb]
    \centering
    \includegraphics[width=0.49\textwidth]{/home/david/Documents/PhDScripts/SixS\_2021\_01/paraview/1414top.png}
    \includegraphics[width=0.49\textwidth]{/home/david/Documents/PhDScripts/SixS\_2021\_01/paraview/1414bottom.png}
    \includegraphics[width=0.49\textwidth]{/home/david/Documents/PhDScripts/SixS\_2021\_01/paraview/1483top.png}
    \includegraphics[width=0.49\textwidth]{/home/david/Documents/PhDScripts/SixS\_2021\_01/paraview/1483bottom.png}
    \includegraphics[width=0.49\textwidth]{/home/david/Documents/PhDScripts/SixS\_2021\_01/paraview/1588top.png}
    \includegraphics[width=0.49\textwidth]{/home/david/Documents/PhDScripts/SixS\_2021\_01/paraview/1588bottom.png}
    \includegraphics[width=0.49\textwidth]{/home/david/Documents/PhDScripts/SixS\_2021\_01/paraview/1675top.png}
    \includegraphics[width=0.49\textwidth]{/home/david/Documents/PhDScripts/SixS\_2021\_01/paraview/1675bottom.png}
    \caption{
        Surface of the reconstructed Amaterasu Pt nanoparticle at \qty{25}{\degreeCelsius} (before the start of the temperature ramp), at \qty{150}{\degreeCelsius} (after the flash annealing), at \qty{450}{\degreeCelsius} and at \qty{600}{\degreeCelsius} (\qty{60}{\minute} after the introduction of \ammonia).
        The surface is coloured by the values of the heterogeneous strain as described in eq. \ref{eq:StrainTensorzz} at each surface voxel, the limit of each facet is delimited by thick white lines.
    }
    \label{fig:Amaterasu}
\end{figure}

\subsection{Shape evolution}

The nanoparticle shape stayed roughly the same during the beginning of the temperature ramp and after the introduction of ammonia, the triangular [111] facet at the top of the particle can be recognized, with the same agency of the surrounding facets (fig. \ref{fig:Amaterasu} - [111] facet surrounded by six facets, [100], [010], [001], [1$\bar{1}$1], [11$\bar{1}$], [$\bar{1}$11]).
The bottom facet, in contact with the substrate, has a [$\bar{1}\bar{1}\bar{1}$] orientation.

% 25 -> 150, one more 113
The nanoparticle shape near the interface has become less round and more faceted from the disappearance of the defect, the surface strain between the interface and the particle has practically disappeared, while just one more [$\bar{1}\bar{1}3$] facet appeared on the side (fig. \ref{fig:AmaterasuFacetsEvolution} - a).
However, the total surface area occupied by the [$\bar{1}\bar{1}\bar{1}$], \{$\bar{1}10$\} and \{100\} facets has increased, the removal of the defect directly increasing the amount of voxels in the neighbouring facets (fig. \ref{fig:AmaterasuDefect}).

\begin{figure}[!htb]
    \centering
    \includegraphics[width=0.49\textwidth]{/home/david/Documents/PhDScripts/SixS_2021_01/paraview/WireframeStrain1414.png}
    \includegraphics[width=0.49\textwidth]{/home/david/Documents/PhDScripts/SixS_2021_01/paraview/WireframeStrain1483.png}
    \caption{
        The dislocation results in a large volume of the particle that is not visible, due to the large strains \textcolor{Important}{Better explain this}.
        Removing the dislocation increase the area of the particle that is then recognized as facets.
    }
    \label{fig:AmaterasuDefect}
\end{figure}

%150->450, loose all 113
If one expects a certain degree of symmetry from the equilibrium Winterbottom shape of a particle on a substrate \parencite{WINTERBOTTOM1967, Boukouvala2021}, this symmetry seems to only be reached at \qty{450}{\degreeCelsius} with three [$\bar{1}1\bar{1}$], [1$\bar{1}\bar{1}$] and [$\bar{1}\bar{1}$1] facets around the substrate for a total of 11 facets, which is also the only reconstruction without any \{113\} facets (fig. \ref{fig:AmaterasuFacetsEvolution}).
The surface occupied by the \{100\} facets was multipied by \qty{50}{\percent} between \qty{25}{\degreeCelsius} and \qty{450}{\degreeCelsius}, which is also the temperature at which most of the particle surface was recognized as faceted (fig. \ref{fig:AmaterasuFacetsEvolution}).

The facets around the bottom of the particle are the most subject to change during this temperature ramp while the facets at the top of particle are more stable (fig. \ref{fig:AmaterasuFacetsEvolution}).
Moreover, it is possible that due to the low imaging resolution of the experiment, it is impossible to distinguish the smallest facets present on the particle surface.
Indeed, fig. \ref{fig:AmaterasuFacetsEvolution} shows that the total surface area not recognized as facets from the algorithm is always near \qty{50}{\percent}.

\begin{figure}[!htb]
    \centering
    \includegraphics[width=\textwidth]{/home/david/Documents/PhDScripts/SixS_2021_01/FacetAnalyser/FacetSizeEvolution.pdf}
    \caption{
        a) Evolution of the number of facet from specific facet families on the particle.
        b) Evolution of the particle surface area occupied by specific facet families.
        c) Evolution of the particle surface area occupied by the $[111]$ top facet, the $[\bar{1}\bar{1}\bar{1}]$ bottom facet, the other $\{111\}$ facets ($\{\bar{1}1\bar{1}\}$ and $\{1\bar{1}1\}$).
        The surface not recognized as part of a facet by the algorithm (e.g. rough areas of the particle surface, edges and corners) is taken into account.
    }
    \label{fig:AmaterasuFacetsEvolution}
\end{figure}

% 450 -> 600, 110 and 113 appear
Interestingly, after the reshaping of the particle at \qty{450}{\degreeCelsius} that shows the disappearance of the \{113\} facets (fig. \ref{fig:AmaterasuFacetsEvolution}), the following rocking curves where impossible to reconstruct.
The disappearance of the \{113\} facets could have been the starting point of a thermally-induced mobility of the Pt atoms on the particle surface, only increasing with the temperature, and making it impossible for the algorithm to determine a fixed support for the reconstruction.

The introduction of \ammonia at \qty{600}{\degreeCelsius} seems to have stabilized the particle, which could again be succesfully reconstructed.
For the first time, a \{110\} facet appeared on the particle surface together with the return of a \{113\} facet, both seem to have formed at the junction of three [100], [$1\bar{1}1$] and [$\bar{1}1\bar{1}$] facets as seen in fig. \ref{fig:Amaterasu110}.
On the other side of the particle, a [$1\bar{1}1$] facet was replaced by a [$1\bar{1}0$] facet, without the appearance of a \{113\} facet.
The change is also visible in fig. \ref{fig:AmaterasuFacetsEvolution}, the relative surface occupied by the [111], [$\bar{1}\bar{1}\bar{1}$], \{110\} and \{113\} facets increasing, whereas the surface occupied by the \{$1\bar{1}1$\} and \{$\bar{1}1\bar{1}$\} decreases at \qty{600}{\degreeCelsius} under ammonia.

\begin{figure}[!htb]
    \centering
    \includegraphics[width=0.49\textwidth]{/home/david/Documents/PhDScripts/SixS\_2021\_01/paraview/110facet1588.png}
    \includegraphics[width=0.49\textwidth]{/home/david/Documents/PhDScripts/SixS\_2021\_01/paraview/110facet1675.png}
    \caption{
        Surface of the reconstructed Amaterasu Pt nanoparticle at \qty{450}{\degreeCelsius} under Argon and at \qty{600}{\degreeCelsius}, \qty{60}{\minute} after the introduction of \ammonia, highlighting the appearance of a [110] facet.
        The surface is coloured by the values of $\epsilon_{zz}$ at each surface voxel, the limit of each facet is delimited by white tubes.
    }
    \label{fig:Amaterasu110}
\end{figure}

When introducing \dioxygen at \qty{600}{\degreeCelsius} to study the oxidation of ammonia, the nanoparticle was definitely lost during the measurement, not to be found again, which underlines the difficulty to study a highly exothermic reaction with BCDI.
Indeed, according to \cite{}, the reaction can heat the catalyst to very high temperatures which in our case could have been the trigger for the loss of the particle during the measurement, especially since the particle had so far resisted to the beam during the temperature ramp to \qty{600}{\degreeCelsius}.
From this first set of results, the measurement of Pt nanoparticles with BCDI was decided to be carried out at \qty{300}{\degreeCelsius} and \qty{450}{\degreeCelsius}.

\subsection{Strain evolution} \label{sec:StrainTempRamp}

\subsubsection{Determination of strain}

Lattice strain in diffraction is usually defined as the difference between the  reference and experimental lattice parameter values, respectively $a_{ref}$ and $a$ (eq. \ref{eq:StrainDiffraction}).

\begin{equation}
    \epsilon = \frac{a - a_{ref}}{a_{ref}}
    \label{eq:StrainDiffraction}
\end{equation}

The values of the lattice parameter can be extracted from the position $\vec{G}$ of the Bragg peaks in reciprocal space \textit{via} eq. \ref{eq:QandD3} and eq. \ref{eq:Interplanarspacing}.
However, the information extracted from the retrieved phase in BCDI is more complex since, by measuring three non-coplanar Bragg peaks, one may retrieve the full strain tensor (eq. \ref{eq:StrainTensor}) from the reconstructed displacement field $\vec{u}_{\hat{q_x}, \hat{q_y}, \hat{q_z}}$ (eq. \ref{eq:DisplacementField}), $\hat{q_x}, \hat{q_y}, \hat{q_z}$ being three orthogonal basis vector in the reciprocal space \parencite{Karpov2019}.

\begin{equation}
    \vec{u}_{\hat{q_x}, \hat{q_y}, \hat{q_z}} =
     \begin{pmatrix}
        \vec{u}_{\hat{q_x}} \\
        \vec{u}_{\hat{q_y}} \\
        \vec{u}_{\hat{q_z}} \\
     \end{pmatrix}
     \label{eq:DisplacementField}
\end{equation}

\begin{multicols}{3}
    \begin{equation}
        \epsilon =
        \begin{bmatrix}
            \epsilon_{xx} & \epsilon_{yx} & \epsilon_{zx}\\
            \epsilon_{xy} & \epsilon_{yy} & \epsilon_{zy}\\
            \epsilon_{xz} & \epsilon_{yz} & \epsilon_{zz}
        \end{bmatrix}
        \label{eq:StrainTensor}
    \end{equation}
    \break
    \begin{equation}
      \epsilon_{ij} = \frac{1}{2}
        \Bigg(
        \frac{\partial \vec{u}_{\hat{q_i}}}{\partial \hat{q_j}}
        +
        \frac{\partial \vec{u}_{\hat{q_j}}}{\partial \hat{q_i}}
        \Bigg)
        \label{eq:StrainTensorIJ}
    \end{equation}
    \break
    \begin{equation}
      \epsilon_{zz} =
        \Bigg(
        \frac{\partial \vec{u}_{\hat{q_z}}}{\partial \hat{q_z}}
        \Bigg)
        \label{eq:StrainTensorzz}
    \end{equation}
\end{multicols}

The idea behind using the strain tensor is to identify shear components in the displacement field, \textit{i.e.} see if components in one direction depend on other directions.
This can also help to identify defects present in nanoparticles \parencite{Lauraux2021}.
Using a single Bragg peak, we can only retrieve one component of the displacement field, obtained from the division of the retrieved phase $\Phi$ of the scattered x-rays by the value of the scattering vector at the center of mass of the 3D Bragg peak (\textit{i.e.} $\vec{q} = \vec{G}$), following the assumptions for phase retrieval detailed in sec. \ref{sec:StrainBCDI}, eq. \ref{eq:FcrystalBCDI3} - \ref{eq:FcrystalBCDI7}.

In our case, the direction of the [111] scattering vector is perpendicular to the sample, along the $\vec{z}$ axis of the laboratory frame.
Therefore, the [111] scattering vector, $\vec{G}_{111}$, is so forth described as $\vec{q}_z$, of magnitude  $|\vec{q}_z|$, with the direction described by the unit vector  $\hat{q_z}$, to be consistent with the equations detailed above.
Our approach to the strain is considerably simplified since we can only correctly derive one component of the strain tensor, $\epsilon_{zz}$ (eq. \ref{eq:StrainTensorzz}) as detailed below in eq. \ref{eq:StrainFromPhase1} - \ref{eq:StrainFromPhase3}.

\begin{align}
    \label{eq:StrainFromPhase1}
    & \Phi =  \vec{q}_z.\vec{u} \\
    \label{eq:StrainFromPhase2}
    & \frac{\Phi}{|\vec{q}_z|} = \frac{\vec{q}_z}{|\vec{q}_z|}.\vec{u} = \hat{q_z}.\vec{u} = \vec{u}_{\hat{q_z}} \\
    \label{eq:StrainFromPhase3}
    & \vec{\nabla} \vec{u}_{\hat{q_z}} = \frac{\partial u_{\hat{q_z}}}{\partial z} = \epsilon_{zz}
\end{align}

\subsubsection{Homogeneous strain}

The deviation of the interplanar spacing from the room temperature value due to the thermal expansion of the crystal is expected to be homogeneous within the particle.
This isotropic \textit{homogeneous} strain ($\epsilon_{hmg}$) in the particle is removed by centering the Bragg peak before phase retrieval, otherwise resulting in a linear phase ramp \parencite{}.
It is of utmost important to study the evolution of the material first under an inert atmosphere to see if there is an evolution of the homogeneous strain not only due to the thermal expansion of the crystal, but also to global relaxation phenomena induced e.g. by the adsorption of molecules involved in the catalytic reaction.

\begin{figure}[!htb]
    \centering
    \includegraphics[width=\textwidth]{/home/david/Documents/PhDScripts/SixS\_2021\_01/Max.png}
    \caption{
        The maximum value of the 3D Bragg peak is used to center the intensity before phase retrieval and to compute the average interplanar spacing $d_{111}$.
    }
    \label{fig:MaxPeak}
\end{figure}

\begin{figure}[!htb]
    \centering
    \includegraphics[width=\textwidth]{/home/david/Documents/PhDScripts/SixS\_2021\_01/LatticeParameter.png}
    \caption{
        Evolution of the $d_{111}$ interplanar spacing and associated homogenous strain values by comparing with the measurement at \qty{25}{\degreeCelsius}.
    }
    \label{fig:HomoStrain}
\end{figure}

The average interplanar spacing was computed from the scattering vector $\vec{q}_z$, \textit{via} eq. \ref{eq:QandD3} (fig. \ref{fig:HomoStrain}).
The interplanar spacing follows approximately a linear increase from \qty{2.2517}{\angstrom} at \qty{25}{\degreeCelsius} to \qty{2.2627}{\angstrom} at \qty{600}{\degreeCelsius} as a function of the temperature.
\textcolor{Important}{Errorbars}

Interestingly, there is no change of the average interplanar spacing after removal of the defect at \qty{150}{\degreeCelsius}.
There is however a very strong increase of the interplanar spacing at \qty{600}{\degreeCelsius}, which decreases after the introduction of ammonia.

\subsubsection{Heterogeneous strain}

The remaining strain after phase retrieval is called the \textit{heterogeneous} strain ($\epsilon_{htg}$) \parencite{GREDIAC1996, FAVIER2007, Atlan2023}, the total strain observed during this experiment being equal to $\epsilon_{tot} = \epsilon_{hmg} + \epsilon_{zz, htg}$

In BCDI, it is the displacement of small unit blocks making up the crystal lattice that is observed.
Depending on the instrumental parameters, these small unit blocks, called \textit{voxels}, have a more or less large size.
In this study, they are approximately \qtyproduct{10x10x10}{\nm} large.
The strain is thus not directly related to the deviation between the interreticular planes, but to the displacement field of these unit blocks from their equilibrium position.
The outermost layers of the crystal only constitute a low percentage of the surface voxels (\qty{\approx 11}{\percent} if we consider 5 atomic layers separated by the value of the interplanar spacing $d_{111}$), which lowers the contribution of the surface strain to the total strain contained in the surface voxels, reducing the ability to properly resolve e.g. surface relaxation effects.
The use of padding during the reconstruction algorithm decreases the voxel size, but without relying on the sampling of high-frequency components of the scattering amplitude and therefore does not increase the strain resolution.

\begin{figure}[!htb]
    \centering
    \includegraphics[width=0.6\textwidth]{/home/david/Documents/PhDScripts/SixS\_2021\_01/FacetAnalyser/Angles.pdf}
    \caption{
        Angles between the $[111]$ direction (normal to the top facet and opposite to the bottom facet of the particle) and other crystallographic directions that describe the normals to the other facets of the particle.
    }
    \label{fig:Angles}
\end{figure}

\begin{figure}[!htb]
    \centering   % left x right y
    \includegraphics[width=\textwidth]{/home/david/Documents/PhD/Figures/introduction/stereographic_projection_bottom.pdf}
    \caption{
        Stereographic projection perpendicular to $[\bar{1}\bar{1}\bar{1}]$ crystallographic orientation.
    }
    \label{fig:StereoBottom}
\end{figure}

The crystallographic orientation of a facet can be determined by computing the angle between the [111] direction and its normal, which is then related to a direction in real space (fig. \ref{fig:Angles}).
Facets with the same orientation (e.g. [100], [010], [001]) are grouped around the same value of the interplanar angle.
To be more precise, the angular value is computed as the average of the angle between the facet normal and the [111] direction plus the angle between the facet normal and the [$\bar{1}\bar{1}\bar{1}$] direction minus \ang{180}, \textit{i.e.} $(\angle (111, \vec{n}) + (\angle (\bar{1}\bar{1}\bar{1}, \vec{n}) -180))/2$.

If equivalent orientation translates into equal surface atomic structures (e.g. [111] and [$\bar{1}\bar{1}\bar{1}$]), the environment of equivalent facets is not always the same.
For example, it is important to differ between [1$\bar{1}$1] and [$\bar{1}$1$\bar{1}$] facets, the higher the value of the interplanar angle, the closer the facet is to the interface with the substrate, and thus the more its influence in prominent.
Moreover, if the [111] facet is at the top of the crystal and the furthest away from the substrate, the [$\bar{1}\bar{1}\bar{1}$] facet is expected to be fully in contact with the substrate.
Therefore, the mean value and standard deviation of the heterogeneous strain distribution on each facet is presented on fig. \ref{fig:AmaterasuStrain}, as a function of the angle between the facets normals and the [111] direction, to highlight this difference.

\begin{figure}[!htb]
    \centering
    \includegraphics[width=\textwidth]{/home/david/Documents/PhDScripts/SixS\_2021\_01/FacetAnalyser/FacetStrainEvolution.pdf}
    \caption{
        Mean value and standard deviation of the heterogeneous strain ($\epsilon_{zz, htg}$) distribution as a function of the angle between the normal of each facet on the particle surface and the $[111]$ direction.
        Upwards and downwards arrow are represented for respectively positive and negative strain.
    }
    \label{fig:AmaterasuStrain}
\end{figure}

% introduce in plane out of plane
It is important to realize that the displacement observed is \textit{only} in the [111] direction, which means that it translates deviations of the crystal structure perpendicular to the [111] and [$\bar{1}\bar{1}\bar{1}$] facets, but parallel to the [$\bar{1}$10] facets, themselves parallel to the [111] direction (fig. \ref{fig:Angles}).
For example, if there existed a displacement of the atoms on the \{100\} facets in the direction perpendicular to the [100] planes, its contribution to the displacement would not be visible.

Moreover, if the $\epsilon_{zz}$ component of the strain tensor on the [111] and [$\bar{1}\bar{1}\bar{1}$] facets is easily assimilated to variation of the interplanar spacing $d_{111}$ (positive strain is tensile strain, negative strain is compressive).
This physical meaning of the strain becomes more complex when observing facets that are neither parallel nor perpendicular to the [111] direction, such as the \{113\} or \{100\} facets.
For this reason, the following analysis of the heterogeneous strain is meant to be qualitative, a quantitative analysis could only be performed with the full strain tensor.
% The relation between in-plane and out-of-plane strain can be rationalized by the Poisson effect \parencite{Atlan2023}, which relates an in-plane tensile (compressive) strain to an out-of-plane compressive (tensile) strain (depending on the origin of the displacement field).

The values of the displacement are in general quite low (always less than \qty{0.2}{\percent}), which puts us far away from the BCDI limit discussed in sec. \ref{sec:StrainBCDI}.
Let's take the example of the particle at \qty{25}{\degreeCelsius} for which the absolute value of the heterogeneous strain is the highest.
We have $|\vec{G}| = \qty{}{\angstrom}$, the furthest value of the scattering vector probed during the measurement is $|\vec{q}| = \qty{}{\angstrom}$ which gives $(\vec{q}-\vec{G}).\vec{u}_k<<1$.
\textcolor{Important}{Get values from data for dq}

% Decribe removal of defect strain evolution 25°C -> 150 °C
The very large standard deviations seen in fig. \ref{fig:AmaterasuStrain} at \qty{25}{\degreeCelsius} for the [$\bar{1}\bar{1}\bar{1}$], [$\bar{1}1\bar{1}$], [$1\bar{1}1$] and [$11\bar{3}$] facets can be explained by the presence of the defect, around which the phase was not well unwrapped as seen in fig. \ref{fig:AmaterasuDefect}.
At \qty{150}{\degreeCelsius}, after removal of the defect, all of the facets show low strain values as well as low standard deviation.

% 150 -> 450
The strain values are still relatively low at \qty{450}{\degreeCelsius} but when comparing to the measurement at \qty{150}{\degreeCelsius}, we can see in fig. \ref{fig:Amaterasu} that the extremities of the [$\bar{1}\bar{1}\bar{1}$] facet are in tension, whereas the center is in compression.
The opposite effect seems to take place around the [111] facet, visible both in fig \ref{fig:Amaterasu} and \ref{fig:AmaterasuStrain}.
\textcolor{Important}{Why ?}

% Decribe strain change between 450°C under Argon and 600°C under ammonia
After the introduction of ammonia at \qty{600}{\degreeCelsius}, the particle could be reconstructed again.

\begin{figure}[!htb]
    \centering
    \includegraphics[width=0.32\textwidth]{/home/david/Documents/PhDScripts/SixS\_2021\_01/paraview/1675_strain_and_facets1.png}
    \includegraphics[width=0.32\textwidth]{/home/david/Documents/PhDScripts/SixS\_2021\_01/paraview/1675_strain_and_facets2.png}
    \includegraphics[width=0.32\textwidth]{/home/david/Documents/PhDScripts/SixS\_2021\_01/paraview/1675_strain_and_facets3.png}
    \caption{
        View of the Amaterasu particle at \qty{600}{\degreeCelsius} after the introduction of ammonia.
        The three \{100\} facets are surrounded by either 4 \{111\} facets and one \{110\} facets (left), by 3 \{111\} facets and one \{110\} facets (middle) or by 4 \{111\} facets (right).
    }
    \label{fig:AmaterasuStrain1675}
\end{figure}

The alternating highly positive and negative strain regions in the middle of the [$\bar{1}\bar{1}\bar{1}$] facet (fig. \ref{fig:Amaterasu}) is the signature of a dislocation network forming in the interface with the substrate \parencite{Dupraz2015}, also responsible for the high strain standard deviation of this facet (fig. \ref{fig:AmaterasuStrain}).
A higher resolution in a 3D displacement field together with simulations of the impact of defect at the interface are needed to fully characterized this network, which is outside the scope of this thesis.

% 111 different effect
% other facets all same
% dif in env of 113 and 111 ?

In comparison with the measurement at \qty{450}{\degreeCelsius} under Argon, the [111] facet goes into compression (negative strain) as seen on fig. \ref{fig:Amaterasu}, \ref{fig:AmaterasuStrain} and \ref{fig:AmaterasuStrain1675}.
Moreover, we can see that the same effect takes place on the neighbouring \{100\} facets, none of which have the same environment (fig. \ref{fig:AmaterasuStrain1675}), while the edges and corners around the [111] and \{100\} facet show a positive strain.
Both [$0\bar{1}1$] and [$01\bar{1}$] facets have a very low strain.

At first sight, all facets that have a similar orientation show a similar strain (e.g. \{1-10\}, \{100\}).
However, the two \{111\} facets that are close to the substrate do not show the same strain than the three \{111\} facets closer to the top of the particle.
The strain being in the [111] direction with the origin of the displacement field in the center of the particle, if all of these facet had experienced the same compressive (tensile) strain perpendicular to their surface \textit{via} the adsorption of ammonia, they would all show a negative (positive) strain (of lower amplitude due to the angle between these facets and the [111] direction).

This behaviour could be linked to a strong effect of the support which has the dual effect of first preventing the [$\bar{1}\bar{1}\bar{1}$] facet from being exposed and secondly forcing them to accomodate the strain linked to the substrate.
There are three possibilities, first ammonia is effectively adsorbed on these facets but the influence of the substrate hides a possible common signature in the strain.
Secondly ammonia is not adsorbed and the strain difference is due to the influence of the substrate on the particle structure.
Thirdly, ammonia is adsorbed but only on the three top facets or only on the 2 bottom facets due to the influence of the substrate that either limitates or facilitates the adsorption process.

% Globally, the top of the particle seems to be in compression along $\vec{z}$ (fig. \ref{fig:Amaterasu}, \ref{fig:AmaterasuStrainSlices}), whereas the bottom of the particle is in tension, with the exception of the one [$1\bar{1}3$] facet (fig. \ref{fig:AmaterasuStrainSlices}) which is in compression.

% \begin{figure}[!htb]
%     \centering
%     \includegraphics[width=0.49\textwidth]{/home/david/Documents/PhDScripts/SixS\_2021\_01/paraview/1675_clipx.png}
%     \includegraphics[width=0.49\textwidth]{/home/david/Documents/PhDScripts/SixS\_2021\_01/paraview/1675_clipy.png}
%     \caption{
%         Particle slices perpendicular to the $\vec{x}$ and $\vec{y}$ directions at center of mass, a white line delimitates the particle surface.
%     }
%     \label{fig:AmaterasuStrainSlices}
% \end{figure}

Overall, it is difficult to conclude on any potential effect of the absorption of ammonia on the particle due to the presence of the dislocation network at the interface, to the lack of comparative reconstruction at \qty{600}{\degreeCelsius} under argon, and more importantly to the fact that we only possess one component of the strain tensor, making it difficult to quantify any relaxation effect on the facets that are not perpendicular to the [111] direction.

% The shape of the particle in the ($\vec{x}, \vec{y}$) plane at \qty{600}{\degreeCelsius} can be roughly approximated as a hexagon with 3 sides occupied by the [$\bar{1}11$], [$1\bar{1}1$] and [$11\bar{1}$] facets.
% The three other sides are occupied by the [100], [010], [001] facets near the top of the particle, under which are situated either 1) one large [$\bar{1}10$] facet, 2) one small [$1\bar{1}0$] facet under which in turn are two smalls [113] and [$\bar{1}1\bar{1}$] facets, 3) a large [$1\bar{1}\bar{1}$] facet.

%%
% The negative strain of the [$1\bar{1}1$] facets and the positive strain on the [$\bar{1}1\bar{1}$] facets (fig. \ref{fig:AmaterasuFacetsEvolution}) follows the respective evolution of the [111] and [$\bar{1}\bar{1}\bar{1}$] facets with a lower amplitude.
% An increase of the negative strain can also be seen on all three [100] facets but, based on the angle between those facets and the [111] direction (\ang{54.7}), the interpretation of this value is difficult to understand.

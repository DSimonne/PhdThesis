\section{Temperature ramp}\label{sec:TempRampBCDI}

% strain field energy: kinda useless
% need to put scan table in appendix
% specific rocking curves

To be able to de-corellate the effect of temperature from the effect of the catalytic reaction on the nanoparticle structure, the alumina-supported nanoparticles were gradually heated from \qtyrange{25}{600}{\degreeCelsius} under a constant \argon-based gas flow (\qty{50}{\ml\per\min}) and at a pressure of \qty{0.3}{\bar}.

The same nanoparticle (so-forth called nanoparticle Amaterasu) was tracked during the heating process, rocking-curves around the (111) Bragg peak were measured at different temperatures to probe its structural evolution, with \qty{201}{steps}, each counting for \qty{5}{\second}, resulting in an angular step of \ang{0.005}.
The measurement of a rocking curve took approximately \qty{22}{\minute}, with \qty{25}{\percent} of dead time.
Two rocking curves were measured at each temperature during the ramp to have improved statistics about the catalyst structure.
The alignement of the sample was performed with the procedure as detailed in sec. \ref{sec:BCDISetup}.

\subsection{Experimental setup for BCDI experiments in the vertical geometry}\label{sec:BCDISetup}

The BCDI experiment was performed at the SixS (Surface Interface X-ray Scattering) beamline of synchrotron SOLEIL, France (sec. \ref{sec:SIXS}).
As detailed in sec. \ref{sec:Gwaihir}, one of the bottlenecks of the BCDI technique is its slow data reduction and analysis process.
Moreover, on $3^{rd}$ generation synchrotrons that offer a lower coherent flux (eq. \ref{eq:CoherentFlux}) than $4^{th}$ generation synchrotrons, the measurement time can also be very long.
At SixS, a rocking curve lasts between \qtyrange{20}{90}{\min} depending on the particle size, the quality of the alignment, the strain of the particle, \textit{etc.}
Once the raw data is obtained, the particle must be \textit{reconstructed} (sec. \ref{sec:PhaseRetrieval}), and - if of interest - the displacement and strain arrays can be retrieved.
The analysis workflow can take up to an hour, which totals to a maximum of two hours from the start of the measurement to the moment when the user has a good idea of the sample shape and structure.

SixS is a beamline that does not only carry out BCDI experiment, but also SXRD experiments, in the same experimental end-station, the multi-environment diffractomer (MED, sec. \ref{sec:MED}).
When aiming at performing \textit{operando} catalysis experiments, switching from one setup to another can take up to a few days.
This leaves only a limited remaining amount of time to align the sample, find a suitable nanoparticle, and carry out the experimental plan.

Li \textit{et al.} \parencite*{Li2020}, who have first shown that the SixS beamline could be used to carry out BCDI experiment, also started to work on improving the BCDI measurement process.
By comparing continuous and step-by-step measurements, they have shown that continous scanning would result in the same data quality while decreasing the measurement dead-time by \qty{30}{\percent}, thereby paving the way for quicker BCDI measurements.

During this thesis, the measurement process was further improved by taking advantage of the new possibility to perform continuous \textit{on-the-fly} scans at SixS, while moving the hexapod holding the sample.
When using the coherence setup (fig. \ref{fig:OpticalSetup}) with the diffractomer in the vertical geometry (fig. \ref{fig:Diffractometer}), the beam focused on the sample is about \qty{1}{\um} large vertically, the horizontal footprint depending on the incident angle between the beam and the sample.
The incoming angle $\mu$ is set to the Bragg angle ($\theta$ in eq. \ref{eq:Bragglaw}) and the scattered x-rays are collected by setting the out-of-plane detector angle $\gamma$ at a position equal to $2\theta$ (similar to fig. \ref{fig:EwaldSphereSpecular}).
Finally, by simultaneously moving the sample with the hexapod and recording the Bragg scattered intensity with the detector, it is possible to map the sample surface with a sub-micron resolution (fig. \ref{fig:SampleMapping}).
A nanoparticle with a width equal to \qty{300}{\nm} was identified with this technique, which is a good estimate of the spatial resolution that can be attained, limited by the hexapod resolution (\qty{\approx 500}{\nm}) and the beam size.
A new generation of motots with a nanometric resolution will further improve this technique in the future.

During this experiment, the simplest possible measurement geometry is to measure the (111) Bragg peak, perpendicular to the sample surface since the nanoparticles have their c-axis oriented along the [111] direction, parallel to the normal of the sample holder.
% ($\vec{z}$ direction in the laboratory frame).
Both in-plane angles are thereby kept to zero during the mapping process.

\begin{figure}[!htb]
    \centering
    \includegraphics[height=5cm]{/home/david/Documents/PhD/Figures/sample/microscope_image.png}
    \includegraphics[height=5cm]{/home/david/Documents/PhD/Figures/sample/microscope_image_photon.png}
    \caption{
        Microscope image of the sample seen through the sapphire window of the PEEK dome (left).
        Map of the sample performed in Bragg condition (right), the high intensity (red) areas correspond to platinum nanoparticles.
        The letters, numbers and isolated nanoparticles in the centre of squares can be recognized on the sample.
        \textcolor{Important}{colorbar}
    }
    \label{fig:SampleMapping}
\end{figure}

On the other hand, \textit{Gwaihir} (sec. \ref{sec:Gwaihir}) was developped primarily for the SixS beamline to counter the long analysis process, which allowed a significant reduction in the analysis time from around an hour to a few minutes.
The following beamtimes profit from the new software by having a more \textit{solution}-driven experimental process.
Indeed, to be measured, a nanoparticle must be isolated, not too small (weak scattered intensity), not too big (loss of coherence, fringes not visible), and not too initially strained (difficult to obtain a good guess of the support).
These conditions are sometimes difficult to assert by simply looking at the diffraction pattern.
Therefore, quick inversion using \textit{Gwaihir} allowed a faster decision process regarding the continuation or not of the nanoparticles measurement.

Successfull measurements by \cite{Lim2021} have permitted the simulatenous use of BCDI measurements from SixS with measurements from other imaging beamlines (ID01 - ESRF, P10 - DESY), designing a robust method to identify defects in the real space with convolutional neural networks (CNN).

In the frame of this thesis, the required beam size was obtained with a Fresnel zone-plate (focal distance of \qty{20}{\cm}), which focused the beam down to \qty{1}{\um} (horizontally) $\times$ \qty{2}{\um} (vertically).
A coherent portion of the beam was selected with high precision slits by matching their horizontal and vertical gaps with the transverse coherence lengths of the beamline: \qty{20}{\um} (horizontally) and \qty{100}{\um} (vertically).
\textcolor{Important}{check consistency}
A circular beam-stop, and a circular order-sorting aperture, were used to block the transmitted beam, and higher diffraction orders, respectively (fig. \ref{fig:OpticalSetup}).
The sample was mounted in the dedicated XCAT reactor with the substrate surface oriented in the vertical plane on a hexapod mounted vertically on the MED diffractometer (fig. \ref{fig:MEDV}).

The BCDI experiment was performed in a vertical specular geometry at a beam energy of \qty{8.5}{\keV} (wavelength of \qty{1.46}{\angstrom}).
Three-dimensional (3D) diffraction data were collected with rocking curves of the rotation angle around the normal of the sample, the diffracted beam was recorded with a 2D MAXIPIX photon-counting detector (\numproduct{515 x 515} square pixels, \qtyproduct{55 x 55}{\um} wide) positioned on the detector arm at a distance of \qty{1.22}{\meter}.
The in-plane ($\omega$) and out-of-plane ($\mu$) angles of the sample at respectively equal to \ang{0} and \ang{18.6} when measuring at room temperature.
The in-plane ($\delta$) and out-of-plane ($\gamma$) angles of the detector are respectively equal to \ang{0} and \ang{37.2}.
The value of the scattering angle $\mu$ ($\theta$ in eq. \ref{eq:Bragglaw}) and the out-of-plane detector angle ($2\theta$) are expected to vary as a function of the temperature during the experiment due to the thermal expansion of the sample.
The alignement of the sample was performed with the same procedure as detailed in sec. \ref{sec:SXRDSetupV}.

\textit{Gwaihir} was used to process the data using the workflow detailed in sec. \ref{sec:Gwaihir}, each dataset was saved as a \textit{cxi} file containing all of the data and metadata related to the data reduction process.
The use of iterative algorithm for phase retrieval is detailed in tab. \ref{tab:ReconstructionProcess}.

\begin{table}[!htb]
\centering
\resizebox{\textwidth}{!}{%
    \begin{tabular}{@{}llll@{}}
    \toprule
    Iteration & Algorithm & PSF & Description \\
    \midrule
    0-199 & HIO & False & Work on the gross identification of the support (typical support voxel \% = 20) \\
    200-599 & RAAR & False & Refining the support (typical support voxel \% = 20) \\
    600 & RAAR & True & \begin{tabular}[c]{@{}l@{}}Activate the use of a point-spread function (PSF) to take into account the partial coherence \\ and the response of the detector.\end{tabular} \\
    601-999 & RAAR & True & Refine the PSF shape, the support must be already well defined to avoid diverging. \\
    1000-1200 & ER & True & Further refine the support by reducing the algorithm flexibility. \\
    \bottomrule
    \end{tabular}%
}
\caption{Example of algorithm chain used in BCDI for the phase retrieval.}
\label{tab:ReconstructionProcess}
\end{table}

\subsection{Shape evolution}

The surface of the Pt nanoparticle coloured by the surface voxel strain values is presented in fig. \ref{fig:AmaterasuA} and \ref{fig:AmaterasuB}, from \qtyrange{150}{600}{\degreeCelsius}, to show the evolution of the particle shape under an inert gas flow as a function of the temperature.
A dislocation identified at the interface with the substrate at \qty{150}{\degreeCelsius} from its strain signature and missing pipe of electronic density \parencite{Dupraz2015} was removed by an instrumental mistake which resulted in a flash heating procedure, the heater going from \qtyrange{150}{800}{\degreeCelsius} for a few minutes before coming back to \qty{150}{\degreeCelsius}.
The sample was realigned with the direct beam after the error of manipulation.

The displacement field could not be perfectly unwrapped at \qty{150}{\degreeCelsius}, \qty{350}{\degreeCelsius} and \qty{450}{\degreeCelsius}, resulting in saturated strain regions on the [$\bar{1}\bar{1}\bar{1}$] facet visible in the bottom view of the nanoparticle.

\begin{figure}[!htb]
    \centering
    \includegraphics[width=0.49\textwidth]{/home/david/Documents/PhDScripts/SixS\_2021\_01/paraview/1414top.png}
    \includegraphics[width=0.49\textwidth]{/home/david/Documents/PhDScripts/SixS\_2021\_01/paraview/1414bottom.png}
    \includegraphics[width=0.49\textwidth]{/home/david/Documents/PhDScripts/SixS\_2021\_01/paraview/1483top.png}
    \includegraphics[width=0.49\textwidth]{/home/david/Documents/PhDScripts/SixS\_2021\_01/paraview/1483bottom.png}
    \includegraphics[width=0.49\textwidth]{/home/david/Documents/PhDScripts/SixS\_2021\_01/paraview/1534top.png}
    \includegraphics[width=0.49\textwidth]{/home/david/Documents/PhDScripts/SixS\_2021\_01/paraview/1534bottom.png}
    \includegraphics[width=0.49\textwidth]{/home/david/Documents/PhDScripts/SixS\_2021\_01/paraview/1563top.png}
    \includegraphics[width=0.49\textwidth]{/home/david/Documents/PhDScripts/SixS\_2021\_01/paraview/1563bottom.png}
    \caption{
        Surface of the reconstructed Amaterasu Pt nanoparticle at \qty{150}{\degreeCelsius} (before and after the flash annealing), \qty{250}{\degreeCelsius}, and \qty{350}{\degreeCelsius}.
        The surface is coloured by the values of the heterogeneous strain as described in eq. \ref{eq:StrainTensorzz} at each surface voxel, the limit of each facet is delimited by thick white lines.
    }
    \label{fig:AmaterasuA}
\end{figure}

\begin{figure}[!htb]
    \centering
    \includegraphics[width=0.49\textwidth]{/home/david/Documents/PhDScripts/SixS\_2021\_01/paraview/1588top.png}
    \includegraphics[width=0.49\textwidth]{/home/david/Documents/PhDScripts/SixS\_2021\_01/paraview/1588bottom.png}
    \includegraphics[width=0.49\textwidth]{/home/david/Documents/PhDScripts/SixS\_2021\_01/paraview/1631top.png}
    \includegraphics[width=0.49\textwidth]{/home/david/Documents/PhDScripts/SixS\_2021\_01/paraview/1631bottom.png}
    \includegraphics[width=0.49\textwidth]{/home/david/Documents/PhDScripts/SixS\_2021\_01/paraview/1675top.png}
    \includegraphics[width=0.49\textwidth]{/home/david/Documents/PhDScripts/SixS\_2021\_01/paraview/1675bottom.png}
    \caption{
        Surface of the reconstructed Amaterasu Pt nanoparticle at \qty{450}{\degreeCelsius} and \qty{600}{\degreeCelsius} under both inert argon atmosphere and \qty{60}{\minute} after the introduction of \ammonia in the reactor.
        The surface is coloured by the values of the heterogeneous strain as described in eq. \ref{eq:StrainTensorzz} at each surface voxel, the limit of each facet is delimited by thick white lines.
    }
    \label{fig:AmaterasuB}
\end{figure}

The nanoparticle shape stayed roughly the same during the beginning of the temperature ramp and after the introduction of ammonia, the triangular [111] facet at the top of the particle can be recognized, with the same agency of the surrounding facets (fig. \ref{fig:AmaterasuA} - [111] facet surrounded by six facets, [100], [010], [001], [1$\bar{1}$1], [11$\bar{1}$], [$\bar{1}$11]).
The bottom facet, in contact with the substrate, has a [$\bar{1}\bar{1}\bar{1}$] orientation.

The low spatial resolution of the reconstructed nanoparticle due to the quality of the measurement (low coherent flux, \textit{i.e.} difficulty to resolve the scattered intensity far from the Bragg peak) resulted in difficulties when trying to identify the smallest facets on the nanoparticles, as well as their orientation.

\begin{figure}[!htb]
    \centering
    \includegraphics[width=\textwidth]{/home/david/Documents/PhD/Figures/introduction/stereographic_projection_bottom.pdf}
    \caption{
        Stereographic projection perpendicular to $[\bar{1}\bar{1}\bar{1}]$ crystallographic orientation.
        The circles describe the angle with the $[\bar{1}\bar{1}\bar{1}]$ direction from \ang{0} (center) to \ang{90} (outer-ring).
    }
    \label{fig:StereoBottom}
\end{figure}

Indeed, the crystal truncation rods perpendicular to small facets are less intense in comparison to large facets, which is at the origin of the direction-dependent spatial resolution in BCDI \parencite{cherukara_anisotropic_2018}.
Moreover, when identifying the facets in real space by the means of a surface recognition algorithm such as \textit{FacetAnalyser} (sec. \ref{sec:FacetAnalysis}), the normal to the smaller facets is more sensitive to deviation from its true value due to the low amount of voxels on the particle surface.

In fig. \ref{fig:StereoBottom} is presented the stereographic projection perpendicular to the $[\bar{1}\bar{1}\bar{1}]$ crystallographic orientation which can be used to identify the type of facets present on the nanoparticles in real or reciprocal space \parencite{Richard2018}.

In the frame of this thesis, the problem of facet identification was often encountered when facets were either too small to be resolved by the recognition algorithm, or when the uncertainty in the direction of the normal to the smallest facets is too important to be certain of the facet type.
For example, the area in orange in fig. \ref{fig:StereoBottom} represents a region in which facets with different structures have a similar orientation and for which it can be complicated to make the difference between $[\bar{1}\bar{1}1]$, $[\bar{2}\bar{2}1]$ and $[\bar{3}\bar{3}1]$ facets.

In this study, facets around the nanoparticle interface with the substrate have been identified as the smallest on the nanoparticle surface and the most subject to structural evolutions.
The orientation of all the facets present of the nanoparticle was performed in real space by the use of \textit{FacetAnalyser} taking into account the expected symmetry of the nanoparticle.

% 150 -> 150
After the flash annealing, the nanoparticle shape near the interface has become less round and more faceted from the disappearance of the defect, the surface strain between the interface and the particle has practically disappeared, a [$10\bar{1}$] facet appeared on the side of the nanoparticle near a [2$\bar{1}\bar{2}$] facet that replaced a [1$\bar{1}\bar{1}$] facet (fig. \ref{fig:AmaterasuDefect}).
A [$\bar{2}\bar{1}2$] facet is still present on the other side of the particle.
The total surface area occupied by the [$\bar{1}\bar{1}\bar{1}$] facet, ($1\bar{1}1$)-type and (100)-type facets has increased (fig. \ref{fig:AmaterasuFacetsEvolution} - a), the removal of the defect directly increasing the amount of voxels in the neighbouring facets (fig. \ref{fig:AmaterasuDefect}), \textit{i.e.} almost \qty{10}{\percent} more of the sample surface was recognized as part of the different facets by the algorithm.

\begin{figure}[!htb]
    \centering
    \includegraphics[width=0.49\textwidth]{/home/david/Documents/PhDScripts/SixS_2021_01/paraview/WireframeStrain1414.png}
    \includegraphics[width=0.49\textwidth]{/home/david/Documents/PhDScripts/SixS_2021_01/paraview/WireframeStrain1483.png}
    \caption{
        The dislocation results in a large volume of the particle that is not visible, due to the large strains \textcolor{Important}{Better explain this}.
        Removing the dislocation increase the area of the particle that is then recognized as facets, especially for the $[\bar{1}\bar{1}\bar{1}]$ facet, at the interface with the substrate.
    }
    \label{fig:AmaterasuDefect}
\end{figure}

%150->450, loose all 212, get -101 that then change side
When heating from \qty{150}{\degreeCelsius} to \qty{250}{\degreeCelsius} , the [$\bar{2}\bar{1}2$] facet present from before the defect removal is transformed towards a [$\bar{1}\bar{1}1$] facet.
When heating to \qty{350}{\degreeCelsius}, the [$10\bar{1}$] facet disappeared and a [$\bar{1}01$] facet appeared on the opposite side of the particle, this transformation is visible in fig. \ref{fig:AmaterasuA}.
If one expects a certain degree of symmetry from the equilibrium Winterbottom shape of a particle on a substrate \parencite{WINTERBOTTOM1967, Boukouvala2021}, this symmetry seems to only be reached at \qty{450}{\degreeCelsius} with three [$\bar{1}1\bar{1}$], [1$\bar{1}\bar{1}$] and [$\bar{1}\bar{1}$1] facets around the substrate for a total of 11 facets, which is also the only reconstruction without any ($\bar{2}\bar{1}2$), ($\bar{1}01$) and ($2\bar{1}\bar{1}$)-type facets (fig. \ref{fig:AmaterasuFacetsEvolution}).

Overall, the surface occupied by the facets around the top of the particle increase with heating, while the surface area that is not recognized as faceted by the algorithm decreases, the respective number of each \{$1\bar{1}1$\}, \{100\}-type facet does not change during the experiment.
% was multipied by \qty{50}{\percent} between \qty{25}{\degreeCelsius} and \qty{450}{\degreeCelsius}, which is also the temperature at which most of the particle surface was recognized as faceted (fig. \ref{fig:AmaterasuFacetsEvolution}).
The facets that are around the bottom of the particle are the most subject to change during this temperature ramp while the facets at the top of particle are more stable (fig. \ref{fig:AmaterasuFacetsEvolution}).
It is possible that due to the low imaging resolution of the experiment, smaller facets that exist on the particle surface cannot be distinguished.
Indeed, fig. \ref{fig:AmaterasuFacetsEvolution} shows that the total surface area not recognized as facets from the algorithm is always near \qty{40}{\percent}.

\begin{figure}[!htb]
    \centering
    \includegraphics[width=\textwidth]{/home/david/Documents/PhDScripts/SixS_2021_01/FacetAnalyser/FacetSizeEvolution.pdf}
    \caption{
        a) Evolution of the number of facet near the bottom (interface with the substrate) of the particle.
        b) Evolution of the particle surface area occupied by specific facets (indicated with []) or facet families (indicated with \{\}).
        The surface not recognized as part of a facet by the algorithm (e.g. rough areas of the particle surface, edges and corners) is taken into account and shown in black.
    }
    \label{fig:AmaterasuFacetsEvolution}
\end{figure}

% 450 -> 600
Interestingly, after the reshaping of the particle at \qty{450}{\degreeCelsius} that shows the disappearance of the higher Miller indices facets (fig. \ref{fig:AmaterasuFacetsEvolution}), the following rocking curves where very difficult to reconstruct until \qty{600}{\degreeCelsius}.
The increase of temperature could have been the trigger of thermally-induced mobility of the Pt atoms on the particle surface, making it impossible for the algorithm to determine a fixed support for the reconstruction.
\textcolor{Important}{find citations here}

A reconstruction at \qty{600}{\degreeCelsius} shows the appearance of a defect at the interface with the substrate (fig. \ref{fig:AmaterasuB}), and the appearance of a [$\bar{2}\bar{1}2$] facet that was present on the nanoparticle at \qty{150}{\degreeCelsius} before the removal of the defect (fig. \ref{fig:Amaterasu110}).
After the introduction of \ammonia at \qty{600}{\degreeCelsius}, the reconstruction of the particle shows the appearance of a [2$\bar{1}\bar{1}$] facet for the first time, together with the appearance of a [$\bar{1}01$] facet and the disappearance of the [$\bar{2}\bar{1}2$] facet, similarily to the changes that occured at \qty{150}{\degreeCelsius} after the removal of the defect (fig. \ref{fig:Amaterasu110}).
The change is also visible in fig. \ref{fig:AmaterasuFacetsEvolution}, for example the relative surface occupied by the \{100\}-type facets decreases, occupied by the [$\bar{1}01$] facet.

\begin{figure}[!htb]
    \centering
    \includegraphics[width=0.32\textwidth]{/home/david/Documents/PhDScripts/SixS\_2021\_01/paraview/110facet1588.png}
    \includegraphics[width=0.32\textwidth]{/home/david/Documents/PhDScripts/SixS\_2021\_01/paraview/110facet1631.png}
    \includegraphics[width=0.32\textwidth]{/home/david/Documents/PhDScripts/SixS\_2021\_01/paraview/110facet1675.png}
    \caption{
        Surface of the reconstructed Amaterasu Pt nanoparticle at \qty{450}{\degreeCelsius} under Argon and at \qty{600}{\degreeCelsius}, \qty{60}{\minute} after the introduction of \ammonia, highlighting the appearance of a [110] facet.
        The surface is coloured by the values of $\epsilon_{zz}$ at each surface voxel, the limit of each facet is delimited by white tubes.
    }
    \label{fig:Amaterasu110}
\end{figure}

When introducing \dioxygen at \qty{600}{\degreeCelsius} to study the oxidation of ammonia, the nanoparticle was definitely lost during the measurement, not to be found again, which underlines the difficulty to study a highly exothermic reaction with BCDI.
Indeed, according to \cite{}, the reaction can heat the catalyst to very high temperatures which in our case could have been the trigger for the loss of the particle during the measurement, especially since the particle had so far resisted to the beam during the temperature ramp to \qty{600}{\degreeCelsius}.
From this first set of results, the measurement of Pt nanoparticles with BCDI during the oxidation of ammonia was decided to be carried out at lower temperatures, e.g. \qty{300}{\degreeCelsius} and \qty{450}{\degreeCelsius}.

\subsection{Strain evolution} \label{sec:StrainTempRamp}

\subsubsection{Determination of strain}

Lattice strain in diffraction is usually defined as the difference between the  reference and experimental lattice parameter values, respectively $a_{ref}$ and $a$ (eq. \ref{eq:StrainDiffraction}).

\begin{equation}
    \epsilon = \frac{a - a_{ref}}{a_{ref}}
    \label{eq:StrainDiffraction}
\end{equation}

The values of the lattice parameter can be extracted from the position $\vec{G}$ of the Bragg peaks in reciprocal space \textit{via} eq. \ref{eq:QandD3} and eq. \ref{eq:Interplanarspacing}.
However, the information extracted from the retrieved phase in BCDI is more complex since, by measuring three non-coplanar Bragg peaks, one may retrieve the full strain tensor (eq. \ref{eq:StrainTensor}) from the reconstructed displacement field $\vec{u}_{\hat{q_x}, \hat{q_y}, \hat{q_z}}$ (eq. \ref{eq:DisplacementField}), $\hat{q_x}, \hat{q_y}, \hat{q_z}$ being three orthogonal basis vector in the reciprocal space \parencite{Karpov2019}.

\begin{equation}
    \vec{u}_{\hat{q_x}, \hat{q_y}, \hat{q_z}} =
     \begin{pmatrix}
        \vec{u}_{\hat{q_x}} \\
        \vec{u}_{\hat{q_y}} \\
        \vec{u}_{\hat{q_z}} \\
     \end{pmatrix}
     \label{eq:DisplacementField}
\end{equation}

\begin{multicols}{3}
    \begin{equation}
        \epsilon =
        \begin{bmatrix}
            \epsilon_{xx} & \epsilon_{yx} & \epsilon_{zx}\\
            \epsilon_{xy} & \epsilon_{yy} & \epsilon_{zy}\\
            \epsilon_{xz} & \epsilon_{yz} & \epsilon_{zz}
        \end{bmatrix}
        \label{eq:StrainTensor}
    \end{equation}
    \break
    \begin{equation}
      \epsilon_{ij} = \frac{1}{2}
        \Bigg(
        \frac{\partial \vec{u}_{\hat{q_i}}}{\partial \hat{q_j}}
        +
        \frac{\partial \vec{u}_{\hat{q_j}}}{\partial \hat{q_i}}
        \Bigg)
        \label{eq:StrainTensorIJ}
    \end{equation}
    \break
    \begin{equation}
      \epsilon_{zz} =
        \Bigg(
        \frac{\partial \vec{u}_{\hat{q_z}}}{\partial \hat{q_z}}
        \Bigg)
        \label{eq:StrainTensorzz}
    \end{equation}
\end{multicols}

The idea behind using the strain tensor is to identify shear components in the displacement field, \textit{i.e.} see if components in one direction depend on other directions, which can also help to identify defects present in nanoparticles \parencite{Lauraux2021}.
Using a single Bragg peak, we can only retrieve one component of the displacement field, obtained from the division of the retrieved phase $\Phi$ of the scattered x-rays by the value of the scattering vector at the position of the Bragg peak (\textit{i.e.} $\vec{q} = \vec{G}$), following the assumptions for phase retrieval detailed in sec. \ref{sec:StrainBCDI}, eq. \ref{eq:FcrystalBCDI3} - \ref{eq:FcrystalBCDI7}.

In our case, the direction of the [111] scattering vector is perpendicular to the sample, along the $\vec{z}$ axis of the laboratory frame.
Therefore, the [111] scattering vector, $\vec{G}_{111}$, is so forth described as $\vec{q}_z$, of magnitude  $|\vec{q}_z|$, with the direction described by the unit vector  $\hat{q_z}$, to be consistent with the equations detailed above.
Our approach to the strain is considerably simplified since we can only correctly derive one component of the strain tensor, $\epsilon_{zz}$ (eq. \ref{eq:StrainTensorzz}) as detailed below in eq. \ref{eq:StrainFromPhase1} - \ref{eq:StrainFromPhase3}.

\begin{align}
    \label{eq:StrainFromPhase1}
    & \Phi =  \vec{q}_z.\vec{u} \\
    \label{eq:StrainFromPhase2}
    & \frac{\Phi}{|\vec{q}_z|} = \frac{\vec{q}_z}{|\vec{q}_z|}.\vec{u} = \hat{q_z}.\vec{u} = \vec{u}_{\hat{q_z}} \\
    \label{eq:StrainFromPhase3}
    & \vec{\nabla} \vec{u}_{\hat{q_z}} = \frac{\partial u_{\hat{q_z}}}{\partial z} = \epsilon_{zz}
\end{align}

\subsubsection{Homogeneous strain}

The deviation of the interplanar spacing from the room temperature value due to the thermal expansion of the crystal is expected to be homogeneous within the particle.
This isotropic \textit{homogeneous} strain ($\epsilon_{hmg}$) in the particle is removed by centering the Bragg peak before phase retrieval, otherwise resulting in a linear phase ramp \parencite{}.
It is of utmost important to study the evolution of the material first under an inert atmosphere to see if there is an evolution of the homogeneous strain not only due to the thermal expansion of the crystal, but also to global relaxation phenomena induced e.g. by the adsorption of molecules involved in the catalytic reaction.

% \begin{figure}[!htb]
%     \centering
%     \includegraphics[width=\textwidth]{/home/david/Documents/PhDScripts/SixS\_2021\_01/Max.png}
%     \caption{
%         The maximum value of the 3D Bragg peak is used to center the intensity before phase retrieval and to compute the average interplanar spacing $d_{111}$.
%     }
%     \label{fig:MaxPeak}
% \end{figure}

\begin{figure}[!htb]
    \centering
    \includegraphics[width=\textwidth]{/home/david/Documents/PhDScripts/SixS\_2021\_01/RockingCurves.pdf}
    \includegraphics[width=\textwidth]{/home/david/Documents/PhDScripts/SixS\_2021\_01/HomoStrain.pdf}
    \caption{
        a) Evolution of the integrated scattered intensity in the detector as a function of the scattering angle during the rocking scans, fitted (b) with a lorentzian profile to retrieve the peak positions and full width at half maxima (FWHM).
        Evolution of the $d_{111}$ interplanar spacing (c), and associated homogenous strain values as a function of the temperature.
        The reference for the computation of homogeneous strain is taken at \qty{25}{\degreeCelsius}.
        Evolution of the Bragg peak FWHM as a function of the temperature (d).
        \textcolor{Important}{Errorbars}
        %(111) Bragg peak angular position (c),
    }
    \label{fig:HomoStrain}
\end{figure}

The average interplanar spacing between (111) crystallographic planes in the particle was computed from the angular position of the Bragg peak \textit{via} eq. \ref{eq:QandD3} (fig. \ref{fig:HomoStrain}).
The interplanar spacing follows approximately two linear increases from \qtyrange{25}{150}{\degreeCelsius} before the defect removal and from \qtyrange{150}{600}{\degreeCelsius} after the defect removal, which had the effect of relaxing the particle, increasing the average interplanar spacing and decreasing the peak FWHM, signature of the decrease of heterogeneous strain inside the particle.
The particle was realigned after the removal of the defect to be sure of the absence of any angular offset, the flash heating process could have moved the sample positions due thermal expansion.
The measurement at \qty{550}{\degreeCelsius} shows a very large FWHM and could be the start of the appearance of a new defect at the interface with the substrate, imaged at \qty{600}{\degreeCelsius} with BCDI.
The measurement at \qty{550}{\degreeCelsius} could not be reconstructed.
At \qty{600}{\degreeCelsius}, the two values of the average interplanar spacing under inert atmosphere are only slightly higher or lower than the value at \qty{550}{\degreeCelsius}, and decrease after the introduction of ammonia.
The FWHM decreases approximately one hour after having reached \qty{600}{\degreeCelsius} (each data point is separated by \qty{\approx30}{\min}) from its value at \qty{550}{\degreeCelsius}, returning to the previous values observed between \qty{150}{\degreeCelsius} and \qty{450}{\degreeCelsius}.
The introduction of ammonia has no visible effect on the FWHM.
The decrease of the interplanar spacing between \qty{550}{\degreeCelsius} and \qty{600}{\degreeCelsius} occurs together with the decrease of the FWHM, as if an equilibrium state has been reached.\textcolor{Important}{Why}

\subsubsection{Heterogeneous strain}

The remaining strain after phase retrieval is called the \textit{heterogeneous} strain ($\epsilon_{htg}$) \parencite{GREDIAC1996, FAVIER2007, Atlan2023}, the total strain observed during this experiment being equal to $\epsilon_{tot} = \epsilon_{hmg} + \epsilon_{zz, htg}$ when considering the changes from a reference (usually room temperature data).

In BCDI, it is the displacement of small unit blocks making up the crystal lattice that is observed.
Depending on the instrumental parameters, these small unit blocks, called \textit{voxels}, have a more or less large size.
In this study, they are approximately \qtyproduct{10x10x10}{\nm} large.
The strain is thus not directly related to the deviation between the interreticular planes, but to the gradient of the displacement field of these unit blocks from their equilibrium position.
The outermost layers of the crystal only constitute a low percentage of the surface voxels (\qty{\approx 11}{\percent} if we consider 5 atomic layers separated by the value of the interplanar spacing $d_{111}$), which lowers the contribution of the surface strain to the total strain contained in the surface voxels, reducing the ability to properly resolve e.g. surface relaxation effects.
The use of padding during the reconstruction algorithm decreases the voxel size, but without relying on the sampling of high-frequency components of the scattering amplitude and therefore does not increase the strain resolution.

% \begin{figure}[!htb]
%     \centering
%     \includegraphics[width=0.6\textwidth]{/home/david/Documents/PhDScripts/SixS\_2021\_01/FacetAnalyser/Angles.pdf}
%     \caption{
%         Angles between the $[111]$ direction (normal to the top facet and opposite to the bottom facet of the particle) and other crystallographic directions that describe the normals to the other facets of the particle.
%     }
%     \label{fig:Angles}
% \end{figure}

If equivalent orientation translates into equal surface atomic structures (e.g. [111] and [$\bar{1}\bar{1}\bar{1}$]), the environment of equivalent facets is not always the same.
For example, it is important to differ between [1$\bar{1}$1] and [$\bar{1}$1$\bar{1}$] facets, the higher the value of the interplanar angle, the closer the facet is to the interface with the substrate, and thus the more its influence in prominent.
Moreover, if the [111] facet is at the top of the crystal and the furthest away from the substrate, the [$\bar{1}\bar{1}\bar{1}$] facet is expected to be fully in contact with the substrate.

Therefore, the mean value and standard deviation of the heterogeneous strain distribution on each facet is presented on fig. \ref{fig:AmaterasuStrain}, as a function of the angle between the facets normals and the [111] direction, to highlight this difference.
The crystallographic orientation of a facet can be determined by computing the angle between the [111] direction and its normal, which is then related to a direction in real space (fig. \ref{fig:StereoBottom}).
Facets with the same orientation (e.g. [100], [010], [001]) are grouped around the same value of the interplanar angle.
To be more precise, the angular value is computed as the average of the angle between the facet normal and the [111] direction plus the angle between the facet normal and the [$\bar{1}\bar{1}\bar{1}$] direction minus \ang{180}, \textit{i.e.} $(\angle (111, \vec{n}) + (\angle (\bar{1}\bar{1}\bar{1}, \vec{n}) -180))/2$.

\begin{figure}[!htb]
    \centering
    \includegraphics[width=\textwidth]{/home/david/Documents/PhDScripts/SixS\_2021\_01/FacetAnalyser/FacetStrainEvolution.pdf}
    \caption{
        Mean value and standard deviation of the heterogeneous strain ($\epsilon_{zz, htg}$) distribution as a function of the angle between the normal of each facet on the particle surface and the $[111]$ direction.
        Upwards and downwards arrow are represented for respectively positive and negative strain.
    }
    \label{fig:AmaterasuStrain}
\end{figure}

% introduce in plane out of plane
It is important to realize that the displacement observed is \textit{only} in the [111] direction, sensitive to deviations of the crystal structure perpendicular to the [111] and [$\bar{1}\bar{1}\bar{1}$] facets, but parallel to [$\bar{1}$10]-type facets, themselves perpendicular to the [111] direction (fig. \ref{fig:StereoBottom}).
On one hand, if there existed an out-of-plane displacement of the atoms on [$\bar{1}$10]-type facets, \textit{i.e.} in the direction perpendicular to the [$\bar{1}$10] planes, its contribution to the displacement field observed in this experiment would not be directly visible.
On the other hand, the in-plane displacement of the atoms on [$\bar{1}$10]-type facets is visible, whereas the in-plane displacement on the [111] and [$\bar{1}\bar{1}\bar{1}$] facets is not visible.

The $\epsilon_{zz}$ component of the strain tensor on the [111] and [$\bar{1}\bar{1}\bar{1}$] facets is easily assimilated to variation of the interplanar spacing $d_{111}$ (positive strain is tensile strain, negative strain is compressive).
However, the physical meaning of the strain becomes more complex when observing facets that are neither parallel nor perpendicular to the [111] direction, such as [$1\bar{1}1$]-type or \{100\} facets.
For this reason, the following analysis of the heterogeneous strain is meant to be qualitative, by comapring the strain evolution of the same facets at different conditions but not between different facets, a quantitative analysis could only be performed with the full strain tensor.
% The relation between in-plane and out-of-plane strain can be rationalized by the Poisson effect \parencite{Atlan2023}, which relates an in-plane tensile (compressive) strain to an out-of-plane compressive (tensile) strain (depending on the origin of the displacement field).

The values of the strain are in general quite low, always below \qty{0.1}{\percent}, while the highest value of the displacement field on the particle facets is of about \qty{1}{\angstrom} (app. \ref{fig:AmaterasuDisplacement}).
It is difficult to be certain of the maximum value of the displacement due to some difficulties when unwrapping the phase for some measurements.
Let's take the example of the particle at \qty{600}{\degreeCelsius} under Argon flow, the magnitude of the scattering vector at the position of the Bragg peak is $|\vec{G}| \qty{\approx 2.772}{\angstrom}$.
The magnitude of the scattering vector probed furthest from the Bragg peak during the measurement is along the [111] crystal truncation rod, $|\vec{q}| \qty{\approx 2.742}{\angstrom}$.
We have $\delta q = \qty{0.03}{\angstrom}$ which gives $(\vec{q}-\vec{G}).\vec{u} = 0.03 <<1$, which puts us far away from the BCDI limit discussed in sec. \ref{sec:StrainBCDI}.
\textcolor{Important}{Add a figure with orthogonalized data}

% Decribe removal of defect strain evolution 150°C -> 150 °C
The very large standard deviations seen in fig. \ref{fig:AmaterasuStrain} at \qty{150}{\degreeCelsius} for the [$\bar{1}\bar{1}\bar{1}$], [$\bar{1}1\bar{1}$], [$1\bar{1}\bar{1}$] and [$11\bar{1}$] facets can be explained by the presence of the defect, around which the phase was not well unwrapped as seen in fig. \ref{fig:AmaterasuDefect}.
Defects are also expected to be at the origin of regions with large strain deviations from their impact on the displacement field, as observed for example in the middle of the [$\bar{1}\bar{1}\bar{1}$] facet in fig. \ref{fig:AmaterasuB} at \qty{600}{\degreeCelsius}.
At \qty{150}{\degreeCelsius}, after removal of the defect, all of the facets show low strain values as well as low standard deviation.

% 250, % 350, % 450
The strain values are relatively low and stable at \qty{250}{\degreeCelsius}, \qty{350}{\degreeCelsius}, and \qty{450}{\degreeCelsius}.

% but when comparing to the measurement at \qty{150}{\degreeCelsius}, we can see in fig. \ref{fig:AmaterasuB} that the extremities of the [$\bar{1}\bar{1}\bar{1}$] facet are in tension, whereas the center is in compression.
% The opposite effect seems to take place around the [111] facet, visible both in fig \ref{fig:Amaterasu} and \ref{fig:AmaterasuStrain}.
% \textcolor{Important}{Why ?}

% 600°C
% Decribe strain change between 450°C under Argon and 600°C under ammonia
The most interesting changes in the facet strain occur at \qty{600}{\degreeCelsius}.
The appearance of a new defect at the interface with the substrate has the effect of \textit{splitting} the particle in two states, with negative strain for the facets that are the closest to the interface and positive strain for the facets that are on the top of the particle.
This effect is translated into respectively tensile and compressive strain on the [111] and [$\bar{1}\bar{1}\bar{1}$] facets, and is reversed by the introduction of ammonia at \qty{600}{\degreeCelsius}, visible also in fig. \ref{fig:AmaterasuB} and \ref{fig:Amaterasu110}.
It is also during the presence of ammonia in the reactor that a [2$\bar{1}\bar{1}$] facet appears for the first time, perpendicular to the [111] direction.
Both facets perpendicular to the [111] direction, [2$\bar{1}\bar{1}$] and [$\bar{1}01$] have low strain values, showing no displacement of the atoms parallel to their surfaces.

\begin{figure}[!htb]
    \centering
    \includegraphics[width=0.32\textwidth]{/home/david/Documents/PhDScripts/SixS\_2021\_01/paraview/1675_strain_and_facets1.png}
    \includegraphics[width=0.32\textwidth]{/home/david/Documents/PhDScripts/SixS\_2021\_01/paraview/1675_strain_and_facets2.png}
    \includegraphics[width=0.32\textwidth]{/home/david/Documents/PhDScripts/SixS\_2021\_01/paraview/1675_strain_and_facets3.png}
    \caption{
        View of the Amaterasu particle at \qty{600}{\degreeCelsius} after the introduction of ammonia.
        The three \{100\} facets are surrounded by either 4 \{111\} facets and one \{110\} facets (left), by 3 \{111\} facets and one \{110\} facets (middle) or by 4 \{111\} facets (right).
    }
    \label{fig:AmaterasuStrain1675}
\end{figure}

The alternating highly positive and negative strain regions in the middle of the [$\bar{1}\bar{1}\bar{1}$] facet (fig. \ref{fig:AmaterasuB}) is the signature of a dislocation network forming in the interface with the substrate \parencite{Dupraz2015}, also responsible for the high strain standard deviation of this facet (fig. \ref{fig:AmaterasuStrain}).
A higher resolution in a 3D displacement field together with simulations of the impact of defect at the interface are needed to fully characterized this network, which is outside the scope of this thesis.

On fig. \ref{fig:AmaterasuStrain1675} we can observe that the three \{100\}-type facets have the same strain values, none of which have the same environment (fig. \ref{fig:AmaterasuStrain1675}), while the edges and corners around the [111] and \{100\} facet show a slightly positive strain.

At first sight, all facets that have a similar orientation show a similar strain (e.g. [1$\bar{1}$1]-type, \{100\}-type).
However, the two \{111\} facets that are close to the substrate do not show the same strain than the three \{111\} facets closer to the top of the particle.
The strain being in the [111] direction with the origin of the displacement field in the center of the particle, if all of these facet had experienced the same compressive (tensile) strain perpendicular to their surface \textit{via} the adsorption of ammonia, they would all show a negative (positive) strain (of lower amplitude due to the angle between these facets and the [111] direction).

This behaviour could be linked to a strong effect of the support which has the dual effect of first preventing the [$\bar{1}\bar{1}\bar{1}$] facet from being exposed and secondly forcing them to accomodate the strain linked to the substrate.
There are three possibilities, first ammonia is effectively adsorbed on these facets but the influence of the substrate hides a possible common signature in the strain.
Secondly ammonia is not adsorbed and the strain difference is due to the influence of the substrate on the particle structure.
Thirdly, ammonia is adsorbed but only on the three top facets or only on the 2 bottom facets due to the influence of the substrate that either limitates or facilitates the adsorption process.

% Globally, the top of the particle seems to be in compression along $\vec{z}$ (fig. \ref{fig:Amaterasu}, \ref{fig:AmaterasuStrainSlices}), whereas the bottom of the particle is in tension, with the exception of the one [$1\bar{1}3$] facet (fig. \ref{fig:AmaterasuStrainSlices}) which is in compression.

% \begin{figure}[!htb]
%     \centering
%     \includegraphics[width=0.49\textwidth]{/home/david/Documents/PhDScripts/SixS\_2021\_01/paraview/1675_clipx.png}
%     \includegraphics[width=0.49\textwidth]{/home/david/Documents/PhDScripts/SixS\_2021\_01/paraview/1675_clipy.png}
%     \caption{
%         Particle slices perpendicular to the $\vec{x}$ and $\vec{y}$ directions at center of mass, a white line delimitates the particle surface.
%     }
%     \label{fig:AmaterasuStrainSlices}
% \end{figure}

Overall, it is difficult to conclude on any potential effect of the absorption of ammonia on the particle due to the presence of the dislocation network at the interface for which it is unclear whether or not it's origin if due to the adsorption of ammonia or from interfacial strain.
The fact that we only possess one component of the strain tensor, makes it difficult to quantify any relaxation effect on the facets that are not perpendicular to the [111] direction.
Bragg peaks from the substrate could have been measured to have a better idea of the interfacial structure.
Additionnal simulations of the particle equilibrium shape and the impact if interfacial defects are needed to have a better understanding of the structural dynamics at work.

% The shape of the particle in the ($\vec{x}, \vec{y}$) plane at \qty{600}{\degreeCelsius} can be roughly approximated as a hexagon with 3 sides occupied by the [$\bar{1}11$], [$1\bar{1}1$] and [$11\bar{1}$] facets.
% The three other sides are occupied by the [100], [010], [001] facets near the top of the particle, under which are situated either 1) one large [$\bar{1}10$] facet, 2) one small [$1\bar{1}0$] facet under which in turn are two smalls [113] and [$\bar{1}1\bar{1}$] facets, 3) a large [$1\bar{1}\bar{1}$] facet.

%%
% The negative strain of the [$1\bar{1}1$] facets and the positive strain on the [$\bar{1}1\bar{1}$] facets (fig. \ref{fig:AmaterasuFacetsEvolution}) follows the respective evolution of the [111] and [$\bar{1}\bar{1}\bar{1}$] facets with a lower amplitude.
% An increase of the negative strain can also be seen on all three [100] facets but, based on the angle between those facets and the [111] direction (\ang{54.7}), the interpretation of this value is difficult to understand.

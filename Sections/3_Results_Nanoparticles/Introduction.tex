\textcolor{red}{This Chapter should demonstrate that you have conducted a thorough and critical investigation of relevant sources.
Apart from a presentation of the sources of your data, this chapter allows you to critically discuss the data (whatever these data are, ‘quantitative’ or ‘qualitative’, primary or secondary), which is proof of good research. You can even do good research with poor data but you must demonstrate that you are aware of the data quality and accordingly are careful in your interpretations. Essentially, there are three aspects to consider:
\begin{enumerate}
\item	Reliability, which, for example, could depend on whether they are estimates or more direct evidence;
\item	Representativity, which is about how typical the data are; for example, you may have arguments why the very few cases are typical or you may carry out statistical tests;
\item Validity, which is about the relevance of the data for your case. Strictly speaking, sometimes no valid data are available but one may argue that there are other data which could be used as ‘proxies’.)
\end{enumerate}
}

\section{Introduction}

Bimetallic nanomaterials have raised more and more significant interest from worldwide researchers in recent years, their new physical and chemical properties deriving from synergistic effects between the two metals being highly desirable for catalytic applications \cite{}.
For example, bimetallic catalysts consisting either of two noble metals (Pt-Rh, Au-Pd) or a noble and a 3d/4d transition metal (e.g. Pt-Cu) can show tremendously higher catalytic activities and altered selectivity in both conventional thermal and electrochemical catalysis \parencite{Resta2020a, Carnis2021b}.

In the case of the oxidation of ammonia, if Pt catalysts were first used when aiming at the production of \nitricoxide, Pt-Rh catalysts have then proven to be more effective since the 1920s when Rhodium was first added to the catalyst \parencite{Handforth1934, Heck1982}.
Recently, the ammonia oxidation reaction over a Pt-Rh binary alloy was studied with a surface science approach by operando techniques such as near-ambient pressure X-ray photoemission spectroscopy (NAP-XPS) and surface x-ray diffraction (SXRD) combined with mass spectrometry \parencite{Resta2020a}.
It was shown that \nitricoxide production coincides with significant changes of the surface structure and the formation of a \ce{RhO_2} surface oxide.
Moreover, changes in the surface relaxation related to the catalyst selectivity as well as the presence of an hysteresis cycle in the reaction were reported.

Recent studies have shown that nanoparticle surface strain can be controlled, opening up a new path to tune and optimise nanoparticle catalysts \parencite{Zhang2014, Sneed2015, Wang2016}.
However, these more conventional techniques traditionally provide ensemble-averaged properties when studying nanoparticles.
With Bragg coherent diffraction imaging (BCDI), we intend to address under \textit{in situ} and \textit{operando} conditions the individual three-dimensional (3D) structural response of an isolated nanoparticle to changes in gas mixtures, enabling us to probe more deeply the details of catalytic phenomena \parencite{Fernandez2019, Passos2020, Dupraz2022}.

In that frame of mind, we aim at investigating in situ the catalytic structure-activity relationships of Pt nanoparticles during the oxidation of ammonia, as a stepping stone to the study of Pt-Rh nanoparticles.
A careful in situ analysis of the properties of the nanoparticles (size, shape, strain, refaceting, support, interaction, oxide formation, etc) in 3D is of essential importance to gain more understanding of the behaviour of these nanocrystals during a catalytic reaction.

\subsection{Synthetizing platinum nanoparticle}\label{sec:PtParticles}

The platinum nanoparticles were synthetized thanks to a collaboration with the Israel Institute of Technology (Technion) \parencite{Dupraz2017}.
A 30 nm thick homogeneous layer of platinum is deposited at room temperature on a (100) oriented alumina ($\alpha-$\ce{Al_2O_3}) substrate.
The Pt nanocrystals have their c-axis oriented along the [111] direction normal to the (0001) sapphire substrate.
A mask is then applied on the sample and a lithographic process route ensures that the platinum layer transform to nanoparticles that are between 100 and 1000 nm large,  epitaxied on the substrate surface.
After dewetting and heating at \qty{1100}{\degreeCelsius} for \qty{30}{\minute}, the platinum nanoparticles exhibit a well-faceted shape as seen in various BCDI measurements \parencite{Dupraz2017}.
\textcolor{Important}{Add more ref with these samples}
The room temperature lattice parameter of platinum (\qty{3.924}{\angstrom}) is close to that of (100) alumina (\qty{4.122}{\angstrom}) resulting in \qty{\approx 5}{\percent} in-plane lattice strain.\textcolor{Important}{Check}
The mask yields isolated nanoparticles in the middle of \qty{100}{\um} large squares (fig. \ref{fig:Mask}).
The position of the squares is designated with arabic numbers (row), letters (column) and roman numbers (large rectangle).
The hole diameter in the mask changes in different rectangle and has a direct impact on the nanoparticle size since more matter is deposited.

\begin{figure}[!htb]
    \centering
    \includegraphics[width=0.49\textwidth]{/home/david/Documents/PhD/Figures/sample/mask.png}
    \includegraphics[width=0.49\textwidth]{/home/david/Documents/PhD/Figures/sample/litho1.png}
    \caption{
        Mask applied during sample preparation (left) and resulting pattern on the sample surface (right).
    }
    \label{fig:Mask}
\end{figure}

All of the position indicators are constituted of platinum nanoparticle as well, which allows the scanning of the sample's surface in Bragg condition to map an area and find the nanoparticles' positions (fig. \ref{fig:SampleMapping}).

\subsection{Experimental setup for BCDI experiments in the vertical geometry}\label{sec:BCDISetup}

The BCDI experiment was performed at the SixS (Surface Interface X-ray Scattering) beamline of synchrotron SOLEIL, France (sec. \ref{sec:SIXS}).
As detailed in sec. \ref{sec:Gwaihir}, one of the bottlenecks of the BCDI technique is its slow data reduction and analysis process.
Moreover, on $3^{rd}$ generation synchrotrons that offer a lower coherent flux (eq. \ref{eq:CoherentFlux}) than $4^{th}$ generation synchrotrons, the measurement time can also be very long.
At SixS, a rocking curve lasts between \qtyrange{20}{90}{\min} depending on the particle size, the quality of the alignment, the strain of the particle, \textit{etc.}
Once the raw data is obtained, the particle must be \textit{reconstructed} (sec. \ref{sec:PhaseRetrieval}), and - if of interest - the displacement and strain arrays can be retrieved.
The analysis workflow can take up to an hour, which totals to an average of two hours from the start of the measurement to the moment when the user has a good idea of the sample shape and structure.

SixS is a beamline that does not only carry out BCDI experiment, but also SXRD experiments, in the same experimental end-station (sec. \ref{sec:MED}).
When aiming at performing \textit{operando} catalysis experiments, switching from one setup to another can take up to a few days.
This leaves only a limited remaining amount of time to align the sample, find a suitable nanoparticle, and carry out the experimental plan.

Li \textit{et al.} \parencite*{Li2020}, who have first shown that the SixS beamline could be used to carry out BCDI experiment, also started to work on improving the BCDI measurement process.
By comparing continuous and step-by-step measurements, they have shown that continous scanning would result in the same data quality while decreasing the measurement dead-time by \qty{30}{\percent}, thereby paving the way for quicker BCDI measurements.

During this thesis, the measurement process was further improved by taking advantage of the new possibility to perform continuous \textit{on-the-fly} scans at SixS, while moving the hexapod holding the sample.
When using the coherence setup (fig. \ref{fig:OpticalSetup}) in the vertical geometry (fig. \ref{fig:Diffractometer}), the beam is focused on the sample and is about \qty{1}{\um} large vertically, the horizontal footprint depending on the incident angle between the beam and the sample.
By satisfying Bragg's law in a specular geometry, the incoming angle is set to the Bragg angle $\theta$ (eq. \ref{eq:Bragglaw}).
The scattered x-rays are then collected by setting the out-of-plane detector angle $\gamma$ at a position $2\theta$ (similar to fig. \ref{fig:EwaldSphereSpecular}).
Finally, by simultaneously moving the sample with the hexapod and recording the Bragg scattered intensity with the detector, it is possible to map the sample surface with a sub-micron resolution (fig. \ref{fig:SampleMapping}).
A nanoparticle with a width equal to \qty{300}{\nm} was identified with this technique, which is a good estimate of the spatial resolution that can be attained, limited by the hexapod resolution (\qty{\approx 500}{\nm}) and the beam size.

Both in-plane angles are kept to zero, this is the simplest possible measurement geometry since the nanoparticles on the sample have their c-axis oriented along the [111] direction, parallel to the normal of the sample holder ($\vec{z}$ direction in the laboratory frame).

\begin{figure}[!htb]
    \centering
    \includegraphics[height=5cm]{/home/david/Documents/PhD/Figures/sample/microscope_image.png}
    \includegraphics[height=5cm]{/home/david/Documents/PhD/Figures/sample/microscope_image_photon.png}
    \caption{
        Microscope image of the sample seen through the sapphire window of the PEEK dome (left).
        Map of the sample performed in Bragg condition (right), the high intensity (red) areas correspond to platinum nanoparticles.
        The letters, numbers and isolated nanoparticles in the centre of squares can be recognized on the sample.
        \textcolor{Important}{colorbar}
    }
    \label{fig:SampleMapping}
\end{figure}

On the other hand, \textit{Gwaihir} (sec. \ref{sec:Gwaihir}) was developped primarily for the SixS beamline to counter the long analysis process, which allowed a significant reduction in the analysis time from around an hour to a few minutes.
The following beamtimes profit from the new software by having a more \textit{solution}-driven experimental process.
Indeed, to be measured, a nanoparticle must be isolated, not too small (weak scattered intensity), not too big (loss of coherence, fringes not visible), and not too initially strained (difficult to obtain a good guess of the support).
These conditions are sometimes difficult to assert by simply looking at the diffraction pattern.
Therefore, quick inversion using \textit{Gwaihir} allowed a faster decision process regarding the continuation or not of the nanoparticles measurement.

Successfull measurements by \cite{Lim2021} have permitted the simulatenous use of BCDI measurements from SixS with measurements from other imaging beamlines (ID01 - ESRF, P10 - DESY), designing a robust method to identify defects in the real space with convolutional neural networks (CNN).

In the frame of this thesis, the required beam size was obtained with a Fresnel zone-plate (focal distance of \qty{20}{\cm}), which focused the beam down to \qty{1}{\um} (horizontally) $\times$ \qty{2}{\um} (vertically).
A coherent portion of the beam was selected with high precision slits by matching their horizontal and vertical gaps with the transverse coherence lengths of the beamline: \qty{20}{\um} (horizontally) and \qty{100}{\um} (vertically).
\textcolor{Important}{check consistency}
A circular beam-stop, and a circular order-sorting aperture, were used to block the transmitted beam, and higher diffraction orders, respectively (fig. \ref{fig:OpticalSetup}).
The sample was mounted in the dedicated XCAT reactor with the substrate surface oriented in the vertical plane on a hexapod mounted vertically on the MED diffractometer (fig. \ref{fig:MEDV}).

The BCDI experiment was performed in a vertical specular geometry at a beam energy of \qty{8.5}{\keV} (wavelength of \qty{1.46}{\angstrom}).
Three-dimensional (3D) diffraction data were collected with rocking curves of the rotation angle around the normal of the sample, the diffracted beam was recorded with a 2D MAXIPIX photon-counting detector (\numproduct{515 x 515} square pixels, \qtyproduct{55 x 55}{\um} wide) positioned on the detector arm at a distance of \qty{1.22}{\meter}.
The in-plane ($\omega$) and out-of-plane ($\mu$) angles of the sample were \ang{0} and \ang{18.6}, respectively at \qty{25}{\degreeCelsius}, when the in-plane ($\delta$) and out-of-plane ($\gamma$) angles of the detector were \ang{0} and \ang{37.2}.
The value of the scattering angle $\mu$ ($\theta$ in eq. \ref{eq:Bragglaw}) and the out-of-plane detector angle ($2\theta$) are expected to vary as a function of the temperature during the experiment due to the thermal expansion of the sample.

The alignement of the beam was performed with the direct beam (all angles at \ang{0}) to ensure that the sample surface is parallel to the direct beam at $\mu=0$ when $\delta = \ang{0}$ and when $\delta = \ang{90}$.
First, to place the sample in the beam, its position was gradually increased in the direction perpendicular to the sample plane while recording the intensity of the direct beam.
The sample was then moved to the position at which the intensity of the direct beam was equal to half of its intensity without the sample in the beam path, so that when incresing the incidence angle to \ang{0.3}, the beam covers the entire sample surface.
Secondly, any possible tilt of the sample surface was corrected by recording the intensity of the \textit{reflected} beam as a function of the $u$ (when $\delta=\ang{0}$) or $v$ (when $\delta=\ang{90}$) angles (fig. \ref{fig:Diffractometer}).

\subsection{Catalysis reactor calibration}

The catalytic activity of the platinum nanoparticles as a function of the temperature was studied to make certain that the nanoparticles were sufficently catalytically active for the reaction products to be detected by the mass spectrometer.
The catalytic activity of the reactor without any sample was also monitored and proven to be nul (sec. \ref{sec:SXRD100}), to make certain that no reaction occurs without the sample.

The heater temperature was first calibrated by measuring the temperature in the reactor at different atmospheres, as a function of the current intensity (fig. \ref{fig:TempRamps} - a).
The experimental data points were then fit using a polynomial of degree four to set the reactor to any temperature from \qtyrange{0}{600}{\degreeCelsius} (fig. \ref{fig:TempRamps} - b) when working under vacuum or at ambient pressure (\qty{0.3}{\bar} or \qty{0.5}{\bar} of \argon).
The thermal conductivity of the gases involved in the oxidation of ammonia is in the similar order of magnitude (tab. \ref{tab:ThermalConductivity}).
Moreover, the same gas - Argon - is used as a carrier gas during the experiments, constituting at least \qty{80}{\percent} of the gas flow, and allowing us to assume that the temperature in the reaction chamber is well approximated.

\begin{table}[!htb]
\centering
    \begin{tabular}{@{}llllllll@{}}
    \toprule
     & \argon & \ammonia & \dioxygen & \nitricoxide & \nitrousoxide & \nitrogen & \water \\
    \midrule
    \qty{300}{\kelvin} & \num{17.7} & \num{25.1} & \num{26.5} & \num{25.9} & \num{17.4} & \num{26.0} & \num{18.6} \\
    \qty{400}{\kelvin} & \num{22.4} & \num{37.2} & \num{34.0} & \num{33.1} & \num{26.0} & \num{32.8} & \num{26.1} \\
    \qty{500}{\kelvin} & \num{26.5} & \num{53.1} & \num{41.0} & \num{39.6} & \num{34.1} & \num{39.0} & \num{35.6} \\
    \qty{600}{\kelvin} & \num{30.3} & \num{68.6} & \num{47.7} & \num{46.2} & \num{41.8} & \num{44.8} & \num{46.2} \\
    \bottomrule
    \end{tabular}%
\caption{Thermal conductivity in \unit{\mW \per \meter \per \kelvin} of gases \parencite{ThermalConductivityOfGases}.}
\label{tab:ThermalConductivity}
\end{table}
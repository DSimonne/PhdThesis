\section{Introduction}

Bimetallic nanomaterials have raised more and more significant interest from worldwide researchers, their new physical and chemical properties deriving from synergistic effects between the two metals being highly desirable for catalytic applications.
For example, bimetallic catalysts consisting either of two noble metals (Pt-Rh, Au-Pd) or a noble and a 3d/4d transition metal (e.g. Pt-Cu) can show tremendously higher catalytic activities and altered selectivity in both conventional thermal and electrochemical catalysis \parencite{Resta2020a, Carnis2021b}.

In the case of the oxidation of ammonia, if Pt catalysts were first used when aiming at the production of \ce{NO}, Pt-Rh catalysts have then proven to be more effective since the 1920s when Rhodium was first added to the catalyst \parencite{Handforth1934, Heck1982}.
Recently, the ammonia oxidation reaction over a Pt-Rh binary alloy was studied with a surface science approach by operando techniques such as near-ambient pressure X-ray photoemission spectroscopy (NAP-XPS) and surface x-ray diffraction (SXRD) combined with mass spectrometry \parencite{Resta2020a}.
It was shown that \ce{NO} production coincides with significant changes of the surface structure and the formation of a \ce{RhO_2} surface oxide.
Moreover, changes in the surface relaxation related to the catalyst selectivity as well as the presence of an hysteresis cycle in the reaction were reported.

Recent studies have shown that nanoparticle surface strain can be controlled, opening up a new path to tune and optimise nanoparticle catalysts \parencite{Zhang2014, Sneed2015, Wang2016}.
However, these more conventional techniques traditionally provide ensemble-averaged properties when studying nanoparticles.
With Bragg coherent diffraction imaging (BCDI), we intend to address under \textit{in situ} and \textit{operando} conditions the individual three-dimensional (3D) structural response of an isolated nanoparticle to changes in gas mixtures, enabling us to probe more deeply the details of catalytic phenomena \parencite{Fernandez2019, Passos2020, Dupraz2022}.
Specifically, the catalytic structure-activity relationships of Pt nanoparticles will be investigated during the oxidation of ammonia, as a stepping stone to the study of Pt-Rh nanoparticles.
A careful in situ analysis of the properties of the nanoparticles (size, shape, strain, re-faceting, support, interaction, oxide formation, etc) in 3D is of essential importance to gain more understanding of the behaviour of these nanocrystals during a catalytic reaction.

\subsection{Synthesising platinum nanoparticle}\label{sec:PtParticles}

The platinum nanoparticles were synthesised thanks to a collaboration with the Israel Institute of Technology (Technion) \parencite{Dupraz2017}.
A 30 nm thick homogeneous layer of platinum is deposited at room temperature on a [001] oriented alumina ($\alpha-$\ce{Al_2O_3}) substrate.
The Pt nanocrystals have their c-axis oriented along the [111] direction normal to the [001] sapphire substrate.
A mask is then applied on the sample and a lithographic process route ensures that the platinum layer transforms to nanoparticles that are between 100 and 1000 nm large, epitaxied on the substrate surface.
The application of a patterned mask during the lithographic process yields a patterned sample with isolated nanoparticles in the middle of \qty{100}{\um} large squares, visible in fig. \ref{fig:Mask}.

The position of the squares is designated with Arabic numbers (row), letters (column) and roman numbers (large rectangle).
All of the position indicators are constituted of platinum nanoparticle as well, which allows the mapping of the sample's surface in Bragg condition to find the nanoparticles' positions (fig. \ref{fig:SampleMapping}).
The hole diameter in the mask changes in different areas of the sample and has a direct impact on the average nanoparticle size.
% The epitaxy relation between the nanoparticle and the substrate was explored in this section with in-plane diffraction measurements that show an average in-plane strain of \qty{1}{\percent}.
After dewetting and heating at \qty{1100}{\degreeCelsius} for \qty{30}{\minute}, the platinum nanoparticles exhibit a well-faceted Winterbottom shape observed in various BCDI measurements \parencite{Dupraz2017}.
% \textcolor{Important}{Add more ref with these samples, detail better, give good strain value}

\begin{figure}[!htb]
    \centering
    \includegraphics[width=0.49\textwidth]{/home/david/Documents/PhD/Figures/sample/mask.png}
    \includegraphics[width=0.49\textwidth]{/home/david/Documents/PhD/Figures/sample/litho1.png}
    \caption{
        Mask applied during sample preparation (left) and resulting pattern on the sample surface (right).
    }
    \label{fig:Mask}
\end{figure}

\subsection{Catalysis reactor calibration}

The catalytic activity of the platinum nanoparticles was first studied as a function of the temperature to (i) be certain that the sample is sufficiently catalytically active for the reaction products to be detected by the mass spectrometer, (ii) identify the catalyst light-off temperature and (iii) observe an evolution of the product selectivity as a function of the temperature.
Similar scans where also performed on the empty reactor/sample holder to ensure the absence of activity towards the oxidation of ammonia.

The heater temperature was first calibrated by measuring the temperature in the reactor at different pressures by the means of a type K thermocouple, as a function of the current intensity (fig. \ref{fig:TempRamps} - a), \qty{10}{\minute} separate consecutive datapoints to ensure the heater stability.
The experimental data points were then fit using a polynomial of degree four to set the reactor to any temperature from \qtyrange{25}{600}{\degreeCelsius} when working under vacuum or at ambient pressure (\qty{0.3}{\bar} or \qty{0.5}{\bar} of \ce{Ar}).
The thermal conductivity of the gases involved in the oxidation of ammonia are in the similar order of magnitude (tab. \ref{tab:ThermalConductivity}).
Moreover, the same inert gas used to set the reactor pressure - Argon - is used as a carrier gas during the experiments, constituting at least \qty{80}{\percent} of the gas flow, and allowing us to assume that the temperature in the reaction chamber is well approximated.

\begin{table}[!htb]
\centering
    \begin{tabular}{@{}llllllll@{}}
    \toprule
     & \ce{Ar} & \ce{NH_3} & \ce{O_2} & \ce{NO} & \ce{N_2O} & \ce{N_2}& \ce{H_2O} \\
    \midrule
    \qty{300}{\kelvin} & \num{17.7} & \num{25.1} & \num{26.5} & \num{25.9} & \num{17.4} & \num{26.0} & \num{18.6} \\
    \qty{400}{\kelvin} & \num{22.4} & \num{37.2} & \num{34.0} & \num{33.1} & \num{26.0} & \num{32.8} & \num{26.1} \\
    \qty{500}{\kelvin} & \num{26.5} & \num{53.1} & \num{41.0} & \num{39.6} & \num{34.1} & \num{39.0} & \num{35.6} \\
    \qty{600}{\kelvin} & \num{30.3} & \num{68.6} & \num{47.7} & \num{46.2} & \num{41.8} & \num{44.8} & \num{46.2} \\
    \bottomrule
    \end{tabular}%
\caption{Thermal conductivity in \unit{\mW \per \meter \per \kelvin} of gases \parencite{ThermalConductivityOfGases}.}
\label{tab:ThermalConductivity}
\end{table}

A temperature ramp at a reactor pressure of \qty{0.3}{\bar} (fig. \ref{fig:TempRamps} - b) was carried out to probe the evolution of the reaction products as a function of the temperature under reacting conditions, \textit{i.e.} a constant gas flow of \qty{41}{\ml\per\min} of \ce{Ar}, \qty{8}{\ml\per\min} of \ce{O_2}, and \qty{1}{\ml\per\min} of \ce{NH_3}.

\begin{figure}[!htb]
    \centering
    \includegraphics[width=0.495\textwidth]{/home/david/Documents/PhDScripts/SixS_2022_01_SXRD_Pt100/gas_analysis/figures/ThermocoupleCalibration.pdf}
    \includegraphics[width=0.495\textwidth]{/home/david/Documents/PhDScripts/Test_Reactor_CO2_2021_01/Figures/TempRamp2.pdf}
    \caption{
        a) Temperature inside the reactor cell measured with a type C thermocouple under vacuum and different \ce{Ar} pressures.
        b) Partial pressures evolution under a constant gas flow (\qty{41}{\ml\per\min} of \ce{Ar}, \qty{8}{\ml\per\min} of \ce{O_2}, \qty{1}{\ml\per\min} of \ce{NH_3}) at a reactor pressure of \qty{0.3}{\bar} during increasing and decreasing (low transparency) temperature ramp to \qty{650}{\degreeCelsius} with 100 steps, each lasting \qty{10}{\second}.
        The partial pressure of oxygen is omitted for simplicity.
    }
    \label{fig:TempRamps}
\end{figure}

The catalyst light off temperatures for the production of \ce{N_2}, \ce{NO} and \ce{N_2O} can respectively be identified to be around \qty{300}{\degreeCelsius}, \qty{450}{\degreeCelsius}, and \qty{550}{\degreeCelsius}.
The excess of oxygen compared to ammonia increases the production of \ce{NO} at high temperatures, with a partial pressure similar to the partial pressure of nitrogen at \qty{650}{\degreeCelsius}.
No decrease of the nitrogen partial pressure can yet be detected as a function of the increasing temperature.
The production of nitrous oxide stays a low values during the temperature ramp, almost imperceptible from the background pressure.

According to these primary results, the study of the oxidation of ammonia with BCDI was originally decided to be carried out at \qtylist{300;500;600}{\degreeCelsius}, temperatures before and after the catalyst light off (tab. \ref{tab:ConditionsNanoparticles}).

\begin{table}[!htb]
\centering
\resizebox{\textwidth}{!}{%
    \begin{tabular}{@{}lll|l|lll|l@{}}
    \toprule
    \multicolumn{3}{l|}{Gas Flow (\unit{\ml\per\min})} & Total pressure (\unit{\milli\bar}) & \multicolumn{3}{l|}{Partial pressures (\unit{\milli\bar})} & Targeted information \\
    \midrule
    \midrule
    \ce{Ar} & \ce{O_2} & \ce{NH_3} &  & \ce{Ar} & \ce{O_2} & \ce{NH_3} &  \\
    \midrule
    \midrule
    50 & 0 & 0 & 500 & 500 & 0 & 0 & Catalyst state without reactants (unactive) \\
    49 & 0 & 1 & 500 & 490 & 0 & 10 & Ammonia adsorption \\
    48.5 & 0.5 & 1 & 500 & 485 & 5 & 10 &  \multirow{4}{*}{\begin{tabular}[c]{@{}l@{}}Influence of (\ce{NH_3}/\ce{O_2}) ratio on the \\ catalyst selectivity and structure \end{tabular}}\\
    48 & 1 & 1 & 500 & 480 & 10 & 10 &  \\
    47 & 2 & 1 & 500 & 470 & 20 & 10 &  \\
    41 & 8 & 1 & 500 & 410 & 80 & 10 & \\
    50 & 0 & 0 & 500 & 500 & 0 & 0 & Returning to pristine state \\
    \bottomrule
    \end{tabular}%
}
\caption{Different atmospheres used to probe the ammonia oxidation on Pt nanoparticles with BCDI.}
\label{tab:ConditionsNanoparticles}
\end{table}
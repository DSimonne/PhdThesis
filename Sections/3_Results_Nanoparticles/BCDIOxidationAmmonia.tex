\section{Measuring the oxidation of ammonia}\label{sec:BCDIAmmoniaOxidation}

% Lattice parameter - homogeneous strain evolution: OK
% 3D plots: OK
% strain field energy: OK
% Peak width :kinda OK
% Field data tables : prob too many so no
% Facet strain evolution plots just like temp ramp: (no disp, useless unless seen in bulk)
% Strain histogram ? useless we want to see the evolution of specific facets, good id not too many facets, also too many scans
% need to do smtg with edges and corners, plot as a fct of the coordination nb ?
% explain how exactly a single voxel was found when getting the surface voxels in python

After having shown that the Pt particles epitaxied on alumina are stable under different atmospheres at \qty{300}{\degreeCelsius}, \qty{500}{\degreeCelsius} and \qty{600}{\degreeCelsius}, the oxidation of ammonia was measured using BCDI starting at \qty{300}{\degreeCelsius}.

The same sample, experimental setup, type of measurements and phase retrieval algorithms were used as detailed in the previous section for the temperature ramp.
Two well faceted nanoparticles were successfully measured and reconstructed at room temperature (fig. \ref{fig:IsanagiSusanooFacets}).
Isanagi exhibits a round shape with many low index facets, \qty{40}{\percent} of its surface covered by \{113\} facets, \qty{40}{\percent} by \{111\} facets, \qty{40}{\percent} by \{110\} facets, and \qty{40}{\percent} by \{100\} facets.
The particle is at the largest \qty{300}{\nm} wide.
On the contrary, Susanoo exhibits a more rectangular shape without any \{113\} facets, \qty{40}{\percent} of its surface is covered by \{111\} facets, \qty{40}{\percent} by \{110\} facets, and \qty{40}{\percent} by \{100\} facets.
The particle is at the largest \qty{800}{\nm} wide.

Together with the Amaterasu nanoparticle, we have measured three nanoparticles that have a different size, shape and facets exhibited on the surface.
The smallest nanoparticle (Isanagi) is the one with the most \{113\} facets, whereas the largest (Susanoo) does not have any.
Amaterasu which is about \qty{600}{\nm} wide does only exhibits small \{113\} facets depending on the temperature.
Overall, the sample is covered with thousands of nanoparticles that all together contribute to the catalytic activity and that probably show an even greater variance of shape and size.
Therefore, this analysis using BCDI can only aim to reveal the structure variation during a catalytic reaction of a few nanoparticles without statistically representing the population on the sample, the measurement process being far too time extensive to probe more than a few nanoparticles.

\textcolor{Important}{Need to say that we did this in a experiment}
proven to be able to bring a more

Three scans were recorded under each condition (tab. \ref{tab:ConditionsNanoparticles}) at \qty{300}{\degreeCelsius} and \qty{400}{\degreeCelsius} for each nanoparticle to probe for any evolution of the particles surface structure during a fixed atmosphere as a function of time, and to otherwise demonstrate the reproducibility of the measurements.

\begin{figure}[!htb]
    \centering
    \includegraphics[width=0.49\textwidth]{/home/david/Documents/PhD/Figures/bcdi_data/B7/B7_facets.png}
    \includegraphics[width=0.49\textwidth]{/home/david/Documents/PhD/Figures/bcdi_data/D6/D6_facets.png}
    \caption{
        3D view of the Isanagi (left) and Susanoo (right) particles measured at room temperature showing a highly faceted surface.
    }
    \label{fig:IsanagiSusanooFacets}
\end{figure}

The 3D diffraction patterns were orthogonalised from the laboratory frame ($\vec{z}$ downstream, $\vec{y}$ vertical up, $\vec{x}$ outboard) to the $\hat{q_x}, \hat{q_y}, \hat{q_z}$ following the cxi convention ($\vec{z}$ perpendicular to the sample holder, $\vec{y}$ parallel to the beam direction and $\vec{x}$ in the sample plane).

\begin{figure}[!htb]
    \centering
    \includegraphics[width=\textwidth]{/home/david/Documents/PhD/Figures/bcdi_data/D6/DP_not_ortho.png}
    \includegraphics[width=\textwidth]{/home/david/Documents/PhD/Figures/bcdi_data/D6/DP_ortho.png}
    \caption{
        Sum of the diffracted intensity in the $\vec{x}$, $\vec{y}$ and $\vec{z}$ directions of the laboratory frame before (top) and after orthogonalisation (bottom) in the $\hat{q_x}, \hat{q_y}, \hat{q_z}$ directions.
    }
    \label{fig:IsanagiOrtho}
\end{figure}

\begin{figure}[!htb]
    \centering
    \includegraphics[width=\textwidth]{/home/david/Documents/PhD/Figures/bcdi_data/B7/DP_not_ortho.png}
    \includegraphics[width=\textwidth]{/home/david/Documents/PhD/Figures/bcdi_data/B7/DP_ortho.png}
    \caption{
        Sum of the diffracted intensity in the $\vec{x}$, $\vec{y}$ and $\vec{z}$ directions of the laboratory frame before (top) and after orthogonalisation (bottom) in the $\hat{q_x}, \hat{q_y}, \hat{q_z}$ directions.
    }
    \label{fig:SusanooOrtho}
\end{figure}

A 3D view of the reconstructed nanoparticles is presented in \ref{sec:3DAmmoniaOxidation}.

After having successfully reconstructed the Pt nanoparticles, the facets at their surface were retrieved using the \textit{FacetAnalyzer} script written for \textit{Paraview} (sec. \ref{sec:FacetAnalysis}).
No surface

\subsection{Mass spectrometry results}

\begin{figure}[!htb]
    \centering
    \includegraphics[width=0.49\textwidth]{/home/david/Documents/PhDScripts/SixS_2021_06_BCDI_NH3/figures/homo_strain_vs_condition.png}
    \includegraphics[width=0.49\textwidth]{/home/david/Documents/PhDScripts/SixS_2021_06_BCDI_NH3/figures/homo_strain_vs_condition_no_25.png}
    \caption{
        Evolution of the homogeneous strain computed from the 3D Bragg peak centre of mass.
    }
    \label{fig:AmaterasuStrainSlices}
\end{figure}

\subsection{Strain field energy}


\parencite{Ellinger2008, Nolte2008, Gustafson2014, Hejral2016, Abuin2019, Kawaguchi2019, Hejral2021, Kim2021, Chung2021}
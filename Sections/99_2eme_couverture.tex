%%%%%%%%%%%%%%%%%%%%%%%%%%%%%%%%%%%%%%%%%%%%%%%%%%%%%%%%%%%%%%%
% \Ifthispageodd{\newpage\thispagestyle{empty}\null\newpage}{}
\thispagestyle{empty}
% \newgeometry{top=1.5cm, bottom=1.25cm, left=2cm, right=2cm}
% \fontfamily{rm}\selectfont

\lhead{}
\rhead{}
\rfoot{}
\cfoot{}
\lfoot{}

\noindent 
%*****************************************************
%***** LOGO DE L'ED À CHANGER IMPÉRATIVEMENT *********
%*****************************************************
\includegraphics[height=2.4cm]{logo/logo_usp_PIF.png}
\vspace{0.5cm}
%*****************************************************
%\fontfamily{cmss}\fontseries{m}\selectfont

\small

\begin{mdframed}[linecolor=Prune,linewidth=1]

\textbf{Titre:} Propriétés catalytiques à l’échelle nanométrique sondées par diffraction des rayons X de surface et imagerie de diffraction cohérente

\noindent \textbf{Mots clés:} Diffraction de rayons X, Catalyse hétérogène, Surface, Structure cristalline, Déformation, Oxydation de l'ammoniac

\vspace{-.5cm}
\begin{multicols}{2}
\noindent \textbf{Résumé:}
Le principal objectif de ce travail est d'étudier des catalyseurs hétérogènes \textit{in situ} et \textit{operando} pendant l'oxydation de l'ammoniac en se rapprochant des valeurs de température et pression industrielles.
Actuellement, ce processus catalytique et les changements structurels associés sont mal compris, et nous proposons d'utiliser différents échantillons en platine, nanoparticules et monocristaux afin de réduire l'écart entre les études scientifiques sur échantillons modèles et les catalyseurs utilisés en agro-industrie.
L'activité catalytique des différents échantillons est mesurée pour lier structure et sélectivité durant la réaction, qui peut être focalisée vers la production d'azote (\ce{N_2}) ou d'oxyde nitrique (\ce{NO}).
La production de protoxyde d'azote (\ce{N_2O}) doit être évitée de part son importante contribution à l'effet de serre.
Le développement d'une catalyse hétérogène avec une sélectivité ciblant les \qty{100}{\percent} est un défi constant, ainsi que la compréhension de la durabilité, du vieillissement et de la désactivation du catalyseur lui-même.
Mesurer la structure de nanoparticules à l'échelle nanométrique permet de révéler les effets de volume, de tension et de compression de surface et d'interface, ainsi que l'existence de différents types de défauts.
En complément des études d'imagerie par diffraction cohérente de rayons X en condition de Bragg sur des nanoparticules individuelles, l'étude d'un ensemble de nanoparticules sera effectuée \textit{via} la diffraction des rayons X à incidence rasante.
La diffraction cohérente de rayons X en condition de Bragg étant une nouvelle technique, une organisation typique de la réduction et analyse des données est proposée.
De plus, chaque type de surface présente sur les nanoparticules (e.g. (111), (100)) est ensuite étudiée à l'aide de monocristaux par diffraction des rayons X en surface et spectroscopie photoélectronique par rayons X.
De ce fait, la structure de surface ainsi que la présence d'espèces adsorbées peut être reliée à l'activité catalytique mesurée, permettant une meilleure compréhension du mécanisme de réaction.
Finalement, l'évolution de la structure d'ensemble et de monocristaux est comparée à celle des nanoparticules uniques pour confirmer/infirmer le lien entre structure moyenne et structure individuelle.

\end{multicols}

\end{mdframed}

\newpage
\thispagestyle{empty}
% \newgeometry{top=1.5cm, bottom=1.25cm, left=2cm, right=2cm}
% \fontfamily{rm}\selectfont

\lhead{}
\rhead{}
\rfoot{}
\cfoot{}
\lfoot{}

\noindent
%*****************************************************
%***** LOGO DE L'ED À CHANGER IMPÉRATIVEMENT *********
%*****************************************************
\includegraphics[height=2.4cm]{logo/logo_usp_PIF.png}
\vspace{0.5cm}
%*****************************************************
%\fontfamily{cmss}\fontseries{m}\selectfont

\small

\begin{mdframed}[linecolor=Prune,linewidth=1]

\textbf{Title:} Catalytic properties at the nanoscale probed by surface X-ray diffraction and coherent diffraction imaging

\noindent \textbf{Keywords:} X-ray diffraction, Heterogeneous catalysis, Surface, Crystal structure, Strain, Ammonia oxidation

\begin{multicols}{2}
\noindent \textbf{Abstract:}
The main objective of this work is to study heterogeneous catalysts \textit{in situ} and \textit{operando} during the oxidation of ammonia by approaching industrial temperature and pressure values.
Currently, this catalytic process and the associated structural changes are poorly understood.
We propose to use different samples in platinum, nanoparticles and single crystals in order to reduce the gap between scientific studies on model samples and catalysts used in the fertiliser industry.
The catalytic activity of the different samples is measured to link structure and selectivity during the reaction, which can be focused towards the production of nitrogen (\ce{N_2}) or nitric oxide (\ce{NO}).
The production of nitrous oxide (\ce{N_2O}) must be avoided due to its significant contribution to the greenhouse effect.
Developing heterogeneous catalysis with selectivity targeting \qty{100}{\percent} is an ongoing challenge, as is understanding the durability, ageing, and deactivation of the catalyst itself.
Measuring the structure of nanoparticles at the nanoscale makes it possible to reveal the effects of volume, surface and interface tension and compression, as well as the existence of different types of defects.
In addition to imaging studies by Bragg coherent X-ray diffraction imaging on individual nanoparticles, the study of a set of nanoparticles will be carried out \textit{via} grazing incidence X-ray diffraction.
Bragg coherent X-ray diffraction imaging being a new technique, a typical workflow for data reduction and analysis is proposed.
In addition, each type of surface present on the nanoparticles (e.g. (111), (100)) is then studied using single crystals by surface X-ray diffraction and X-ray photoelectron spectroscopy.
Therefore, the surface structure as well as the presence of adsorbed species can be linked to the measured catalytic activity, allowing a better understanding of the reaction mechanism.
Finally, the evolution of the nanoparticle ensemble structure and of single crystals is compared to that of single nanoparticles to confirm/refute the link between average structure and individual structure.
\end{multicols}
\end{mdframed}

\normalsize

%************************************
\vspace{\fill} % ALIGNER EN BAS DE PAGE
%************************************

\newpage\thispagestyle{empty}\null\newpage
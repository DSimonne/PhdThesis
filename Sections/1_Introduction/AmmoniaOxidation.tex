% \textcolor{red}{(The Introduction chapter should contain background information as appropriate, plus definitions of all special and general terms. Your topic should be: clearly stated and defined; have a clear overall purpose; and have clear, relevant and coherent aims and objectives. It is also informative to give a brief description of the contents of the remaining chapters of the thesis. This alerts the reader and prepares them for the rest of the thesis.)}

The use of catalysts has several advantages, including faster, selective, and more energy-efficient chemical reactions, enabling them to direct reactions towards producing higher amounts of the desired product while reducing unwanted byproducts \parencite{Schlogl2015}.
Over the years, scientists have developed specialized catalysts for various real-world applications, today 90\% of chemical processes involve catalysts in at least one of their steps \parencite{WEINER1998915, DeVries2012}.
Notable advancements in catalysis have led to the production of biodegradable plastics, novel pharmaceuticals, and eco-friendly fuels and fertilizers \parencite{FECHETE20122}.

Several new challenges have emerged in the field of catalysis related to improving efficiency, reducing environmental impact, and developing sustainable processes.
First, environmental challenges concern minimizing and/or managing by-products, reducing contamination in effluents/wastewaters, and using sustainable sources of raw materials \parencite{LUDWIG2017313, Lange2021} and energy supplies.
Secondly, economical challenges which imply using cheaper, readily available raw materials, increased productivity, and decreased lag-time between discovery to commercialization \parencite{Keisuke2019, Gunay2021}.
For example, recent studies suggest that alternative, more economical catalysts, such as non-noble metals \parencite{Zhong2021} and other derived metal-based compounds, need to be tested as possible substitutes for the most frequently used noble metals, which are very efficient but expensive.

\begin{figure}[!htb]
    \centering
    \includegraphics[width=\textwidth]{/home/david/Documents/PhD/Figures/ammonia/ParisNO2.png}
    \caption{
        $NO_2$ levels in Paris are on average twice superior to the annual limit of $40 \, \mu g / m^3$ \parencite{AirParis}.
    }
    \label{fig:NO2Paris}
\end{figure}

Catalysis also has a role to play to combat pollution and create cleaner energy with for example the development of efficient water-splitting technologies \parencite{AHMAD2015599}, and enhancing the use of biomass and other energy vectors such as ammonia \parencite{Fang2022}.
Finally, challenges also arise in automotive exhaust where catalysts participate in the reduction of the emissions of toxis gases and particles \parencite{WHOAirPollution, GANDHI2003433}.
Some of the major air pollutants such as nitrogen oxides, ($NO_x$), and particulate matter (PM) are emitted by road traffic (65\% of $NO_x$, $\approx 35\%$ of PM), mainly by diesel vehicles, and directly inhaled by nearby major city inhanbitants.
To set a striking example, in Paris in 2018, 700 000 inhabitants were exposed to $NO_2$ concentrations exceeding the regulations (fig. \ref{fig:NO2Paris}), 60 000 inhabitants for PM$_{10}$, and all Parisians were concerned by exceeding the WHO recommendations for PM$_{2.5}$ \parencite{AirParis}.
Air quality is the main environmental concern of Ile-de-France residents (65\% of total mentions) ahead of climate change (63\%) and food (38\%) \parencite{AirParis}.

\section{The oxidation of Ammonia}

The story of ammonia begins in the early 20th century with the discovery of the Ostwald process in 1902 that permitted the synthesis of fertilizers from ammonia.
Wilhelm Ostwald later received the Nobel price in 1909 \textit{"in recognition of his work on catalysis and for his investigations into the fundamental principles governing chemical equilibria and rates of reaction"}.
Seven years later, in 1909, Fritz Haber designed a process for the synthesis of ammonia which was later improved by Carl Bosch.
It is today known as the Haber-Bosch process and is at the origin of the mass production of ammonia using metallic catalysts.
First Haber received the Nobel prize in 1918 \textit{"for the synthesis of ammonia from its elements"} \parencite{Alexander1920} and Carl Bosch in 1931 \textit{“in recognition of their contributions to the invention and development of chemical high pressure methods”}.

\begin{figure}
\centering
    \begin{tikzpicture}
        \node (image) [anchor=south west, inner sep=0pt] {\includegraphics[width=0.95\textwidth]{/home/david/Documents/PhD/Presentations/Slides/PhdSlides/Figures/worldindata/world-population-with-and-without-fertilizer.png}};
        \begin{scope}[x={(image.south east)}, y={(image.north west)}]
            \node [text width=4cm,align=right] at (0.18, 0.7) (OP) {\textcolor{Ostwald}{Ostwald} process\\(1902)\\ \textrightarrow Nobel prize\\(1909)};
            \draw [-latex, ultra thick, Ostwald] (0.10,0.75) to (0.10,0.12);
            \node [text width=4cm,align=right] at (0.19, 0.45) (HBP) {\textcolor{Haber}{Haber-Bosch}\\process (1908)\\ \textrightarrow Nobel prize\\(1918)};
            \draw [-latex, ultra thick, Haber] (0.13,0.4) to (0.13,0.12);
        \end{scope}
    \end{tikzpicture}
    \caption{
    Since the discovery of the Ostwald and Haber-Bosch process that allowed the mass production of nitrogen-based fertilizers, the world population increase has been relying on their production and use for agriculture.
    Today, about 50\% of the worlds population relies on nitrogen-based fertilizer to produce the food necessary to their alimentation.
    Taken from \cite{WorldDataFertilizer}.
    }
    \label{fig:FertilizerWID}
\end{figure}

The oxidation of ammonia can be described by three stochiometric equations that, depending on the stoechiometric ratio between \ammonia and \dioxygen, have different products.

\begin{align}
    \label{eq:AmmoniaOxidationNitrogen}
    4 \ammonia (g) + 3 \dioxygen (g) & \rightarrow 6 \water (g) + 2 \nitrogen (g) \\
    \label{eq:AmmoniaOxidationNitrousOxide}
    4 \ammonia (g) + 4 \dioxygen (g) & \rightarrow 6 \water (g) + 2 \nitrousoxide (g) \\
    \label{eq:AmmoniaOxidationNitricOxide}
    4 \ammonia (g) + 5 \dioxygen (g) & \rightarrow 6 \water (g) + 4 \nitricoxide (g)
\end{align}

The first equation (eq. \ref{eq:AmmoniaOxidationNitrogen}) yields nitrogen (\nitrogen), a naturally occuring gas that does not pollute the environment nor shows any toxic behavious towards humans (tab. \ref{tab:NitrogenGases}).
The second equation (eq. \ref{eq:AmmoniaOxidationNitrousOxide}) yields nitrous oxide (\nitrousoxide), a powerful greenhouse effect gas (tab. \ref{tab:NitrogenGases}).
For an equal amount of \nitrousoxide and \carbondioxide, the amount of \nitrousoxide will trap 298 (298 Ceq) times more heat than the amount of \carbondioxide over the next 100 years \parencite{MITCLIMATE}.
The third equation (eq. \ref{eq:AmmoniaOxidationNitricOxide}) yields nitric oxide, also called nitrogen monoxide (\nitricoxide), which is the main interest for the production of nitrogen-based fertilizers (tab. \ref{tab:NitrogenGases}) via the Ostwald process.

\begin{align}
    \label{eq:SideReactions1}
    4 \ammonia (g) + 6 \nitricoxide (g) & \rightarrow 5 \nitrogen (g) + 6 \water (g)\\
    \label{eq:SideReactions2}
    2 \nitricoxide (g) & \rightarrow \nitrogen (g) + \dioxygen (g)
\end{align}

Known side reactions to the oxidation of ammonia are first the recombination of nitric oxide with unreacted ammonia that leads to the production of water and nitrogen (eq. \ref{eq:SideReactions1}), and secondly the thermal decomposition of nitric oxide (eq. \ref{eq:SideReactions2}), that both lower the total yield of the reaction when aiming at the production of nitric oxide.
Reactions \ref{eq:AmmoniaOxidation1}, \ref{eq:AmmoniaOxidation2}, \ref{eq:AmmoniaOxidation3} and \ref{eq:SideReactions1} are strongly exothermic.
At low temperature or in the absence of catalyst, the production if \nitrogen is favoured \parencite{Hatscher2008}.

The second (eq. \ref{eq:Stage2}) and third (eq. \ref{eq:Stage2}) stages of the Ostwald process stem from \nitricoxide.
It is therefore important to achieve important selectivity towards the production of \nitricoxide when aiming at the production of fertilizers.
Since the 1930s, the presence of platinum-rhodium ($\approx 10 \% Rh$) knitted gauzes catalyst and favourable reaction conditions (900 °C - 5 bar - excess \dioxygen) allowed a 98 \% yield to be achieved \parencite{Handforth1934, Heck1982}.
The reaction pathways has been proven to follow a step-by-step decomposition of ammonia on the catalyst surface by stripping the hydrogen atoms of adsorbed ammonia by dissociated surface oxygen atoms, which confirms the existence of a heterogenous catalytic process \parencite{Bradley1995,PEREZRAMIREZ2004}.
The final pathway towards the production of \nitrogen, \nitrousoxide or \nitricoxide depends on the oxygen density and binding on the surface.
However the exact mechanism of action is not yet exactly understood.
The role of metal-oxides has been underlined by several studies, after a few hours of catalytic reaction, the catalyst undergoes an \textit{activation process} which leads to an increase selectivity towards \nitricoxide while the gauzes undergo a transformation towards a roughened surface with the presence of \ce{PtO_2} and \ce{Rh_2O_3}.
The rough surface has an increased area compared to the smooth surface and is responsible for the formation of metallic oxides in step and pits, \ce{PtO_2} increases the reaction rate while \ce{Rh_2O_3} leads towards the deactivation of the catalyst.
The volatile platinum oxide is considered to be primary responsible for mass losses and the progressive deactivation of the catalyst due to the ever increasing presence of \ce{Rh_2O_3} \parencite{McCabe1986}.


\begin{align}
    \label{eq:Stage2}
    2 \nitricoxide (g) + \dioxygen (g) & \rightarrow \nitrogendioxide (g) \\
    3 \nitrogendioxide (g) + \water (l) & \rightarrow 2\nitricacid (aq) + \nitricoxide (g)
\end{align}

\begin{align}
    \label{eq:Stage3}
    \nitricacid + \ammonia & \rightarrow \ammoniumnitrate
\end{align}

Nevertheless, it can also be interresting to be able to remove \ammonia, a colourless gas with a pungent odour, from the atmosphere since it irritates the eyes, nose, throat, and respiratory tract if inhaled in small amounts due to its corrosive nature and is poisonous in large quantities.
It also pollutes and contributes to the eutrophication and acidification of terrestrial and aquatic ecosystems (cite), and forms secondary particulate matter (PM2.5) when combined with other pollutants in the atmosphere (cite).
In that case, the reaction must be tuned towards the production of \nitrogen which is the only non polluant and toxic gas.

Today, the ammonia oxidation is an essential catalytic reaction used in the production of artificial fertilizers and in environmental applications.
In both cases, particular focus is on two products of the reaction, namely, \nitricoxide and \nitrogen.
The selectivity toward either one is dictated by reaction parameters, that is, by temperature, \ammonia and \dioxygen partial pressures, and the type of catalyst.

Excessive nitrogen can cause environmental issues.
The most important contribution to the nitrous oxide in the atmosphere comes from growing crops using nitrogen-based fertilizers (fig. \ref{fig:PieGreenhouseNO2}).
Nitrogen-based fertilizers and other human activities can lead to nitrogen runoff into water bodies, contributing to eutrophication (excessive growth of algae) and causing harm to aquatic ecosystems.
Moreover, approximately half of the production of ammonia is lost to the environment \parencite{ERISMAN2007}

Excessive use of fertilizers can contribute to pollution, and one of the consequences is the formation of nitrous oxide (N2O), a potent greenhouse gas. Fertilizers can be either ammonium- or nitrate-based, and when plants don't fully absorb all the nutrients, a series of microbe-mediated transformations occur. These processes lead to the release of nitrogen back into the atmosphere, primarily as nitrogen gas (N2) and, to a lesser extent, as N2O. Nitrous oxide is known for its strong impact on climate change as a greenhouse gas.

\begin{figure}[!htb]
    \centering
    \includegraphics[width=\textwidth]{/home/david/Documents/PhD/Presentations/Slides/PhdSlides/Figures/ammonia/NO2pie.pdf}
    \caption{Pie charts underlining the contribution of nitrogen-based fertilizers to the greenhouse gases in the USA.
    Adapted from \cite{EPAGreenhouseGases}.}
    \label{fig:PieGreenhouseNO2}
\end{figure}

\begin{table}[]
\centering
\resizebox{\textwidth}{!}{%
    \begin{tabular}{@{}l|l|lll|l|l|ll@{}}
    \toprule
    Formula & \ammonia & \nitrogen & \nitrousoxide & \nitricoxide & \nitrogendioxide & \nitricacid & \ammoniumnitrate & \urea \\
    \midrule
    Name & Ammonia & Nitrogen & \begin{tabular}[c]{@{}l@{}}Nitrous oxide,\\ Laughing gas\end{tabular} & \begin{tabular}[c]{@{}l@{}}Nitrogen oxide,\\ Nitric oxide\\ Nitrogen monoxide\end{tabular} & Nitrogen dioxide & Nitric acid & \begin{tabular}[c]{@{}l@{}}Ammonium\\ nitrate\end{tabular} & Urea \\
    Origin & \begin{tabular}[c]{@{}l@{}}Haber-Bosch\\ process\end{tabular} & \begin{tabular}[c]{@{}l@{}}Naturally present\\ in the atmosphere,\\ Ammonia \\ oxidation,\\ Selective catalytic\\ reaction (SCR)\end{tabular} & \begin{tabular}[c]{@{}l@{}}Ammonia \\ oxidation,\\ Emissions from\\ nitrogen-based\\ fertilizers\end{tabular} & \begin{tabular}[c]{@{}l@{}}Ammonia oxidation,\\ Anthropogenic sources \\ (combustion process, \\ industry, agriculture, ...)\\ Naturally produced \\ from lightning or\\ volcanoes\end{tabular} & \begin{tabular}[c]{@{}l@{}}Ostwald process (step 1)\\ Anthropogenic sources \\ (combustion process, \\ industry, agriculture, ...)\\ Naturally produced \\ from lightning or \\ volcanoes\end{tabular} & \begin{tabular}[c]{@{}l@{}}Ostwald process\\ (step 2)\end{tabular} & \begin{tabular}[c]{@{}l@{}}Nitric acid\\ neutralization \\ with ammonia\end{tabular} & Ammonia \\
    Major use & \begin{tabular}[c]{@{}l@{}}Ostwald process\\ (fertilizers),\\ Direct use in soil,\\ Fuel,\\ Hydrogen \\ carrier,\\ Cooling in liquid\\  form\end{tabular} & \begin{tabular}[c]{@{}l@{}}Ammonia \\ production\end{tabular} & \begin{tabular}[c]{@{}l@{}}Medicine,\\ Propellant\end{tabular} & Production of nitric acid & Production of nitric acid & \begin{tabular}[c]{@{}l@{}}Fertilizer production,\\ Nitration (explosives,\\ dyes, ...),\\ Propellant,\\ Etching\end{tabular} & \begin{tabular}[c]{@{}l@{}}Fertilizing \\ agent,\\ Explosives\end{tabular} & \begin{tabular}[c]{@{}l@{}}Fertilizing \\ agent,\\ SCR to reduce\\ NOx into N2\end{tabular} \\
    Toxicity & \begin{tabular}[c]{@{}l@{}}Dangerous for\\ the environment\\ Toxic, \\ Corrosive\end{tabular} & \begin{tabular}[c]{@{}l@{}}Asphyxiation by\\ displacing O2\end{tabular} & \begin{tabular}[c]{@{}l@{}}Anesthetic, \\ euphoric\end{tabular} & \begin{tabular}[c]{@{}l@{}}Oxidizing, Corrosive,\\ Toxic\end{tabular} & \begin{tabular}[c]{@{}l@{}}Oxidizing, Corrosive,\\ Toxic, Health hazard\end{tabular} & Oxidizing, Corrosive &  &  \\
    Pollution & PM formation &  & \begin{tabular}[c]{@{}l@{}}Ozone\\ depletion\end{tabular} & \begin{tabular}[c]{@{}l@{}}Smog, acid rains, \\ ozone depletion,\\ Precursor to NO2 in\\ the atmosphere\end{tabular} & \begin{tabular}[c]{@{}l@{}}Smog, acid rains, \\ ozone formation\end{tabular} & \begin{tabular}[c]{@{}l@{}}Decomposes towards\\ NO2\end{tabular} & \begin{tabular}[c]{@{}l@{}}Decomposes \\ into NO2 when\\ used as fertilizer,\\ Eutrophication\end{tabular} & \begin{tabular}[c]{@{}l@{}}Decomposes \\ into NH3 when\\ used as fertilizer,\\ Eutrophication\end{tabular} \\
    \begin{tabular}[c]{@{}l@{}}Greenhouse \\ effect\end{tabular} &  &  & \begin{tabular}[c]{@{}l@{}}Very important\\ (298 CO2 eq.)\end{tabular} &  &  &  & \begin{tabular}[c]{@{}l@{}}Nitrogen-based\\ fertilizer release\\ N2O in the \\ atmosphere\end{tabular} & \begin{tabular}[c]{@{}l@{}}Nitrogen-based\\ fertilizer release\\ N2O in the \\ atmosphere\end{tabular} \\ \bottomrule
    \end{tabular}%
    }
    \caption{
        Nitrogen based molecules that are involved in the oxidation of ammonia, the Haber-Bosch process, the Ostwald process or nitrogen-based fertilizers.
        Information compiled from various sources\parencite{Thiemann2000, HARRISON2001, Baerns2005, Imbihl2007, Hatscher2008, Davidson2009, Resta2020a,Borodin2021,Pottbacker2022}
    }
    \label{tab:NitrogenGases}
\end{table}

\section{From industry to model catalysis}

"A long standing conundrum in the catalysis community emerged at the interface between surface science and heterogeneous catalysis, better known as the pressure and materials gap."

Nature Catalysis editorial, 2018.

The most efficient catalyst for the oxidation of ammonia is a Pt-Rh alloy.


This reaction is considered as a classic example of a strongly exothermic, heterogeneous, catalytic reaction [4].
Due to the very fast kinetics of oxidation reactions, a direct experimental investigation of several reaction steps is difficult at realistic conditions.

nitrous oxide is a powerful oxidiser similar to molecular oxygen.

Selectivity to nitrous oxide at low temperature was reported in the following order: $Pt > Pd > Ni > Fe > W > Ti $[6].

N2O selectivity for us ?

NH3 oxidation requires surface sites for the adsorption of two ammonia molecules and two oxygen atoms.

Furthermore, the importance of availability of oxygen vacant sites near N-containing adspecies was demonstrated by the decrease of the reaction rate when a surface oxide was formed [31–33].
Finally, adsorbed oxygen did not block the ammonia adsorption [12].
All these facts led to the conclusion that a dual-site mechanism is operative.
A similar conclusion on the reaction mechanism was made for ammonia oxidation on a supported ruthenium catalyst [34].

has been well optimized in industry reaching up to 99\% efficiency towards the production of nitrous oxide,

Depending on the application of the ammonia oxidation, the catalytic reaction must be tuned towards a specific product.
This \textit{selectivity} is controlled by the reaction temperature, pressure, the  reactant ratio, and the type of catalyst


However, there has been a gap between the size of the nanoparticles studied for heterogeneous catalysis by computer methods (a few nm - linked to computational limits) and the size of the particles studied with synchrotron techniques ($>100 nm$) for a simple reason which is that in both diffraction and spectroscopy techniques, the outcoming photon or electron flux is proportionnal to the sample volume (e.g. eq. \ref{eq:ScatteredIntensity}).

In the frame of this thesis, the difference in activity between different crystalline facets for the oxidation of ammonia will be studied with different samples.
BCDI allows the 3D exploration of a single faceted nanoparticle which comes with a compromise in the strain resolution, whereas SXRD yields information about specific facets since the samples are large monocrystals.

\begin{table}[!htb]
    \centering
    \begin{tabular}{l|l|l|l}
    \toprule
                & Pressure    & Material       &     Temperature \\
    \midrule
    Industry {\color{DarkOrange}[[Insert references]]}  & 1-12 bar & Wires (diameter $\approx 80 \, \mu m$) & \textgreater 1000 K \\
    \midrule
    Literature {\color{DarkOrange}[[Insert references]]} & UHV, mbar & Single crystals & RT - 1000 K \\ \midrule
    This study & Near ambient    & Single crystals  & $\approx$ 750 K \\
               & pressure (0.5 bar)  & and nanoparticles & \\
    \bottomrule
    \end{tabular}
    \caption{Material and pressure gap in heterogenous catalysis.}
    \label{tab:gap}
\end{table}

\begin{figure}[!htb]
    \centering
    \includegraphics[height=4cm]{/home/david/Documents/PhD/Presentations/Slides/PhdSlides/Figures/sample/pt_gazes.png}
    \includegraphics[trim=0 100 0 75, clip, height=4cm]{/home/david/Documents/PhD/Presentations/Slides/PhdSlides/Figures/bcdi_data/B7/B7_facets.png}
    \includegraphics[height=4cm]{/home/david/Documents/PhD/Presentations/Slides/PhdSlides/Figures/sample/sxrd_sample.png}
    \caption{Platinum gazes used in industry (left), Reconstructed phase of a Pt particle measured at SixS, its diameter is of about 300 nm (middle), Pt 111 single crystal used in SXRD and XPS experiments, its diameter is of about 8 mm (right).}
\end{figure}

\section{Aim and Scope}

The oxidation of ammonia is a catalytic reaction that has had an extremely high impact on the 19th and 20th century, being at the origin of dramatic changes in the world demography with the fertilizer industry, and being today part of numerous industrial process that not only contribute to the fast climate change with greenhouse gases, but also to the ever growing pollution of our ecosystems.

In this first chapter, the importance of the oxidation of ammonia has been underlined.
It was shown that despite being a major catalytic process in a multi-billion industry, the exact mechanisms of action are not yet understood.
If in the frame of this thesis, the focus will be set on the sole oxidation of ammonia that evidently, the goal is also to develop experimental methods that can speak to and attract the synchrotron neophyte to the study of heterogeneous catalysis with synchrotron radiation, together with efficient and reliable computer methods that allow good data reduction and analysis.

Indeed, if it is of prime interest for a scientist to always be able to understand the fundamentals and to be ready to answer to the question \textit{why ?}, one must also realize that the impact one has strongly depends on the application and reach of his work.
Keeping this idea in mind, this thesis aims first at bridging the material and pressure gap to bring forth the possibility for synchrotron users to study catalytic reactions at conditions tending to 'real' industry conditions.
Secondly, a focus has also been set on explaining the origin, advantages, drawbacks and workflows of each technique, especially for Bragg coherent diffraction imaging, a technique that has yet to reach its full potential through the development of 4th generation synchrotrons and powerful computing clusters.

\section{Outline of the Thesis}

The initial chapter of this thesis will provide a concise explanation of heterogeneous catalysis and if the fundamental principles governing the interaction between x-rays and matter.
This will serve to underscore the origins, benefits, and limitations of each technique employed in the study of the ammonia oxidation.
Additionally, it will explore how these catalytic reactions can be indirectly observed using x-rays, leveraging the unique signatures they leave on the materials involved.
The SixS synchrotron beamline, which served as the primary location for conducting the majority of experiments, will be presented.
Notably, the latest advancements in experimental techniques specific to this beamline will be covered.
Finally, the latest computer programs used for the analysis of the collected data and the challenges of data analysis will be presented so that the reader of this thesis has complete knowledge of every step undertaken to obtain the final results.

In a second chapter, the results obtained with three different techniques, surface x-ray diffraction, Bragg coherent diffration imaging and x-ray photoelectron spectroscopy, will be presented.
Each step will be discussed, sample preparation, the choice of experimental conditions, the reproducibility of the results, the quality of the final data as well as the different difficulties encountered during data collection.

Finally, in the last chapter of this thesis will be discussed quantitatively and qualitatively the relation between the result of each technique, their comparison to literature findings, as well as their reliability, representativity and validity, paving the way for any future study.
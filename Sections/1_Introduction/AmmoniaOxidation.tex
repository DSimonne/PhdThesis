\textcolor{red}{(The Introduction chapter should contain background information as appropriate, plus definitions of all special and general terms. Your topic should be: clearly stated and defined; have a clear overall purpose; and have clear, relevant and coherent aims and objectives. It is also informative to give a brief description of the contents of the remaining chapters of the thesis. This alerts the reader and prepares them for the rest of the thesis.)}

\section{The oxidation of Ammonia}

Ammonia oxidation is an essential catalytic reaction used in the production of artificial fertilizers and in environmental applications. In both cases, particular focus is on two products of the reaction, namely, \nitricoxide and \nitrogen. The selectivity toward either one is dictated by reaction parameters, that is, by temperature, \ammonia and \dioxygen partial pressures, and the type of catalyst.

Detail and literature about the oxidation of Ammonia on Platinum nano-catalyst can be found here \parencite{Resta2020a}.

\subsection{From industry to model catalysis}

\subsection{Crystal structures}

\subsection{Pt 111}
\subsection{Pt 100}
\subsection{Nanoparticles}
\lipsum

\section{Aim and Scope}


\lipsum


\section{Outline of the Thesis}


\lipsum


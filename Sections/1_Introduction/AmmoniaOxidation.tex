\textcolor{red}{(The Introduction chapter should contain background information as appropriate, plus definitions of all special and general terms. Your topic should be: clearly stated and defined; have a clear overall purpose; and have clear, relevant and coherent aims and objectives. It is also informative to give a brief description of the contents of the remaining chapters of the thesis. This alerts the reader and prepares them for the rest of the thesis.)}

\section{The oxidation of Ammonia}

Ammonia oxidation is an essential catalytic reaction used in the production of artificial fertilizers and in environmental applications. In both cases, particular focus is on two products of the reaction, namely, \nitricoxide and \nitrogen. The selectivity toward either one is dictated by reaction parameters, that is, by temperature, \ammonia and \dioxygen partial pressures, and the type of catalyst.

Detail and literature about the oxidation of Ammonia on Platinum nano-catalyst can be found here \parencite{Resta2020a}.

\subsection{From industry to model catalysis}

"A long standing conundrum in the catalysis community emerged at the interface between surface science and heterogeneous catalysis, better known as the pressure and materials gap."

Nature Catalysis editorial, 2018.

\begin{table}[!htb]
    \centering
    \begin{tabular}{l|l|l|l}
    \toprule
                & Pressure    & Material       &     Temperature \\
    \midrule
    Industry {\color{DarkOrange}[[Insert references]]}  & 1-12 bar & Wires (diameter $\approx 80 \, \mu m$) & \textgreater 1000 K \\
    \midrule
    Literature {\color{DarkOrange}[[Insert references]]} & UHV, mbar & Single crystals & RT - 1000 K \\ \midrule
    This study & Near ambient    & Single crystals  & $\approx$ 750 K \\
               & pressure (0.5 bar)  & and nanoparticles & \\
    \bottomrule
    \end{tabular}
    \caption{Material and pressure gap in heterogenous catalysis. {\color{DarkOrange}[[I prefer the word diameter and not the symbol]]}}
    \label{tab:gap}
\end{table}

\begin{figure}[!htb]
    \centering
    \includegraphics[height=3cm]{/home/david/Documents/PhD/Presentations/Slides/PhdSlides/Figures/sample/pt_gazes.png}
    \includegraphics[height=3cm]{/home/david/Documents/PhD/Presentations/Slides/PhdSlides/Figures/bcdi_data/B7/B7_facets.png}
    \includegraphics[height=3cm]{/home/david/Documents/PhD/Presentations/Slides/PhdSlides/Figures/sample/sxrd_sample.png}
    \caption{Platinum gazes used in industry (left), {\color{DarkOrange}Reconstructed phase of a} Pt particle measured at SixS, {\color{DarkOrange}its diameter is} of about 300 nm (middle), Pt 111 single crystal used in SXRD and XPS experiments, {\color{DarkOrange}its diameter} is of about 8 mm (right).}
\end{figure}

\subsection{Crystal structures}

\subsection{Pt 111}

\subsection{Pt 100}

\subsection{Nanoparticles}

\section{Aim and Scope}

\section{Outline of the Thesis}

In the first chapter of this thesis we will come back to the basics behind each X-ray technique that was used to study the catalyst, as well as define the exact meaning of heterogeneous catalysis and how

In the second chapter we will present and discuss the results that we obtained with three different techniques

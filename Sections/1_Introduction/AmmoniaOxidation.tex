The use of catalysts has several advantages such as faster, selective, and more energy-efficient chemical reactions, directed towards producing higher amounts of the desired product while reducing undesired byproducts.
Reactions impossible without catalysts can also be enabled.
Over the years, scientists have developed specialised catalysts for various applications, today \qty{90}{\percent} of chemical processes involve catalysts in at least one of their steps \parencite{Weiner1998, DeVries2012}.
Notable advancements in catalysis have led to the production of plastics, pharmaceuticals, fuels and fertilisers \parencite{Fechete2012}.

\begin{figure}[!htb]
    \centering
    \includegraphics[width=\textwidth]{/home/david/Documents/PhD/Figures/ammonia/ParisNO2English.png}
    \caption{
        \ce{NO2} levels in Paris near the main traffic roads remained on average twice superior to the annual limit of \qty{40}{\ug \per \m^3} \parencite{AirParis} between 2012 and 2018, despite a global decrease since 2012.
    }
    \label{fig:NO2Paris}
\end{figure}

Several new challenges have emerged in the field of catalysis related to reducing environmental impact, and developing sustainable processes.
First, environmental challenges concern minimising and/or managing by-products, reducing contamination in effluents/wastewaters, using sustainable sources of raw materials, and energy supplies \parencite{Ludwig2017, Lange2021}.
Secondly, economical challenges imply using cheaper, readily available raw materials, increased productivity, and decreased lag-time between discovery to commercialisation \parencite{Keisuke2019, Gunay2021}.
For example, recent studies suggest that alternative, more economical catalysts, such as non-noble metals, and other derived metal-based compounds, must be tested as possible substitutes for the most frequently used noble metals, very efficient but expensive \parencite{Zhong2021, Ruan2022}.
For green catalysis, the best material it not only the cheapest but most importantly the least impactful on the environment, e.g. in terms of material extraction, life span, and selectivity towards reducing pollutants and greenhouses gases \parencite{Lange2021}.

Catalysis also has a role to play to combat pollution and create cleaner energy with for example the development of efficient water-splitting technologies \parencite{Ahmad2015}, and enhancing the use of biomass and other energy vectors such as ammonia \parencite{Fang2022}.
Challenges also arise in automotive exhaust where catalysts participate in the reduction of the emissions of toxic gases and nanoparticles \parencite{WHOAirPollution, Heck2001, Gandhi2003}.
Some of the major air pollutants such as nitrogen oxides, (\ce{NO_x}), and particulate matter (PM) are emitted by road traffic (\qty{65}{\percent} of \ce{NO_x}, \qty{\approx 35}{\percent} of PM), mainly by diesel vehicles, and directly inhaled by nearby major city inhabitants.
To set a striking example, in Paris in 2018, 700 000 inhabitants were exposed to \ce{NO_2} concentrations exceeding the regulations (fig. \ref{fig:NO2Paris}), 60 000 inhabitants for PM$_{10}$, and all Parisians were concerned by exceeding the World Health Organisation (WHO) recommendations for PM$_{2.5}$ \parencite{AirParis}.
Air quality is the main environmental concern of Île-de-France residents (\qty{65}{\percent} of total mentions) ahead of climate change (\qty{63}{\percent}) and food (\qty{38}{\percent} - \cite{AirParis}).
Diesel engines are also used in the heavy industry impacting the surrounding areas in terms of atmosphere quality.

\section{The oxidation of ammonia}

\subsection{The Haber-Bosch and Ostwald processes}

\epigraph{Today, about \qty{50}{\percent} of the world population relies on nitrogen-based fertilisers to produce the food necessary to their alimentation.}%{\textit{Nature Catalysis editorial, \cite*{NatureEditorial2018}.}}

The story of ammonia begins in the early $20^{th}$ century with the discovery in 1902 of the Ostwald process that permitted the synthesis of nitric acid from the oxidation of ammonia \parencite{Ostwald1902, Ostwald1902a}.
Wilhelm Ostwald later received the Nobel price in 1909 \textit{"in recognition of his work on catalysis and for his investigations into the fundamental principles governing chemical equilibria and rates of reaction"}.

Seven years after the discovery, in 1909, Fritz Haber designed a process for the synthesis of ammonia which was later improved by Carl Bosch.
It is today known as the Haber-Bosch process, and is at the origin of the mass production of ammonia using metallic catalysts \parencite{Hosmer1917, Parsons1919}.
Fritz Haber received the Nobel prize in 1918 \textit{"for the synthesis of ammonia from its elements"} \parencite{Alexander1920} and Carl Bosch in 1931 \textit{“in recognition of his contributions to the invention and development of chemical high pressure methods”}.

Since the discovery of the Ostwald and Haber-Bosch processes that allowed the mass production of nitrogen-based fertilisers, the world population increase has been relying on their production and use for agriculture \parencite{Erisman2008}.
Today, about \qty{50}{\percent} of the world population relies on nitrogen-based fertilisers to produce the food necessary to their alimentation (fig. \ref{fig:FertilizerWID}).

\begin{figure}[!htb]
\centering
    \begin{tikzpicture}
        \node (image) [anchor=south west, inner sep=0pt] {\includegraphics[width=0.95\textwidth]{/home/david/Documents/PhD/Presentations/Slides/PhdSlides/Figures/WorldInData/world-population-with-and-without-fertilizer.png}};
        \begin{scope}[x={(image.south east)}, y={(image.north west)}]
            \node [text width=4cm,align=right] at (0.18, 0.7) (OP) {\textcolor{Ostwald}{Ostwald} process\\(1902)\\ \textrightarrow Nobel prize\\(1909)};
            \draw [-latex, ultra thick, Ostwald] (0.10,0.75) to (0.10,0.12);
            \node [text width=4cm,align=right] at (0.19, 0.45) (HBP) {\textcolor{Haber}{Haber-Bosch}\\process (1908)\\ \textrightarrow Nobel prize\\(1918-1931)};
            \draw [-latex, ultra thick, Haber] (0.13,0.4) to (0.13,0.12);
        \end{scope}
    \end{tikzpicture}
    \caption{
    Figure adapted from Our World In Data \parencite{WorldDataFertilizer}.
    }
    \label{fig:FertilizerWID}
\end{figure}

The oxidation of ammonia (\ce{NH_3}) can be described by three equations that, depending on the stoechiometric ratio between \ce{NH_3} and \ce{O_2}, have different products.

\begin{align}
    \label{eq:AmmoniaOxidationNitrogen}
    4 \ammonia + 3 \ce{O_2} & \rightarrow 6 \water + 2 \nitrogen \\
    \label{eq:AmmoniaOxidationNitrousOxide}
    4 \ammonia + 4 \ce{O_2} & \rightarrow 6 \water + 2 \nitrousoxide \\
    \label{eq:AmmoniaOxidationNitricOxide}
    4 \ammonia + 5 \ce{O_2} & \rightarrow 6 \water + 4 \nitricoxide
\end{align}

The first equation (eq. \ref{eq:AmmoniaOxidationNitrogen}) yields nitrogen (\nitrogen), a naturally occurring gas that does not pollute the environment nor shows any toxic behaviour towards humans.
The second equation (eq. \ref{eq:AmmoniaOxidationNitrousOxide}) yields nitrous oxide (\nitrousoxide), a powerful greenhouse effect gas, and thus an often unwanted by-product.
The third equation (eq. \ref{eq:AmmoniaOxidationNitricOxide}) yields nitric oxide, also called nitrogen monoxide (\nitricoxide), which is the main desired product for the subsequent production of nitric acid (\nitricacid) with the Ostwald process.
The characteristics of the many different gases linked to the oxidation of ammonia are recapitulated in tab. \ref{tab:NitrogenGases}.

\begin{align}
    \label{eq:SideReactions1}
    2 \ammonia + 8 \nitricoxide & \rightarrow 5 \nitrousoxide + 3 \water\\
    \label{eq:SideReactions2}
    4 \ammonia + 6 \nitricoxide & \rightarrow 5 \nitrogen + 6 \water\\
    \label{eq:SideReactions3}
    2 \nitricoxide & \rightarrow \nitrogen + \ce{O_2}\\
    \label{eq:SideReactions4}
    2 \ammonia + 3 \nitrousoxide & \rightarrow 4 \nitrogen + 3 \water\\
    \label{eq:SideReactions5}
    2 \ammonia & \rightarrow \nitrogen + 3 \ce{H_2}
\end{align}

Known side reactions to the oxidation of ammonia are the recombination of nitric oxide (\ce{NO}) with unreacted ammonia that leads to the production of water, nitrogen, or nitrous oxide (eq. \ref{eq:SideReactions1} - \ref{eq:SideReactions2}).
The thermal decomposition of nitric oxide (eq. \ref{eq:SideReactions3}), also lowers the total yield of the reaction when aiming at the production of nitric oxide.

\begin{landscape}
\begin{table}[!htb]
\centering
\resizebox{\columnwidth}{!}{%
    \begin{tabular}{@{}l|l|lll|l|l|ll@{}}
    Formula & \ammonia & \nitrogen & \nitrousoxide & \nitricoxide & \nitrogendioxide & \nitricacid & \ammoniumnitrate & \urea \\
    \toprule
    Name & Ammonia & Nitrogen & \begin{tabular}[c]{@{}l@{}}Nitrous oxide,\\ Laughing gas\end{tabular} & \begin{tabular}[c]{@{}l@{}}Nitrogen oxide,\\ Nitric oxide\\ Nitrogen monoxide\end{tabular} & Nitrogen dioxide & Nitric acid & \begin{tabular}[c]{@{}l@{}}Ammonium\\ nitrate\end{tabular} & Urea \\
    Origin & \begin{tabular}[c]{@{}l@{}}Haber-Bosch\\ process\end{tabular} & \begin{tabular}[c]{@{}l@{}}Naturally present\\ in the atmosphere,\\ Ammonia \\ oxidation,\\ Selective catalytic\\ reaction (SCR)\end{tabular} & \begin{tabular}[c]{@{}l@{}}Ammonia \\ oxidation,\\ Emissions from\\ nitrogen-based\\ fertilisers\end{tabular} & \begin{tabular}[c]{@{}l@{}}Ammonia oxidation,\\ Anthropogenic sources \\ (combustion process, \\ industry, agriculture, ...)\\ Naturally produced \\ from lightning or\\ volcanoes\end{tabular} & \begin{tabular}[c]{@{}l@{}}Ostwald process (step 1)\\ Anthropogenic sources \\ (combustion process, \\ industry, agriculture, ...)\\ Naturally produced \\ from lightning or \\ volcanoes\end{tabular} & \begin{tabular}[c]{@{}l@{}}Ostwald process\\ (step 2)\end{tabular} & \begin{tabular}[c]{@{}l@{}}Nitric acid\\ neutralisation \\ with ammonia\end{tabular} & Ammonia \\
    Major use & \begin{tabular}[c]{@{}l@{}}Ostwald process\\ (fertilisers),\\ Direct use in soil,\\ Fuel,\\ Hydrogen \\ carrier,\\ Cooling, SCR \end{tabular} & \begin{tabular}[c]{@{}l@{}}Ammonia \\ production\end{tabular} & \begin{tabular}[c]{@{}l@{}}Medicine,\\ Propellant\end{tabular} & Production of nitric acid & Production of nitric acid & \begin{tabular}[c]{@{}l@{}}Fertiliser production,\\ Nitration (explosives,\\ dyes, ...),\\ Propellant,\\ Etching\end{tabular} & \begin{tabular}[c]{@{}l@{}}Fertilising \\ agent,\\ Explosives\end{tabular} & \begin{tabular}[c]{@{}l@{}}Fertilising \\ agent,\\ SCR to reduce\\ \ce{NO_x} into \ce{N_2}\end{tabular} \\
    Toxicity & \begin{tabular}[c]{@{}l@{}}Dangerous for\\ the environment\\ Toxic, \\ Corrosive\end{tabular} & \begin{tabular}[c]{@{}l@{}}Asphyxiation by\\ displacing \ce{O_2}\end{tabular} & \begin{tabular}[c]{@{}l@{}}Anaesthetic, \\ euphoric\end{tabular} & \begin{tabular}[c]{@{}l@{}}Oxidising, Corrosive,\\ Toxic\end{tabular} & \begin{tabular}[c]{@{}l@{}}Oxidising, Corrosive,\\ Toxic, Health hazard\end{tabular} & Oxidising, Corrosive &  &  \\
    Pollution & PM formation &  & \begin{tabular}[c]{@{}l@{}}Ozone\\ depletion\end{tabular} & \begin{tabular}[c]{@{}l@{}}Smog, acid rains, \\ ozone depletion,\\ Precursor to \ce{NO_2} in\\ the atmosphere\end{tabular} & \begin{tabular}[c]{@{}l@{}}Smog, acid rains, \\ ozone formation\end{tabular} & \begin{tabular}[c]{@{}l@{}}Decomposes towards\\ \ce{NO_2}\end{tabular} & \begin{tabular}[c]{@{}l@{}}Decomposes \\ into \ce{NO_2} when\\ used as fertiliser,\\ Eutrophication\end{tabular} & \begin{tabular}[c]{@{}l@{}}Decomposes \\ into \ce{NH_3} when\\ used as fertiliser,\\ Eutrophication\end{tabular} \\
    \begin{tabular}[c]{@{}l@{}}Greenhouse \\ effect\end{tabular} &  &  & \begin{tabular}[c]{@{}l@{}}Very important\\ (298 \ce{CO_2} eq.)\end{tabular} &  &  &  & \begin{tabular}[c]{@{}l@{}}Nitrogen-based\\ fertiliser release\\ \ce{N_2O} in the \\ atmosphere\end{tabular} & \begin{tabular}[c]{@{}l@{}}Nitrogen-based\\ fertiliser release\\ \ce{N_2O} in the \\ atmosphere\end{tabular} \\ \bottomrule
    \end{tabular}%
    }
    \caption{
        Nitrogen based species involved in the oxidation of ammonia, the Haber-Bosch process, the Ostwald process or nitrogen-based fertilisers.
        Information compiled from various sources: \cite{Thiemann2000, Harrison2001, Baerns2005, Imbihl2007, Hatscher2008, Davidson2009, Resta2020a, Borodin2021, Pottbacker2022}.
    }
    \label{tab:NitrogenGases}
\end{table}
\end{landscape}

\ce{N_2O} may also react with ammonia to produce nitrogen and water (eq. \ref{eq:SideReactions4}).
It is also possible to observe the dissociation of ammonia resulting in the production of \ce{N_2} and \ce{H_2} (eq. \ref{eq:SideReactions5}) when outside industrial reacting conditions.

Reactions \ref{eq:AmmoniaOxidationNitrogen}, \ref{eq:AmmoniaOxidationNitrousOxide}, \ref{eq:AmmoniaOxidationNitricOxide} and \ref{eq:SideReactions1} are strongly exothermic, leading to a significant increase of the catalyst temperature during the reaction \parencite{Hatscher2008}.

\begin{align}
    \label{eq:Stage2}
    2 \nitricoxide + \ce{O_2} & \rightarrow \nitrogendioxide \\
    \label{eq:Stage3}
    3 \nitrogendioxide + \water & \rightarrow 2\nitricacid + \nitricoxide
\end{align}

The second (eq. \ref{eq:Stage2}) and third (eq. \ref{eq:Stage3}) stages of the Ostwald process stem from the production of \ce{NO} \textit{via} the first stage, \textit{i.e.} the oxidation of ammonia (eq. \ref{eq:AmmoniaOxidationNitricOxide}).
Nitrogen dioxide (\nitrogendioxide) is produced from \ce{NO} which then reacts with water in liquid phase to form nitric acid, an important actor in multiple industrial processes (tab. \ref{tab:NitrogenGases}).
Nitric acid is for example used to produce fertilisers such as ammonium nitrate (eq. \ref{eq:AmmoniumNitrate}).

\begin{align}
    \label{eq:AmmoniumNitrate}
    \nitricacid + \ammonia & \rightarrow \ammoniumnitrate
\end{align}

Today, the ammonia oxidation is an essential catalytic reaction, \qty{80}{\percent} of the world production of nitric acid is used in the production of nitrogen-based fertilisers, a production expected to continuously grow in the future years \parencite{Lim2021a}.
Nitric acid is also extremely important for nitration, \textit{i.e.} the introduction of a nitro group into a chemical compound \parencite{Hughes1950}, crucial for the production of inks, dyes, explosives, and pharmaceuticals \parencite{Lee2005, Ouellette2014}.
Nitric acid can also be used as a rocket propellant \parencite{Mason1956, Oommen1999}.

Ammonia has also been investigated as a potential energy vector for hydrogen fuel cells, which has reignited the interest in understanding the complex system drawn by the many simultaneous reactions \parencite{Afif2016, Georgina2021}.
Moreover, ammonia can also be used for the reduction of nitrogen oxides (eqs. \ref{eq:SideReactions1} - \ref{eq:SideReactions4}), aiming at the production of nitrogen.
In all cases, particular focus is on two products of the reaction, namely, \ce{NO} and \ce{N_2}, which is where heterogeneous catalysis plays an important role.

\section{The importance of heterogeneous catalysis}\label{sec:AmoOxiHC}

\subsection{Industry conditions and catalysts}

\begin{figure}[!htb]
    \centering
    \includegraphics[width=\textwidth]{/home/david/Documents/PhD/Figures/introduction/Umicore.pdf}
    \caption{
    a) Temperature dependence product distribution during ammonia oxidation at industrial conditions (\qty{12}{\bar} - excess \ce{O_2}).
    b) Dependence of \ce{NO} yield on initial ammonia concentration.
    Figures adapted from literature \parencite{Heck1982, Hatscher2008}.
    }
    \label{fig:Products}
\end{figure}

When aiming at the production of nitric acid, selectivity towards the production of \ce{NO} must be achieved during the ammonia oxidation (eq. \ref{eq:AmmoniaOxidationNitricOxide}), \textit{i.e.} during the first stage of the Ostwald process.
The production of nitrogen oxide is possible \textit{via} direct methods, \textit{i.e.} by reacting nitrogen and oxygen above \qty{2000}{\degreeCelsius}.
However, such processes are not used due to their low energy efficiency and product selectivity.
This is at the origin of the selective, heterogeneous catalytic oxidation of ammonia towards the production of nitrogen oxide \parencite{Hatscher2008}.
Only at high temperature, high \ce{O_2}/\ce{NH_3} ratios, and by the use of catalysers, is the production of \ce{NO} favoured over \ce{N_2} and \ce{N_2O} (eq. \ref{eq:AmmoniaOxidationNitrogen} - fig. \ref{fig:Products}).

Since the 1930s, the presence of rhodium (\qty{\approx 10}{\percent} Rh) in knitted Pt-Rh gauzes catalysts at favourable reaction conditions (e.g. \qty{900}{\degreeCelsius} - \qty{12}{\bar} - excess \ce{O_2}) allowed a \qty{98}{\percent} \ce{NO} yield to be achieved, \qty{4}{\percent} higher than for pure platinum, while also increasing the catalyst lifetime, and decreasing the material loss \parencite{Kaiser1909, Handforth1934, Heck1982, Hatscher2008}.

In order to understand the reaction mechanism occurring on the catalyst surface, and the role of Pt and Rh in the catalyst stability and selectivity, scientists have been studying the reaction by different methods since the beginning of last century.
A comprehensive review of the ammonia oxidation is given by Hatscher et al. \parencite*{Hatscher2008}.
Relevant findings of the past 100 years will be resumed below.

% roughening, oxides
The importance of high temperature for the selective production of \ce{NO} was demonstrated early by temperature dependant studies \parencite{Nutt1968, Pignet1974, Li1997}, while a restructuring of the catalyst, also called catalytic \textit{etching}, was put into evidence by the means of \textit{ex-situ} SEM (scanning electron microscopy) imaging of industrial samples \parencite{McCabe1974, FlytzaniStephanopoulos1979, McCabe1986}.
This \textit{activation process} leads to an increase selectivity towards \ce{NO} while the gauzes undergo a transformation towards a roughened surface, composed of pits, facets, and large \textit{cauliflower} patterns (fig. \ref{fig:Gauzes}).
The roughening of the catalyst surface can double the active surface area \parencite{Hatscher2008}.

\begin{figure}[!htb]
    \centering
    \includegraphics[width=\textwidth]{/home/david/Documents/PhD/Figures/sample/EtchedGauzeAndReconstructedGauze.pdf}
    \caption{
    SEM images of Pt-Rh reconstructed gauzes with cauliflower patterns after use in industry, taken from Bergene et al. \parencite*{Bergene1996}.
    The horizontal bar is \qty{0.1}{\mm} wide.
    }
    \label{fig:Gauzes}
\end{figure}

The deactivation of Pt-Rh industrial catalysts after long exposure times have been explained by rhodium enrichment, and the presence of rhodium oxides \parencite{Fierro1990, Fierro1992, Bergene1996}.
However, a deactivation process was also reported on pure Pt catalyst \parencite{Ostermaier1974}, linked to the presence of platinum oxides \parencite{Ostermaier1976}.

The role of volatile surface oxides in the roughening and etching of the catalyst surface was theorised by Wei et al. \parencite*{Wei1996}, and confirmed experimentally by Nilsen et al. \parencite*{Nilsen2001}.
The transport of Pt and Rh was found to be permitted by the simultaneous presence of surface oxides and high temperature gradient areas, \textit{i.e.} \qtyrange{800}{1400}{\degreeCelsius} over \qty{100}{\mm}, decreasing with increasing Rh content and decreasing oxygen pressure \parencite{Hannevold2005a}.

\textit{Ex-situ} characterisation of Pt and Pt-Rh industrial gauzes by x-ray powder diffraction and electron microscopy allowed the identification of defect sites at the origin of high temperature gradient areas during reaction.
The existence of such areas allows the restructuring process to occur nearby initial defects by the formation of \ce{PtO_2} and \ce{RhO_2} oxides, depositing metallic atoms on colder regions, and also leading to the loss of some of the precious metals constituting the catalyst \parencite{Hannevold2005}.

The progressive deactivation of the catalyst due to the ever increasing presence of \ce{Rh_2O_3} \parencite{McCabe1986} was refuted by Hannevold et al. \parencite*{Hannevold2005}, explaining that such oxide could only form during the cooling of the catalyst, which is a good example of the limitation of \textit{ex-situ} works.
A \ce{Pt_3O_4} catalyst used for the oxidation of ammonia was proven to be unstable at working temperature above \qty{690}{\degreeCelsius}, decomposing into a pure Pt phase after \qty{7}{\hour} of operation \parencite{Zakharchenko2001}.
Overall, the role of platinum and rhodium oxides in the catalyst activation/deactivation process is not fully understood.

Nevertheless, the presence of rhodium in the catalysts limits the loss of platinum during operation, thus reducing the cost of industrial scale ammonia oxidation.
The replacement of the material loss constitutes the second biggest expense in the production of fertilisers after the production/purchase of ammonia \parencite{Hatscher2008}.

\subsection{Reaction mechanism}\label{sec:Mechanism}

First studies performed at low pressure and different temperatures have supported a Langmuir-Hinshelwood mechanism during ammonia oxidation \parencite{Nutt1969, Pignet1974, Ostermaier1974, Pignet1975, Gland1978a}.
Both reactants are adsorbed and decomposed on top of the catalyst surface, the final pathway towards the production of nitrogen or nitric oxide depending on the ratio between the \ce{O_2} and \ce{NH_3} partial pressures.

The importance of adsorbed atomic oxygen (\ce{O_a}), and adsorbed hydroxyl groups (\ce{OH_a}) in the de-hydrogenation of ammonia on both Pt(111) \parencite{Mieher1995} and Pt(100) surface was put into evidence by molecular beam studies under various oxygen coverage \parencite{Bradley1995, Bradley1997, vandenBroek1999, Kim2000}.
High oxygen coverage on Pt(100) has shown to reduce the production of nitrogen, mainly produced by the dissociation of nitric oxide (\ce{NO}), favoured at lower temperatures over \ce{NO} desorption \parencite{Bradley1995}.

Asscher et al. \parencite*{Asscher1984} have reported the existence of a mechanism involving oxygen in the gas phase reacting with adsorbed \ce{NH_x} species on Pt(111) for the production of \ce{NO}, evidence of which was not observed in recent studies.

At ultra-high vacuum (UHV), the rotated hexagonal reconstruction of clean Pt(100) \parencite{Hammer2016} was found to impinge on \ce{NH_3} de-hydrogenation at low temperature (\qty{-123}{\degreeCelsius}) \parencite{Bradley1997}.
Between \qty{125}{\degreeCelsius} and \qty{350}{\degreeCelsius}, the ammonia oxidation stabilises the (1x1) phase by the formation of \ce{NH_x} intermediates \parencite{Rafti2007}.

% steps
The importance of atomic steps in the catalytic activity was first revealed by Gland et al. \parencite*{Gland1978, Gland1980}.
A more recent study of the oxidation of ammonia on several model catalysts (Pt(533), Pt(443), Pt(865), Pt(100), Pt foil) to investigate the structure selectivity by Yingfeng \parencite*{Yingfeng2008} linked the presence of steps and kinks with higher catalytic activity in the \qtyrange{1e-9}{1e-5}{\bar} range.
Similar results have revealed that the Pt(533) is 2 to 4 times more active than Pt(443) below \qty{1e-5}{\bar} \parencite{Scheibe2005}.
Moreover, the production of \ce{N_2} was confirmed to be promoted by lower temperatures, and a reduced \ce{O_2}/\ce{NH_3} ratio in the incoming gas flow, while higher temperatures and an elevated \ce{O_2}/\ce{NH_3} ratio tend to result in a higher selectivity towards the formation of \ce{NO} \parencite{Zeng2009}.
%No production of \ce{N_2O} was observed within the studied pressure range.
The importance of steps at ambient pressure was revealed to be most important when aiming at producing \ce{N_2} at low temperature, but could not be correlated to an increase in \ce{NO} production at high temperature in a comparative study of the Pt(111) and Pt(211) surfaces \parencite{Ma2019}.
Moreover, it was shown that the adsorbed \ce{NH_3} hopping rate (that describes the mobility or diffusion of \ce{NH_3}) is close to its desorption rate on Pt(111) terrace sites, making it unlikely to reach the steps where it may react rather than desorb from the catalyst surface \parencite{Borodin2021}.

% tap, kinetic studies
Rebrov et al. \parencite*{Rebrov2002} detailed the reaction kinetics and mechanism with a 13 step, temperature dependent model, the parameters of which have been refined with data collected from the reactants and products partial pressures evolution in a micro-reactor.
A wide range of conditions was explored, including ambient pressures of \ce{NH3} (\qtyrange{0.01}{0.12}{\bar}) and \ce{O_2} (\qtyrange{0.10}{0.88}{\bar}), in a large temperature range (\qtyrange{250}{400}{\degreeCelsius}).
A dual adsorption site mechanism was proposed, with a preference for hollow site for oxygen species, and for top or bridge sites for nitrogen species, while \ce{NO} species have been reported to exist on both top and bridge sites.

Pérez-Ramirez et al. \parencite*{PerezRamirez2004} have also attempted to study the kinetics of the reaction by directly analysing the selectivity of catalysts used in industry (\textit{i.e.} Pt-Rh and Pt gauzes), above \qty{700}{\degreeCelsius}, by the means of reactant gas pulses.
The heterogeneous catalysis reaction mechanism was found to be similar on both sample.
A pre-exposition of the catalysts to oxygen facilitated the de-hydrogenation of \ce{NH_3}, the decomposition towards \ce{N_2} was not detected without oxygen.
A high oxygen coverage of the catalyst was linked to the formation of \ce{NO}, strongly bound oxygen favours the production of \ce{N_2}, whereas weakly bound oxygen was associated to \ce{NO} selectivity.
The importance of adsorbed oxygen species to prevent spontaneous \ce{NO} dissociation was confirmed, confirming previous works \parencite{Bradley1995}.
In a second study, increasing the \ce{O_2}/\ce{NH_3} ratio to 10 pushed \ce{NO} selectivity to almost \qty{100}{\percent}, \ce{N_2} and \ce{N_2O} production being both suppressed by favouring \ce{NO} desorption \parencite{PerezRamirez2009}.

The advent of density functional theory (DFT) brought forward theoretical explanations of the reaction mechanism for the first time.
For example, a study combining DFT calculations and temporal analysis of products (TAP) by Baerns et al. \parencite*{Baerns2005} confirmed the importance of adsorbed surface oxygen (\ce{O_a}) and hydroxyl groups (\ce{OH_a}) for the de-hydrogenation of \ce{NH_3} on the catalyst surface.
\ce{N_2O} could only be detected at ambient pressures, \ce{N_2} is favoured at lower temperature, whereas \ce{NO} is favoured at higher temperatures.
Importantly for the use of model catalysts, no difference in the temperature dependant production of \ce{N_2} and \ce{NO} between a Pt(533) single crystal, Pt foil, and knitted Pt gauzes could be observed.
Surface roughening of the catalyst at ambient pressure was linked to activation and selectivity change for a Pt foil.
Interestingly, the deactivation of Pt catalysts due to the adsorption of nitrogen species below \qty{115}{\degreeCelsius} was put into evidence, but with a reactivation above that temperature \parencite{Sobczyk2004}.

The first studies performed under industrially relevant conditions brought forward two recurrent problems when carrying out low pressure studies.
(i) the production of \ce{N_2O} was rarely discussed because undetected.
Van den Broek et al. \parencite*{vandenBroek1999} first hypothesised that \ce{NO} was a precursor in the production of nitrous oxide, produced \textit{via} an additional reaction with adsorbed oxygen.
\ce{N_2O} was then detected also by Pérez-Ramirez et al. \parencite*{PerezRamirez2004}, Baerns et al. \parencite*{Baerns2005} and Kondratenko et al. \parencite*{Kondratenko2007}, which confirmed the precursor role of \ce{NO} in its production.
\ce{N_2} was hypothesised to be produced from the association of two adsorbed nitrogen atoms, after the de-hydrogenation process.
Previous works mentioning oxidation of adsorbed ammonia by gas phase atoms was contradicted.
(ii) the known roughening process of the catalyst at working conditions could not be reproduced without long working times, and high pressures.
Kinetic studies on poly-crystalline Pt up to \qty{10}{\bar} but at temperatures below \qty{385}{\degreeCelsius} observed the roughening transition \parencite{Kraehnert2008}, which proved difficult to fit with developped kinetic models, attributed to local increase in temperature and surface area.

% DFT
\begin{figure}[!htb]
    \centering
    \includegraphics[width=\textwidth]{/home/david/Documents/PhD/Figures/ammonia/ReactionMechanism.png}
    \caption{
    Example of \ce{NH_3} stripping process by adsorbed \ce{O}, taken from Imbihl et al. \parencite*{Imbihl2007}.
    From grey to black colours are lower layer of Pt(111), hydrogen atoms, upper layer of Pt(111), oxygen atoms, and nitrogen atoms.
    }
    \label{fig:ReactionMechanism}
\end{figure}

Novell-Leruth et al. \parencite*{NovellLeruth2005} confirmed the adsorption of \ce{NH_3} and \ce{NH_2} to occur respectively on top and bridge sites for both Pt(100) and Pt(111), but with a more favourable adsorption process on Pt(100).
Similar adsorption energies have been reported for \ce{NH} and \ce{N} that both adsorb on hollow sites \
Additional DFT studies of the reaction pathways and kinetics confirmed a mechanism following a step-by-step decomposition of ammonia on the catalyst surface.
The stripping of the hydrogen atoms from adsorbed ammonia was reported to be facilitated by the presence of oxygen species, confirming the key role of oxygen in the production of not only nitric oxide, but also nitrogen \parencite{Offermans2006}.
A comparative study between the Pt(100), Pt(111) and Pt(211) surfaces did not reveal a strong structure sensitivity of the \ce{NH_3} stripping process \parencite{Offermans2007}.
%(\textit{i.e.} stepped)

Imbihl et al. \parencite*{Imbihl2007} have further improved the understanding of the production of \ce{N_2O}, which happens not only by the recombination of two adsorbed \ce{NO} species but also \textit{via} the reaction between adsorbed \ce{NO} and \ce{NH_x} species.
The first de-hydrogenation step was found to be the slowest, while the desorption of \ce{NO} is the rate limiting-step when aiming at the selective production of nitric oxide \parencite{NovellLeruth2008}.
Interestingly, for the first time different oxygen species were reported to be responsible for the de-hydrogenation process depending on the surface structure, respectively \ce{O} for Pt(111) (fig. \ref{fig:ReactionMechanism}) and \ce{OH} for Pt(100).
A high energy barrier for \ce{NO} desorption and \ce{N_2O} formation explain the high temperature needed for their production in comparison with \ce{N_2}.

% conclude
Despite the large amount of work in the previous and current century, the mechanism of the oxidation of ammonia is still unclear.
Studies performed at ultra high vacuum have paved the way for the understanding of model catalyst systems, followed by works at progressively higher pressures, but without reaching industrial conditions during \textit{operando} studies on model systems.
Moreover, many characterisation of used catalysts have been performed \textit{ex-situ}.

This shows that the complete understanding of the mechanisms ruling heterogeneous catalysis comes from developing novel methods, compatible with very high pressures and temperatures, especially when aiming at the study of a complex system with many competing reactions.
For example, Pottbacker et al. \parencite*{Pottbacker2022} have presented a new characterisation method to facilitate the study of the reaction kinetics at industrial conditions, by precisely measuring the temperature and compositional gradients present on the catalyst surface.
Cross checking such information with results gathered on model systems at industrious conditions, e.g. active sites and structure, will aid the comprehension of the reaction mechanism.

Replacing precious metals in industrial reactor can be thought as the natural step that will follow the comprehension of the reaction mechanism.
For example, the performance of new material has already been explored with \ce{V_2O_5} catalyst, reaching promising \ce{NO} yields at atmospheric pressure between \qtyrange{300}{650}{\degreeCelsius} \parencite{Ruan2022}.
Enabling selective mechanisms to function at lower working temperature can also result in a large economy of energy.

\section{Environmental impact}

\subsection{Greenhouse effect}

As illustrated earlier in fig. \ref{fig:FertilizerWID}, nitrogen-based fertilisers have permitted an industrial development of agriculture.
Fertilisers can be either ammonium- or nitrate-based.
When plants do not fully absorb all the nutrients, a series of microbe-mediated transformations occur which leads to the release of nitrogen back into the atmosphere, primarily as nitrogen gas (\ce{N_2}) and, to a lesser extent, as \ce{N_2O}.

For an equal amount of \ce{N_2O} and \ce{CO_2}, nitric oxide will trap \num{298} times more heat than the carbon dioxide over the next 100 years \parencite{MITCLIMATE}, responsible for \qty{6.2}{\percent} of the total U.S.A. greenhouse gases emissions in 2021 (fig. \ref{fig:PieGreenhouseNO2}).
Moreover, the production of the necessary ammonia which is in turn used for fertilisers often relies on natural gas to provide hydrogen, meaning that each ton of \ce{NH_3} produced is equivalent to the emission of \qty{1.9}{\tonne} of \ce{CO_2} \parencite{Rafiqul2005, Chen2018}.

\begin{figure}[!htb]
    \centering
    \includegraphics[width=\textwidth]{/home/david/Documents/PhD/Presentations/Slides/PhdSlides/Figures/ammonia/NO2pie.pdf}
    \caption{
    Pie charts underlining the importance of different gas in the total US greenhouse gas emissions in 2021 (a) and the specific contribution of nitrogen-based fertilisers to the total \ce{N_2O} emissions in 2021.
    LULUCF means Land Use, Land-Use Change, and Forestry.
    Adapted from \cite{EPAGreenhouseGases}.
    }
    \label{fig:PieGreenhouseNO2}
\end{figure}

The presence of \ce{N_2O} in the atmosphere can be linked to the development of agriculture towards a productivity model, helped by the means of nitrogen-based fertilisers.
Thus, the amount of nitric oxide, which has an atmospheric lifetime of 114 years, is expected to decrease in the future years in the northern hemisphere, but increase in the southern hemisphere following such transitions \parencite{Solomon2007, Davidson2009}.

In a review of the presence of \ce{N_2O} in the atmosphere linked to human activities, Pérez-Ramirez et al. \parencite*{PerezRamirez2003} have shown that the most important contribution to nitrous oxide in the atmosphere is not only from unused volumes of nutrients, but also from nitric acid manufacture.
Understanding and limiting the origin of \ce{N_2O} by controlling the process selectivity is thus capital to limit the amount of nitric oxide released in the atmosphere.

\subsection{Pollution}

Nevertheless, it can also be interesting to be able to remove \ce{NH_3} from the atmosphere, a colourless gas with a pungent odour, that irritates the eyes, nose, throat, and respiratory tract if inhaled in small amounts due to its corrosive nature; and poisonous in large quantities.
Ammonia also pollutes and contributes to the eutrophication (excessive growth of algae) and acidification of terrestrial and aquatic ecosystems, and forms secondary particulate matter (PM2.5) when combined with other pollutants in the atmosphere (tab. \ref{tab:NitrogenGases}).
Nitric acid can also be linked to acid rains \parencite{Galloway1981}.
Moreover, nitrogen-based fertilisers and other human activities can lead to nitrogen runoff into water bodies, contributing to eutrophication and causing harm to aquatic ecosystems (tab. \ref{tab:NitrogenGases}).
Approximately half of the production of ammonia is lost to the environment \parencite{Erisman2007}.

Finally, the important effect of nitrogen oxides (\ce{NO_x}, \textit{i.e.} \ce{NO}, \ce{NO2}, and \ce{N_2O}) on the environment has brought forward the necessity to control their emissions, especially from the exhaust of diesel engines that are responsible for \qty{65}{\percent} of their emissions.
The selective catalytic reaction (SCR) using urea or ammonia as reductant (tab. \ref{tab:NitrogenGases}) has proven to be effective and to reach \qty{95}{\percent} efficiency \parencite{MitsubishiSCR}.
However, there can be a subsequent problem of unreacted ammonia \textit{slipping} from the reaction, which is also an important subject of study \parencite{Thermofischer}.

To efficiently remove ammonia, the selectivity of the ammonia oxidation reaction must be tuned towards the production of \ce{N_2}, which is the only non pollutant and toxic gas.

Lim et al. \parencite*{Lim2021a} have underlined that if the current methodologies for producing ammonia and nitric acid remain unchanged, carbon emissions originating from the manufacturing of fixed fertiliser feedstocks could exceed 1300 MtCO2eq/yr, highlighting the pressing necessity for sustainable alternatives.

Therefore, depending on the application of the ammonia oxidation, the catalytic reaction must be tuned towards a specific product, this \textit{selectivity} is controlled by the reaction temperature, pressure, the reactant ratio, and the type of catalyst.
To be able to drive the reaction, the impact of each parameter \textit{at relevant industrial conditions} on the product pressure must be studied.

\section{From industry to model catalysis}\label{sec:LiteratureAmmonia}

\epigraph{"A long standing conundrum in the catalysis community emerged at the interface between surface science and heterogeneous catalysis, better known as the pressure and materials gap."}{\textit{Nature Catalysis editorial, \cite*{NatureEditorial2018}.}}

% techniques
Pressure and material gap are limiting the understanding of catalyst operation (tab. \ref{tab:Gap}).
X-rays are intrinsically well suited when working at high pressures thanks to their high penetration in gases, which made them a promising probe to access sample environments approaching industrial condition.

\begin{table}[!htb]
    \centering
    \begin{tabular}{l|l|l|l}
    \toprule
                & Pressure    & Material                         &     Temperature \\
    \midrule
    Industry   & \qtyrange{1}{12}{\bar} & Knitted gauzes wires   & \qtyrange{750}{900}{\degreeCelsius} \\
               &              & (diameter \qty{\approx 80}{\um}) & \\
    \midrule
    Literature & UHV, mbar    & Single crystals                  & \qtyrange{25}{900}{\degreeCelsius} \\
    \midrule
    This study & Near ambient & Single crystals                  & \qtyrange{25}{600}{\degreeCelsius} \\
               & pressure (\qty{0.5}{\bar})  & and nanoparticles & \\
    \bottomrule
    \end{tabular}
    \caption{
        Ammonia oxidation conditions: comparison between industry and model catalysts.
        Reproducing the same exact industrial reaction conditions and sample can be difficult in a laboratory due to the nature of the probe, the sensitivity of the technique, and the design of reactor cells for synchrotrons \parencite{Hatscher2008}.
        This is the so-called material and pressure gap in heterogeneous catalysis.
    }
    \label{tab:Gap}
\end{table}

Bridging the pressure and material gap by the development of new x-ray techniques has been the subject of several dissertations in recent years.
Ackermann \parencite*{Ackermann2007} has for example pushed forward the use of \textit{operando} surface x-ray diffraction (SXRD) for the study of heterogeneous catalysis.
This effort has also been focused towards spectroscopy techniques such as x-ray photoelectron spectroscopy (XPS) or x-ray absorption spectroscopy (XAS) \parencite{Dann2019}.
Progress in bridging the pressure gap and material gap has been preceded by the development of catalysis reactors compatible with synchrotrons beamlines.
Such reactors are closed environments penetrable by x-rays, and accommodating a wide range of temperature, total pressure and gas compositions \parencite{VanRijn2010, Richard2017, CastanGuerrero2018}.

% single crystals
The samples often used during SXRD and XPS experiments are single crystals, much larger than industrial samples.
For example, the industrial gauzes used during the oxidation of ammonia have a diameter of about \qty{80}{\um}, whereas single crystals are typically \unit{\cm} large (fig. \ref{fig:SamplesIntro}).
These kind of samples only exhibit a single type of structure on their surface (e.g. (111), (100)), reducing the complexity of the catalyst surface.
The reason behind their use is a large surface area exposed to the reacting gases, and thus an increased surface signal, most important to understand heterogeneous catalysis which is a surface process \parencite{Goodman1994}.
Nevertheless, they can show some limitations, for example large Pt(111) single-crystals exhibit a significant amount of steps on their surface also contributing to the catalytic activity \parencite{CalleVallejo2017}.

\begin{figure}[!htb]
    \centering
    \includegraphics[height=4.3cm]{/home/david/Documents/PhD/Presentations/Slides/PhdSlides/Figures/sample/pt_gazes.png}
    \includegraphics[height=4.3cm]{/home/david/Documents/PhD/Presentations/Slides/PhdSlides/Figures/bcdi_data/B7/B7_facets_cropped.png}
    %trim=140 100 0 75, clip,
    \includegraphics[height=4.3cm]{/home/david/Documents/PhD/Presentations/Slides/PhdSlides/Figures/sample/sxrd_sample.png}
    \caption{
    Left: platinum gauzes used in industry, diameter \qty{\approx 80}{\um}, image taken from industry website \parencite{PtRhGauze}.
    Middle: reconstructed Pt particle, surface coloured by the displacement of surface layers from their equilibrium positions, diameter of about \qty{300}{\nm}.
    The orientation of each facet on the particle surface is indicated.
    Right: Pt$(111)$ single crystal used in SXRD and XPS experiments, diameter \qty{\approx8}{\mm}, thickness \qty{\approx2}{\mm}.
    }
    \label{fig:SamplesIntro}
\end{figure}

% nanoparticles
Only recently have supported platinum nanoparticles (\qty{\approx 1}{\nm} large) been used during the oxidation of ammonia, showing a remarkable selectivity towards \ce{NO} at high temperature and atmospheric pressure \parencite{Schaffer2013}.
Reducing Pt supported nanoparticles by \ce{H_2} showed to improve their catalytic activity below \qty{200}{\degreeCelsius} in a study aiming at the production of \ce{N_2} \parencite{Svintsitskiy2020}.
Similarly, supported palladium nanoparticles have shown high selectivity towards production of \ce{N_2} in the selective catalytic oxidation of \ce{NH_3} below \qty{200}{\degreeCelsius} \parencite{Dann2019}.
Nanoparticles allow to better understand heterogeneous catalysis since they exhibit a various amount of active sites such as different facets (e.g. (111), (110), (100), ...), edges, corners and defects (fig. \ref{fig:SamplesIntro}).
They are thus considered a good approximation of real catalysts \parencite{Somorjai2007, Molenbroek2009, Cuenya2010, Kwangjin2012, Schauermann2013}.
Studies have shown that nanoparticle surface strain can be controlled, opening up a new path to tune and optimise nanoparticle catalysts \parencite{Zhang2014, Sneed2015, Wang2016}.

% introduce bcdi
Bragg coherent diffraction imaging (BCDI) is a young technique \parencite{Robinson2001}, only recently applied to catalysis \parencite{Ulvestad2016}, that can only be applied to sub-micron samples due to instrumental limitations \parencite{Marchesini2003a}.
Indeed, the sample can not be larger than the coherence lengths of the beam, about \qty{1}{\um} at $3^{rd}$ generation synchrotrons.
The x-ray beam, focused by some optical elements to micrometer size to respect this requirement, must fully illuminate the sample during the experimental process.
Therefore, BCDI can contribute to reducing the material gap in heterogeneous catalysis.

However, samples below a certain size are not easily measured experimentally with other diffraction techniques because the outcoming photon flux is proportional to the sampled volume \parencite{Willmott}.
This limitation also exists in BCDI, despite the highly focused beam, which draws a limit between experimental samples (so far, the smallest imaged nanoparticle is \qty{20}{\nm} large - \cite{Bjorling2019, Carnis2021}) and the few \unit{\nm} large nanoparticles typically used in density-functional theory (DFT) or molecular dynamics (MD).
DFT and MD are theoretical approaches to molecular adsorption and particle relaxation only limited by the current computational power.
Moreover, the ability to resolve the surface signal with BCDI is still limited in comparison with SXRD, which is why both techniques are here used together to obtain a better understanding of the dynamics at play during the catalytic reaction.

\section{Aim and Scope}

The catalytic oxidation of ammonia has exerted a profound influence in the $20^{th}$ and $21^{st}$ century, playing a pivotal role in the fertiliser industry and associated demographic shifts.
Despite its critical industrial significance, the various industrial applications linked to the Ostwald process have significantly contributed to both climate change, and to the ever growing pollution of our ecosystems.

In this first chapter, the importance of the oxidation of ammonia has been underlined.
It was shown that despite being a major catalytic process in a multi-billion industry, the exact mechanisms of action are not yet understood.
Obtaining a comprehensive understanding of these mechanisms is essential for controlling reaction selectivity, thereby mitigating pollution arising from \ce{NH_3} and \ce{NO_x}, as well as reducing the greenhouse effect induced by \ce{NO_2}.
Moreover, a deeper understanding of the reaction mechanism holds promise for the development of novel catalysts that can move away from the reliance on expensive precious metals.

This thesis aims at studying the catalytic reaction at near ambient pressure by the means of diffraction and spectroscopy synchrotron techniques.
The reactor used to study heterogeneous catalytic reaction at the SixS beamline (SOLEIL) is already compatible with highly oxidative environments \parencite{VanRijn2010, Resta2020a}, and will permit high pressure surface x-ray diffraction and Bragg coherent diffraction imaging experiments.
Optical elements necessary for the use of a focused coherent beam have been implemented and characterised.
Extending the panel of available techniques will help bridging the material and pressure gap by bringing forth the possibility for synchrotron users to study catalytic reactions at the same exact conditions and environment, but with different techniques.

X-ray photoelectron spectroscopy has been carried out at the B07 beamline (Diamond), also compatible with high pressures \parencite{Held2020}.
Both the material and pressure gap have been partly bridged by operating at temperatures above the catalyst light-off, at almost industrial pressure.

The study of the reaction is performed \textit{operando}, by exploring the large parameter space defined by the operating temperature, total pressure, and \ce{O_2}/\ce{NH_3} ratio.
The utilisation of diverse samples will contribute to a comprehensive understanding of the underlying structural dynamics.
Integrating the evolution of the catalyst surface structure with the nature of adsorbed species during the reaction will help validate the reaction mechanism.

\section{Thesis outline}

The first chapter has briefly presented the history of the oxidation of ammonia, and its societal impacts, setting the stage of this thesis.

The second chapter will provide a concise overview of heterogeneous catalysis, and the fundamental principles governing the interaction between x-rays and matter.
This section emphasises the origins, advantages, and constraints of each technique used in the study, highlighting how x-rays can indirectly observe catalytic reactions by detecting unique markers left on the materials.
The SixS beamline at synchrotron SOLEIL, the primary tool for most experiments, is introduced, emphasising the latest advancements in experimental techniques and beamline-specific hardware.
Additionally, the thesis presents the software used for data reduction and analysis developed during this work, focusing on a comprehensive workflow for Bragg coherent diffraction imaging, a technique that has yet to reach its full potential through the development of $4^{th}$-generation synchrotrons and powerful computing clusters.
A description of the software \textit{Gwaihir} \parencite{Simonne2022}, which aims to facilitate the reduction and analysis of data in the \textit{Python} programming language, while encouraging broader adoption of the technique through a user-friendly graphical interface, is provided.

The third and fourth chapters present results from the \textit{operando}, near-ambient pressure study of catalytic reactions using synchrotron techniques.
Pt nanoparticles and single crystals are employed to explore the structure-selectivity relationship at ambient pressure and high temperature, by the means of a vast array of \ce{O_2}/\ce{NH_3} partial pressure ratios, favouring either the production of \ce{N_2} or \ce{NO}.
The presence of platinum oxides is monitored in order to understand its importance in the reaction mechanism, as well as in potential catalyst reconstructions.
Thus, the difference in activity between different crystalline facets is studied with different samples that, together, offer a good compromise between industrial catalysts and compatibility with the experimental setup of synchrotron beamlines.
Complementary studies with x-ray photoelectron spectroscopy are performed at the same \ce{O_2}/\ce{NH_3} ratio and temperature.
Information about the presence or not of nitrogen and oxygen species on the catalyst surface will improve the understanding of the reaction mechanism.
Finally, to explore the correlation between surface structure and catalytic activity, mass spectrometry data is collected simultaneously to all measurements.

The last chapter will serve as a general conclusion, resuming the results obtained from each technique, and detailing the perspectives enabled by this thesis.
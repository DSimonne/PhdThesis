%%%%%%%%%%%%%%%%%%%%%%%%%%%%%%%%%%%%%%%%%%%%%%%%%%%%%%%%%%%%%%%
\Ifthispageodd{\newpage\thispagestyle{empty}\null\newpage}{}
\thispagestyle{empty}
\newgeometry{top=1.5cm, bottom=1.25cm, left=2cm, right=2cm}
\fontfamily{rm}\selectfont

\lhead{}
\rhead{}
\rfoot{}
\cfoot{}
\lfoot{}

\noindent 
%*****************************************************
%***** LOGO DE L'ED À CHANGER IMPÉRATIVEMENT *********
%*****************************************************
\includegraphics[height=2.4cm]{logo/logo_edpif_4eme.png}
\vspace{0.5cm}
%*****************************************************
\fontfamily{cmss}\fontseries{m}\selectfont

\small

\begin{mdframed}[linecolor=Prune,linewidth=1]

\textbf{Titre:} Propriétés catalytiques à l’échelle nanométrique sondées par diffraction des rayons X de surface et imagerie de diffraction cohérente

\noindent \textbf{Mots clés:} Diffraction, Catalyse, Surface, Structure, Déformation

\vspace{-.5cm}
\begin{multicols}{2}
\noindent \textbf{Résumé:} Le principal objectif est d'imager des nanostructures pour sonder les conditions in situ et operando ; mesurer la structure à l'échelle nanométrique et révéler également les effets de masse, de surface et d'interface ainsi que les défauts. Viser à terme à comprendre les phénomènes structurels importants pour les nanocatalyseurs et les relier à leur activité, sélectivité, réutilisabilité et durabilité. En complément des études cohérentes aux rayons X sur des particules individuelles, des techniques étudiants la moyenne des ensembles comme la diffraction des rayons X à incidence rasante seront employées pour voir si l'évolution des formes d'ensemble sont similaires à celles des nanoparticules uniques et sondent s'il y a une déconnexion entre les particules uniques et l'activité catalytique sur des billions de particules. La catalyse des nanomatériaux est apparue comme un moyen efficace d'exposer une surface plus élevée et d'accélérer les processus catalytiques en maximisant le rapport surface-volume. Le développement d'une catalyse hétérogène avec une sélectivité ciblant les 100\% est un défi constant ainsi que la compréhension de la durabilité et du vieillissement du catalyseur lui-même. Dans un procédé réel (réacteur d'usine de catalyseur, échappement de voiture, pile à combustible), l'évolution de la forme et de la déformation des nanoparticules catalytiques dans des conditions de réaction contribue au vieillissement du catalyseur et a un impact sur la durée de vie du dispositif. Cependant, le processus catalytique et les changements structurels associés restent encore mal compris. Comprendre comment la structure du catalyseur est affectée par la couche adsorbée dans des conditions de réaction est donc de la plus haute importance pour formuler des relations de performance de structure de catalyseur qui guident la conception de meilleurs catalyseurs.
\end{multicols}

\end{mdframed}

\begin{mdframed}[linecolor=Prune,linewidth=1]

\textbf{Title:}Catalytic properties at the nanoscale probed by surface X-ray diffraction and coherent diffraction imaging

\noindent \textbf{Keywords:} Diffraction, Catalysis, Surface, Structure, Strain

\begin{multicols}{2}
\noindent \textbf{Abstract:} The main objective is to image nanostructures to probe in situ and operando conditions; measure the structure at nanoscale and to reveal bulk, surface and interface effects, as well as defects. Ultimately aiming to understand the structural phenomena important for the working nanocatalysts and link them to their activity, selectivity, reusability and sustainability. In complement to coherent x-ray studies on individual particles, ensemble-averaging techniques like grazing incidence x-ray diffraction will be employed to see if the evolution of ensemble shapes is similar to the one of single nanoparticles and probe if there is a disconnect between single particles and the catalytic activity over trillions of particles. Catalysis of nanomaterials has emerged as an efficient way to expose higher surface area and accelerate catalytic processes by maximizing the surface-volume ratio. The development of heterogeneous catalysis with selectivity targeting the 100\% is a constant challenge as well as understanding the durability and ageing of the catalyst itself. In a real process (catalyst plant reactor, car exhaust, fuel cell) the shape and strain evolution of catalytic nanoparticles under reaction conditions contributes to the ageing of the catalyst and impact the lifetime of the device. However, the catalytic process and the associated structural changes remain poorly understood. Understanding how catalyst structure is affected by the adsorbed layer under reaction conditions is therefore of utmost importance to formulate catalyst structure performance relations that guide the design of better catalysts.
\end{multicols}
\end{mdframed}

%************************************
\vspace{\fill} % ALIGNER EN BAS DE PAGE
%************************************

\noindent
\color{Prune} \footnotesize Maison du doctorat, Université Paris-Saclay\\
2$^{\mathrm{e}}$ étage, aile ouest, École normale supérieure Paris-Saclay\\
4 avenue des Sciencs\\
91190 Gif-sur-Yvette, France
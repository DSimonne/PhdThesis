\begin{titlepage}

\newgeometry{left=6cm,bottom=1cm, top=1cm, right=1cm}

\tikz[remember picture,overlay] \node[opacity=1,inner sep=0pt] at (-13mm,-135mm){\includegraphics{/home/david/Documents/PhD/Figures/Logos/Frame-ups.pdf}};

% fonte sans empattement pour la page de titre
\fontfamily{fvs}\fontseries{m}\selectfont

%*****************************************************
%******** NUMÉRO D'ORDRE DE LA THÈSE À COMPLÉTER *****
%******** POUR LE SECOND DÉPOT                   *****
%*****************************************************

\color{white}

\begin{picture}(0,0)
\put(-152,-743){\rotatebox{90}{\Large \textsc{THESE DE DOCTORAT}}} \\
\put(-120,-743){\rotatebox{90}{NNT : 2024UPASP001}}
\end{picture}

% \vspace{-12mm}
% \flushright \includegraphics[height=2.1cm]{/home/david/Documents/PhD/Figures/Logos/cea.jpg} \hspace{7mm}
% \includegraphics[height=2.1cm]{/home/david/Documents/PhD/Figures/Logos/SOLEIL.png}

%*****************************************************
%******************** TITRE **************************
%*****************************************************

\flushright
\vspace{10mm}
\color{Prune}
\fontfamily{cmss}\fontseries{m}\fontsize{22}{26}\selectfont
Catalytic properties at the nanoscale probed by surface x-ray diffraction and coherent diffraction imaging

\normalsize
\color{black}
\Large{\textit{Propriétés catalytiques à l'échelle nanométrique sondées par diffraction des rayons X de surface et imagerie de diffraction cohérente}}
%*****************************************************

\fontsize{8}{12}\selectfont

\vspace{2cm}

\normalsize
\textbf{Thèse de doctorat de l'université Paris-Saclay}

\vspace{6mm}

École doctorale n$^{\circ}$ 564, Physique en Île-de-France (PIF)\\
\small Spécialité de doctorat: Physique\\
\small Graduate School: Physique. Référent: Faculté des sciences d’Orsay\\
\vspace{6mm}

\footnotesize Thèse préparée dans les unités de recherche \textbf{Synchrotron Soleil (Université Paris-Saclay)} et \textbf{Modélisation et Exploration des Matériaux (Université Grenoble Alpes, CEA)}, sous la direction d'\textbf{Alessandro COATI}, Docteur, la co-direction de \textbf{Marie-Ingrid RICHARD}, Directrice de recherche, et le co-encadrement de \textbf{Andrea RESTA}, Docteur\\
\vspace{15mm}
\textbf{Thèse soutenue à Paris-Saclay, le 12 janvier 2024, par}\\
\bigskip
\Large {\color{Prune} \textbf{David SIMONNE}}

%************************************
\vspace{\fill} % ALIGNER LE TABLEAU EN BAS DE PAGE
%************************************

\bigskip
\flushleft
\small {\color{Prune} \textbf{Composition du jury}}\\
{\color{Prune} \scriptsize {Membres du jury avec voix délibérative}} \\
\vspace{2mm}
\scriptsize
\begin{tabular}{|p{10cm}l}
\arrayrulecolor{Prune}
\textbf{Sylvain RAVY} & Président \\
Directeur de recherche, LPS, Université Paris-Saclay & \\
\textbf{Andreas STIERLE} & Rapporteur \& Examinateur \\
Professeur, University of Hamburg / DESY & \\
\textbf{Thomas CORNELIUS} & Rapporteur \& Examinateur \\
Directeur de recherche, IM2NP, Aix-Marseille Université & \\
\textbf{Gerardina CARBONE} & Examinatrice \\
Dr., Lund University & \\
\textbf{Virginie CHAMARD} & Examinatrice \\
Directrice de recherche, Institut Fresnel, Aix-Marseille Université & \\
\end{tabular}
\vspace{6mm}

\end{titlepage}
\section{Experimental setup}

Past ammonia oxidation experiments on \ce{Pt_{25}Rh_{75}} (100) \parencite{Resta2020} single-crystals gave significant new insights on the relaxation dynamics when the system crosses the catalyst activation barrier.
The experiment focused on measuring before and after the catalyst light off temperature, it was found that the transition from a nitrogen-rich to oxygen-rich input mixture coincides with a relaxation change in the topmost metallic layer: moving from outward to inward, but only for temperatures above the light off.
Thanks to NAP-XPS experiments, nitrogen rich species (such as \nitrogen) were associated to positive relaxation of the topmost layer, expansion; while oxidative species (such as NO and oxygen) were linked to negative relaxation, contraction.
This change in relaxation has an impact on the band structure and therefore on the catalytic properties as proposed in the d-band model as detailed in sec. \ref{sec:Catalysis}.
It is therefore our intent to study if this modification in the relaxation is present only in the \ce{PtRh} alloy or if it has a more general value also on the constituent elements of the alloy to fully understand the impact of the Pt surface orientation on the catalyst selectivity.
Moreover, different relaxation state and transient structures are expected to exist as a function of the ammonia to oxygen ratio.

Moreover, we have recently probed the relaxation of single multi-faceted nanoparticles at SixS under the same conditions using Bragg Coherent Diffraction Imaging.
The 3D displacement and strain evolution of two different facets (\{100\} and \{111\}) will be compared to our results obtained on Pt 100 and 111 crystal surfaces.

\section{Crystal structures}

Platinum crystallizes in a face-centered cubic structure with a lattice parameter $|\vec{a}|$ equal to \qty{3.9254}{\angstrom} at room temperature.
The structure of the [001] and [111] planes differ greatly as shown in fig. \ref{fig:Cubic100Hex111}.
The shortest distance between the atoms on the [001] and [111] surfaces is $a/\sqrt{2} = \qty{2.78}{\angstrom}$.
However, the atoms on the [001] plane still follow a cubic structure whereas the atoms on the [111] plane follow a hexagonal structure.

\begin{figure}[!htb]
    \centering
    \includegraphics[width=0.49\textwidth]{/home/david/Documents/PhD/Figures/introduction/100.png}
    \includegraphics[width=0.49\textwidth]{/home/david/Documents/PhD/Figures/introduction/111.png}
    \caption{
        Face-centered cubic unit cell of Pt with two [001] crystallographic planes drawn in green at z=0 and z=1 (left).
        The $\vec{a}$, $\vec{b}$, and $\vec{c}$ vectors are used classically to decribe the cubic structure, the $\vec{a}_s$ and $\vec{b}_s$ in-plane vectors separated by \ang{90} and of magnitude \qty{2.78}{\angstrom} can also be used specifically to describe the structure of the [001] planes.
        Face-centered cubic unit cell of Pt with $[111]$ crystallographic plane drawn in green (right).
        The arrangement of the Pt atoms on the $[111]$ crystal planes is hexagonal, which leads to a new definition of the unit cell with the $\vec{a}$ and $\vec{b}$ in-plane vectors separated by \ang{60} and of magnitude \qty{2.78}{\angstrom} and $\vec{c}$ perpendicular to the $[111]$ plane of magnitude \qty{6.81}{\angstrom}.
        There are three $[111]$ planes spanned by the magnitude of $\vec{c}$ (blue, red and green on the figure).
    }
    \label{fig:Cubic100Hex111}
\end{figure}

\subsection{XPS results} \label{sec:XPS111}

On the Pt-Rh alloy an important body of work has been
already carried out, revealing two oxide structures present in the temperature and
pressure ranges we intend to study [6].
It is our goal to retrieve the surface moieties on PtRh nanoparticles as function of the reactive atmosphere’s composition.
The particles are lithography patterned on the sapphire surface, allowing to retrieve each particle position across different setups.
Through Bragg coherent diffraction imaging (BCDI) particles will be studied under reactive condition, then we plan to measure those very same particles at B07. The BCDI measurements give the possibility to follow the particle internal stresses as a function of the gas composition and the behavior of each facet at the same time as difference to the single crystal approach that can probe only one facet at the time.
The B07 setup is sensitive to the surface moieties on the particle across the different for each gas condition.
The colored insert in In Figure 1 is one particle from lithographed sample in operando condition: SixS beamline data (SOLEIL synchrotron).
The particles are sub-micron, like the grains in the metallic gauzes industrially used for this reaction. We plan to expose the particles to five ammonia to oxygen ratios, spanning from oxygen rich to oxygen poor conditions.

The goal of the experiment is to provide new insights on the PtRh catalysts during the NH3 oxidation.
Core level photoemission spectroscopy provides different signatures for many moieties adsorbed on the surface during reactions and therefore the possibilityto strengthen the links between structure, surface moieties and reaction products.
Our recently published results from both surface x-ray diffraction experiments (SXRD) and NAP-XPS [7] link surface relaxation to changes in surface moieties.
The intent of the experiment is to extent such concept to a particle and link internal stresses/surface-relaxations obtained from BCDI with the chemistry on its surfaces.

The B07 NAP-PXS end station can control the xyz positioning of the sample with a precision in the order of 10um, allowing to position the 50um beam[5] in the center of
any of the 100x100um squares of the sample.
At the center of each square is located one isolated particle.
This approach allows to study single particle behavior as function of the NAP-XPS atmosphere but also to retrieve the very same particle studied with BCDI on the SixS beamline.
As for the experiment in Ref. [7], we intent to apply the same temperatures and pressures for BCDI and NAP-XPS.
Temperatures and gas pressures will range between 450, and 650 K and from 0.5 to few tens of mBar respectively.
For each combination of temperature and ammonia to oxygen ratio, we will measure: O1s, N1s, Pt4d, Pt4f and Rh3d core levels that should carry information of the surface state.
Essential for the experiment is also to collect the gas phase composition for each condition to compare particles reactivity.

We expect to establish a connection between selectivity and surface moieties but also between surface moieties and particle internal stress. We also hope to be able to detect variations in the rhodium surface concentration related to the imposed gas conditions

B07 and its recent ambient pressure setup [5], that has now benefit from years of experience, push the high-pressure limits well into the mBar regime.
This new limit and the wide range of achievable temperatures/energies/pressures make it a unique tool for modern surface science and catalysis.
Timewise we expect to be able to characterize the nanoparticle(s) of interest as soon as they entered the analysis chamber from atmosphere and take reference scans after few oxidation/reduction cycles.
Reference scans will be taken on “as inserted” samples and on freshly oxidized/reduced particles with and without adsorbates reactants/products.
Similar scan will also be collected for the bare sapphire substrate. It is estimated that the first 2/3 days will suffice to collect references.
The operando experiments will be performed during the remaining 3/4 days of the beamtime.

\subsubsection{Experimental setup}

Since the XPS measurements were performed at different total pressures, the raw data had first to be reduced in order to analyse the different spectra in a systematic way.
The adopted workflow for the analysis of XPS data is to first align the recorded spectra on the Fermi edge that corresponds to the kinetic energy of the first electron that escapes the sample.
By doing so, one can be confident that any shift in the peak positions is due to chemical changes, such as the oxidation state of the sample, and not to charging effects of the sample (cite).

Secondly, to be able to quantify and compare the evolution of the peak intensity, one must normalize the intensity of the detected electron beam since the electron mean free path depends on the pressure in the reaction chamber \parencite{Willmott}.
The range of kinetic energy just before the absorption edge of Pt 4f was chosen since it had the best signal to noise ratio and does not depend on any experimental parameter besides the pressure.

Finally, for the peaks that showed a good signal to noise ratio, the fitting of the peak shape was realised thanks to the \textit{lmfit} \parencite{Newville2016} package by the means of the Doniach-equation which is the best approximation of the asymmetric peak shape resulting from the convolution of the analyser function and the photoelectron process in metals \parencite{Doniach_1970}.

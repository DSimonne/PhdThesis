\section{Pt(100) single crystal studied at \qty{450}{\degreeCelsius}} \label{sec:SXRD100}

A similar experiment was carried out on a different surface of platinum, namely the Pt(100) surface.
The arrangement of the Pt atoms on the (100) surface is square, the distance between in-plane neighbouring Pt atoms is smaller than between out-of-plane atoms.
A surface unit cell must be derived, shown in fig. \ref{fig:SurfaceUnitCellPt100}, to represent the surface arrangement of the Pt atoms with the smallest unit vectors possible.
The in-plane vectors $\vec{a}_{(100)}$ and $\vec{b}_{(100)}$ are of equal magnitude ($a_{Pt} / \sqrt{2} = \qty{2.775}{\angstrom})$, separated by \ang{90}.
The out-of-plane vector $\vec{c}_{(100)}$ is perpendicular to the (100) plane, and of magnitude $a_{Pt} = \qty{3.9242}{\angstrom}$.

\begin{SCfigure}
    \centering
    \includegraphics[trim=0 2cm 0 2cm, clip, width=0.70\textwidth]{/home/david/Documents/PhD/Figures/introduction/100.pdf}
    \caption{
        Face-entered cubic unit cell of Pt with $(100)$ crystallographic plane drawn in green.
        $\vec{a}_{(100)}$, $\vec{b}_{(100)}$ and $\vec{c}_{(100)}$ are the $(100)$ surface unit cell vectors.
    }
    \label{fig:SurfaceUnitCellPt100}
\end{SCfigure}

Reciprocal space in-plane maps were collected using the same experimental setup and atmospheres as for Pt(111), detailed in tab. \ref{tab:ConditionsSXRD}, to probe the structural evolution of the sample during ammonia oxidation.
Considering the square symmetry in the position of the Bragg peaks, the in-plane reciprocal space maps were collected by rotating the in-plane sample and detector angles ($\omega$ and $\gamma$) from \ang{0} to \ang{90} to collect a quarter of the reciprocal space in the ($\vec{q}_x$, $\vec{q}_y$) plane

The reciprocal space in-plane map were computed in both $q$-space and ($hkl$)-space to visualise the arrangement of surface structures or surface relaxations in comparison with the structure of the Pt atoms on the (100) surface, the $h$ and $k$ values being computed using the square surface unit cell of the Pt(100) surface.

\subsection{Oxide growth under \qty{80}{\milli\bar} of oxygen}

\begin{figure}[!htb]
    \centering
    \includegraphics[width=0.495\textwidth]{/home/david/Documents/PhDScripts/SixS_2022_01_SXRD_Pt100/figures/Map_hkl_surf_or_1335-1375.pdf}
    \includegraphics[width=0.495\textwidth]{/home/david/Documents/PhDScripts/SixS_2022_01_SXRD_Pt100/figures/Map_hkl_surf_or_1596-1635_patched.pdf}
    \caption{
        Reciprocal space in-plane maps collected under different atmospheres measured at \qty{450}{\degreeCelsius}, computed using the surface lattice of Pt(100).
    }
    \label{fig:MapsPt100A}
\end{figure}

The first map was collected under \qty{500}{\milli\bar} of Ar, after the cleaning of the sample by sputtering and annealing (fig. \ref{fig:MapsPt100A} - a).
The (200), (1$\bar{1}$0) and (0$\bar{2}$0) Bragg peaks can be observed, together with the bottom of the [0, 1, L] and [0, $\bar{1}$, L] CTRs (circled in white).

The sample was then first exposed to \qty{80}{\milli\bar} of oxygen, while keeping the total pressure to \qty{500}{\milli\bar} by compensation with argon (fig. \ref{fig:MapsPt100A} - b)
The signals generated by the rich oxygen condition can be divided into two groups, identified by red and green circles.
The red peaks are situated approximately at [H, K] = [1, -0.5], [H, K] = [1, -1.5], [H, K] = [0.5, $\bar{1}$] and [H, K] = [0.5, -1.5].
These signal follow a Pt(100)-p(2x2)-R\ang{0} arrangement, keeping in mind that it is related to surface unit cell if Pt(100).
A p(2x2) reconstruction of the Pt(100) surface has been reported before at an oxygen pressure of \qty{1e-3}{\milli\bar} by the use of environmental TEM \parencite{Li2016}, much lower than in this experiment.

The green peaks are slightly shifted from the red peaks by the same amount $\delta = 0.07$ in either H or K.
No simple surface unit cell could be derived from their in-plane positions.

Two out-of-plane measurements were performed on each group of signal, up to $L=3.5$, to probe the related out-of-plane structure.
The background-subtracted intensity was integrated as a function of $L$ using the \textit{fitaid} module of \textit{binoculars} (fig. \ref{fig:LScansHighOxygenPt100}), following the same kind of procedure performed on Pt(111).

\begin{figure}[!htb]
    \centering
    \includegraphics[width=\textwidth]{/home/david/Documents/PhDScripts/SixS_2022_01_SXRD_Pt100/figures/l_scans_high_oxygen_no_map.pdf}
    \caption{
        Out-of plane measurements for four different positions under \qty{80}{\milli\bar} of \ce{O_2}, and \qty{420}{\milli\bar} of \ce{Ar}.
    }
    \label{fig:LScansHighOxygenPt100}
\end{figure}

The intensity as a function of $L$ for the shifted peaks was found to quickly decrease down to zero, which shows they do not correspond to 3D structures but are more characteristic of monolayers with no out-of-plane periodicity (simulation example in fig. \ref{fig:SimROD}, Robinson et al. \cite*{Robinson1991}).

However, four peaks at different $L$ values are visible on both $L$-scans corresponding to the Pt(100)-p(2x2)-R\ang{0} arrangement, including at $L=0$, which is characteristic of a bulk structure, \textit{i.e.} more than a few unit cells thick (simulation example also in fig. \ref{fig:SimROD}, Robinson et al. \cite*{Robinson1991}).
The peaks were fitted using a model with four Gaussian peaks and a constant background, the same full width at half maxima was used for all four peaks since they are assumed to be linked to the same structure (fig. \ref{fig:FitPt100LScans}).
A $r^2$ value of \num{0.86} and \num{0.98} was reached for the $L$-scans at [H, K] = [0.5, $\bar{1}$], and [H, K] = [0.5, -1.5], respectively.
In the second fit, the maximum value of each peak is better represented which explains the different in the $r^2$ values.

Corresponding interplanar spacings were computed from the position of the peaks, which were found to coincide with a slightly distorted cubic unit cell of in-plane lattice parameter equal to \qty{5.60}{\angstrom}, and out-of-plane lattice parameter equal to \qty{5.64}{\angstrom}, used to index each Bragg peak in fig. \ref{fig:FitPt100LScans}.

\begin{figure}[!htb]
    \centering
    \includegraphics[width=\textwidth]{/home/david/Documents/PhDScripts/SixS_2022_01_SXRD_Pt100/figures/ctr_reconstructions_fitting_result}
    \caption{
        Fit result for the two $L$-scans performed at $[H, K] = [0.5, \bar{1}]$ and $[H, K] = [1, -1.5]$.
        The different peaks were indexed using a distorted cubic structure, corresponding to bulk \ce{Pt_3O_4}.
    }
    \label{fig:FitPt100LScans}
\end{figure}

\ce{Pt_3O_4} has a cubic structure (space group $P_{m3n}$) with a lattice parameter equal to \qty{5.65}{\angstrom} \parencite{Galloni1941, Galloni1952, Muller1968}, more recent studies have proposed a theoretical value of \qty{5.59}{\angstrom} \parencite{Seriani2006}, already presented in tab. \ref{tab:PtOxides}, similarly to the parameters reported in this study.
Moreover, the (212) and (232) reflections are not allowed which is also validated in fig. \ref{fig:FitPt100LScans}.
The 3D structure of bulk \ce{Pt_3O_4} is presented in fig. \ref{fig:Pt3O4}.

\begin{SCfigure}
    \centering
    \includegraphics[trim=0 1cm 0 1cm, clip, width=0.35\textwidth]{/home/david/Documents/PhD/Figures/introduction/Pt3O4.pdf}
    \caption{
        \ce{Pt_3O_4} bulk unit cell.
        Platinum atoms are situated on the faces on the cubic unit cell (e.g. $(0, 1/2, 1/4)$, $(0, 1/2, 3/4)$), while the eight oxygen atoms are inside the unit cell at the positions $(1/4, 1/4, z)$, $(1/4, 2/4, z)$, $(2/4, 1/4, z)$, $(2/4, 2/4, z)$ for $z=1/4$ and $z=3/4$.
    }
    \label{fig:Pt3O4}
\end{SCfigure}

The epitaxial relationship verifies \ce{Pt_3O_4}[001]||Pt[001] between the Pt(100) surface and \ce{Pt_3O_4}.
The lattice parameter of \ce{Pt_3O_4} is approximately twice that of the Pt(100) surface lattice parameter, which allows the formation of a cubic on cubic coherent interface.
The nature of the atoms at the interface is addressed later in this chapter.
The corresponding Wood notation is Pt(100)-p(2x2)-R\ang{0}.
Assuming that the in-plane lattice parameter of \ce{Pt_3O_4} is the same at the interface, the interfacial strain between both lattice is equal to \qty{0.85}{\percent} (eq. \ref{eq:StrainDiffraction}).
Such tensile strain can be the reason for the distorted cubic structure observed in this study.
To match perfectly with the Pt(100) surface, the in-plane lattice parameter of \ce{Pt_3O_4} would need to reach the value of $2\times2.775=\qty{5.55}{\angstrom}$.

The possibility to perform out-of-plane measurements with SXRD allows us to link this reconstruction with certainty to \ce{Pt_3O_4}.

A first estimate of the oxide layer thickness can be obtained by measuring the FWHM $\sigma$ of each peak in the $\vec{z}$ direction, and using  the equation $2\pi/\sigma$ \parencite{Warren1990}.
A thickness of \qty{62.65}{\angstrom} and \qty{60.18}{\angstrom} is found for the first and second measurement.
For bulk \ce{Pt_3O_4} this would correspond to approximately \num{11} unit cells.

\subsection{Near ambient pressure ammonia oxidation cycle}

\begin{figure}[!htb]
    \centering
    \includegraphics[width=0.495\textwidth]{/home/david/Documents/PhDScripts/SixS_2022_01_SXRD_Pt100/figures/Map_hkl_surf_or_1880-1902_patched.pdf}
    \includegraphics[width=0.495\textwidth]{/home/david/Documents/PhDScripts/SixS_2022_01_SXRD_Pt100/figures/Map_hkl_surf_or_1930-1936_patched.pdf}
    \caption{
        Reciprocal space in-plane maps collected under different atmospheres measured at \qty{450}{\degreeCelsius}, computed using the surface lattice of Pt(100).
    }
    \label{fig:MapsPt100B}
\end{figure}

The combination of high temperature, oxygen and ammonia define a very harsh environment that can result in the oxidation of the screws responsible for the contact on the sample holder.
Fig \ref{fig:MapsPt100B} (a) shows a map in which the heating system failed after the introduction of ammonia in the reactor, reaching reacting conditions.
Nevertheless, half of the large reciprocal space in-plane maps could be measured, the red circled peaks corresponding to \ce{Pt_3O_4} are still visible, showing that \ce{Pt_3O_4} is not immediately removed during reacting condition.
The two other peaks at (0.93, -0.5) and (0.5, -0.93) have disappeared, the related structure is not stable during the reaction.

After repair of the sample heater, the sample was cleaned by sputtering and annealing cycles, and replaced in the reactor.
In order to probe for the reproducibility of the \ce{Pt_3O_4} growth on the surface, the partial pressure of oxygen was set to \qty{80}{\milli\bar}, while measuring a small area of the reciprocal space to detect the same peaks as in fig. \ref{fig:MapsPt100A} (b).
Such peaks could indeed be detected as shown in fig. \ref{fig:MapsPt100B} (b), but with a shift between the green and red groups of peaks equal to $0.09$, more important than the previously measured value of $0.07$.
The order of the second structure is possibly linked to the exposure to oxygen atmosphere, the smaller in-plane map was collected faster, and directly after the introduction of oxygen in the reactor.

Ammonia was subsequently introduced in the cell to probe the relation between surface structure and selectivity during ammonia oxidation.

\begin{figure}[!htb]
    \centering
    \includegraphics[width=0.53\textwidth]{/home/david/Documents/PhDScripts/SixS_2022_01_SXRD_Pt100/figures/Map_hkl_surf_or_1953-1981_patched.pdf}
    \includegraphics[width=0.46\textwidth]{/home/david/Documents/PhDScripts/SixS_2022_01_SXRD_Pt100/figures/l_scans_high_oxygen_ammonia_no_map.pdf}
    \caption{
        Reciprocal space in-plane maps collected under different atmospheres measured at \qty{450}{\degreeCelsius}, computed using the surface lattice of Pt(100).
        Out-of plane measurements for four different positions under \qty{80}{\milli\bar} of \ce{O_2}, \qty{10}{\milli\bar} of \ce{NH_3}, and \qty{410}{\milli\bar} of \ce{Ar}.
    }
    \label{fig:MapsAndLScansPt100HighOxAmmonia}
\end{figure}

Signals separated by $0.1$ in H or in K were observed around the platinum Bragg peaks in a square arrangement, compatible with a Pt(100)-p(10x10)-R\ang{0} surface reconstruction.
Most of the peaks are observed at $H={0,1,2}$ or $K={0, -1}$ (fig. \ref{fig:MapsAndLScansPt100HighOxAmmonia}).
The other features are observed in rows or columns \textit{shifted} by $0.09$ with respect to the (1, $\bar{1}$, 0) Bragg peak, following the shift of the green circled peaks observed while only oxygen is present in the cell (fig. \ref{fig:MapsPt100B} - b).

Out-of-plane measurements were performed at the positions of four peaks, presented in fig. \ref{fig:MapsAndLScansPt100HighOxAmmonia} (b).
$L$-scans at [H, K] = [1.9, 0], [H, K] = [0.5,  $\bar{1}$], and [H, K] = [0, -1.2] are performed on the \textit{unshifted} Pt(100)-p(10x10)-R\ang{0} reconstructions (fig. \ref{fig:MapsAndLScansPt100HighOxAmmonia} - b).
The peaks measured earlier at [H, K] = [0.5, $\bar{1}$] at $L=0.7$, $L=1.4$, and $L=2.1$ corresponding to bulk \ce{Pt_3O_4} are no longer visible.
For [H, K] = [1.9, 0], two peaks are visible near $L=0.15$ and $L=2.15$.
The reconstructed signal effectively disappears near $L=1.9$, but reappears with a strong intensity at $L=2$, following the same decrease as a function of $L$ as from $L=0$.
For [H, K] = [0, -1.2], large oscillations are visible with a minimum at $L=1.2$, which could be the signature of a bi-layer structure (e.g. simulated \ce{Pt_3O4} in fig. \ref{fig:SimROD} - a).
Large oscillations are also visible for [H, K] = [0.5, -1], with a minimum between $L=1$ and $L=1.7$.

The $L$-scan at [H, K] = [0.5, -0.91] (fig. \ref{fig:MapsAndLScansPt100HighOxAmmonia} - b) is performed on the shifted Pt(100)-p(10x10)-R\ang{0} reconstructions.
The intensity decreases gradually with increasing $L$, characteristic of a monolayer, no signal can be detected above $L=2$.

It is not certain that the same thickness of \ce{Pt_3O_4} was present on the catalyst surface at the beginning of the reacting conditions.
Indeed, the sample was exposed to \qty{80}{\milli\bar} of oxygen for \qty{6}{\hour} before $L$-scans could be measured.
The second exposition only lasted for \qty{1}{\hour}, during which the same in-plane peaks could effectively be measured, but without out-of-plane information.
The thickness of \ce{Pt_3O_4} is very likely to be related to the time spent under only \qty{80}{\milli\bar} of \ce{O_2}.
However, it is clear that i) new signals appear during reacting conditions, related to structures that themselves differ depending on the thickness of \ce{Pt_3O_4}, ii) the bulk oxide cannot grow under reacting conditions.
Therefore, different types of structures coexist on the Pt(100) surface when the \ce{O_2}/\ce{NH_3} ratio is equal to \num{8}, possibly linked to different adsorbed species participating in the catalytic reaction.

\begin{figure}[!htb]
    \centering
    \includegraphics[width=0.53\textwidth]{/home/david/Documents/PhDScripts/SixS_2022_01_SXRD_Pt100/figures/Map_hkl_surf_or_2227-2283_patched.pdf}
    \includegraphics[width=0.46\textwidth]{/home/david/Documents/PhDScripts/SixS_2022_01_SXRD_Pt100/figures/l_scans_low_oxygen_ammonia.pdf}
    \caption{
        Reciprocal space in-plane maps collected under different atmospheres measured at \qty{450}{\degreeCelsius}, computed using the surface lattice of Pt(100).
        Out-of plane measurements for four different positions under \qty{5}{\milli\bar} of \ce{O_2}, \qty{10}{\milli\bar} of \ce{NH_3}, and \qty{485}{\milli\bar} of \ce{Ar}.
    }
    \label{fig:MapsAndLScansPt100LowOxAmmonia}
\end{figure}

Lowering the partial pressure of oxygen from \qty{80}{\milli\bar} to \qty{5}{\milli\bar} in the reactor has completely removed the Pt(100)-p(10x10)-R\ang{0} reconstruction phenomena (fig. \ref{fig:MapsAndLScansPt100LowOxAmmonia} - a).

The existence of co-existing domains with the same hexagonal arrangements on top on the Pt(100) surface is revealed, with an in-plane lattice parameter equal to \qty{2.685}{\angstrom}, \qty{\approx 3.36}{\percent} lower than the distance between neighbouring Pt atoms on the Pt(100) surface (\qty{2.775}{\angstrom}).
Second order peaks can also be seen at the edge of the reciprocal space in-plane map, each domain has one axis parallel to either $\vec{a}_{(100)}$ or $\vec{b}_{(100)}$, \textit{i.e.} respectively in the [011] and [01$\bar{1}$] directions.
The reconstruction can be described as follows: Pt(100)-p$\begin{pmatrix} 0.97 & 0.00\\ 0.84 & 0.48 \end{pmatrix}$, the closest commensurate structure found is Pt(100)-p(33x25)-R\ang{0}.
A sketch describing the real space unit cell can be found in appendix \ref{fig:Pt100UnitCellsReconstruction}.
% precursor to bulk is chemisorbed \parencite{BradleyShumbera2007}

Different hexagonal surface reconstructions on the Pt(100) surface have been reported also in the [110] direction at UHV conditions, summarised in Hammer et al. \parencite*{Hammer2016}, based on an important body of work \parencite{Heilmann1979, Vanhove1981, Heinz1982, Mase1992, Kuhnke1992, Borg1994, VanBeurden2004, Havu2010}.
They have been measured to evolve to rotated hexagonal reconstructions with angles between \ang{0.77} and \ang{0.94} depending on the sample temperature, and the previous temperature treatment.
The unit cell describing those reconstructions with respect to the Pt(100) surface varies, if first contained to (5XN) where N = 20–30, the latest study reports a commensurate c(26x118) superstructure.
Exposition of the rotated hexagonal structure to oxygen at \qty{450}{\degreeCelsius} studied by low energy electron diffraction (LEED) has been found to remove the hexagonal structure and precipitate the growth of surface oxides \textit{via} different phases \parencite{BradleyShumbera2007, BradleyShumbera2007a}, a similar conclusion was reached by \cite{Deskins2005} by DFT studies.
Exposition to \ce{NO} has been reported to stabilise the clean Pt(100) phase \parencite{Heinz1982}, while exposition to \ce{CO} removes the hexagonal reconstruction, an oscillatory behaviour between a clean Pt(100)-(1x1)-R\ang{0} surface and the rotated hexagonal reconstruction was reported by Cox et al. \parencite*{Cox1983}.

Subsurface oxygen has also been predicted to exist on Pt(100) \parencite{Gu2007}, reported at a pressure of \qty{0.133}{\milli\bar} \parencite{McMillan2005}, and participating in the catalytic oxidation of \ce{CO}.
It was identified as a precursor to a stable surface oxide during the catalytic oxidation of \ce{CO} by Dicke et al. \parencite*{Dicke2000}, at an oxygen pressure of \qty{0.09}{\milli\bar}.
The appearance of subsurface oxygen was linked to the lifting of surface reconstructions on the clean Pt(100) surface from the adsorption of \ce{CO}, which then allowed oxygen atoms to penetrate under the topmost layer of platinum \parencite{Rotermund1993, Lauterbach1994}.

Overall, few works at ambient pressure have been found to exist, even less during the catalytic oxidation of ammonia.
Out-of-plane measurements were performed at the position of the two peaks, also shown in fig. \ref{fig:MapsAndLScansPt100LowOxAmmonia}, no bulk structure could be detected.
A different behaviour is measured when lowering the \ce{O_2}/\ce{NH_3} ratio from \num{8} to \num{0.5}, which also has an impact on the product selectivity.
The partial pressure of \ce{NO} and \ce{N_2O} decreases quickly after the change of condition (fig. \ref{fig:RGA450Pt100Cycle}), whereas the partial pressure of \ce{N_2} is more stable.

\begin{figure}[!htb]
    \centering
    \includegraphics[width=0.495\textwidth]{/home/david/Documents/PhDScripts/SixS_2022_01_SXRD_Pt100/figures/Map_hkl_surf_or_2520-2570_patched.pdf}
    \includegraphics[width=0.495\textwidth]{/home/david/Documents/PhDScripts/SixS_2022_01_SXRD_Pt100/figures/Map_hkl_surf_or_2719-2767.pdf}
    \caption{
        Reciprocal space in-plane maps collected under different atmospheres measured at \qty{450}{\degreeCelsius}, computed using the surface lattice of Pt(100).
    }
    \label{fig:MapsPt100C}
\end{figure}

After removing oxygen from the reactor, only the first order reflections corresponding to the hexagonal structures could be detected during the measurement of the reciprocal space in-plane map (fig. \ref{fig:MapsPt100C} - a), the following measurement under Argon showed a clean Pt(100) surface (fig. \ref{fig:MapsPt100C} - b).

\subsection{Oxide growth under \qty{5}{\milli\bar} of oxygen}

\begin{figure}[!htb]
    \centering
    \includegraphics[width=0.495\textwidth]{/home/david/Documents/PhDScripts/SixS_2022_01_SXRD_Pt100/figures/Map_hkl_surf_or_2905-2953_patched.pdf}
    \includegraphics[width=0.495\textwidth]{/home/david/Documents/PhDScripts/SixS_2022_01_SXRD_Pt100/figures/Map_hkl_surf_or_3154-3169.pdf}
    \caption{
        Reciprocal space in-plane maps collected under different atmospheres measured at \qty{450}{\degreeCelsius}, computed using the surface lattice of Pt(100).
        The area covered in (b) is drawn by a black rectangle in (a).
    }
    \label{fig:MapsPt100D}
\end{figure}

To make sure that the hexagonal structure was related to the reaction conditions and not only to a lower pressure of oxygen in the reactor, a partial pressure of \qty{5}{\milli\bar} of oxygen was set.
The following measurement showed the presence of many peaks that appeared in the first hour of the measurement (fig. \ref{fig:MapsPt100C} - a)
A second map was measured directly after the end of the first map, in which none of the newly detected peaks could be detected (fig. \ref{fig:MapsPt100C} - c).
The colorbar was adjusted to highlight the presence of weak signals in the H and K directions in the second map (fig. \ref{fig:MapsPt100C} - c).

These two measurements showed first that the hexagonal structure observed under reacting conditions when the \ce{O_2}/\ce{NH_3} ratio is equal to \num{0.5} is linked to the \textit{simultaneous} presence of ammonia and oxygen in the reactor.
Secondly, the dynamics of some of the observed phenomena under \qty{5}{\milli\bar} of oxygen are too short to be effectively measured with the current time-resolution of in-plane reciprocal space maps.
Weak signals in the H and K directions can be seen in the second map (fig. \ref{fig:MapsPt100C} - c).

\subsection{Surface roughness and surface relaxation effects}

\begin{figure}[!htb]
    \centering
    \includegraphics[width=\textwidth]{/home/david/Documents/PhDScripts/SixS_2022_01_SXRD_Pt100/figures/ctr_a.pdf}
    \includegraphics[width=\textwidth]{/home/david/Documents/PhDScripts/SixS_2022_01_SXRD_Pt100/figures/ctr_b.pdf}
    \caption{
        Evolution of crystal truncation rods under different atmospheres.
    }
    \label{fig:CTRPt100}
\end{figure}

The ($\bar{1}\bar{1}L$), ($\bar{1}0L$) and ($\bar{2}1L$) crystal truncation rods have been measured \qty{6}{\hour} after the start of each condition, each measurement lasted for \qty{2}{\hour}.
The background-subtracted intensity of the CTR was integrated using the \textit{fitaid} module of \textit{binoculars} as a function of $L$ with the same integration range.
The CTR intensity is displayed in fig. \ref{fig:CTRPt100}, additional peaks are detected under exposition to \qty{80}{\milli\bar} of \ce{O_2}, confirming that bulk \ce{Pt_3O_4} is epitaxied on the Pt(100) surface in Pt(100)-p(2x2)-R\ang{0} arrangement.

Those peaks prevent us from resolving the minimal position of the CTR intensity, highly sensitive to surface relaxation effects.
The absence of oscillations in the evolution of the CTR intensity as a function of $L$ also shows that \ce{Pt_3O_4} is not homogeneously covering the Pt(100) substrate (such signal is presented in fig. \ref{fig:SimROD} - b).
This phenomena can instead be explained by \ce{Pt_3O_4} islands of different heights covering the substrate, the fringes linked to islands of one thickness being smeared out from the contribution of others.

The catalyst surface can be divided in three different components as a function of $\vec{z}$.
First, there is bulk Pt(100).
Secondly, there is the interface between bulk platinum and the different islands of \ce{Pt_3O_4}, in which the platinum atoms are expected to be displaced from their equilibrium positions due to the interaction with the oxide layer, and where there could also be the presence of some subsurface oxygen atoms.
Thirdly, there are the \ce{Pt_3O_4} islands, on top of the previous layers, with different heights.
A first approach to approximate the catalyst surface is discussed below.

\begin{figure}[!htb]
    \centering
    \includegraphics[width=\textwidth]{/home/david/Documents/PhDScripts/SixS_2022_01_SXRD_Pt100/figures/fit_8o2.pdf}
    \caption{
        Fitting results for crystal truncation rods collected under a \qty{80}{\milli\bar} of \ce{O_2}.
    }
    \label{fig:CTRFitHighOxygen}
\end{figure}

The structure factors $F_i$ resulting from the presence of one to nine unit cells of \ce{Pt_3O_4} on the Pt(100) surface was simulated with \textit{ROD}, the same interface with the substrate is used in each simulation.
The fitting routine consists in minimising the square root difference between the CTR structure factors $F_{obs}$, and the square root of the coherent sum of the squared simulated structure factors by adjusting the weight $W_i$ of each signal in the total signal $F_{calc}$ (eq. \ref{eq:Fcalc}).

\begin{equation}
    F_{calc} = \sqrt{\sum_{i=1}^{9} W_i F_i^2}
    \label{eq:Fcalc}
\end{equation}

The structure factor resulting from different domains are not allowed to coherently interfere since the distance between each domain may be larger than the beam coherence lengths.
In this first hypothesis, the topmost Pt(100) layers were assumed to be strain free.
Different $\beta$ roughness parameters (explained in fig. \ref{fig:CTRSimulation}) were used, the best fit was found to be with $\beta = \num{0.6}$.
The $\beta$ roughness parameter only lowers the intensity of the bulk signal and not that of the surface layers.
Moreover, it is assumed that \ce{Pt_3O_4} islands have the same out-of-plane and in-plane lattice parameter in each layer, independently of its thickness.
For example, the 7$^{th}$ unit cell in a \num{8} unit cell thick island has the same lattice parameters than the second unit cell in a \num{4} unit cell thick island.

The relation between each weight was adjusted by following a Gaussian distribution, expecting the different islands to not differ too much in height.
The result of the fitting routine is shown in fig. \ref{fig:CTRFitHighOxygen}, with a Gaussian distribution centred around $3.6$ unit cells, and with a standard deviation $\sigma$ equal to $1.3$ unit cells.
The presence and width of the \ce{Pt_3O_4} peaks is well adjusted, see for example the small peak at $L=3.5$ in the [4, 0, L] CTR.
The intensity shape near the Bragg peak is also quite accurate, but tends to fall too quickly.
This can be adjusted by lowering the $\beta$ roughness but low intensity regions becomes then very difficult to represent.
Indeed, the simulation struggles to accurately reproduce the intensity when below a certain threshold.
The position of some peaks is very well reproduced, but some peaks are a little shifted, such as the peak at $L=1.4$ in the [4, 0, L], and [2, $\bar{2}$, L] CTR.
The ratio of intensity between the \ce{Pt_3O_4} peaks is also not perfect, and could possibly be further adjusted by changing the nature of the Pt(100)||\ce{Pt_3O_4} interface, as well as the distance between the oxide and surface.
The positions of the Pt atoms at the interface are explained by a sketch in appendix \ref{fig:Pt3O4onPt100}.
The distance between the Pt atoms in the Pt(100) bulk and \ce{Pt_3O_4} was chosen equal to \num{0.5} unit cell lattice length following the spacing between Pt atoms in {100} planes.

There is a large difference when comparing the results with the average thickness of \qty{\approx 60}{\angstrom} given by fitting the $L$-scans in fig. \ref{fig:LScansHighOxygenPt100}, since each unit cell is expected to be \qty{\approx5.64}{\angstrom} thick.
Overall, the existence of bulk \ce{Pt_3O_4} islands is confirmed to remove the coherent fringes visible when a homogeneous layer is present of the sample surface (fig. \ref{fig:SimROD}).
\textcolor{Important}{try to give an explanation}

When observing the CTR intensity under other atmospheres, a clear evolution in the position of the minimum intensity between both reacting conditions (in red and green in fig. \ref{fig:CTRPt100}) is visible.
The CTR recorded after exposition to reacting conditions under ammonia or under argon have similar intensities and are almost indistinguishable.

All three CTR were fitted together using \textit{ROD} at each atmospheres besides the oxygen rich atmosphere.
Different models were tested, adding the possibility of in-plane lattice displacement did not show any improvement of the fit quality, and a simple surface model was kept consisting of two Pt(100) layers, each sharing an out-of-plane lattice displacement parameter, on top of bulk Pt(100).
The fitting results are shown in fig. \ref{fig:CTRFit100}.

\begin{figure}[!htb]
    \centering
    \includegraphics[width=\textwidth]{/home/david/Documents/PhDScripts/SixS_2022_01_SXRD_Pt100/figures/fit_comparison.pdf}
    \caption{
        Fitting results for roughness parameter $\beta$ (a) and out-of-plane strain $\sigma_z$ (b) as a function of the experimental conditions.
        The innermost layer is named $A$, which has the same lattice parameter as in the bulk.
        The second and first topmost layers of the Pt(100) single crystal are respectively named $B$ and $C$.
    }
    \label{fig:CTRFit100}
\end{figure}

The surface roughness evaluated \textit{via} the $\beta$ parameter in fig. \ref{fig:CTRFit100} (a) is consistent with the evolution of the CTR intensity in fig. \ref{fig:CTRPt100}.
The initial value for the surface roughness is equal to \num{0.44} when exposed to a total pressure of argon equal to \qty{500}{\milli\bar}.
The average roughness after introducing a partial pressure of \qty{80}{\milli\bar} of oxygen was estimated to increase to \num{0.6}, coherent with the creation of a bulk oxide layer on the sample surface.
The minimal intensity can be seen to decrease after introduction of oxygen, see for example at $L=1$ for the [$\bar{1}$, $\bar{1}$, L] CTR in fig. \ref{fig:CTRPt100}.

The following crystal truncation rods having been measured after the second introduction of ammonia in the cell, \textit{i.e.} after cleaning of the sample and a short exposition to a high oxygen atmosphere.
The roughness under reacting conditions, when observing the Pt(100)-p(10x10)-R\ang{0} reconstructions (fig. \ref{fig:MapsAndLScansPt100HighOxAmmonia}), decreases only slightly in comparison to having only oxygen in the cell, and is still more important than under inert atmosphere.
Lowering the amount of oxygen in the cell decreases the surface roughness, coming back to the initial value.
In this condition, the hexagonal reconstructions are observed (fig. \ref{fig:MapsAndLScansPt100LowOxAmmonia}).

The lowest roughness value is reached when only ammonia is present in the reactor, without oxygen.
The hexagonal reconstruction was observed to slowly disappear (fig. \ref{fig:MapsPt100C}), which could mean that the presence of different surface reconstruction are linked to increased roughness on the sample surface.
Removing ammonia does not change the surface roughness, which shows that oxygen plays a key role in the increased surface roughness.

Finally, the presence of \qty{5}{\milli\bar} of oxygen after the ammonia oxidation cycle increases the surface roughness to higher values, almost equal to the maximum value reached under the presence of \qty{80}{\milli\bar} of oxygen.
Bulk \ce{Pt_3O_4} was not observed in this condition, small peaks were visible linked to transient structures (fig. \ref{fig:MapsPt100D}).

% strain
The strain of the second topmost layer, ($B$ in fig. \ref{fig:CTRFit100} - b) is always very close to \qty{0}{\percent} besides interestingly under \qty{5}{\milli\bar} of oxygen.
The strain of the topmost layer ($C$ in fig. \ref{fig:CTRFit100} - b) is always compressive and seen to increase during reacting conditions in comparison with the initial values under argon atmosphere.
A slight increase is reported when changing the \ce{O_2}/\ce{NH_3} ratio from \num{8} to \num{0.5}.
Keeping only ammonia in the reactor after the oxidation reaction brings the lowest strain observed, both layers almost exhibit the bulk lattice parameters.

The presence of ammonia can be associated with reduced roughness and strain, whereas the presence of oxygen is associated with increased roughness and strain.
Different reacting conditions, which can be linked to a different selectivity in the product, show in larger difference in the surface roughness than in the surface strain.
For example, when observing the [$\bar{1}$, 0, L] CTR in fig. \ref{fig:CTRFit100}, the local minima positions do not show a significant change.
However, the change in intensity indicates different surface roughness.
The shape of the [$\bar{2}$, 1, L] CTR when the \ce{O_2}/\ce{NH_3} ratio is equal to \num{0.5} is interesting.
The intensity minimum at $L=2.1$ is very low, whereas the decrease of intensity around the Bragg peaks is not very strong, in comparison with the same CTR but when the \ce{O_2}/\ce{NH_3} ratio is equal to \num{8}.
It is possible that low intensity peaks are present near $L=1.8$ or $L=2.5$ that would explain this shape, linked to the hexagonal reconstruction observed in fig. \ref{fig:MapsAndLScansPt100LowOxAmmonia}.

\subsection{Surface species presence}

In order to link surface structure, surface moieties and reaction products, the Pt 4f, N 1s and O 1s XPS spectra were recorded at near ambient pressure at the B07 beamline (Diamond synchrotron), at \qty{450}{\degreeCelsius}.
The same order in the ammonia oxidation cycle was repeated as during the SXRD experiment, with the same ratio between reaction products.
No carrier gas is used to keep the total pressure constant, the reactant pressure is lowered to \qty{11}{\percent} of the pressure during the SXRD experiment as a compromise for surface photoelectron detection.
The conditions have been resumed in tab. \ref{tab:ConditionsXPS}.
The mass spectrometer available at the B07 beamline allows us to monitor the presence of the reactants and products close to the sample surface.
The pressure of gases going through the same aperture of the electron analyser is measured, presented in fig. \ref{fig:XPS100RGA}.

\begin{figure}[!htb]
    \centering
    \includegraphics[width=\textwidth]{/home/david/Documents/PhDScripts/B07_2022_04_XPS/Figures/pt100_time.pdf}
    \caption{
        Evolution of reaction product partial pressures as a function of time during the XPS experiment on the Pt(100) single crystals at \qty{450}{\degreeCelsius}.
        Transition between conditions are indicated with dashed vertical lines.
    }
    \label{fig:XPS100RGA}
\end{figure}

% Describe rga
Overall, the same products are observed as with Pt(111).
A high \ce{O_2}/\ce{NH_3} ratio equal to \num{8} favours the production of \ce{NO} as expected, accompanied by a high amount of water
The production of \ce{NO} is higher than for Pt(111), no pressure of \ce{N_2} and \ce{N_2O} can be measured, showing a higher selectivity of the Pt(100) surface towards \ce{NO} that could not be measured from the RGA in SXRD experiments.
Approximately half of the pressure of ammonia is still detected, which means that the oxygen cannot be considered to be in excess, and that the complete oxidation of ammonia is probably limited by the availability of active sites.

Lowering the amount of oxygen by reducing the \ce{O_2}/\ce{NH_3} ratio to \num{0.5} has the remarkable effect of shifting the reaction selectivity entirely towards \ce{N_2}, just like for the Pt(111) surface.
Water is also detected.
\ce{H_2} coming from the simultaneous dissociation of \ce{NH_3} can be measured, not observed under a higher pressure of oxygen, which means that this reaction is not favoured when oxygen is present in the reactor.
Oxygen being undetected by the mass spectrometer when the \ce{O_2}/\ce{NH_3} ratio is equal to \num{0.5}, all of the introduced oxygen dissociates on the catalyst surface and participates in the production of \ce{N_2} and \ce{H_2O} via the oxidation reaction.
Ammonia can be thus considered to be in excess, and partly decomposing towards \ce{N_2}.
The surface sites are probably occupied mainly by nitrogen-rich species that cannot find a nearby oxygen or OH to react with, eventually decomposing towards nitrogen.

The removal of oxygen shows that more ammonia is consumed but without producing water.
Only the dissociation of ammonia happens on the catalyst surface, the production of nitrogen decreases even though more ammonia is consumed, but less than for the Pt(111) surface.
It is not clear why more nitrogen is produced under the presence of oxygen, when more ammonia is used for less production of \ce{N_2} after the removal of oxygen.

\subsubsection{N 1s and O 1s levels}

N 1s and O 1s levels were recorded to probe for the existence of specific surface species, allowing us to obtain more information about the reaction mechanism, and the link between surface state and selectivity.
The evolution of the N 1s and O 1s XPS spectra for different atmospheres is presented in fig. \ref{fig:O1sN1sPt100}.
Binding energy are given with reference to the Fermi level, all the reported peaks and corresponding species are detailed in tab. \ref{tab:XPSPt100}.

During the study of the oxidation of the Pt(100) by Derry et al. \parencite*{Derry1984}, O 1s peaks between \qty{530.4}{\eV} and \qty{530.5}{\eV} are reported as a function of the oxygen surface coverage.
Sugai et al \parencite*{Sugai1993} report a low intensity peak at \qty{531.5}{\eV} during the dissociation of \ce{NO} at \qty{400}{\degreeCelsius}, which they cannot specifically assign to either \ce{NO_a} or \ce{O_a}.
Different types of adsorbed oxygen species are observed during the dissociation of \ce{NO} by Rienks et al. \parencite*{Rienks2003} at \qty{527.7}{\eV} and \qty{529.7}{\eV}, the latter only shortly distant from \ce{NO_a} reported at \qty{529.8}{\eV}.
Kondratenko et al. \parencite*{Kondratenko2006} report \ce{O_a} and \ce{OH_a} at \qty{529.7}{\eV} and \qty{532}{\eV} respectively during the decomposition of \ce{N_2O} after exposition to \ce{H_2}, based on previous works by Wild et al. \parencite*{Wild2000}.

\begin{figure}[!htb]
    \centering
    \includegraphics[width=\textwidth]{/home/david/Documents/PhDScripts/B07_2022_04_XPS/Figures/Pt100/O1sN1s_700.pdf}
    \caption{
        Spectra collected at the O 1s (a) and N1 s (b) levels under different atmospheres at \qty{450}{\degreeCelsius} with an incoming photon energy of \qty{700}{\eV}.
        The spectra are normalised by the pre-edge intensity and shifted to highlight the presence of different peaks.
    }
    \label{fig:O1sN1sPt100}
\end{figure}
\begin{table}[!htb]
\centering
\resizebox{\textwidth}{!}{%
    \begin{tabular}{@{}ll|lllllll@{}}
    \toprule
    \multirow{3}{*}{Partial pressures (\unit{\milli\bar})} & \ce{Ar}   & 1 & 0   & 0   & 0    & 0   & 0    & 1    \\
                                              & \ce{NH_3} & 0 & 0   & 1.1 & 1.1  & 1.1 & 0    & 0    \\
                                              & \ce{O_2}  & 0 & 8.8 & 8.8 & 0.55 & 0   & 0.55 & 0    \\
    \midrule
    Gas presence (decreasing & & Ar & \ce{O_2} & \ce{O_2}, \ce{H_2O}, \ce{NO}   & \ce{H_2O}, \ce{NH_3} & \ce{H_2}, \ce{NH_3} & \ce{O_2} & Ar \\
    pressure order)          & &    &          & \ce{NH_3}, \ce{N_2}, \ce{N_2O} & \ce{N_2}, \ce{H_2}   & \ce{N_2}            &          &    \\
    \midrule
    \multicolumn{2}{l|}{N 1s: peak positions}
        & \qty{403.6}{\eV} & No peak          & \qty{404.3}{\eV} & \qty{404.3}{\eV} & \qty{404.9}{\eV} & No peak          & No peak        \\
     &  &                  &                  &                  & \qty{400.0}{\eV} & \qty{400.6}{\eV} &                  &                \\
    \multicolumn{2}{l|}{Attributed surface species}
        & Not assigned     &                  & \ce{N_{2,g}} \& \ce{NO_g} & \ce{N_{2,g}}     & \ce{N_{2,g}}     &                  &       \\
     &  &                  &                  &                           & \ce{N_a}         & \ce{NH_{x,a}}    &                  &       \\
    \midrule
    \multicolumn{2}{l|}{O 1s: peak positions}
        & \qty{531.6}{\eV} & \qty{538.3}{\eV} & \qty{538.4}{\eV} & \qty{534.3}{\eV} & \qty{532.4}{\eV} & \qty{538.3}{\eV} & \qty{532}{\eV} \\
     &  &                  & \qty{537.2}{\eV} & \qty{537.3}{\eV} & \qty{532.1}{\eV} &                  & \qty{537.2}{\eV} &                \\
     &  &                  & \qty{529.7}{\eV} & \qty{533.7}{\eV} &                  &                  & \qty{531.4}{\eV} &                \\
     &  &                  &                  & \qty{529.7}{\eV} &                  &                  & \qty{530.6}{\eV} &                \\
     &  &                  &                  &                  &                  &                  & \qty{529.7}{\eV} &                \\
    \multicolumn{2}{l|}{Attributed surface species}
        & \ce{H_2O_a}      & \ce{O_{2,g}}     & \ce{O_{2,g}}     & \ce{H_2O_g}      & \ce{H_2O_a}      & \ce{O_{2,g}}     & \ce{H_2O_a}    \\
     &  &                  & \ce{O_{2,g}}     & \ce{O_{2,g}}     & \ce{H_2O_a}      &                  & \ce{O_{2,g}}     &                \\
     &  &                  & \ce{O_a}         & \ce{H_2O_g}      &                  &                  & \ce{O_a}         &                \\
     &  &                  &                  & \ce{O_a}         &                  &                  & \ce{O_a}         &                \\
     &  &                  &                  &                  &                  &                  & \ce{O_a}         &                \\
    \bottomrule
    \end{tabular}%
    }
    \caption{Indexing of peaks measured during ammonia oxidation of the Pt(100) surface.}
\label{tab:XPSPt100}
\end{table}

% argon

% high ox
The Pt(100) surface was first exposed to \qty{8.8}{\milli\bar} of oxygen after argon.
The peaks at \qty{538.3}{\eV} and \qty{537.2}{\eV} can safely be attributed to \ce{O_{2,g}}, the positions are shifted in energy with respect to literature \parencite{Avval2022}.
By comparing with the reported literature values, the large peak at \qty{529.7}{\eV} is probably due to atomic oxygen adsorbed in different sites, explaining the wide peak shape.
No stable surface of bulk oxide was observed in SXRD at an oxygen pressure of \qty{5}{\milli\bar} which reduces the possibility of surface oxide peaks in the O 1s level.
\ce{Pt_3O_4} was only observed under \qty{80}{\milli\bar} of oxygen.
Some transient peaks (fig. \ref{fig:MapsPt100C} - c) were measured with SXRD but disappeared after a few hours of measurements.
It is possible that the oxygen atoms within that surface structure yield some intensity around \qty{530.5}{\eV}, where a small region of higher intensity is observed.
No peak near \qty{527.7}{\eV} was observed throughout the experiment.

% ratio 8
Interestingly, adding ammonia in the cell does not remove the \ce{O_a} peak, but only introduces a \ce{H_2O_g} peak at \qty{533.7}{\eV} \parencite{Weststrate2006, Linford2019}, proving the catalytic activity.
The \ce{O_a} peak has a lower intensity which is partly due to the oxidation reaction utilising oxygen to produce almost only water and nitric oxide in this condition.
A peak at \qty{404.3}{\eV} is reported, linked mainly to \ce{NO_{g}} and possibly slightly to \ce{N_{2, g}}.
\ce{NO_{g}} is expected between \qty{404.5}{\eV} and \qty{406.7}{\eV}, while \ce{N_{2,g}} between \qty{403.9}{\eV} and \qty{404.8}{\eV} \parencite{Ivashenko2021}, which could possibly explain the broad peak shape.
% The fact that the partial pressure of \ce{NO} is higher than for Pt(111) tends to support this hypothesis.
Since gas phase peaks are shifting when changing the partial pressure of oxygen, it is difficult to be certain of the peak nature.
The presence of \ce{O_a} shows that oxygen is easily adsorbed in Pt(100), even when ammonia is present, \textit{i.e.} during reacting conditions.
Moreover, Novell-Leruth et al. \parencite*{NovellLeruth2008} have reported that the de-hydrogenation process of ammonia is favoured to occur \textit{via} \ce{O_a} rather than \ce{OH_a} on Pt(100) in comparison with Pt(111).
This could explain the surface high selectivity towards the production of \ce{NO}.
Furthermore, no additional peak that could be linked to \ce{OH_a} is observed, reported at \qty{532}{\eV} on Pt(100).
All the \ce{NH_x} species are probably quickly oxidised towards nitrogen, which then reacts with a neighbouring oxygen to form \ce{NO_a}.
Adsorbed \ce{NO} at \qty{401.3}{\eV} was not observed during the dissociation of \ce{NO} on Pt(100) above room temperature by XPS \parencite{Rienks2003}.
\ce{NO} probably desorbs too quickly from the catalyst surface to be detected in this study, as reported by Ivashenko et al. \parencite*{Ivashenko2021}.

% ratio 05
Reducing the partial pressure of oxygen induces a remarkable change in the O 1s spectrum, removing the \ce{O_a} peak, introducing a \ce{H_2O_a} peak at \qty{532.1}{\eV} and shifting the \ce{H_2O_g} peak by \qty{0.6}{\eV}.
It is difficult to differ between \ce{OH_a} and \ce{H_2O_a}.
If \ce{OH_a} is reported at \qty{532.0}{\eV}, such a peak also exists under argon atmosphere, characteristic of water contamination in the reactor.
An energy of \qty{400.0}{\eV} in the N 1s spectrum is linked to adsorbed atomic nitrogen on Pt(100) \parencite{Sugai1993, vandenBroek1999}.
Since the only product of the ammonia oxidation is now nitrogen, the peak at \qty{404.3}{\eV} can safely be assigned to \ce{N_{2,g}}, similar to reported values \parencite{Ivashenko2021}.
Ammonia being in excess, all of the oxygen is used during ammonia oxidation, yielding only \ce{H_2O} since no \ce{NO} or \ce{N_2O} can be detected, which explains the removal of the \ce{O_a} peak.
\ce{H_2} is also produced from the dissociation of \ce{NH_3} on the catalyst, adsorbed hydrogen atoms then recombine and desorb from the surface.
This reaction is only observed without oxygen in the reactor.
All of the oxygen is used in the first steps of the ammonia oxidation to remove hydrogen atoms from \ce{NH_x} species.
When \ce{OH_a} groups recombine to produce water, one oxygen atom is left on the catalyst surface.
It seems that the oxygen atoms are preferably used for the de-hydrogenation process rather than the formation of \ce{NO} by reacting with adsorbed nitrogen.
The recombination of two nitrogen atoms on the catalyst surface is probably a slow process since \ce{N_a} can be detected, but faster than the reaction with adsorbed oxygen at low \ce{O_2}/\ce{NH_3} ratios.
At this condition, most of the surface sites are hypothesised to be occupied by adsorbed water or nitrogen.
\ce{NH_{3,g}} reported between \qty{400.4}{\eV} and \qty{400.7}{\eV} \parencite{Ivashenko2021}, was not clearly detected.
Nevertheless, it is possible that it also gives a contribution to the peak at \qty{400.0}{\eV}.

% ammonia
Removing oxygen from the reactor removes the gas phase water peak from the O 1s level, adsorbed water is still present but shifted in energy, with the same peak shape, maybe from a lower presence of \ce{OH_a} groups since no more oxygen is present.
The dissociation of ammonia towards \ce{N_2} and \ce{H_2} is measured with the mass spectrometer.
\ce{N_{2,g}} is measured, shifted by \qty{0.6}{\eV}.
The peak at \qty{400.6}{\eV} can tentatively be linked to a superposition of gas phase ammonia and adsorbed \ce{NH_{x,a}} species.
No \ce{N_a} peak can be observed which supports a slow dissociation process of ammonia of the Pt(100) surface.

% low o2
Removing ammonia and adding \qty{0.55}{\milli\bar} of oxygen in the reactor shows at least five peaks in the O 1s level.
Both peaks at \qty{538.3}{\eV} and \qty{537.2}{\eV} are from gas phase oxygen.
At least three different oxygen species are present on the catalyst.
The peak at \qty{529.7}{\eV} exhibits a high intensity and low width, attributed to the presence of chemisorbed oxygen on Pt(100), mainly responsible for the de-hydrogenation steps.
At least two other peaks can be detected at higher energy, around \qty{530.6}{\eV} and \qty{531.4}{\eV}.
If only one peak was present at \qty{531.4}{\eV} the intensity would drop between at \qty{530.6}{\eV} which is not the case here.
\ce{OH_a} reported at \qty{532}{\eV} \parencite{Kondratenko2006} is too far from the energies indexed here to be linked to the peak at \qty{531.4}{\eV}.
Thus, various oxygen species, adsorbed on different sites (e.g. hollow, on-top, bridge) or in different layers of the catalyst may be linked to the peaks at \qty{532.0}{\eV}, \qty{531.4}{\eV}.
The oxygen specie at \qty{529.7}{\eV} is predominant compared to the other species, only that specie could be resolved during reacting conditions.
Thus, the oxidation of ammonia may prevent the other species from existing due to the high mobility of adsorbed atoms on the catalyst surface.

\subsubsection{Pt 4f level}

The Pt 4f level was also measured to report possible differences in the electronic configuration of surface platinum atoms.
No bulk oxide could be measured during reacting conditions or under the presence of \qty{5}{\milli\bar} of oxygen.
However, many different surface structures were identified under reacting conditions, changing as a function of the \ce{O_2}/\ce{NH_3} ratio (fig. \ref{fig:MapsAndLScansPt100HighOxAmmonia} - \ref{fig:MapsAndLScansPt100LowOxAmmonia}).

\begin{figure}[!htb]
    \centering
    \includegraphics[width=\textwidth]{/home/david/Documents/PhDScripts/B07_2022_04_XPS/Figures/Pt100/Pt4f_550_no_fit_merged.pdf}
    \caption{
        Spectra collected at the Pt 4f level under different atmospheres at \qty{450}{\degreeCelsius} with an incoming photon energy of \qty{550}{\eV}.
        A Shirley-type background has been subtracted from all XPS spectra.
        Normalisation performed first by the background intensity and secondly by the maximum intensity to allow a qualitative comparison between different total pressures.
        Spectra before normalisation are shown on the top left.
    }
    \label{fig:Pt4fPt100}
\end{figure}

The introduction of \qty{8.8}{\milli\bar} of oxygen in the reactor induces a shift of \qty{0.1}{\eV} in the main peak position.
Two new peaks can be detected, at \qty{71.8}{\eV} and \qty{71.4}{\eV}, probably linked to the presence of oxygen atoms on top of the catalyst surface.
One new peak was clearly identified to appear in the O 1s level during this condition at \qty{529.7}{\eV}, linked to chemisorbed oxygen based on reported literature values \parencite{Rienks2003, Kondratenko2006}.
One of both peaks at \qty{71.8}{\eV} and \qty{71.4}{\eV} is probably from the same specie.
The other Pt 4f peak may be linked to the moieties yielding a small peak near \qty{530.5}{\eV} in the O 1s level.
The presence of \ce{OH_a} on the surface is likely since it is involved in the reaction mechanism, and could correspond to this second specie.

Introducing ammonia shifts the main peak to lower binding energy, possibly from the appearance of a new peak linked to nitrogen species.
Both peaks identified under oxygen do not disappear but have a decreased intensity, similarly to what has been observed in the O 1s level, as a result of the participation of oxygen species in the catalytic reaction.

Reducing the partial pressure of oxygen to \qty{0.55}{\milli\bar} has the effect of removing the oxygen induced features at high binding energy.
The peak tail near \qty{71.5}{\eV} closely resembles that observed under argon atmosphere.
The \ce{O_a} peaks in the O 1s level also disappeared during this condition.
Interestingly, the peak shape is broader and extended towards lower binding energy, where a contribution of adsorbed nitrogen atoms detected in the N 1s level, is possible.
If \ce{N_a} could not be detected when the \ce{O_2}/\ce{NH_3} = 8, it is not possible to rule out its presence due to the low intensity of the XPS spectrum.
Thus, the shift towards lower energy in the Pt 4f level may be due to adsorbed nitrogen for both conditions, more important when \ce{O_2}/\ce{NH_3} = 0.5.
Indeed, removing oxygen from the reactor which removed the \ce{N_a} peak from the N 1s level also reverts this broadening effect.

The same reasoning can be applied to adsorbed water, which appears at the same condition but does not disappear when removing oxygen.
However, the energy position in the O 1s level is slightly shifted, from \qty{532.1}{\eV} to \qty{532.4}{\eV}, which underlines different electronic environments.

A new peak near \qty{71.6}{\eV} under the sole presence of ammonia can safely be attributed to the presence of adsorbed \ce{NH_{x,a}} species, also identified in the N 1s level.
The peak intensity with respect to the maximum intensity is the highest measured, translating a large coverage of the platinum surface by \ce{NH_{x,a}}.

Changing the reactor atmosphere to \qty{0.55}{\milli\bar} of oxygen has removed the peak at \qty{71.6}{\eV}.
Overall the peak shape is similar to that under \qty{8.8}{\milli\bar} of oxygen.
The weaker intensity at higher binding energy is probably linked to a lower oxidised state of the Pt surface.

Reverting to argon atmosphere removed the peaks linked to \ce{O_a}.
However, the peak has not completely reverted to its shape before the oxidation cycle, and is very similar to the shape under a \ce{O_2}/\ce{NH_3} ratio equal to \num{0.5}.
The adsorbed nitrogen is absent from the Pt surface, only adsorbed water is present at both conditions, which was also present before the oxidation cycle but in a lower amount.
Therefore, adsorbed water may be the specie giving rise to the peak at lower binding energy.

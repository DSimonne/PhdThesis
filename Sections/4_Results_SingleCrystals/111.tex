\section{Pt(111) single crystal studied at \qty{450}{\degreeCelsius}} \label{sec:SXRD111}

Platinum crystallises in a face-centred cubic structure with a lattice parameter $a_{Pt}$ equal to \qty{3.92}{\angstrom} at room temperature \parencite{Waseda1975}.
Its structure was first presented in sec. \ref{sec:ScatCrystal} to introduce the notions of crystals.

The arrangement of the Pt atoms on the (111) surface is hexagonal, which leads to the definition of the surface unit cell shown in fig. \ref{fig:SurfaceUnitCellPt111} to be able to better represent the surface arrangement of the Pt atoms.
The in-plane vectors $\vec{a}_{(111)}$ and $\vec{b}_{(111)}$ are of equal magnitude ($a_{Pt} / \sqrt{2} = \qty{2.775}{\angstrom})$, separated by \ang{120}.
The out-of-plane vector $\vec{c}_{(111)}$ is perpendicular to the (111) plane, and of magnitude $3 a_{Pt} / \sqrt{3} = \qty{6.797}{\angstrom}$.

\begin{SCfigure}
    \centering
    \includegraphics[trim=0 2cm 0 2cm, clip, width=0.70\textwidth]{/home/david/Documents/PhD/Figures/introduction/111.pdf}
    \caption{
        Face-centred cubic unit cell of Pt with (111) crystallographic plane drawn in green.
        $\vec{a}_{(111)}$, $\vec{b}_{(111)}$ and $\vec{c}_{(111)}$ are the $(111)$ surface unit cell vectors.
        There are three \{111\} planes spanned by the magnitude of $\vec{c}_{(111)}$ (blue, red and green on the figure).
    }
    \label{fig:SurfaceUnitCellPt111}
\end{SCfigure}

\subsection{Oxide growth under \qty{80}{\milli\bar} of oxygen}

To identify the presence of surface reconstructions and/or surface oxides, in-plane reciprocal space maps were collected at the atmospheres detailed in tab. \ref{tab:ConditionsSXRD} by rotating the in-plane sample and detector angles ($\omega$ and $\gamma$) from \ang{0} to \ang{120} to collect a third of the reciprocal space in the ($\vec{q}_x$, $\vec{q}_y$) plane, considering a hexagonal symmetry in the position of the Bragg peaks.

The reciprocal space in-plane maps were computed in both $q$-space (to obtain the interplanar spacing related to the observed signals) and ($hkl$)-space (fig. \ref{fig:MapsPt111A} and \ref{fig:MapsPt111B}) to visualise the arrangement of surface structures or surface relaxations in comparison with the hexagonal structure of the Pt atoms on the (111) surface, the $H$ and $K$ values being computed using the hexagonal surface unit cell described above.

\begin{figure}[!htb]
    \centering
    \includegraphics[width=0.495\textwidth]{/home/david/Documents/PhDScripts/SixS_2023_04_SXRD_Pt111/figures/map_hkl_76-115.pdf}
    \includegraphics[width=0.495\textwidth]{/home/david/Documents/PhDScripts/SixS_2023_04_SXRD_Pt111/figures/map_hkl_285-320.pdf}
    \includegraphics[width=0.495\textwidth]{/home/david/Documents/PhDScripts/SixS_2023_04_SXRD_Pt111/figures/map_hkl_481-516_patched.pdf}
    \includegraphics[width=0.495\textwidth]{/home/david/Documents/PhDScripts/SixS_2023_04_SXRD_Pt111/figures/map_hkl_681-689_patched.pdf}
    \caption{
        Reciprocal space in-plane maps collected under different atmospheres measured at \qty{25}{\degreeCelsius} under UHV and at \qty{450}{\degreeCelsius} under different atmospheres, with a total pressure equal to \qty{500}{\milli\bar}, computed using the hexagonal lattice of Pt(111).
    }
    \label{fig:MapsPt111A}
\end{figure}

The first map was collected at low pressure (\qty{<1e-5}{\milli\bar}) after the cleaning of the sample by sputtering and annealing (fig. \ref{fig:MapsPt111A} - a).
The ($\bar{1}$$\bar{1}$0) and ($\bar{2}$10) Bragg peaks can be observed, together with the bottom of [111]-oriented crystal truncation rods going through the [0, $\bar{2}$], [0, $\bar{1}$], [$\bar{1}$, 0], [$\bar{1}$, 1], and [$\bar{2}$, 0] positions in the (H, K) plane.
No difference can be observed when introducing \qty{500}{\milli\bar} of argon in the reactor (fig. \ref{fig:MapsPt111A} - b).

New peaks can first be observed under \qty{80}{\milli\bar} of oxygen, identified by different colour circles in fig. \ref{fig:MapsPt111A} (c), the corresponding interplanar spacings are given in tab. \ref{tab:InterplanarSpacingsPt111Oxygen}.
The angle between the peaks circled in grey and white and the peaks circled in red and purple is equal to \ang{60}, which could be the signature of the existence of two hexagonal surface structures, each rotated by $\pm \ang{6}$ with respect to the Pt(111) surface unit cell and with a larger in-plane lattice parameter (angles measured in q-space, fig. \ref{fig:481QSpace}).
No apparent second order peak can be linked to those rotated hexagonal structures.
Summing the two vectors going from the centre of the reciprocal space to the white and purple circled peaks yields the position of the black circled peak, the angle between both vectors is then equal to \ang{42.65} (angle measured in q-space, fig. \ref{fig:481QSpace}).

The measurement of the entire reciprocal space in-plane map took about \qty{3}{\hour}\qty{35}{\minute}, the entire map being a concatenation of multiple $\omega$ scans during which the plane of the sample is rotated, the first scan starting near the centre of the reciprocal space.
The contribution of each scan during the measurement is visible from their different background.
Therefore, there is approximately \qty{1}{\hour} between the measurement of the peaks circled in red, grey, purple and white, and the peak circled in black.

A second map was measured in a smaller region of the reciprocal space \qty{9}{\hour}\qty{30}{\minute} after the introduction of oxygen in the cell to see if the intensity and position of the newly found peaks changed (fig. \ref{fig:MapsPt111A} - d).
The measurement of this map took about \qty{1}{\hour}\qty{15}{\minute}.
Three additional peaks (circled in black in fig. \ref{fig:MapsPt111A} - d) appeared on a (8x8) coincidence with comparison to the hexagonal lattice of Pt(111), describing a hexagonal lattice of lattice parameter equal to $|\vec{a_{hex}}| = \qty{3.118}{\angstrom}$ (drawn in black dotted lines in fig. \ref{fig:MapsPt111A} - d).

The grey circled peak seems to have disappeared from the reciprocal map (fig. \ref{fig:MapsPt111A}, c, d), but an additional peak circled in red is revealed at \ang{120} from the other peak circled in red, and \qty{60} from the peak circled in grey, at the same distance from the centre (fig. \ref{fig:MapsPt111A} - d).
From these first measurement, a first hypothesis to explain the position of the observed peaks is given.
First, two rotated hexagonal structures appear with a larger in-plane lattice parameter in comparison to Pt(111) (fig. \ref{fig:MapsPt111A} - c).
Secondly, a (8x8) hexagonal structure appears, detected while measuring the large reciprocal space in-plane map, with a larger in-plane lattice parameter, but in the same direction as the in-plane unit cell vectors of Pt(111) (fig. \ref{fig:MapsPt111A} - d), explaining the position of the black circled peak in the first map.
It is possible that the disappearance of the grey circle peak is linked to some alignment problems, visible from the large low intensity region in fig. \ref{fig:MapsPt111A} (d).

Out-of-plane measurements were performed perpendicular to peaks belonging to the (8x8) hexagonal structure, [H, K] = [-0.89, 0] (fig. \ref{fig:LScans80} - a), and rotated hexagonal structures, [H, K] = [-0.15, -0.74] (fig. \ref{fig:LScans80} - b), to probe their 3D structures.
A second out-of-plane measurement was performed at [H, K] = [-1.78, 0] to probe the intensity of a potential second-order peak for the (8x8) hexagonal structure (fig. \ref{fig:LScans80} - c), \qty{10}{\hour} after the measurement perpendicular to [H, K] = [-0.89, 0].
A peak is detected, which supports the existence of a (8x8) structure.
The background-subtracted scattered intensity was integrated as a function of $L$ using the \textit{fitaid} module of \textit{binoculars} (fig. \ref{fig:LScans80}).

\begin{figure}[!htb]
    \centering
    \includegraphics[width=\textwidth]{/home/david/Documents/PhDScripts/SixS_2023_04_SXRD_Pt111/figures/l_scans_high_oxygen.pdf}
    \caption{
        2D detector slices at $L=0$ during out-of plane measurements for three different [H, K] positions under \qty{80}{\milli\bar} of \ce{O_2} (a-b-c).
        The integrated intensity is presented in (d) as a function of $L$.
    }
    \label{fig:LScans80}
\end{figure}

The integrated intensity perpendicular to [H, K] = [-0.15, -0.74] shows a small peak at $L\approx 0.2$ followed by a constant intensity, characteristic of non-periodic out-of-plane structures such as monolayers (fig. \ref{fig:SimROD}, Robinson et al. \cite*{Robinson1991}).

Four different bulk platinum oxides have been reported to exist in literature, $\alpha$-\ce{PtO_2}, $\beta$-\ce{PtO_2}, \ce{Pt_3O_4} and \ce{PtO}.
The corresponding experimental structures are described in tab. \ref{tab:PtOxides} together with the expected equilibrium structure found by Seriani et al \parencite*{Seriani2006, Seriani2008} from first-principles atomistic thermodynamics calculations and molecular dynamics simulations based on density functional theory.

\begin{table}[!htb]
\centering
% \resizebox{\textwidth}{!}{%
    \begin{tabular}{@{}lllll@{}}
    \toprule
     & \ce{PtO} & \ce{Pt_3O_4} & $\beta$-\ce{PtO_2} & $\alpha$-\ce{PtO_2} \\ \midrule
    Bravais lattice & Tetragonal & Cubic & Orthorhombic & Hexagonal \\
    a (\unit{\angstrom}) & 3.10 (3.08) & 5.65 (5.59) & 4.49 (4.48) & 3.14 (3.10) \\
    b (\unit{\angstrom}) & 3.10 (3.08) & 5.65 (5.59) & 4.71 (4.52) & 3.14 (3.10) \\
    c (\unit{\angstrom}) & 5.41 (5.34) & 5.65 (5.59) & 3.14 (3.14) & 5.81 (4.28-4.40) \\
    $\alpha$ & \ang{90} & \ang{90} & \ang{90} & \ang{90} \\
    $\beta$ & \ang{90} & \ang{90} & \ang{90} & \ang{90} \\
    $\gamma$ & \ang{90} & \ang{90} & \ang{90} & \ang{120}\\
    \bottomrule
    \end{tabular}%
    %}
    \caption{
    Experimental values taken from \cite{McBride1991} for \ce{PtO}, from \cite{Muller1968} for \ce{Pt_3O_4} and $\alpha$-\ce{PtO_2}, and from \cite{McBride1991} for $\beta$-\ce{PtO_2}.
    Theoretical values and table adapted from Seriani et al. \parencite*{Seriani2006}.
    Similar values have been reported by Nomiyama et al. \parencite*{Nomiyama2011} from DFT calculations.
    }
\label{tab:PtOxides}
\end{table}

The lack of peaks at $L=0$ in fig. \ref{fig:LScans80} for the (8x8) hexagonal structure shows that there is no bulk structure on the catalyst surface.
However, Seriani et al. \parencite*{Seriani2006} have also found that the most stable surface oxide on the Pt(111) surface is expected to be hexagonal $\alpha$-\ce{PtO_2}.
A high uncertainty regarding the lattice parameter in the $\vec{c}$ direction is reported in experimental and theoretically studies due to a poor crystallisation of the bulk oxide \parencite{Muller1968}, and to underestimation of weak interlayer Van der Waals forces in theoretical studies \parencite{Li2005}.

The formation of hexagonal on hexagonal surface $\alpha$-\ce{PtO_2} on Pt(111) is expected to induce a high amount of in-plane compressive strain in the oxide layer from the large difference between the Pt-Pt distance on the (111) surface (\qty{2.775}{\angstrom}) and the in-plane lattice parameter of $\alpha$-\ce{PtO_2} (\qty{3.14}{\angstrom}).
A rotation of \ang{30} is expected to reduce the in-plane strain, resulting in a (2X2) arrangement.
The presence of $\alpha$-\ce{PtO_2} on the Pt(111) surface was also shown to provide favourable special sites that could contribute to the catalytic activity \parencite{Li2005}.
The 3D structure of bulk $\alpha$-PtO$_2$ is presented in fig. \ref{fig:AlphaPtO2}.

\begin{SCfigure}
    \centering
    \includegraphics[trim=0 2.5cm 0 2.5cm, clip, width=0.5\textwidth]{/home/david/Documents/PhD/Figures/introduction/AlphaPtO2.pdf}
    \caption{
        $\alpha$-\ce{PtO_2} bulk unit cell.
        Platinum atoms are situated on the unit cell corners while the two oxygen atoms are at the positions $(1/3, 2/3, 1/4)$ and $(2/3, 1/3, 3/4)$.
    }
    \label{fig:AlphaPtO2}
\end{SCfigure}

Ellinger et al. \parencite*{Ellinger2008} reported the existence of surface $\alpha$-\ce{PtO_2} on Pt(111) in an experimental study using SXRD at similar conditions (\qty{500}{\milli\bar} of oxygen, at a temperature between \qty{245}{\degreeCelsius} and \qty{635}{\degreeCelsius}).
The in-plane lattice parameter was found equal to \qty{3.15}{\angstrom}, close to the reported bulk value of $\alpha$-PtO$_2$ (\qty{3.14}{\angstrom}).
Similar results were found by Ackermann \parencite*{Ackermann2007}.
Ellinger et al. \parencite*{Ellinger2008} proposed a surface model for surface $\alpha$-\ce{PtO_2} consisting of a full unit cell, with an out-of-plane lattice parameter reduced by \qty{15}{\percent} in comparison with the bulk structure (\qty{3.62}{\angstrom}), and with a lateral displacement of \qty{\approx33}{\percent} in the [110] direction of the Platinum atoms on the top layers with respect to the Pt atoms of the bottom layer, in contact with the Pt(111) surface.
The question of the epitaxial relation between both structures remains, since the proposed structure does come with a large misfit strain (\qty{13.5}{\percent}) when in a hexagonal on hexagonal arrangement.
% Ellinger et al. \parencite*{Ellinger2008} proposed
% As our fit is not compatible with domain formation, we are forced to conclude that oxide formation is initiated at step edges.
Similarly to the current experiment, an (8x8) coincidence is reported between the Pt(111) hexagonal surface and surface $\alpha$-\ce{PtO_2}.
The out-of-plane measurements also performed by Ellinger et al. \parencite*{Ellinger2008} perpendicular to [H, K] = [-0.89, 0] presents a peak at $L=0.65$ ($L$ computed using the distorted $\alpha$-\ce{PtO_2} unit cell).

The two $L$-scans performed in the current study were fitted using two Gaussian peaks of same full width at half maximum (FWHM) and a constant background (fig. \ref{fig:LScans80Fit}), the second peak near $L=1$ in the $L$-scan perpendicular to [H, K] = [-0.89, 0] comes from a Pt powder signal.

\begin{figure}[!htb]
    \centering
    \includegraphics[width=\textwidth]{/home/david/Documents/PhDScripts/SixS_2023_04_SXRD_Pt111/figures/ctr_reconstructions_fitting_result.pdf}
    \caption{
        Out-of-plane measurements and fit result under \qty{80}{\milli\bar} of \ce{O_2}.
    }
    \label{fig:LScans80Fit}
\end{figure}

A peak at [-0.89, 0, 0.65] is reported, but using the Pt(111) surface unit cell, with an out-of-plane lattice parameter of \qty{6.797}{\angstrom}, incompatible with the results presented in Ellinger et al \parencite*{Ellinger2008}.
It seems that a different out-of-plane structure is present in the current work.

Additional simulations are needed to fully understand the structures in this study, with different surface models including the \textit{spoked-wheel} superstructure identified \textit{via} scanning tunnelling microscopy (STM) by \cite{VanSpronsen2017, Boden2022} which also exhibits a (8x8) coincidence with the Pt(111) lattice.
Different surface oxides have also been identified by Miller et al \parencite*{Miller2011, Miller2014}, Fantauzzi et al \parencite*{Fantauzzi2017}, as well as Farkas et al. \parencite*{Farkas2017}, respectively under \qty{6.66}{\milli\bar} of \ce{O_2} at \qty{450}{\degreeCelsius} and under \qty{1}{\milli\bar} of oxygen between \qty{160}{\degreeCelsius} and \qty{410}{\degreeCelsius}.
\textcolor{Important}{which one..., mention stripe stuff}

\begin{figure}[!htb]
    \centering
    \includegraphics[width=\textwidth]{/home/david/Documents/PhDScripts/SixS_2023_04_SXRD_Pt111/figures/reflecto_80.pdf}
    \caption{
    	X-ray reflectivity curves measured in a specular geometry (full lines) under \qty{80}{\milli\bar} of oxygen.
    	Curves fitted using \text{GenX} are shown an order of magnitude below as dotted lines.
    }
    \label{fig:Reflecto80}
\end{figure}

Reflectivity curves measured in a specular geometry and fitted using \textit{GenX} \parencite{Bjorck2007, Glavic2022} are presented in fig. \ref{fig:Reflecto80}.
%The roughness is adjusted by the means of the surface root mean square roughness $\sigma$ in \unit{\angstrom}, as defined by Bennet et al. \parencite*{Bennett1961}.
The specular X-ray reflectivity simulations are conducted using the Parratt algorithm \parencite{Parratt1954}.
The classical way of implementing roughness is based on a Gaussian distribution, introducing corrective factors to the electric field amplitudes at the interfaces in accordance with the Nevot-Croce model \parencite{Nevot1980}.
For the computation of material refractive indices, the Henke tables \parencite{Henke1993} are employed.

A simple model was used consisting of a semi-infinite slab of platinum, on top of which is present a homogeneous layer of platinum oxide.
The density of the layer was set free during the minimisation process between zero and \qty{0.021172}{FU\per\cubic\angstrom} (formula unit per cubic Angström, computed from unit cell scattering length and volume), the value for $\alpha$-PtO$_2$.
The fitted density values is divided by two between the first (\qty{0.00727}{FU\per\cubic\angstrom}) and second (\qty{0.00358}{FU\per\cubic\angstrom}) measurement, and is always an order of magnitude below the expected value for a homogeneous layer of $\alpha$-PtO$_2$.
The layer was found to be \qty{13.89}{\angstrom} and \qty{14.185}{\angstrom} thick in the first and second reflectivity curves, with a very low roughness (\qty{<1e-7}{\angstrom}) in comparison with the roughness of the Pt surface.

By combining the information obtained from in-plane, out-of-plane and reflectivity measurements, it is probable that the platinum surface is first covered by two types of domains corresponding to two rotated hexagonal structures.
Secondly, a $\alpha$-\ce{PtO_2} surface oxide grows on the surface, with an average thickness of \qty{\approx 14}{\angstrom}, \textit{i.e.}, a few monolayers thick, confirmed by the presence of peaks on the L-scans, in contrast with the rotated structures.

% \begin{figure}[!htb]
%     \centering
%     \includegraphics[trim=0 1cm 0 1cm, clip, width=\textwidth]{/home/david/Documents/PhD/Figures/introduction/EpitaxyPtO2Pt111.pdf}
%     \caption{
%     	Epitaxy relationship between Pt(111) surface unit cell (in red) and $\alpha$-\ce{PtO_2} basal unit cell.
%     }
%     \label{fig:PtO2OnPt111}
% \end{figure}

\subsection{Near ambient pressure ammonia oxidation cycle}

Ammonia was subsequently introduced in the reactor cell to investigate the evolution of the platinum oxide structures, as well as the presence of additional surface structures or surface reconstructions during the oxidation of ammonia on the platinum surface (fig. \ref{fig:MapsPt111B} a-b).
The presence of $\alpha$-PtO$_2$ on Pt(111) has been shown to not hinder the oxidation of \ce{CO} by Ackermann et al. \parencite*{Ackermann2007}, that would then occur \textit{via} a Mars-Van Krevelen mechanism \parencite{Mars1954}, inducing a progressive roughening of the platinum surface.

\begin{figure}[!htb]
    \centering
    \includegraphics[width=0.495\textwidth]{/home/david/Documents/PhDScripts/SixS_2023_04_SXRD_Pt111/figures/map_hkl_830-865.pdf}
    \includegraphics[width=0.495\textwidth]{/home/david/Documents/PhDScripts/SixS_2023_04_SXRD_Pt111/figures/map_hkl_1041-1076.pdf}
    \includegraphics[width=0.495\textwidth]{/home/david/Documents/PhDScripts/SixS_2023_04_SXRD_Pt111/figures/map_hkl_1413-1448.pdf}
    \includegraphics[width=0.495\textwidth]{/home/david/Documents/PhDScripts/SixS_2023_04_SXRD_Pt111/figures/map_hkl_1458-1493.pdf}
    \caption{
        Reciprocal space in-plane maps collected under different atmospheres measured at \qty{450}{\degreeCelsius}, computed using the hexagonal lattice of Pt(111).
    }
    \label{fig:MapsPt111B}
\end{figure}

During the oxidation of ammonia, none of the previously observed peaks could not be discerned, while no additional peaks emerged.
Even when the oxygen pressure within the cell was reduced to \qty{5}{\milli\bar} (favouring alternate products) or eliminated at \qty{0}{\milli\bar}, there was no induction of surface reconstructions or surface structures.
This result shows that the reaction removes the oxide layer present of the surface until the structure disappears.

No transition between the reaction products is observed in the first hours following the introduction of ammonia in the reactor (fig. \ref{fig:RGA450Pt111_2}, the presence of the surface oxide at the beginning of the condition seemingly not influencing the product selectivity.
All reciprocal space in-plane maps following the introduction of ammonia to a final atmosphere of \qty{500}{\milli\bar} of Argon showed a pristine surface structure (fig. \ref{fig:MapsPt111B} - c, d), proving that the oxidation cycle is reproducible, and that ammonia effectively removed the different surface oxides that grew under \qty{80}{\milli\bar} of oxygen.
The low roughness of the platinum surface seems to be in contradiction with a possible Mars-Van Krevelen mechanism during the oxidation of ammonia at ambient pressure.

\begin{figure}[!htb]
    \centering
    \includegraphics[width=\textwidth]{/home/david/Documents/PhDScripts/SixS_2023_04_SXRD_Pt111/figures/reflecto_cycle.pdf}
    \caption{
    	X-ray reflectivity curves measured in a specular geometry (full lines) under different atmospheres during the oxidation of ammonia.
    	Curves fitted using \text{GenX} are shown an order of magnitude below as dotted lines.
    }
    \label{fig:ReflectoCycle}
\end{figure}

Reflectivity curves were also measured and fitted with \textit{GenX} under pure Argon atmosphere (before and after the oxidation cycle), under different reacting conditions, and with only ammonia and argon in the reactor, presented in fig. \ref{fig:ReflectoCycle}.
In accordance with the results of the in-plane reciprocal space maps in fig. \ref{fig:MapsPt111B}, no oxide layer could be detected on the platinum surface.
The roughness of the [111]-oriented platinum crystal decreases following the introduction of ammonia after having increased to \qty{\approx 2.61}{\angstrom} when being exposed to \qty{80}{\milli\bar} of oxygen for more than \qty{20}{\hour}, almost reaching its value before the start of the oxidation cycle.
The largest decrease in roughness is seen during the reacting conditions, underlining the importance of the mobility of adsorbed species on the catalyst surface in the roughness evolution.

\subsection{Oxide growth under \qty{5}{\milli\bar} of oxygen}

Following the complete ammonia oxidation cycle, the sample was put under \qty{5}{\milli\bar} of oxygen to see if any of the detected surface structures would come to grow again, and to de-correlate the effect of the simultaneous presence of ammonia and oxygen in the cell on the surface when favouring the production of nitrogen at lower oxygen pressures.
The sample was cleaned with two sputtering and annealing cycles before the introduction of oxygen.
Large and small reciprocal space in-plane maps were measured to have the best compromise between a higher temporal resolution of the oxide growth, and the possibility to detect other peaks from corresponding surface unit cells.
The larger reciprocal space in-plane maps are shown in fig. \ref{fig:LargeMapsPt111LowOxygen}.

\begin{figure}[!htb]
    \centering
    \includegraphics[width=\textwidth]{/home/david/Documents/PhDScripts/SixS_2023_04_SXRD_Pt111/figures/large_maps_05O2.pdf}
    \caption{
        Reciprocal space in-plane maps collected under \qty{495}{\milli\bar} of argon and \qty{5}{\milli\bar} of oxygen at \qty{450}{\degreeCelsius}, for different exposure times.
    }
    \label{fig:LargeMapsPt111LowOxygen}
\end{figure}

In the first map, that started directly after the stabilisation of the oxygen pressure at \qty{5}{\milli\bar} in the reactor cell, three low intensity peaks can be detected (fig. \ref{fig:LargeMapsPt111LowOxygen} - a).
Those peaks correspond to the same peaks detected previously shortly after having a pressure of \qty{80}{\milli\bar} in the cell (fig. \ref{fig:MapsPt111A} - c).
Only the grey circled peak cannot be detected until \qty{4}{\hour} of measurement, which is also when some peaks start to split in a similar pattern, exhibiting three distinct peaks around a more diffuse region (fig. \ref{fig:LargeMapsPt111LowOxygen} - c).

This splitting of the peaks could be the signature of a Pt(111) surface covered by different domains exhibiting similar hexagonal structures.
These domains are rotated by $\pm \ang{6}$ with respect to the Pt(111) surface unit cell (angles measured in q-space, fig. \ref{fig:2064QSpace}), with a higher magnitude of the in-plane lattice parameter, similarly to what has been observed under an oxygen pressure of \qty{80}{\milli\bar} (fig. \ref{fig:MapsPt111A} - c).

A peak can be observed at [-0.89, -0.89] in the only large reciprocal space in-plane map collected between \qty{41}{\minute} and \qty{4}{\hour}\qty{3}{\minute} after the start of the condition (fig. \ref{fig:LargeMapsPt111LowOxygen} - b).
The peak position is the same as in the first large reciprocal space in-plane map measured under \qty{80}{\milli\bar} of oxygen (fig. \ref{fig:MapsPt111A} - d).

The peak position corresponds to the ($\bar{1}\bar{1}0$)$_{\alpha-PtO_2}$ Bragg peak of bulk $\alpha-PtO_2$, but no peaks at [-0.89, 0, 0] and [0, -0.89, 0] can be detected (respectively ($\bar{1}00$)$_{\alpha-PtO_2}$ and ($0\bar{1}0$)$_{\alpha-PtO_2}$).
The nature of the peak situated at [-0.89, -0.89] is not easily determined since it appears before the [-0.89, 0] and [0, -0.89] peaks, all three positions verify the (8x8) hexagonal structure.
As mentioned before, it is possible to draw a surface unit cell including this peak together with the white and purple circled peak, but with an in-plane angle $\gamma^*$ equal in this case to \ang{42.13} (angle measured in q-space, fig. \ref{fig:2064QSpace}).

The area sampled during the measurement was extended in the two last maps which allows us to observe more signals around the ($1\bar{1}0$) region (fig. \ref{fig:LargeMapsPt111LowOxygen} - e, f)
For each map, two peaks circled by the same colour draw two vectors of the same magnitude going through the centre of the reciprocal space and separated by \ang{120}, in the last two maps there are three doublet of peaks separated by \ang{120}, in red, grey and purple ($q$-map available in appendix \ref{fig:2064QSpace}).
Moreover, the purple and red circled peaks are separated by \ang{60}, likewise for the grey and white circled peaks.
During this set of measurement, the grey circled peak did not disappear on the contrary to the measurements carried out under \qty{80}{\milli\bar} of oxygen, which furthermore supports the existence of at least two rotated hexagonal structures.

The ($\bar{1}0$0)$_{\alpha-PtO_2}$, ($0\bar{1}$0)$_{\alpha-PtO_2}$ and ($\bar{1}\bar{1}$0)$_{\alpha-PtO_2}$ peaks corresponding to surface ${\alpha-PtO_2}$ are detected in the large maps after \qty{\approx 24}{\hour} of measurements, circled in black.
Smaller maps taken with a shorter time interval are presented in fig. \ref{fig:SmallMapsPt111LowOxygen} in which the ($\bar{1}0$0)$_{\alpha-PtO_2}$ peak is detected after \qty{\approx 23}{\hour} of measurements, when it was already detected at the very least after \qty{10}{hour} under \qty{80}{\milli\bar} of oxygen (fig. \ref{fig:MapsPt111A} - d), highlighting the importance of the oxygen pressure in the surface oxidation.

\begin{figure}[!htb]
    \centering
    \includegraphics[width=\textwidth]{/home/david/Documents/PhDScripts/SixS_2023_04_SXRD_Pt111/figures/small_maps_05O2.pdf}
    \caption{
        Reciprocal space in-plane maps collected under \qty{495}{\milli\bar} of argon and \qty{5}{\milli\bar} of oxygen at \qty{450}{\degreeCelsius}, for different exposure times.
    }
    \label{fig:SmallMapsPt111LowOxygen}
\end{figure}

A more quantitative analysis of the different structures appearing during the exposition to oxygen was performed by integrating the scattered intensity around the white, red and black circled peak, present in the ($\bar{1}$10) region.
The average background was subtracted to each reciprocal space voxel before integration.
%, while the intensity of the crystal truncation rod ($\bar{1}$10) Bragg peak was used for normalisation to correct possible miss-alignement effects on the integrated intensity.
The starting time of each reciprocal space in-plane map is used as estimate for the time since the introduction of oxygen, the evolution of each peak is presented in fig. \ref{fig:HexBraggPeaks}.
The growth of the ${\alpha-PtO_2}$ surface oxide peak at [-0.89, -0.89] seems to be exponential after \qty{23}{\hour} of exposition.

\begin{figure}[!htb]
    \centering
    \includegraphics[width=\textwidth]{/home/david/Documents/PhDScripts/SixS_2023_04_SXRD_Pt111/figures/intensity_comparison_hex_reconstructions.pdf}
    \caption{
        Intensity evolution for peaks corresponding to the hexagonal surface structure and to the surface oxide.
    }
    \label{fig:HexBraggPeaks}
\end{figure}

The white peak is visible from the start of the exposition to oxygen (fig. \ref{fig:SmallMapsPt111LowOxygen}), whereas the red peak is only observed \qty{7}{hour} later with a lower intensity.
The intensity of both peaks (related to the other rotated hexagonal structures) quickly increases in the first hours and then plateaus until \qty{\approx 23}{\hour}, at which it increases together with the appearance of the $\alpha$-\ce{PtO_2} surface oxide.
Both peak intensity then starts to decreases after \qty{\approx 25}{\hour} of exposition, a few hours after the increase of the $\alpha$-\ce{PtO_2} peak intensity.

Several out-of-plane measurements have been carried out perpendicular to the white and red circled peaks, presented in appendix \ref{fig:LScans05}.
The peak intensity is constant as a function of $L$, similarly to the out-of-plane measurements presented in fig. \ref{fig:LScans80}, which shows that the rotated hexagonal structures are not three-dimensional.
No $L$-scan was measured perpendicularly to the ($0\bar{1}$0)$_{\alpha-PtO_2}$ peak.

Lowering the oxygen partial pressure in the reactor cell from \qtyrange{80}{5}{\milli\bar} has demonstrated that similar in-plane structures appear, but with a slower growth rate, which were both absent in the simultaneous presence of oxygen and ammonia in the reactor.
Additional studies are needed to understand if whether or not the rotated hexagonal structures act only as precursors to the appearance of surface $\alpha$-\ce{PtO_2} oxide, or if those peaks are still present after an extended exposure to oxygen.

% Finally, reflectivity curves were measured before the introduction of oxygen, as well as \qty{14}{\hour}\qty{30}{\minute} and \qty{23}{\hour}\qty{30}{\minute} after the introduction of oxygen, shown in fig. \ref{fig:reflecto}.
% The curves were fitted with \textit{GenX} following the method described earlier.
% The hexagonal structure was not yet detected when measuring the second reflectivity curve, a very low density oxide layer can be
% There is a significant drop of intensity in the second reflectivity curve,

% \begin{figure}[!htb]
%     \centering
%     \includegraphics[width=\textwidth]{/home/david/Documents/PhDScripts/SixS_2023_04_SXRD_Pt111/figures/reflecto_05.pdf}
%     \caption{
%     	X-ray reflectivity curves measured in a specular geometry (full lines) under different atmospheres during the oxidation of ammonia.
%     	Curves fitted using \text{GenX} are shown an order of magnitude below as dotted lines.
%     }
%     \label{fig:reflecto_5}
% \end{figure}

\subsection{Surface roughness and surface relaxation effects}

\begin{figure}[!htb]
    \centering
    \includegraphics[width=\textwidth]{/home/david/Documents/PhDScripts/SixS_2023_04_SXRD_Pt111/figures/ctr_a.pdf}
    \includegraphics[width=\textwidth]{/home/david/Documents/PhDScripts/SixS_2023_04_SXRD_Pt111/figures/ctr_b.pdf}
    \caption{
        Evolution of [111] crystal truncation rods measured perpendicular to three different Bragg peaks under different atmospheres.
    }
    \label{fig:CTRPt111}
\end{figure}

Crystal truncation rods have been measured perpendicularly to the ($\bar{2}10$), ($\bar{1}00$) and ($\bar{1}\bar{1}0$) positions \qty{6}{\hour} after the start of each condition, each measurement lasted for \qty{2}{\hour}.
The background-subtracted intensity of the CTR was integrated using the \textit{fitaid} module of \textit{binoculars} as a function of $L$ with the same integration range.

The CTR intensity is presented in fig. \ref{fig:CTRPt111}, no additional peak could be detected at any condition, besides the Pt Bragg peaks at higher $L$ values.
At first sight, the roughness seems to evolve during the exposition to different atmospheres, visible from the increase and decrease of the  intensity minimum near $L=1.5$ (or $L=2.5$ for the CTR recorded perpendicular to the ($\bar{1}$00) Bragg peak).
The intensity of the CTR under \qty{500}{\milli\bar} of argon before the oxidation cycle was progressively lost during the measurements due to problems with the sample heater, also visible in the reciprocal space in-plane map under the same condition in fig. \ref{fig:MapsPt111A}, and is therefore not shown in the data besides for the ($\bar{1}\bar{1}$0) Bragg peak.

\begin{figure}[!htb]
    \centering
    \includegraphics[trim=0 5cm 0 4.5cm, clip, width=0.45\textwidth]{/home/david/Documents/PhD/Figures/introduction/Pt111HexA.pdf}
    \includegraphics[trim=0 5cm 0 4.5cm, clip, width=0.45\textwidth]{/home/david/Documents/PhD/Figures/introduction/Pt111HexB.pdf}
    \caption{
        View from above (a) and from the side (b) of the Pt(111) surface with the atoms belonging to the A, B and C layers respectively coloured in green, red and blue.
        The size of the Pt atoms has been tuned from (a) to (b) to be able to visualise the arrangement of the A, B and C layers.
    }
    \label{fig:Pt111StructureSideAndTop}
\end{figure}

To investigate potential surface relaxation effects, the three CTR were fitted together to increase the number of data points at each condition using \textit{ROD}, with a simple model consisting of three ABC layers of [111] oriented platinum, C being the topmost layer (fig. \ref{fig:Pt111StructureSideAndTop}).
These three layers are on top of the rest of the crystal, so forth denominated as the \textit{bulk}, separating it from the \textit{surface} of the crystal in which relaxation effects can be detected.

Atoms on the same layer always share the same out-of-plane position and atomic displacement, introduced to see if surface relaxations effect could be detected at different atmospheres.
Four different models have been tested to fit the CTR intensity as a function of $L$.
In the first model, only the Pt atoms in the first topmost layer have a common out-of-plane displacement parameter.
In the second and third models, the two and three topmost layers have a unique out-of-plane atomic displacement, while in the fourth model the two topmost layers share the same out-of-plane atomic displacement.

During the fitting process, in-plane atomic displacements were excluded as one of the parameters to be adjusted because the presence of too many parameters posed challenges in achieving a successful convergence of the fitting routine.
The roughness parameter $\beta$ was set free between 0 and 0.5.% (unitless parameter, see $\beta$ roughness model detailed in sec. \ref{sec:CTR}).

\begin{figure}[!htb]
    \centering
    \includegraphics[width=\textwidth]{/home/david/Documents/PhDScripts/SixS_2023_04_SXRD_Pt111/figures/fit_comparison_1_last_layer_free.pdf}
    \caption{
        Fitting results for roughness parameter $\beta$ (a) and out-of-plane strain $\sigma_z$ (b) as a function of the experimental conditions.
    }
    \label{fig:CTRFit111}
\end{figure}

The best fit was found for the first model in which only the C layer was allowed to have a common out-of-plane atomic displacement $\delta_z$ between \qty{-0.05}{\angstrom} and \qty{0.05}{\angstrom}.
The position of the Pt atoms on the A and B layers were fixed following the position of the atoms in the bulk.

The strain with respect to the bulk was computed following eq. \ref{eq:StrainDiffraction}, the reference was set to the magnitude of the Pt(111) out-of-plane vector, \textit{i.e.} $|\vec{c}_{(111)}| = \qty{6.797}{\angstrom}$.
The evolution of the CTR roughness and of the strain of each layer is shown in fig. \ref{fig:CTRFit111}.

The roughness of the Pt(111) surface (fig. \ref{fig:CTRFit111} - a) increases with the introduction of Argon in the cell at \qty{450}{\degreeCelsius}, which is probably due to the presence of impurities in the gas flow.
The introduction of oxygen in the cell further increases the surface roughness, as expected from the formation of the different surface oxides visible in the in-plane reciprocal space maps.
Adding ammonia in the reactor cell, which was seen to remove the different surface oxides, has also the effect of decreasing the surface roughness.

The surface roughness increases slightly again when oxygen is removed, but reaches a value of 0 when both gases are removed from the reactor, falling back to an inert atmosphere, with a lower surface roughness than at the beginning of the measurement (visible also in fig. \ref{fig:CTRPt111} and consistent with the reflectivity results inf fig. \ref{fig:ReflectoCycle}).
It seems that the ammonia oxidation cycle has effectively \textit{cleaned} the surface from the presence of impurities or surface oxides.
Finally, the re-introduction of \qty{5}{\milli\bar} of oxygen increases the surface roughness again, in accordance with the formation of surface oxides detected during in-plane reciprocal space maps.

Overall, a very low amount of strain is detected on the surface, almost imperceptible when observing position of the CTR minimum in fig. \ref{fig:CTRPt111}.
From the fitting results, the topmost layer is already under tensile strain at UHV, further increased by the presence of Argon and possible impurities from the opening of the gas valves.

The largest evolution in the strain values comes from the high oxygen atmosphere, which has the effect of decreasing the surface strain, with an out-of-plane lattice parameter almost equal to the bulk value.
The formation of surface oxides observed under this atmosphere does not seem to have a very important effect on the surface relaxation state.
The introduction of ammonia increases again the surface strain, higher ammonia to oxygen ratio coincides with higher tensile strain.

The reacting conditions and the sole presence of ammonia have had the effect of removing the different surface oxides present on the platinum surface.
The roughness quantified \textit{via} reflectivity measurements and the evolution of the $\beta$ parameter in fig. \ref{fig:CTRFit111} (a) also decreases.
It is possible that the tensile strain in the last layer under Argon after the oxidation cycle corresponds to the equilibrium state of a clean Pt(111) surface.

Finally, the re-introduction of \qty{5}{\milli\bar} of oxygen decreases slightly the tensile strain on the topmost layer, a weaker but similar effect to the presence of \qty{80}{\milli\bar} of oxygen.
To conclude, no changes from tensile to compressive strain are observed during the reaction, the presence of oxygen alone in the reactor cell during which the growth of surface oxides has been monitored has the effect of lowering the surface strain, while the presence of ammonia has the opposite effect.
Different reacting conditions are related to the same direction of displacement but with a lower magnitude when lowering the partial pressure of oxygen.

\subsection{Surface species presence}

In order to link surface structure, surface moieties and reaction products, the Pt 4f, N 1s and O 1s XPS spectra were recorded at near ambient pressure at the B07 beamline (Diamond synchrotron), at \qty{450}{\degreeCelsius}.
The same order in the ammonia oxidation cycle was repeated as during the SXRD experiment, with the same ratio between reaction products.
No carrier gas is used to keep the total pressure constant, the reactant pressure is lowered to \qty{11}{\percent} of the pressure during the SXRD experiment as a compromise between high pressure and surface photoelectron detection.
The conditions have been resumed in tab. \ref{tab:ConditionsXPS}.
The mass spectrometer available at the B07 beamline allows us to monitor the presence of the reactants and products close to the sample surface.
The pressure of gases going through the same aperture of the electron analyser is measured, shown in fig. \ref{fig:XPS111RGA}.

\begin{figure}[!htb]
    \centering
    \includegraphics[width=\textwidth]{/home/david/Documents/PhDScripts/B07_2022_04_XPS/Figures/pt111_time.pdf}
    \caption{
        Evolution of reaction product partial pressures as a function of time during the XPS experiment on the Pt(111) single crystals at \qty{450}{\degreeCelsius}.
        Transition between conditions are indicated with dashed vertical lines.
    }
    \label{fig:XPS111RGA}
\end{figure}

% Describe rga
A high \ce{O_2}/\ce{NH_3} ratio equal to \num{8} favours the production of \ce{NO} as expected, accompanied by a high amount of water, small amounts of \ce{N_2} and \ce{N_2O} can also be detected.
Approximately half of the pressure of ammonia is still detected, which means that the oxygen cannot be considered to be in excess, and that the complete oxidation of ammonia is probably limited by the availability of active sites.

Lowering the amount of oxygen by reducing the \ce{O_2}/\ce{NH_3} ratio to \num{0.5} has the remarkable effect of shifting the reaction selectivity entirely towards \ce{N_2}, water is also detected.
\ce{H_2} coming from the simultaneous dissociation of \ce{NH_3} can be measured, not observed under a higher pressure of oxygen, which means that this reaction is not favoured when oxygen is present in the reactor.
Oxygen being undetected by the mass spectrometer when the \ce{O_2}/\ce{NH_3} ratio is equal to \num{0.5}, all of the introduced oxygen dissociates on the catalyst surface and participates in the production of \ce{N_2} and \ce{H_2O} via the oxidation reaction.
Ammonia can be thus considered to be in excess, and partly decomposing towards \ce{N_2}.
The surface sites are probably occupied mainly by nitrogen-rich species that cannot find a nearby oxygen or OH to react with, eventually decomposing towards nitrogen.

The removal of oxygen shows that more ammonia is consumed but without producing water.
Only the dissociation of ammonia happens on the catalyst surface, the production of nitrogen decreases even though more ammonia is consumed.
It is not clear why more nitrogen is produced under the presence of oxygen, when more ammonia is used for less production of \ce{N_2} after the removal of oxygen.
Both the oxidation and dissociation of ammonia yield 0.5 nitrogen for each molecule of ammonia.
The transition around \qty{28.5}{\hour} is due to a problem in the monitoring of the reactor total pressure which was not correctly set to \qty{1.1}{\milli\bar}.

\subsubsection{N 1s and O 1s levels}

N 1s and O 1s levels were recorded to probe for the existence of specific surface species, allowing us to obtain more information about the reaction mechanism, as well as the link between surface state and selectivity.
The evolution of the N 1s and O 1s XPS spectra for different atmospheres is presented in fig. \ref{fig:O1sN1sPt111}.
Binding energy are given with reference to the Fermi level, all the reported peaks and corresponding species are detailed in tab. \ref{tab:XPSPt111}.

% high ox
The oxidation cycle is started by introducing \qty{8.8}{\milli\bar} of oxygen in the reactor.
The presence of gas phase oxygen (\ce{O_{2,g}}) can logically be confirmed in the O 1s spectra by a characteristic peak doublet around \qty{538}{\eV}.
The positions are shifted in energy with respect to literature (\qty{539.3}{\eV}, \qty{540.4}{\eV}, Avval et al. \cite{Avval2022}).
This signal is also present when the \ce{O_2}/\ce{NH_3} ratio is equal to 8, but not when equal to 0.5, confirming that all the oxygen is used in the reactor.
\ce{O_{2,g}} is also detected when only \qty{0.55}{\milli\bar} of oxygen is in the cell.
A very broad signal extending from \qtyrange{528}{532}{\eV} is probably hiding a various amount of peaks of similar intensity, linked to the presence of oxygen-rich species.

Fisher et al \parencite*{Fisher1980} report adsorbed oxygen (\ce{O_{a}}) to give a peak at \qty{529.8}{\eV} on Pt(111), with adsorbed hydroxyl groups (\ce{OH_{a}}) at \qty{531}{\eV} when exposing the surface to water.
Peuckert et al. \parencite*{Peuckert1984} have studied various oxidised Pt surfaces and indexed a peak at \qty{530.2}{\eV} for \ce{O_{a}} on Pt(111), and \ce{OH_{a}} at \qty{531.5}{\eV} for polycristalline Pt.
Derry et al. \parencite*{Derry1984} report a peak at \qty{530.8}{\eV} for \ce{O_{a}} on Pt(111) during its exposition to oxygen, while Zhu et al. \parencite*{Zhu2003} report \ce{O_{a}} at \qty{529.9}{\eV} when probing the dissociation of \ce{NO} on the Pt(111) surface.
Fantauzzi et al. \parencite*{Fantauzzi2017} report oxygen surface species at \qty{529.7}{\eV} during the oxidation of Pt(111) at \qty{225}{\degreeCelsius}, similarly to Miller et al. \parencite*{Miller2014}.

During a recent study of the oxidation of ammonia at different pressures and \ce{O_2}/\ce{NH_3} ratio on Pt(111), (2x2) chemisorbed oxygen and hydroxyl groups were reported respectively at \qty{529.7}{\eV} and \qty{531.4}{\eV} in \qty{1}{\milli\bar} of oxygen at \qty{325}{\degreeCelsius} \parencite{Ivashenko2021}.

In the current XPS study, the signal to noise ratio is too low under the presence of \qty{8.8}{\milli\bar} of oxygen to characterise the exact surface state, it is probable however than hydroxyl groups as well as atomic oxygen are adsorbed on the surface around \qty{529.7}{\eV} and \qty{531.4}{\eV} respectively.
Similar O 1s peaks can be seen when the oxygen pressure is equal to \qty{0.55}{\milli\bar} and without ammonia, but with a higher apparent amount of atomic oxygen species in comparison to hydroxyl groups.

\begin{figure}[!htb]
    \centering
    \includegraphics[width=\textwidth]{/home/david/Documents/PhDScripts/B07_2022_04_XPS/Figures/Pt111/O1sN1s_700.pdf}
    \caption{
        Spectra collected at the O 1s (a) and N1 s (b) levels under different atmospheres at \qty{450}{\degreeCelsius} with an incoming photon energy of \qty{700}{\eV}.
        The spectra are normalised and shifted in intensity to highlight the presence of different peaks.
    }
    \label{fig:O1sN1sPt111}
\end{figure}
\begin{table}[!htb]
\centering
\resizebox{\textwidth}{!}{%
    \begin{tabular}{@{}ll|lllllll@{}}
    \toprule
    \multirow{3}{*}{Partial pressures (mbar)} & \ce{Ar}   & 1 & 0   & 0   & 0    & 0   & 1 & 0    \\
                                              & \ce{NH_3} & 0 & 0   & 1.1 & 1.1  & 1.1 & 0 & 0    \\
                                              & \ce{O_2}  & 0 & 8.8 & 8.8 & 0.55 & 0   & 0 & 0.55 \\
    \midrule
    Gas presence (decreasing & & Ar & \ce{O_2} & \ce{O_2}, \ce{H_2O}, \ce{NO}   & \ce{H_2O}, \ce{NH_3} & \ce{H_2}, \ce{NH_3} & Ar & \ce{O_2} \\
    pressure order)          & &    &          & \ce{NH_3}, \ce{N_2}, \ce{N_2O} & \ce{N_2}, \ce{H_2}   & \ce{N_2}            &    &          \\
    \midrule
    \multicolumn{2}{l|}{N 1s: peak positions}
        & No data          & No peak          & \qty{402.6}{\eV} & \qty{404.1}{\eV} & \qty{405.3}{\eV} & \qty{400.4}{\eV} & No peak          \\
     &  &                  &                  &                  & \qty{399.8}{\eV} & \qty{400.8}{\eV} & \qty{398.4}{\eV} &                  \\
     &  &                  &                  &                  & \qty{397.5}{\eV} & \qty{398.4}{\eV} &                  &                  \\
    \multicolumn{2}{l|}{Attributed surface species}
        &                  &                  & Not assigned     & \ce{N_{2,g}}     & \ce{N_{2,g}}     & \ce{NH_{3,a}}    &                  \\
     &  &                  &                  &                  & \ce{NH_{3,a}}    & \ce{NH_{3,g}}    & \ce{NH_{x,a}}    &                  \\
     &  &                  &                  &                  & \ce{N_a}         & \ce{NH_{x,a}}    &                  &                  \\
    \midrule
    \multicolumn{2}{l|}{O 1s: peak positions}
        & \qty{532.4}{\eV} & \qty{538.2}{\eV} & \qty{538.5}{\eV} & \qty{534.0}{\eV} & \qty{532.4}{\eV} & \qty{534.0}{\eV} & \qty{538.3}{\eV} \\
     &  &                  & \qty{537.1}{\eV} & \qty{537.5}{\eV} & \qty{532.0}{\eV} &                  & \qty{532.4}{\eV} & \qty{537.2}{\eV} \\
     &  &                  & \qty{531.4}{\eV} & \qty{534.0}{\eV} &                  &                  &                  & \qty{531.4}{\eV} \\
     &  &                  & \qty{529.7}{\eV} &                  &                  &                  &                  & \qty{529.7}{\eV} \\
    \multicolumn{2}{l|}{Attributed surface species}
        & \ce{H_2O_a}      & \ce{O_{2,g}}     & \ce{O_{2,g}}     & \ce{H_2O_g}      & \ce{H_2O_a}      & \ce{H_2O_g}      & \ce{O_{2,g}}     \\
     &  &                  & \ce{O_{2,g}}     & \ce{O_{2,g}}     & \ce{H_2O_a}      &                  & \ce{H_2O_a}      & \ce{O_{2,g}}     \\
     &  &                  & \ce{OH_a}        & \ce{H_2O_g}      &                  &                  &                  & \ce{OH_a}        \\
     &  &                  & \ce{O_a}         &                  &                  &                  &                  & \ce{O_a}         \\
    \bottomrule
    \end{tabular}%
    }
    \caption{Indexing of peaks measured during the oxidation of ammonia of the Pt(111) surface.}
\label{tab:XPSPt111}
\end{table}

% ratio 8
No clear nitrogen species can be detected in the N 1s spectra when introducing \qty{1.1}{\milli\bar} of ammonia in the reactor while keeping the pressure of oxygen equal to \qty{8.8}{\milli\bar}.
The total pressure in the cell is probably too high to resolve the signals corresponding to different adsorbed nitrogen species.
The presence of gas phase nitric oxide (\ce{NO_{g}}), ammonia (\ce{NH_{3,g}}), nitrogen (\ce{N_{2,g}}), and nitrous oxide (\ce{NO_{2,g}}) is also expected from the mass spectrometer data but could not be detected.
The main product, \ce{NO_{g}}, is expected between \qty{404.5}{\eV} and \qty{406.7}{\eV}, observed during reacting conditions with an equal amount of oxygen and ammonia, at a total pressure of \qty{1}{\milli\bar}, and temperature of \qty{325}{\degreeCelsius} \parencite{Ivashenko2021}.
The same study reports \ce{N_{2,g}} between \qty{403.9}{\eV} and \qty{404.8}{\eV} and \ce{NH_{3,g}} at \qty{400.4}{\eV}, while \ce{NH_{3,g}} was also detected at \qty{400.7}{\eV} under \qty{1}{\milli\bar} of ammonia.
The peak at \qty{402.6}{\eV} in this study could not be indexed.

Likewise, no clear adsorbed nitrogen species could be detected in the O 1s level.
For example, molecular \ce{NO} is expected to yield peaks between \qty{530}{\eV} and \qty{532}{\eV} \parencite{Kiskinova1984, Zhu2003, Gunther2008}.
Nevertheless, the presence of those adsorbed species cannot be ruled out due to the high pressure in the chamber lowering the signal to noise ratio.
Gas phase water (\ce{H_2O_{g}}) is visible in the O 1s spectra by a peak at \qty{534}{\eV}, as reported during the oxidation of ammonia by Weststrate et al. \parencite*{Weststrate2006}, proving the catalytic activity.
The energy difference between the low energy peak of \ce{O_{2,g}} and \ce{H_2O_{g}}, \qty{3.5}{\eV}, is close to the difference reported in literature for pure gas phases, equal to \qty{3.3}{\eV} \parencite{Linford2019, Avval2022}.

% It is possible that the two peaks of gas phase \ce{O_2} are hiding the signal of gas phase \ce{NO}, expected to be near \qty{538}{\eV}.
% Indeed, the ratio between both \ce{O_{g}} peaks goes from to \num{2.02} to \num{2.54}.

% ratio 0.5
When lowering the pressure of oxygen to \qty{0.55}{\milli\bar}, a condition under which nitrogen-rich products are favoured, three peaks can be detected in the N 1s level.
The peak at \qty{397.5}{\eV} is characteristic of adsorbed atomic nitrogen (\ce{N_{a}}) on Pt(111) \parencite{vandenBroek1999, Zhu2003}.
The peaks at \qty{399.8}{\eV} and \qty{404.1}{\eV} are attributed to adsorbed ammonia (\ce{NH_{3,a}}) and \ce{N_{2,g}}.
The reason why we attribute the peak at \qty{399.8}{\eV} to \ce{NH_{3,a}} is because a peak at \qty{400.8}{\eV} is detected when only ammonia is in the cell, which corresponds to the aforementioned \ce{NH_{3,g}}.
The energy difference between \ce{N_{a}}, \ce{NH_{a}}, and \ce{NH_{2,a}} is reported to be approximately \qty{0.95}{\eV}, \qty{1.9}{\eV} in total on Pt(111) \parencite{Ivashenko2021}, which is too few to link the peak at \qty{399.8}{\eV} to \ce{NH_{2,a}} with respect to the \ce{N_{a}} peak.
% Moreover, \ce{NH_{3,a}} is reported at \qty{400.0}{\eV} on Pt(410) by Günther et al. \parencite*{Gunther2008}.

At this condition, no oxygen is anymore detected in the reactor, as observed with the mass spectrometer.
No \ce{OH_{a}} and \ce{O_{a}} peaks can be detected even though the total pressure was divided by \num{6}, increasing the detection of photo-electrons.
Adsorbed water groups (\ce{H_2O_{a}}) are reported between \qty{532.2} and \qty{532.9} eV on Pt(111) depending on the surface coverage \parencite{Fisher1980, Kiskinova1985}, to which we tentatively attribute the peak at \qty{532}{\eV}.

The presence of adsorbed water supports a Langmuir-Hinshelwood mechanism with quick stripping of hydrogen from \ce{NH_{x,a}} species by \ce{OH_a} and \ce{O_a}, eventually forming adsorbed water.
The de-hydrogenation process must be limiting the catalytic activity due to the lack of available adsorbed oxygen species near adsorbed ammonia, which could be why we observe \ce{NH_{3,a}} but not \ce{NH_{x,a}}.
Even though we cannot measure the presence of \ce{OH_a} and \ce{O_a}, an Elley-Rideal mechanism in which the \ce{O_2} dissociation happens without adsorbing on the catalyst surface is highly unlikely.
Moreover, it is clear that oxygen adsorbs on the surface from the peaks visible without ammonia present in the cell.
The presence of adsorbed nitrogen could be due to a long residual time on the catalyst before recombination and desorption of \ce{N_2}.
This hypothesis is in accordance with the commonly accepted reaction mechanism detailed in sec. \ref{sec:Mechanism}.

% ammonia
Once we remove oxygen, gas phase water disappears from the O 1s level as the oxidation reaction can no longer happen.
Adsorbed water is still visible, possibly from long desorption time before producing water, or from contaminants.
The energy level differs by \qty{0.5}{\eV} from adsorbed water during reacting conditions, signifying different electronic environments.

As observed in the mass spectrometer, the dissociation of ammonia towards nitrogen still occurs, a slightly shifted \ce{N_{2,g}} peak is reported at \qty{405.3}{\eV}, explained by a change in the work function for gas species exclusively \parencite{Starr2021}.
The large peak linked to atomic nitrogen has disappeared, a peak linked to \ce{NH_{3,g}} is reported as well as a large and weak peak probably linked to \ce{NH_x} groups.
The dissociation of ammonia without oxygen is reported to be slow without the help of oxygen species on Pt(111) \parencite{Offermans2006,Offermans2007, Imbihl2007, NovellLeruth2008}.
This could explain why such a large peak is observed and why \ce{NH_{3,g}} is observed rather than \ce{NH_{3,a}}, since most of the adsorption sites are probably occupied by \ce{NH_x} species.
The difficulty to fully dissociate ammonia is also correlated to the production of hydrogen, it is possible that the combination of two hydrogen removed from ammonia to produce \ce{H_2} is slow, and thus occupies part of the adsorption sites.

% argon
The introduction of argon and removal of ammonia increases the \ce{H_2O_{g}} signal, possibly from contaminants.
Some nitrogen rich species are still visible, possibly from long desorption time.
Removing argon and introducing \qty{0.55}{\milli\bar} of oxygen removes all the N 1s peaks, probably by the oxidation of the leftover \ce{NH_x} species.
The same peaks are observed in the O 1s level as under \qty{8.8}{\milli\bar} of oxygen, linked to hydroxyl groups, \ce{O_{2,g}}, and different atomic oxygen states.

\subsubsection{Pt 4f level}

\begin{figure}[!htb]
    \centering
    \includegraphics[width=\textwidth]{/home/david/Documents/PhDScripts/B07_2022_04_XPS/Figures/Pt111/Pt4f_550_no_fit_merged.pdf}
    \caption{
        Spectra collected at the Pt 4f level under different atmospheres at \qty{450}{\degreeCelsius} with an incoming photon energy of \qty{550}{\eV}.
        A Shirley-type background has been subtracted from all XPS spectra.
        Normalisation performed first by the background intensity and secondly by the maximum intensity to allow a qualitative comparison between different total pressures.
        Spectra before normalisation are shown on the top left.
    }
    \label{fig:Pt4fPt111}
\end{figure}

% For the peaks that showed a good signal to noise ratio, the fitting of the peak shape was realised thanks to the \textit{lmfit} \parencite{Newville2016} package by the means of the Doniach-equation which is the best approximation of the asymmetric peak shape resulting from the convolution of the analyser function and the photoelectron process in metals \parencite{Doniach1970}.

The Pt 4f level was also measured to report possible differences in the electronic configuration of surface platinum atoms.
From the previous SXRD measurements, it was seen that a surface $\alpha$-\ce{PtO_2} oxide grows under the presence of \qty{5}{\milli\bar} of oxygen, but only after at least \qty{22}{\hour}.
Another structure that was determined to be a precursor of surface $\alpha$-\ce{PtO_2} oxide was measured in the minutes following the introduction of oxygen.
This structure could correspond to the oxide stripe hypothesised to be a precursor for surface $\alpha$-\ce{PtO_2} oxide on Pt(111) by Hawkins et al. \parencite*{Hawkins2009}.
Therefore, if surface $\alpha$-\ce{PtO_2} oxide is not expected here since the duration of each condition is below \qty{5}{\hour}, the precursor may be associated to peaks in the O 1s level.

Miller et al. \parencite*{Miller2011} have measured two peaks at \qty{72.1}{\eV} and \qty{73.5}{\eV} under \qty{6.66}{\milli\bar} of oxygen at \qty{450}{\degreeCelsius}.
The \qty{72.1}{\eV} peak is assigned to (i) oxygen between the metallic surface and surface $\alpha$-\ce{PtO_2} oxide as reported by Ellinger et al. \parencite*{Ellinger2008}.
The \qty{73.5}{\eV} peak is attributed to (ii) Pt atoms within the trilayer oxide structure.

However, both peaks are absent under a pressure of \qty{0.66}{\milli\bar} at \qty{350}{\degreeCelsius}, for which the presence of (iii) p(2x2) chemisorbed oxygen and (iv) "4O" oxide surface stripes are linked to two other peaks, respectively \qty{71.1}{\eV} and \qty{71.6}{\eV}-\qty{71.7}{\eV}.

In a following study, slightly shifted high intensity peaks at \qty{72.2}{\eV} and \qty{73.6}{\eV} are linked to the presence of surface $\alpha$-\ce{PtO_2} oxide fully covering a Pt(111) crystal \parencite{Miller2014}.
The photon energy for both experiments is equal to \qty{275}{\eV}, whereas the photon energy is here equal to \qty{550}{\eV}.
Thus, both studies by the Miller group are more sensitive to the surface structure since the photons do not penetrate as deep in the catalyst.

Interestingly the preparation of the crystal surface to grow surface $\alpha$-\ce{PtO_2} oxide is not too far from this study.
The Pt(111) crystal was exposed to \qty{13.33}{\milli\bar} of oxygen for \qty{10}{\minute} while cycling the temperature from \qtyrange{25}{525}{\degreeCelsius} four times.

A small peak can be identified near \qty{71.6}{\eV} in the current experiment under \qty{8.88}{\milli\bar} of oxygen that could correspond to the oxide stripe structure.
Its presence is not certain since the intensity is very low, corresponding potential peaks in the O 1s level cannot be resolved either.

No clear peak could be detected near \qty{73.5}{\eV} in the Pt 4f$_{5/2}$ level, but a very small peak can be seen at \qty{72.1}{\eV}.
Since the presence of surface $\alpha$-\ce{PtO_2} oxide was linked to very high intensity peaks by Miller et al. \parencite{Miller2014}, we can safely assume that this structure is not present in this study.
It seems that the temperature cycling is crucial to grow $\alpha$-\ce{PtO_2} on Pt(111).

The peak position is shifted by \qty{0.09}{\eV} after the introduction of \qty{8.88}{\milli\bar} of oxygen in the cell compared to the presence of \qty{1}{\milli\bar} of argon.
Overall, the lack of clear peak corresponding the "4O" oxide stripe and $\alpha$-\ce{PtO_2} phases shows that the oxygen on the Pt(111) surface is mostly chemisorbed, which could also be why the peak is slightly shifted towards \qty{71.1}{\eV}.

When introducing \qty{1}{\milli\bar} of ammonia in the reactor, the Pt 4f$_{5/2}$ peak is shifted back to the position under inert atmosphere, while a new component is measured at \qty{70.5}{\eV}.
Reducing the pressure of oxygen to \qty{0.55}{\milli\bar} further increases the intensity of this component compared to the maximum peak intensity.
Removing oxygen but keeping ammonia in the reactor removes this peak.
Since the intensity of this peak increases when the \ce{O_2}/\ce{NH_3} ratio decreases, \textit{i.e.} when all the oxygen is consumed, it is probably linked to the presence of adsorbed nitrogen species on the platinum surface.
The absence of this peak under the presence of ammonia in the cell, for which adsorbed nitrogen can not be detected, supports a link with \ce{N_a}.
Multiple peaks may also be present.

Only a smaller shift is repeated when \qty{0.55}{\milli\bar} of oxygen are introduced after \qty{1}{\milli\bar} or argon, approximately equal to \qty{0.2}{\eV} after the ammonia oxidation cycle, both spectra are very similar, some nitrogen species possibly left on the surface are probably removed by oxygen which explains the lower intensity near \qty{70.5}{\eV}.

% Parkinson et al. \parencite*{Parkinson2003} measured a peak at \qty{76.8}{\eV} after oxygen adsorption at room temperature, which disappeared after annealing at \qty{500}{\degreeCelsius}.
% This peak was linked to a peak between \qty{530.2}{\eV} and \qty{530.8}{\eV} in the O 1s level, and asigned to a sub-surface oxide.
% Weaver et al. \parencite*{Weaver2005} have reported at peak at \qty{76.9}{\eV} at \qty{175}{\degreeCelsius} after deposition of oxygen on the surface.
\textcolor{red}{This Chapter should demonstrate that you have conducted a thorough and critical investigation of relevant sources.
Apart from a presentation of the sources of your data, this chapter allows you to critically discuss the data (whatever these data are, ‘quantitative’ or ‘qualitative’, primary or secondary), which is proof of good research. You can even do good research with poor data but you must demonstrate that you are aware of the data quality and accordingly are careful in your interpretations. Essentially, there are three aspects to consider:
\begin{enumerate}
\item	Reliability, which, for example, could depend on whether they are estimates or more direct evidence;
\item	Representativity, which is about how typical the data are; for example, you may have arguments why the very few cases are typical or you may carry out statistical tests;
\item Validity, which is about the relevance of the data for your case. Strictly speaking, sometimes no valid data are available but one may argue that there are other data which could be used as ‘proxies’.) 
\end{enumerate}
}

\section{Operando and in-situ experiments at SixS}



\section{BCDI on isolated Pt nanoparticles}

\textbf{Here, we will evaluate the metal-support interaction during reaction by monitoring the activity and structure evolution of the Pt NPs dewetted on three different support materials: sapphire, MgO and TiO2 [6]. This will allow examining the effect of the support on the metal catalyst. A careful in situ analysis of the properties of the nanoparticles (size, shape, strain, refaceting, support interaction, etc) in 3D is of essential importance to gain more understanding of the behaviour of these nanocrystals during a catalytic reaction.}

We want to evaluate the strain evolution of Pt particles and probe the impact of the support during CO oxidation.

\subsection{Preliminary test on gas reactor}

We performed several test on the small gas reactor to see how high we could go in temperature, which reactions happen at which temperature, etc ...
The data is in $PhDScripts/test\_reactor\_cell$

On 20/01, we confirmed that batch were usefull ?  (long batch stochio) and that products of both reactions (for the production of CO2, and then of nitrogen oxides respectively from CO and NH3 can be detected !)

Atmospheric pressure : $\approx$ 1 bar = 1 013.25 mbar = 1 013.25 hPa


uhv = 10 -9 mbar, 10-7 pascal

near ambient pressure ($>$ 1 mbar) 

DATA ACQUISITION

Rotating the sample results in rotating the reciprocal space
Whenever this happens, a diffracted beam is originated in the center of the Ewald sphere and passes through the reciprocal point that lies on the Ewald spherical surface... In these circumstances the so-called Bragg law is fulfilled. The set of all diffracted beams constitute the so-called diffraction pattern, which is subject to detection and evaluation. The reader should be aware that a complete diffraction pattern can contain a highly variable number of diffraction beams, from hundreds (simple inorganic compounds) to hundreds of thousands (proteins or viruses).
 
Ewald sphere is centered on the sample, reciprocal lattice on where $\vec{k_i}$ (from the sample) meets the sphere

Larger q -> higher hkl indexes

\subsection{Measurements}



\lipsum


\subsection{Reaction}


\lipsum



\section{SXRD on Pt 100 and Pt 111}
\subsection{Surface reconstruction}
\lipsum

\subsection{CTR and roughness}

\lipsum

\section{Ambient pressure XPS}
\subsection{Multi component analysis at ambient pressure}


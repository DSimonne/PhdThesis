\section{X-ray interaction with matter}

\subsection{Scattering from electrons and atoms}

We begin our discussion of x-ray scattering by first considering scattering from a single free electron using classical electromagnetic theory. During elastic scattering, the oscillating electric field of the x-ray wave exerts forces ($\vec{F} = q \vec{E_i}$ , $q$ represents the charge) on the electron, causing it to accelerate and oscillate in the same direction as the incident field.

The oscillating electron then emits a spherical wave with the same wavelength as the incident beam (Thomson scattering) and this is the scattered field.

If we consider how the incident x-ray wave will interact with the different charge elements relative to the origin of the atom, you can see from Figure 1.4 that there is a path length difference of   where  =  is the scattering vector and is equal to the change experienced by the wavevector during scattering.

\subsection{Diffraction}



\lipsum


\subsection{Reciprocal space}



\lipsum


\subsection{Brillouin zone}

The importance of the Brillouin zone stems from the description of waves in a periodic medium given by Bloch's theorem, in which it is found that the solutions can be completely characterized by their behavior in a single Brillouin zone.

The first Brillouin zone is the locus of points in reciprocal space that are closer to the origin of the reciprocal lattice than they are to any other reciprocal lattice points (see the derivation of the Wigner–Seitz cell). Another definition is as the set of points in k-space that can be reached from the origin without crossing any Bragg plane.

A related concept is that of the irreducible Brillouin zone, which is the first Brillouin zone reduced by all of the symmetries in the point group of the lattice (point group of the crystal).

k-vectors exceeding the first Brillouin zone (red) do not Brillouin any more information than their counterparts (black) in the first Brillouin zone. k at the Brillouin zone edge is the spatial Nyquist frequency of waves in the lattice, because it corresponds to a half-wavelength equal to the inter-atomic lattice spacing a.



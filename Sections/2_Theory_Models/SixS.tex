\section{Synchrotron radiation}

Presentation of the MED environment ... 

\subsection{Tailored for the study of materials}

\subsection{The SixS beamline}

\subsubsection{Diffractometer}
theta two theta is mu gamma or omega delta

\subsubsection{Mass flow controller}

\subsubsection{Residual Gas Analyser}


\subsection{Data collection}

Explain here how we actually collect the data

Ewald sphere is quite important here.

\subsection{Intensity of a Bragg peak}
Neglecting absorption, the intensity of a Bragg peak $I_{nuc}$ as a function of its Miller indices \textit{hkl} and of $2\theta$ can be written as:

\begin{equation}
    \label{eq:PeakIntensity}
    I_{nuc} = A \times |F_{hkl}|^2  \times j_{hkl} \times L(2\theta) \times \exp{(-2W)}
\end{equation}
where A is an instrument constant.

\subsubsection{Structure factor}
$F_{hkl}$ is known as the structure factor, it is given by:
\begin{equation}
    \label{eq:StrucFactor}
    F_{hkl} = \sum_{j=0}^n b_j \exp{(-2\pi i \vec{Q}. \vec{r_{j0}})}
\end{equation}

The structure factor is the summation of the contribution to the scattering energy of each atoms at the position $\vec{r_{j0}}$ of scattering length $b_j$ in our unit cell for a given $\vec{Q}$.
The position of the atom $\vec{r_{j0}}$ is given by:

\begin{equation}
    \label{eq:AtomPos}
    \vec{r_{j0}} = x_j\vec{a} + y_j\vec{b} + z_j\vec{c}
\end{equation}

\newpage
\section{BCDI} \label{sec:BCDI}

Bragg Coherent Diffraction Imaging (BCDI) \parencite{robinson_coherent_2009} is a powerful technique for the non-destructive characterisation of material structure in three dimensions with unparalleled spatial and strain resolution, few nanometers \parencite{labat_inversion_2015,cherukara_anisotropic_2018} and $10^{-4 }$ respectively \parencite{Newton2010,Lauraux2020}.

The BCDI method is reliant on the coherence (sec. \ref{sec:Coherence}) of the light available, thus only began to be exploited at third generation synchrotron sources \parencite{Miao1999,Miao2000,Robinson2001}.
Since then it was developed into a characterisation tool for \textit{in situ / operando} studies of materials structure \parencite{ulvestad_situ_2016,Kim2019,Carnis2021} and will gain significantly (several orders of magnitude) from further source and beamline improvements as many synchrotrons have recently completed or are in the process of an upgrade from $3^{rd}$ to $4^{th}$ generation synchrotrons.
Focusing optics that allowed the development of coherent imaging techniques by increasing the coherent flux on the probed sample are discussed in sec. \ref{sec:SIXS}.

\subsection{Scattering by strained crystals}

In this section, the scattering intensity of a single crystal of dimensions below the coherent volume of the source is considered, the waves scattered from the crystal interfere with each other and maximum intensity is observed when the scattering vector satisfies the Laue condition.

Note that the following derivations only apply if the diffraction pattern is viewed at a distance far away from the diffracting object in a region known as the far-field or Fraunhofer region where the scattered waves can be considered as plane waves.
Moreover, the scattering volume is assumed to be in the focal plane of the focusing optics and far from the source so that the waves in the scattering volume can also be described as plane waves.
Finally, the scattering volume is assumed to be small with dimensions below the x-ray extinction length to ignore multiple scattering and stay in the kinematical approach of diffraction.
The amplitude of the incident field is considered to be homogeneous in the scattering volume.

A dynamical approach to BCDI has been under recent study by the community \parencite{Yan2014, Shabalin2017, Hu2018, Gao2022} by the means of the Takagi-Taupin approach to dynamical diffraction \parencite{Takagi1962, Takagi1969} which tend to validate the kinematical approach when working with slightly strained sub-micron sized crystals \parencite{Karpov2019, Barringer2021}.
The phase of the incoming and scattered waves must however be corrected for refraction and absorption \parencite{Harder2007, Gao2022} which does not contradict the main hypothesis of the kinematical approach which is that in the Born approximation, the structure factor of the crystal is equal to the Fourier transform of its electronic density \parencite{Paganin}.

Another approach is to write the structure factor as the sum over each atom present in the crystal of the atomic scattering factor multiplied by a phase component (eq. \ref{eq:FCrystal1}).
In general, the Fourier transform approach is prefered since one can use quick algorithms such as the Fast Fourier Transform \parencite{Cooley1965, Cochran1967} to compute the structure factor as described below, whereas the atomistic approach relies on knowing the position of each atom in the crystal.

As seen in sec. \ref{sec:StructureFactor}, when considering the crystal to be made of repeating unit cells that organize themselves on the crystal lattice, the structure factor can also be written as the convolution between the Lattice factor $F_{lat}$ and the unit cell structure factor $F_{uc}$.
The shape of the scattering intensity around Bragg peaks is then given by $F_{lat}$, which is the Fourier transform of the crystal's lattice, whereas $F_{uc}$ is the Fourier transform of the crystal's unit cell electronic density.

\begin{figure}[!htb]
    \centering
    \includegraphics[width=0.66\textwidth]{/home/david/Documents/PhD/Figures/introduction/StrainedCrystal.pdf}
    \caption{
    Simplified representation of a crystal in which each dot corresponds to a unit cell.
    The green dots represent displaced unit cells from their ideal position by a displacement vector $\vec{u}_k$.
    }
    \label{fig:StrainedCrystals}
\end{figure}

In the case of strained crystals, eq. \ref{eq:Fcrystal} can be rewritten as follows

\begin{align}
    \label{eq:FcrystalBCDI1}
    F_{crystal}(\vec{q}) & = \sum_j^{N_{atoms, uc}} f_j(\vec{q}) e^{i\vec{q}.(\vec{r}_{0,j} + \vec{u}_{j})} \sum_k^{N_{uc}} e^{i\vec{q}.\vec{R_k}}\\
    \label{eq:FcrystalBCDI2}
    F_{crystal}(\vec{q}) & = \sum_j^{N_{atoms, uc}} f_j(\vec{q}) e^{i\vec{q}.\vec{r}_{0,j}} \sum_k^{N_{uc}} e^{i\vec{q}.(\vec{R_k} + \vec{u}_{k})}
\end{align}

where $\vec{r}_{0,j}$ is the ideal position of the atom $j$ in the unit cell and $\vec{u}_{j}$ is the displacement from that position.
In eq. \ref{eq:FcrystalBCDI2}, all atoms in the unit cell are considered to have the \textit{same displacement} to be able to shift the displacement parameter from the atomic position to the unit cell position, $\vec{u}_{k}$ becomes the displacement of the unit cell $k$.

To reduce the complexity in the second part of eq. \ref{eq:FcrystalBCDI2}, only the intensity in a volume situated near the Bragg peak is considered so that $\vec{q} \approx \vec{G}$ where $\vec{G}$ designates the reciprocal space vector that correponds to the position of the Bragg peak in reciprocal space \parencite{Pfeifer2006, Minkevich2007, Harder2007}.

\begin{align}
    \label{eq:FcrystalBCDI3}
    F(\vec{q}) & = \sum_j^{N_{atoms, uc}} f_j(\vec{q}) e^{i\vec{q}.\vec{r}_{0,j}} \sum_k^{N_{uc}} e^{i\vec{q}.\vec{R_k}} e^{i\vec{q}.\vec{u}_k}\\
    \label{eq:FcrystalBCDI4}
    F(\vec{q}) & = F_{uc}^k(\vec{q}) \sum_k^{N_{uc}} e^{i\vec{q}.\vec{R_k}} e^{i(\vec{G}+\vec{q}-\vec{G}).\vec{u}_k}\\
    \label{eq:FcrystalBCDI5}
    F(\vec{q}) & = F_{uc}^k(\vec{q}) \sum_k^{N_{uc}} e^{i\vec{q}.\vec{R_k}} e^{i\vec{G}.\vec{u}_k} \times e^{i(\vec{q}-\vec{G}).\vec{u}_k}\\
    \label{eq:FcrystalBCDI6}
    F(\vec{q} \approx \vec{G}) & = \sum_k^{N_{uc}} F_{uc}^k(\vec{G}) e^{i\vec{q}.\vec{R_k}} e^{i\vec{G}.\vec{u}_k}\\
    \label{eq:FcrystalBCDI7}
    F(\vec{q} \approx \vec{G}) & = DFT[\rho_k(\vec{r}) e^{i\vec{G}.\vec{u}_k}]
\end{align}

The first part of eq. \ref{eq:FcrystalBCDI6} can be considered to be the Fourier transform of the unit cell $k$ of electronic density $\rho_k(\vec{r})$ in which all atoms have the same displacement vector $\vec{u}_k$, in this equation the electronic density is allowed to vary from one unit cell to another.
Moreover, one can recognize the expression of a discrete Fourier transform (DFT) of frequency $\vec{G}$ in which $\rho_k(\vec{r})$ is the $k-th$ sample of the series that consist in $N_{uc}$ samples \parencite{Cooley1965, Cochran1967, favre-nicolin_semiconductor_2011, Godard2021}.
The reason for the approximation $\vec{q} \approx \vec{G}$ is simply to be able to use a Fast Fourier Transform (FFT) algorithm to compute the DFT that has a linearithmic time complexity $O(n \, log(n))$.
This approximation breaks down far away from the Laue condition or if the strain is so important that the condition $(\vec{q}-\vec{G}).\vec{u}_k<<1$ is not anymore respected to move from eq. \ref{eq:FcrystalBCDI5} to eq. \ref{eq:FcrystalBCDI6}.

In the case of highly strained crystal, or when the intensity must be computed far away from Bragg peaks, it is still possible to resort to so-called \textit{atomistic} simulations by computing the scattering intensity from the position of each atom in the crystal (eq. \ref{eq:FCrystal1}).
However, the position of each atom must be known and, for a large crystal, the sum becomes too large to handle even with high-perfomance clusters due to a quadratic time complexity $O(n^2)$.
Atomistic simulations are usually used when studying crystals of dimensions below 100 nm \parencite{Dupraz2022}.

For a non-strained crystal, the scattering intensity distribution around every Bragg peak is the same, equal to the Fourier transform of the crystal's electronic density.
In the case of a strained crystal, the additional phase factor introduced in eq. \ref{eq:FcrystalBCDI7} breaks the centrosymmetry around each Bragg peak due to the inhomogeneous displacement field in the crystal.
The effect of the displacement can be seen in the distortion of the scattering intensity around Bragg peaks.
It is important to notice that the phase in eq. \ref{eq:FcrystalBCDI7} is equal to the dot product between the displacement vector and the reciprocal space vector, which means that to possess the complete displacement field, one must collect the scattered intensity around three non-coplanar Bragg peaks that will yield the displacement field over an orthogonal base ($\vec{q_1}, \vec{q_2}, \vec{q_3}$).

\subsubsection{Data collection} \label{sec:DataCollectionBCDI}

The data collection during a BCDI measurement is performed by continuously rocking the sample around the incident angle, which will result in a tilt of the associated Ewald sphere and in the detector probing a different slice of the reciprocal space for each step.
This measurement is called a rocking curve (fig. \ref{fig:3DDP} - left), the associated 3D scattering intensity is illutrated in fig. \ref{fig:3DDP} (right).

\begin{figure}[!htb]
    \centering
    \includegraphics[width=0.39\textwidth]{/home/david/Documents/PhD/Figures/introduction/RockingCurve.png}
    \includegraphics[width=0.59\textwidth]{/home/david/Documents/PhD/Figures/bcdi_data/3D_DP/DP_white_1.png}
    \caption{
    (Left) Rocking curve for the measurement of the 3D scattered intensity around a Bragg peak.
    (Right) 3D diffraction pattern collected at SixS (SOLEIL).
    Different iso-surfaces from high (red) to low (yellow) intensity are plotted, highlighting the drop of the scattered intensity as a function of $q_{hkl}+\delta q$ where $\delta q$ is the distance from the center of the Bragg peak.
    }
    \label{fig:3DDP}
\end{figure}

The collected volume is usually in the angular space and can be transformed to the q-space to correct for any possible visual distortion linked to how the detector cuts the Ewald sphere, and to be able to compare different Bragg peaks in the reciprocal space.
However, this step is usually avoided since the intermediate interpolation process can be unaccurate, especially when the scattering intensity is low.

\subsection{Phase retrieval}\label{sec:PhaseRetrieval}

As seen in sec. \ref{sec:DataCollectionSXRD}, the diffracted intensity collected by the detector is proportional to the squared modulus of the structure factor $F(\vec{q})$ (eq. \ref{eq:ScatteredIntensity} - \ref{eq:DetectorIntensity}), the phase of the scattered x-rays is lost during the measurement.

\begin{equation}
    \label{eq:DetectorIntensity}
    I(\vec{q}) \propto |F(\vec{q})|^2
\end{equation}

Without the phase information, it is impossible to directly reconstruct the electron density distribution with the help of an inverse Fourier transform (eq. \ref{eq:IDFT}) and obtain a detailed structural model of the crystal.

\begin{equation}
    \label{eq:IDFT}
    \rho_k(\vec{r}) e^{i\vec{G}.\vec{u}_k} = IDFT [F(\vec{q} \approx \vec{G})]
\end{equation}

It has been demonstrated that when fulfilling the \textit{oversampling} condition, it is possible to retrieve the phase lost during the measurement \parencite{Shannon1949, sayre_implications_1952}.
The oversampling condition refers to the practice of sampling a signal at a higher sampling rate than the Nyquist rate, which is twice the highest frequency present in the signal, increasing the sampling rate in reciprocal space results in a denser set of data points in real space.

In the case of faceted crystals, the oversampling $\sigma$ can be derived by computing the distance between the fringes on rods in directions opposite to the facets, it was shown that the interfringe $\delta q$ observed on the rods is proportional to the size of the crystal in that direction, $t$, following the relation $\delta q = \frac{2\pi}{t}$ (sec. \ref{sec:LatticeFactor}).
For small crystals ($t<um$), the interfringe is not too small and it thus becomes possible to oversample the fringes.
The instrumental requirement for oversampling in the direction of a rod is thus simply that the detector must have at least two pixels between each fringe: $\sigma >2$.
In three dimensions, the oversampling condition becomes $\sigma>2^{1/3}$ \parencite{Miao1998, Miao2000, miao_possible_2000}.
When it comes to Fourier transforms, higher oversampling can offer several advantages such as reducing the possibility of aliasing (high-frequency components being mistakenly represented as lower frequencies due to insufficient sampling), spectral leakage (frequency components spread into neighboring frequency bins in the Fourier transform, minimized if the frequency bins become narrower).
Moreover, with more densely spaced samples, the interpolation process can more faithfully reconstruct the original signal without introducing artifacts

\cite{fienup_reconstruction_1978, Fienup1982, Fienup1986} have laid the ground for computer methods that permit phase retrieval with computer algorithms based on real-space and reciprocal space constraints, once the oversampling condition has been satisfied.
The error-reduction (ER) algorithm works by iterating between real and reciprocal space by the means of fast Fourier transforms (FFT) and inverse fast Fourier transforms (IFFT) methods.
The process starts by assigning random phases to the square root of the measured reciprocal space data to create a first guess of the structure factor: $F(\vec{q}) = \sqrt{I(\vec{q})}e^{i\Phi(\vec{q})}$.
The inverse Fourier transform of the structure factor yields the Bragg electronic density of the crystal (eq. \ref{eq:IDFT}), a real space constraint is then applied by selecting a region above a certain threshold $t$ of the modulus of the electronic density which is called the \textit{support}.
The electronic density outside the support is set to zero (eq. \ref{eq:ElectronicDensity}).
The structure factor is then computed with the Fourier transform of the electronic density (eq. \ref{eq:FcrystalBCDI7}), the reciprocal space constraint then consists in replacing the amplitude of the computed structure factor by the square root of the measured intensity.

\begin{equation}
    \label{eq:ElectronicDensity}
    \rho(\vec{r}) =
        \begin{cases}
            0  & \text{if} \; |\rho(\vec{r})| < t \\
            \rho(\vec{r}) & \text{if}  \; |\rho(\vec{r})| >= t
        \end{cases}
\end{equation}

By iterating between the real and reciprocal space, the algorithm converges towards a final solution for the electronic density (fig. \ref{fig:PRAlgo})

\begin{figure}[!htb]
   \centering
   \includegraphics[width=0.66\textwidth]{/home/david/Documents/PhD/Figures/introduction/PRAlgo.png}
   \caption{
   Iterative phase retrieval algorithm converge towards a solution by the means of real and reciprocal space constraints.
   Taken from \cite{Willmott}
   }
   \label{fig:PRAlgo}
\end{figure}

The error-metric used to follow the convergence of the algorithm is simply the mean squared error between the square root of the measured intensity and the structure factor computed from the Fourier transform of the electronic density $A(\vec{q})$ (eq. \ref{eq:MSE}).
The main drawback of this error-metric is that each voxel is given the same weight, in this assumption is not a problem when working near the Bragg peak, 4th generation synchrotron that possess a higher flux make it possible to sample the signal in region further away from the Bragg peak.
A detailed study of the impact of trying to improve the image resolution by increasing the distance from the center of the Bragg peak by combining simulations and experimental reconstructions would be of great interest to the community to understand the impact of the $(\vec{q}-\vec{G}).\vec{u}_k<<1$ approximation.

\begin{equation}
    \label{eq:MSE}
    E = \frac{\sum \big( F(\vec{q}) - \sqrt{I(\vec{q})}  \big)^2}{I(\vec{q})}
\end{equation}

More algorithms have since then been developped that try to help converge towards the final solution without being stuck in local minima such as the hybrid input-output (HIO) and relaxed averaged alternating reflections (RAAR) algorithms, by introducing a new parameter, $\beta$, which relaxes the threshold condition in real space from the ER algorithm.
Additional details can be found in \cite{Marchesini2003,Luke2005,Marchesini2007}.

Finally, a major recent improvement of the phase retrieval algorithm is the possibility to take into account the partial coherence of the beam together with the point-spread function of the detector by using the deconvolution Richardson-Lucy algorithm \parencite{Richardson1972, LucyAlgo1974, Fish1995, Clark2012}.
The point-spread function (PSF) must first be approximated with a 3D Gaussian, Lorentzian or pseudo-voigt shape depending on the expected coherence of the beam and is refined during the last steps of the phase retrieval when the support is already well determined, to avoid any divergence from simultaneously having to refine both the support and the PSF.

Moreover, new methods such as convolution neural networks (CNN) are under study and have started to show some encouraging results, but are not yet sufficiently robust, particularly in the case of strained particles \parencite{cherukara_real-time_2018,chan_rapid_2021,Wu2021}.

\subsection{Resolution}

The resolution of the reconstructed object, defined according to multiple parameters in the literature, is critical for an imaging technique.
The voxel size in real space $2\pi / \delta q_x, 2\pi / \delta q_y, 2\pi / \delta q_z$ depends on the extent of the collected volume in the reciprocal space $\delta q_x, \delta q_y, \delta q_z$.
However the voxel size would only be a good criteria to determine the resolution if the signal to noise ratio of the scattering intensity was high enough in all of the probed reciprocal space volume, so that all of the frequencies in the q-space could be effectively retrieved with the FFT algorithm.
If this is the case in simulated data, it is not with experimental data for which the noise starts to become important far away from the Bragg peak, keeping in mind that the intensity decreases as a function of $q^{-4}$ \parencite{Marchesini2003a} as a function of the distance from the Bragg peak.
The community has therefore relied so far on three methods to estimate the spatial resolution of the reconstructed electronic density.

First, the Phase Retrieval Transfer Function (PRTF) \parencite{chapman_high-resolution_2006} is the \textit{ratio} of the calculated structure factor $|F_k(\vec{q})|$ to the square-root of the measured intensity $\sqrt{I_k(\vec{q})}$ as a function of the resolution shell.
The resolution shell $k$ is defined as all voxels of position $q=\sqrt{q_x^2 + q_y^2 +q_z^2}$ that satify the condition $q \in [q_k, q_{k+1}]$.
$\delta q$ is the width of the resolution shell which divides the extent $[q_{min}, q_{max}]$ of the sampled reciprocal space in $N$ shells so that $q_{max} = q_{min} + N \times \delta q$ and $q_{k+1} - q_{k} = \delta q$.

\begin{equation}
    PRTF(q_k, q_{k+1}) = \frac{|<F_k(\vec{q})>|}{<\sqrt{I_k(\vec{q})}>}
\end{equation}

Secondly, the Fourier shell correlation (FSC) \parencite{van_heel_fourier_2005} measures the \textit{normalized cross-correlation coefficient} between the structure factor of two independant reconstructions, also as a function of the resolution shell but from two independant datasets collected with the same instrumental parameters.

\begin{equation}
    FSC(q_k, q_{k+1}) = \frac{ |\sum F_{1,k} F_{2,k}^*| }{\sqrt{ \sum F_{1,k} F_{1,k}^* \times \sum F_{2,k} F_{2,k}^*}}
\end{equation}

Both of these methods allow us to quantify the distance from the center of the Bragg peak after which one of the criteria diverges, which yields a corresponding real space resolution.
There are several limitations to these methods, first the resolution curve as a function of $|\vec{q}|$ strongly depends on the width of the bins used to create the resolution shells, the number of voxels in each bin is dependent on the distance from the center $q_{min}$ and the voxel size is usually not isotropic.
Finally and most importantly, it supposes that the resolution is isotropic in reciprocal space, which is false since the signal to noise ratio will be much higher around the rods perpendicular to the crystal's facets \parencite{cherukara_anisotropic_2018}.
Indeed, the scattered intensity along the rods is proportional to the size (\textit{i.e.} the number of unit cells) of the crystal in the direction perpendicular to the facet.
Thus, for large facets, the rod intensity will be much more intense and increase the resolution in that direction.

In real space, the spatial resolution can be quantified by differentiating line profiles of electron density amplitude across the object-air interface and fitting these with a Gaussian profile.
The average 3D spatial resolution is taken as 2$\sigma$ of the fitted Gaussian \parencite{hofmann_nanoscale_2020}.

Concerning the resolution of the retrieved atomic displacement, \cite{labat_inversion_2015} have demonstrated a displacement field accuracy of few pm along preferential directions with BCDI by comparing with simulated displacement fields.
Another method is to work in a similar way to the Fourier shell correlation by replacing the shells of structure factor as a function of $\vec{q}$ by shells of strain as a function of the strain magnitude and see if there is a strain threshold below which the correlation drops.
This method is then called the strain shell correlation \parencite{Girard2020}.

However, one must keep in mind that the upper threshold for the strain resolution exists based on the $(\vec{q}-\vec{G}).\vec{u}_k<<1$ condition, which mostly affect the quality of the phase retrieval, highly strained crystals are very difficult to reconstruct and represent the next milestone in Bragg coherent diffraction imaging.
One could imagine a future in which the atomistic computation of the scattering factor has been so optimized that phase retrieval algorithms would only be needed to provide an initial guess to the electronic density, and be replaced in the final steps by a simple fitting of the computed structure factor to the square root of the measured intensities.

\subsection{Facet analysis}

After phase retrieval, it is possible to manipulate the electronic density of the reconstructed crystal.
The surface of the crystal corresponds to a layer of voxels in the three dimensional array defined by a threshold of its maximum density.
As seen in the histogram in fig. \ref{fig:histo} that represents the number of voxels per bin of electronic density, an empiric criterion to select the contour threshold is to take the value at the lowest foot of the peak corresponding to bulk voxels, here around $0.75$, resulting in a clear distinction between surface and bulk.
Guidelines on how to select this threshold are given in \cite{Carnis2019_scientific_reports}, the objective being to select correctly the object's surface voxels.

\begin{figure}[!htb]
   \centering
   \includegraphics[width=0.66\textwidth]{/home/david/Documents/PhD/Figures/gwaihir/norm_dist.png}
   \caption{Histogram representing the distribution of the amplitude of the electronic density.
   An empirical criterion to select the contour threshold is to take the base of the peak ($\mu - 2\sigma$), here around 0.75.
   Note that most points in the array have values close to zero, for they are not part of the reconstructed object, and are here ignored.
   }
   \label{fig:histo}
\end{figure}

Once the threshold for the iso-surface is selected, it is possible to visualize a contour of the object with specialized solutions such as \textit{Paraview} (\cite{ahrens2005paraview} - fig. \ref{fig:facets_paraview}) or \textit{Gwaihir} (\cite{Simonne2022} - fig. \ref{fig:3D_object}).
Crystallographic facets can be identified on the surface of the reconstructed object when studying faceted objects with a highly coherent beam \parencite{richard_crystallographic_2018}, allowing in-depth studies of facet dependent strain and displacement.
As described in sec. \ref{sec:Catalysis}, lattice strain and displacement are key factors in fields such as heterogeneous catalysis \parencite{ulvestad_situ_2016,kim_active_2018,fernandez_situ_2019,Passos2020,Carnis2021} or electrochemistry \parencite{vicente_bragg_2021}.

Retrieving the facets can be achieved by analysing the probability distributions of the orientations of triangle normals on a mesh representation of the object \parencite{Grothausmann2012}.
This method is used in the \textit{Paraview} plugin FacetAnalyser \parencite{GrothausmannBeare2015}, and yields a list of features detailed in tab. \ref{tab:facets}.

In the case of highly resoluted measurement, another approach could be inspired from  image processing and convolutional neural networks in which 3D convolutions kernels could help identify the object surface.
By tuning the kernel size and values, it is possible to be sensitive to boundaries in specific directions.

In this example, we use the FacetAnalyser plugin (fig. \ref{fig:facets_paraview}). Note that the edges and corners, sites of particular interest for heterogeneous catalysis \parencite{Taylor1925}, are also retrieved together as voxels not belonging to any facets.
The result depends on the algorithm input parameters, such as the minimum relative facet size, the angular acceptance for the facet normals, \textit{etc.}

\begin{figure}
    \centering
    \includegraphics[width=0.66\textwidth]{/home/david/Documents/PhD/Figures/gwaihir/facets_D6.png}
    \caption{On this screenshot taken from \textit{Paraview}, the particle facets are identified by a white contour. The coloring represents the phase of the reconstructed object, proportional to the lattice displacement. This particle has a diameter of about 800 nm.}
    \label{fig:facets_paraview}
\end{figure}

\begin{table}
    \begin{center}
    \resizebox{0.95\textwidth}{!}{%
    \begin{tabular}{@{}lllll@{}}
    %\toprule
    Facet id & Facet normal & Relative facet size & Average displacement (\AA)& Average strain ($10^{-4}$)\\
    %\midrule
    1 & $(1, 1, 1)$ & 0.106 & $0.080 \pm 0.173$      & $-0.28 \pm 0.51$\\
    2 & $(\bar{1}, \bar{1}, \bar{1})$ & 0.223 & $0.052 \pm 0.26$     & $-1.17 \pm 0.64$\\
    3 & $(1, \bar{1}, 0)$ & 0.106 & $0.080 \pm 0.173$    & $-0.28 \pm 0.51$\\
    4 & $(1, 0, 0)$ & 0.096 & $0.137 \pm 0.192$ & $0.27 \pm 0.49$\\
    0 & Edges and corner & NaN & $-0.224 \pm 0.259$ & $0.20 \pm 0.86$\\
    %\bottomrule
    \end{tabular}
    }
    \end{center}
    \caption{The output of facet analysis is a list of values for each facet. The accessible features are the facet size, the average strain, the average displacement, the facet center and the facet normal. The uncertainty on the average displacement and strain corresponds to the standard deviation of  the displacement and strain distribution, respectively.}
    \label{tab:facets}
\end{table}{}

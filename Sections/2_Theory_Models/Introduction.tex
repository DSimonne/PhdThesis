\section{Introduction}

Understanding the different mechanisms at play when photons interact with matter is of crucial importance to decide how to use x-rays as a probe in material science.
Different phenomena are at the source of different techniques (tab. \ref{tab:techniques} - fig. \ref{fig:TechniquesDataExamples}), e.g. scattering brings surface x-ray diffraction (SXRD) and Bragg coherent diffraction imaging (BCDI).
Both techniques are used during this thesis, yielding complementary information about the sample atomic structure.

X-ray absorption together with the photoelectric effect explained by Einstein in \cite*{Einstein1905} are at the origin of the third technique used during this work, x-ray photoelectron spectroscopy, sensitive to the chemical environment of the adsorbates present on the sample's surface.

\begin{table}[!htb]
\centering
\resizebox{\textwidth}{!}{%
    \begin{tabular}{l|l|l|l}
        Phenomena & Technique used & Studied samples & Information \\
        \toprule
        Scattering & Bragg Coherent & Pt particles & Shape, 3D strain \\
        & Diffraction Imaging & (111), (110), (100), ... & and displacement \\
        & (BCDI) & & \\
        \midrule
        Scattering & Surface x-ray & Pt single crystals & Roughness, relaxation, atomic \\
        & Diffraction (SXRD) & (111), (100) & and crystallographic phases \\
        \midrule
        Absorption & X-ray Photoelectron & Pt single crystals & Species presence, \\
        & Spectroscopy (XPS) & (111), (100) & quantity, oxidation state \\
    \end{tabular}
    }
    \caption{
        X-ray techniques carried out in the frame of this thesis.
    }
    \label{tab:techniques}
\end{table}

\begin{figure}[!htb]
    \centering
    \includegraphics[height=0.28\textwidth]{/home/david/Documents/PhD/Presentations/Slides/PhdSlides/Figures/gwaihir/dp_pr.png}
    \includegraphics[height=0.28\textwidth]{/home/david/Documents/PhDScripts/SixS_2022_01_SXRD_Pt100/figures/Map_hkl_surf_or_2227-2283_patched.pdf}
    \includegraphics[height=0.28\textwidth]{/home/david/Documents/PhD/Presentations/Slides/PhdSlides/Figures/xps_data/transition_xps.png}
    \caption{
    Left: 2D diffraction pattern collected using BCDI at the SixS beamline (SOLEIL synchrotron)
    Middle: reciprocal space in-plane map collected using SXRD at the SixS beamline.
    Right: near ambient pressure XPS spectra measured during a transition between two atmospheres (\ce{Ar}  and \ce{O2}), collected using XPS at the B07 beamline (Diamond synchrotron).
    }
    \label{fig:TechniquesDataExamples}
\end{figure}

This chapter will discuss heterogeneous catalysis, focusing on its key principles and the associated primary reaction mechanisms.
The concept of active sites will be introduced together with the link between adsorption and surface strain, crucial for the study of catalytic reaction with x-ray diffraction techniques, intrinsically sensitive to lattice strain in crystals.
The origin of each technique from fundamentals that not only tailor their experimental design but also the information one can gain from their use will be detailed (tab \ref{tab:techniques}).

Finally, the importance of open-source, user-friendly, comprehensive, and detailed computer programs will be addressed with the example of \text{Gwaihir}, a \textit{Python} software developed during this thesis not only to bring new users to BCDI, but also to facilitate the data analysis workflow.
A brief description of the other softwares used during this thesis will also be given.
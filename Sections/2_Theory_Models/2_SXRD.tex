\section{SXRD}

\subsection{Crystal truncation rods}

Thus the diffraction intensity of the finite-sized crystal has diffuse streaks connecting all the Bragg points. The diffuse intensity far from the nodes is of order of magnitude $N^4$ compared with $N^6$ at the nodes.

Scattering that is sharp in two directions and diffuse in the third (referred to as a "rod" of scattering) must arise from a crystalline object that is localized in one dimension and extended in the other two.

We are then left with only the sixth component due to the sharply truncated surface. We will call these features "crystal truncation rods. '

We now wish to estimate the strength of the truncation rods in the Bragg geometry. We must first modify Eqs. (1) and (2) by including the x-ray coherence length, $m$ (measured in unit cells), of the experimental configuration. This broadens all the diffraction features to $\frac{1}{m}$ reciprocal units. The Bragg points then have intensity of order $N_1 N_2 N_3 m^3$, while the diffuse intensity is $N_1 N_2 m^2$

A typical penetration depth is $1 \mu m$ so $N_3 =10^3$ unit cells (perpendicular to the face). With $m \approx 100$ unit cells, this gives a relative intensity
$\frac{I(Bragg peak)}{I(truncation rod)} = N_3 m \approx 10^5$

More roughness means wider Bragg peaks and deeper valleys between the BP.

It is not clear that different detailed models of roughness could be distinguished at this level of accuracy. One central concept to all descriptions of crystal truncation rods, however, is the continuation of the crystal lattice into the roughened region (not defects, keeping symmetry).

\subsection{Reciprocal space mapping}


\lipsum


\subsection{Reflectively}


\lipsum


\subsection{Computer programs}


\lipsum


\subsubsection{RefNX}


\lipsum


\subsubsection{ROD}

\lipsum


\section{Crystal structures}\label{sec:Structures}

\subsection{Platinum}

Platinum crystallises in a face-centred cubic structure with a lattice parameter $|\vec{a}|$ equal to \qty{3.94}{\angstrom} at room temperature.
Its structure was first presented in sec. \ref{sec:ScatCrystal} to introduce the notions of crystals.
Bragg's law tells us that diffraction occurs when x-rays are scattered at discrete angles with the crystal that depend on the crystallographic planes present on the sample.
It can be interesting to study those crystallographic planes with techniques such as surface x-ray diffraction to reveal direction-dependent effects from the crystal.
In that case, the unit cell used to describe the planes differs from the FCC unit cell more commonly used in diffraction.

In fig. \ref{fig:Cubic100Hex111} is presented the unit cell of platinum, with a particular attention to the structure of the [001] (a) and [111] (b) planes.
The FCC unit cell is described by the $\vec{a}$, $\vec{b}$ and $\vec{c}$ vectors.
The $\vec{a}_{[001]}$, and $\vec{b}_{[001]}$ vectors are used to describe the structure of the Pt [001] planes since the distance between the closest Pt atoms has become closer.
The $\vec{c}$ vector is still used to describe the out-of-plane distance between successive [001] planes, that has not changed.
Likewise, the $\vec{a}_{[111]}$, and $\vec{b}_{[111]}$ vectors are used to describe the structure of the Pt [111] planes.
The shortest distance between the atoms on the [001] and [111] surfaces becomes $a/\sqrt{2}$, which is the magnitude of the \textit{in-plane} vectors used to describe the two surfaces.
The atoms on the [001] plane follow a cubic structure, whereas the atoms on the [111] plane follow a hexagonal structure, therefore the angle $\gamma$ between the two in-plane vectors is equal to \ang{90} for the [100] surface and to \ang{120} for the [111] surface.

The $\vec{c}_{[111]}$ vector is also introduced to describe the distance between three successive Pt [111] planes (fig. \ref{fig:Cubic100Hex111}).
The arrangement of the Pt atoms on the [111] surface is reproduced exactly every three consecutive planes, and is described by the ABC letters (respectively the blue, green and orange atoms in fig. \ref{fig:Cubic100Hex111} - b).
For example, a perfect arrangement of six consecutive [111] planes will be named ABCABC.
The \textit{out-of-plane} vector $\vec{c}_{[111]}$ is equal to $3a/\sqrt{3}$.
The different structures are resumed in tab. \ref{tab:PtStructures}.

\begin{figure}[!htb]
    \centering
    \includegraphics[width=0.49\textwidth]{/home/david/Documents/PhD/Figures/introduction/100.pdf}
    \includegraphics[width=0.49\textwidth]{/home/david/Documents/PhD/Figures/introduction/111.pdf}
    \caption{
        Face-centred cubic unit cell of Pt with two [001] crystallographic planes drawn in green at z=0 and z=1 (a).
        %The $\vec{a}$, $\vec{b}$, and $\vec{c}$ vectors are used to decribe the FCC cubic structure, the $\vec{a}_s$ and $\vec{b}_s$ in-plane vectors separated by \ang{90} and of magnitude \qty{2.78}{\angstrom} can also be used specifically to describe the structure of the [001] planes.
        Face-centred cubic unit cell of Pt with $[111]$ crystallographic plane drawn in green (b).
        %The arrangement of the Pt atoms on the $[111]$ crystal planes is hexagonal, which leads to a new definition of the surface unit cell with the $\vec{a}$ and $\vec{b}$ in-plane vectors separated by \ang{120} of magnitude \qty{2.78}{\angstrom}, and $\vec{c}$ perpendicular to the $[111]$ plane of magnitude \qty{6.81}{\angstrom}.
        There are three $[111]$ planes spanned by the magnitude of $\vec{c}$ (blue, red and green on the figure).
    }
    \label{fig:Cubic100Hex111}
\end{figure}

\begin{table}[!htb]
    \centering
    \begin{tabular}{@{}llllll@{}}
    \toprule
     & $\alpha$ & $\beta$ & $\gamma$ & $|\vec{a}| = |\vec{b}|$ & $|\vec{c}|$ \\
    \midrule
    FCC & \ang{90} & \ang{90} & \ang{90} & \qty{3.9242}{\angstrom} & \qty{3.9242}{\angstrom} \\
    $[100]$ & \ang{90} & \ang{90} & \ang{90} & \qty{2.7748}{\angstrom} & \qty{3.9242}{\angstrom} \\
    $[111]$ & \ang{90} & \ang{90} & \ang{120} & \qty{2.7748}{\angstrom} & \qty{6.7969}{\angstrom} \\
    \bottomrule
    \end{tabular}%
    \caption{
        Values for the lattice parameters and angles detailing the FCC unit cell of platinum as well as the two surface unit cells used for the study of the [001] and [111] crystallographic planes of platinum.
        }
    \label{tab:PtStructures}
\end{table}

\subsection{Platinum oxides}

During the exposition of pure platinum to oxygen, different platinum oxides have been reported to appear on the platinum surface \parencite{Galloni1952, MULLER1968, Seriani2006, Ackermann2007, Ellinger2008, VanSpronsen2017, Fantauzzi2017}.
In the frame of this thesis, we are mostly concerned with the hexagonal structure of $\alpha$-\ce{PtO_2} and the cubic structure of \ce{Pt_3O_4}, presented in fig. \ref{fig:PtOxides} and detailed in tab. \ref{tab:PtOxideStructures}.

\begin{figure}[!htb]
    \centering
    \includegraphics[trim=0 5.5cm 0 4.5cm, clip, width=\textwidth]{/home/david/Documents/PhD/Figures/introduction/PtOxides.pdf}
    \caption{
    $\alpha$-\ce{PtO_2}: platinum atoms situated on the unit cell corners while the two oxygen atoms are at the positions (1/3, 2/3, 1/4) and (2/3, 1/3, 3/4).
    \ce{Pt_3O_4}: platinum atoms situated on the sides of the unit cell, e.g. position (1/4, 2/2, 1) and (3/4, 1/2, 1) for the [001] facet, oxygen atoms situated on the corners of a cube of length $0.5\times a$, with an offset of (1/4, 1/4, 1/4) from the origin of the unit cell.
    }
    \label{fig:PtOxides}
\end{figure}

\begin{table}[!htb]
    \centering
    \begin{tabular}{@{}llllll@{}}
    \toprule
     & $\alpha$ & $\beta$ & $\gamma$ & $|\vec{a}| = |\vec{b}|$ & $|\vec{c}|$ \\
    \midrule
    $\alpha$-\ce{PtO_2} & \ang{90} & \ang{90} & \ang{120} & \qty{3.14}{\angstrom} (\qty{3.10}{\angstrom}) & \qty{4.34}{\angstrom} (\qty{5.735}{\angstrom}) \\
    \ce{Pt_3O_4}        & \ang{90} & \ang{90} & \ang{90} & \qty{5.65}{\angstrom} (\qty{5.59}{\angstrom}) & \qty{5.65}{\angstrom} (\qty{5.59}{\angstrom}) \\
    \bottomrule
    \end{tabular}%
    \caption{
        Experimental (computed) values reported in \cite{Seriani2006} for the lattice parameters of $\alpha$-\ce{PtO_2} and \ce{Pt_3O_4}, with angles between unit cell vectors.
        }
    \label{tab:PtOxideStructures}
\end{table}



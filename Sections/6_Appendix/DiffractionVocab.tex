\begin{table}[!htb]
\centering
\resizebox{\textwidth}{!}{%
    \begin{tabular}{@{}ll@{}}
    \toprule
    Symbol & Description \\
    \midrule
    $\vec{a},\vec{b},\vec{c}$ & Real space unit cell vectors. \\
    a, b, c & Magnitude of real space unit cell vectors.\\
    $\alpha, \beta, \gamma$ & Real space unit cell angles, respectively [$\angle (\vec{b}, \vec{c})$],  [$\angle (\vec{c}, \vec{a})$], [$\angle (\vec{a}, \vec{b})$]. \\
    $\vec{a}^*,\vec{b}^*,\vec{c}^*$ & Reciprocal space unit cell vectors. \\
    $a*, b*, c*$ & Magnitude of reciprocal space unit cell vectors. \\
    $\alpha^*, \beta^*, \gamma^*$ & Reciprocal space unit cell angles, respectively [$\angle (\vec{b}^*, \vec{c}^*)$],  [$\angle (\vec{c}^*, \vec{a}^*)$], [$\angle (\vec{a}^*, \vec{b}^*)$] \\
    (x, y, z) & Points coordinates within the unit cell, expressed in terms of a, b and c units.\\
    & z is negative when situated below the surface. \\
    {[}u v w{]} & Direction in real space. \\
    % \textless{}u v w\textgreater{} & Set of equivalent directions. \\
    (hkl) & Miller indices detailing orientation of a family of parallel crystalline planes. \\
    \{hkl\} & Equivalent families of crystallographic planes by lattice symmetry. \\
    $d_{hkl}$ & Interplanar spacing between (hkl) crystal planes (\unit{\angstrom}). \\
    (h, k, l) & Coordinates of a point in reciprocal space, also used for designing the direction of crystal truncation rods (L is then used).\\
    \end{tabular}%
}
\caption{
    Symbols used in this thesis for the description of real and reciprocal space structures.
    Subscript $_s$ used to distinguish surface from underlying structure  \parencite{Wood1964, Willmott}.
    }
\label{tab:Vocab}
\end{table}
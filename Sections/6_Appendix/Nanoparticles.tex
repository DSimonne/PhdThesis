\begin{figure}[!htb]
    \centering
    \includegraphics[width=0.8\textwidth]{/home/david/Documents/PhD/PhDScripts/SixS_2021_03_SXRD_NH3/figures/epitaxy/200.pdf}
    \caption{
        Integrated intensity in a \ang{0.1} range around the value of the (200) scattering angle, as a function of the in-plane sample angle $\omega$.
    }
    \label{fig:Epitaxy200}
\end{figure}

\begin{figure}[!htb]
    \centering
    \includegraphics[width=0.8\textwidth]{/home/david/Documents/PhD/PhDScripts/SixS_2021_03_SXRD_NH3/figures/epitaxy/111.pdf}
    \caption{
        Integrated intensity in a \ang{0.1} range around the value of the (111) scattering angle, as a function of the in-plane sample angle $\omega$.
    }
    \label{fig:Epitaxy111}
\end{figure}

\begin{figure}[!htb]
    \centering
    \includegraphics[width=\textwidth]{/home/david/Documents/PhD/PhDScripts/SixS_2021_03_SXRD_NH3/figures/rga/rga_300.pdf}
    \caption{
        Time dependent partial pressures recorded from a leak in the reactor output by a residual gas analyser (RGA) during the SXRD experiment on the non-patterned sample containing Pt nanoparticles at \qty{300}{\degreeCelsius}.
        Vertical dotted lines indicate transitions between two conditions for which the \ce{NH_3} and \ce{O_2} flow is indicated in the legend.
    }
    \label{fig:RGA300SXRDNanoparticles}
\end{figure}

\begin{figure}[!htb]
    \centering
    \includegraphics[width=\textwidth]{/home/david/Documents/PhD/PhDScripts/SixS_2021_03_SXRD_NH3/figures/rga/rga_500.pdf}
    \caption{
        Time dependent partial pressures recorded from a leak in the reactor output by a residual gas analyser (RGA) during the SXRD experiment on the non-patterned sample containing Pt nanoparticles at \qty{500}{\degreeCelsius}.
        Vertical dotted lines indicate transitions between two conditions for which the \ce{NH_3} and \ce{O_2} flow is indicated in the legend.
    }
    \label{fig:RGA500SXRDNanoparticles}
\end{figure}

\begin{figure}[!htb]
    \centering
    \includegraphics[width=\textwidth]{/home/david/Documents/PhD/PhDScripts/SixS_2021_03_SXRD_NH3/figures/rga/rga_600.pdf}
    \caption{
        Time dependent partial pressures recorded from a leak in the reactor output by a residual gas analyser (RGA) during the SXRD experiment on the non-patterned sample containing Pt nanoparticles at \qty{600}{\degreeCelsius}.
        Vertical dotted lines indicate transitions between two conditions for which the \ce{NH_3} and \ce{O_2} flow is indicated in the legend.
    }
    \label{fig:RGA600SXRDNanoparticles}
\end{figure}

\begin{figure}[!htb]
    \centering
    \includegraphics[width=\textwidth]{/home/david/Documents/PhD/PhDScripts/SixS\_2021\_01/FacetAnalyser/FacetDispEvolution.pdf}
    \caption{
        Mean value and standard deviation of the displacement ($\vec{u}_{\hat{q_z}}$) distribution as a function of the angle between the normal of each facet on the particle surface and the [111] direction.
        Upwards and downwards arrow are represented for respectively positive and negative displacement.
    }
    \label{fig:AmaterasuDisplacement}
\end{figure}

\begin{figure}[!htb]
    \centering
    \includegraphics[width=\textwidth]{/home/david/Documents/PhD/Figures/Introduction/stereographic_projection_bottom.pdf}
    \caption{
        Stereographic projection perpendicular to $[\bar{1}\bar{1}\bar{1}]$ crystallographic orientation.
        The circles describe the angle with the $[\bar{1}\bar{1}\bar{1}]$ direction from \ang{0} (centre) to \ang{90} (outer-ring).
    }
    \label{fig:StereoBottom}
\end{figure}

\begin{figure}[!htb]
    \centering
    \includegraphics[width=\textwidth]{/home/david/Documents/PhD/PhDScripts/SixS_2021_06_BCDI_NH3/figures/B-7/Fit3D.png}
    \caption{
        Sum of Bragg peak intensity perpendicular to $\vec{q}_x$ and $\vec{q}_y$ and $\vec{q}_z$, fitted by Lorentzian profiles.
    }
    \label{fig:FitB73D}
\end{figure}

\begin{figure}[!htb]
    \centering
    \includegraphics[width=\textwidth]{/home/david/Documents/PhD/PhDScripts/SixS_2021_06_BCDI_NH3/gas_analysis/figures/rga_300.pdf}
    \caption{
        Time dependent partial pressures recorded from a leak in the reactor output by a residual gas analyser (RGA) during the BCDI experiment on the patterned sample containing Pt nanoparticles at \qty{300}{\degreeCelsius}.
        Vertical dotted lines indicate transitions between two conditions for which the \ce{NH_3} and \ce{O_2} flow is indicated in the legend.
    }
    \label{fig:RGA300BCDINanoparticles}
\end{figure}

\begin{figure}[!htb]
    \centering
    \includegraphics[width=\textwidth]{/home/david/Documents/PhD/PhDScripts/SixS_2021_06_BCDI_NH3/gas_analysis/figures/rga_400.pdf}
    \caption{
        Time dependent partial pressures recorded from a leak in the reactor output by a residual gas analyser (RGA) during the BCDI experiment on the patterned sample containing Pt nanoparticles at \qty{400}{\degreeCelsius}.
        Vertical dotted lines indicate transitions between two conditions for which the \ce{NH_3} and \ce{O_2} flow is indicated in the legend.
    }
    \label{fig:RGA400BCDINanoparticles}
\end{figure}

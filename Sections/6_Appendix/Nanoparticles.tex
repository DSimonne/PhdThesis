\section{Facet data during temperature ramp}

The particle Amaterasu was studied during a temperature ramp, 3D views of the particle can be found in fig. \ref{fig:Amaterasu} in sec. \ref{sec:TempRampBCDI}.

\begin{table}[!htb]
        \centering
        \resizebox{\textwidth}{!}{%
        \begin{tabular}{lSSSSSSSS}
        \toprule
        {} & {Facet} & {$<\epsilon_{zz}>$} & {$\sigma_{\epsilon_{zz}}$} & {$\vec{u}_{\hat{q_z}}$ (\unit{\angstrom})} & {$\sigma_{\vec{u}_{\hat{q_z}}}$ (\unit{\angstrom})} & {Angle with [111] direction (\unit{\degree})} & {Absolute facet size (in voxels)} & {Relative facet size} \\
        \midrule
        0 & 0 & -0.000126 & 0.000819 & -0.935006 & 0.937775 & 0.000000 & nan & nan \\
        1 & 1 & 0.000148 & 0.000140 & -0.755531 & 0.167763 & 0.000000 & 44397.663467 & 0.044881 \\
        2 & 2 & 0.000026 & 0.000112 & -1.017636 & 0.179658 & 54.735610 & 28554.191678 & 0.028865 \\
        3 & 3 & 0.000147 & 0.000111 & -1.258746 & 0.319739 & 58.372940 & 42252.889700 & 0.042712 \\
        4 & 4 & -0.000067 & 0.000063 & -0.457895 & 0.248747 & 61.113265 & 53275.838905 & 0.053855 \\
        5 & 5 & 0.000092 & 0.000095 & -1.231178 & 0.270344 & 72.272200 & 37115.885233 & 0.037520 \\
        6 & 6 & -0.000143 & 0.001033 & -1.548569 & 0.943487 & 71.266574 & 50106.894497 & 0.050652 \\
        7 & 7 & -0.000042 & 0.000041 & -0.553660 & 0.409718 & 79.225781 & 29784.140334 & 0.030108 \\
        8 & 8 & 0.000124 & 0.000038 & -1.661702 & 0.216682 & 96.642581 & 13676.155558 & 0.013825 \\
        9 & 9 & -0.000056 & 0.000558 & -0.366485 & 0.571493 & 100.218987 & 7184.353885 & 0.007262 \\
        10 & 10 & -0.000478 & 0.001634 & -1.908002 & 0.723495 & 103.737939 & 21640.156224 & 0.021876 \\
        11 & 11 & -0.000356 & 0.000223 & -0.213895 & 0.753181 & 117.452457 & 17727.454403 & 0.017920 \\
        12 & 12 & -0.000212 & 0.000820 & -0.361291 & 1.091395 & 178.494144 & 148119.668614 & 0.149731 \\
        \bottomrule
        \end{tabular}
        }
        \caption{
        Output of the \textit{FacetAnalyzer} plugin used in \textit{Paraview} when extracting the facets on different particle surfaces for the scan 1414 at \qty{25}{\degreeCelsius} under Argon atmosphere.
        }
        \label{tab:FieldData1414}
\end{table}

\begin{figure}[!htb]
    \centering
    \includegraphics[width=\textwidth]{/home/david/Documents/PhDScripts/SixS\_2021\_01/FacetAnalyser/FacetDispEvolution.pdf}
    \caption{
        Mean value and standard deviation of the displacement ($\vec{u}_{\hat{q_z}}$) distribution as a function of the angle between the normal of each facet on the particle surface and the [111] direction.
        Upwards and downwards arrow are represented for respectively positive and negative displacement.
    }
    \label{fig:AmaterasuDisplacement}
\end{figure}


\section{Mass spectrometer data during ammonia oxidation on Pt nanoparticles}

\subsection{Non patterned sample} \label{sec:RGANanoparticlesNonPatterned}

\begin{figure}[!htb]
    \centering
    \includegraphics[width=\textwidth]{/home/david/Documents/PhDScripts/SixS_2021_03_SXRD_NH3/figures/rga/rga_300.png}
    \caption{
        Time dependent partial pressures recorded from a leak in the reactor output as detailed in sec. \ref{sec:XCAT}, at \qty{300}{\degreeCelsius}.
    }
    \label{fig:RGA300SXRDNanoparticles}
\end{figure}

\begin{figure}[!htb]
    \centering
    \includegraphics[width=\textwidth]{/home/david/Documents/PhDScripts/SixS_2021_03_SXRD_NH3/figures/rga/rga_500.png}
    \caption{
        Time dependent partial pressures recorded from a leak in the reactor output as detailed in sec. \ref{sec:XCAT}, at \qty{500}{\degreeCelsius}.
    }
    \label{fig:RGA500SXRDNanoparticles}
\end{figure}

\begin{figure}[!htb]
    \centering
    \includegraphics[width=\textwidth]{/home/david/Documents/PhDScripts/SixS_2021_03_SXRD_NH3/figures/rga/rga_600.png}
    \caption{
        Time dependent partial pressures recorded from a leak in the reactor output as detailed in sec. \ref{sec:XCAT}, at \qty{600}{\degreeCelsius}.
    }
    \label{fig:RGA600SXRDNanoparticles}
\end{figure}

\subsection{Non patterned sample} \label{sec:RGANanoparticlesPatterned}

\section{3D views of reconstructed Pt nanoparticles during the oxidation of ammonia}\label{sec:3DAmmoniaOxidation}

The following images represent the surface of the D-6 and B-7 nanoparticles measured during two ammonia oxidation cycles at 300°C and 400°C.
The surface is created by using the Marching-Cubes algorithm \parencite{Lorensen1987} in \textit{Paraview} \parencite{Ahrens2001}, which works by first assigning a scalar value to each voxel of the data (in our case the amplitude of the retrieved complex Bragg electronic density), and secondly selecting an isovalue which acts as a threshold, separating the regions of the volume that have values above it from those with values below it.
Finally, by "marching" over the volume of the array, the algorithm derives a surface representation by assigning a configuration to each point in the array that depends on whether or not the 8 neighbouring cubes have values above or below the isovalue (fig. \ref{fig:MarchingCubes}).

As mentionned in sec. \ref{sec:BCDI}, the isovalue must be carefully chosen since it depends on the amplitude of the electronic density, \textit{i.e.} theoretically on the structure factor of each voxel in real space.
Therefore, there should be a clean drop of amplitude when a voxel is not in the reconstructed object, in the case of large strain or bad measurements this is not as clear as seen in the surface of particle B-7.

The view perpendicular to the three axes of the laboratory frame is then represented, the particle is tilted due to the incoming angle $\theta$ for the measurement of the [111] Bragg peak.
Both the retrieved displacement and strain fields are represented with the same respective colorbar.
The ammonia to oxygen ration is represented on the right part of the figure.

\begin{figure}[!htb]
    \centering
    \includegraphics[width=0.66\textwidth]{/home/david/Documents/PhD/Presentations/Slides/PhdSlides/Figures/bcdi_data/MarchingCubes.png}
    \caption{
    The Marching-cubes algorithm generates triangles connecting the vertices to form the mesh surface.
    Each selected configuration has a specific arrangement of vertices that dictate how to form triangles to create a smooth surface.
    Image taken from \url{https://fr.wikipedia.org/wiki/Marching_cubes}
    }
    \label{fig:MarchingCubes}
\end{figure}

\includepdf[pages=-]{/home/david/Documents/PhD/Presentations/Slides/PhdSlides/Figures/bcdi_data/3D_B7.pdf}\label{ref:AppendixB7}
\includepdf[pages=-]{/home/david/Documents/PhD/Presentations/Slides/PhdSlides/Figures/bcdi_data/3D_D6.pdf}\label{ref:AppendixD6}
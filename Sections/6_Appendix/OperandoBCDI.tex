\section{3D views of reconstructed Pt nanoparticles during the oxidation of ammonia}\label{sec:3DAmmoniaOxidation}

The following images represent the surface of the D-6 and B-7 nanoparticles measured during two ammonia oxidation cycles at 300°C and 400°C.
The surface is created by using the Marching-Cubes algorithm \parencite{Lorensen1987} in \textit{Paraview} \parencite{Ahrens2001}, which works by first assigning a scalar value to each voxel of the data (in our case the amplitude of the retrieved complex Bragg electronic density), and secondly selecting an isovalue which acts as a threshold, separating the regions of the volume that have values above it from those with values below it.
Finally, by "marching" over the volume of the array, the algorithm derives a surface representation by assigning a configuration to each point in the array that depends on whether or not the 8 neighbouring cubes have values above or below the isovalue (fig. \ref{fig:MarchingCubes}).

As mentionned in sec. \ref{sec:BCDI}, the isovalue must be carefully chosen since it depends on the amplitude of the electronic density, \textit{i.e.} theoretically on the structure factor of each voxel in real space.
Therefore, there should be a clean drop of amplitude when a voxel is not in the reconstructed object, in the case of large strain or bad measurements this is not as clear as seen in the surface of particle B-7.

The view perpendicular to the three axes of the laboratory frame is then represented, the particle is tilted due to the incoming angle $\theta$ for the measurement of the [111] Bragg peak.
Both the retrieved displacement and strain fields are represented with the same respective colorbar.
The ammonia to oxygen ration is represented on the right part of the figure.

\begin{figure}[!htb]
    \centering
    \includegraphics[width=0.66\textwidth]{/home/david/Documents/PhD/Presentations/Slides/PhdSlides/Figures/bcdi_data/MarchingCubes.png}
    \caption{
    The Marching-cubes algorithm generates triangles connecting the vertices to form the mesh surface.
    Each selected configuration has a specific arrangement of vertices that dictate how to form triangles to create a smooth surface.
    Image taken from \url{https://fr.wikipedia.org/wiki/Marching_cubes}
    }
    \label{fig:MarchingCubes}
\end{figure}

\includepdf[pages=-]{/home/david/Documents/PhD/Presentations/Slides/PhdSlides/Figures/bcdi_data/3D_B7.pdf}\label{ref:AppendixB7}
\includepdf[pages=-]{/home/david/Documents/PhD/Presentations/Slides/PhdSlides/Figures/bcdi_data/3D_D6.pdf}\label{ref:AppendixD6}
% \begin{figure}[!htb]
%     \centering
%     \includegraphics[width=0.8\textwidth]{/home/david/Documents/PhDScripts/SixS_2023_04_SXRD_Pt111/figures/hscan_fit_768.pdf}
%     \caption{
%     }
%     \label{fig:HScan}
% \end{figure}

\begin{figure}[!htb]
    \centering
    \includegraphics[width=\textwidth]{/home/david/Documents/PhDScripts/SixS_2023_04_SXRD_Pt111/figures/map_q_481-516_patched.pdf}
    \caption{
        Reciprocal space maps collected under \qty{420}{\milli\bar} of argon and \qty{80}{\milli\bar} of oxygen at \qty{450}{\degreeCelsius}after the introduction of oxygen.
        The angle between the gray vectors is equal to \ang{42.65}.
    }
    \label{fig:481QSpace}
\end{figure}

\begin{figure}[!htb]
    \centering
    \includegraphics[width=\textwidth]{/home/david/Documents/PhDScripts/SixS_2023_04_SXRD_Pt111/figures/map_q_2064-2072_patched_hex.pdf}
    \caption{
        Reciprocal space maps collected under \qty{495}{\milli\bar} of argon and \qty{5}{\milli\bar} of oxygen at \qty{450}{\degreeCelsius}, from \qty{24}{\hour}\qty{22}{\minute} to \qty{25}{\hour}\qty{43}{\minute} after the introduction of oxygen.
        The angle between vectors of the same color is \ang{120}.
        The angle between vectors going from the center to neighbouring peaks at the same magnitude of the scattering vector (e.g. purple and gray peaks) is equal to \ang{12}.
    }
    \label{fig:2064QSpace}
\end{figure}

\begin{figure}[!htb]
    \centering
    \includegraphics[width=\textwidth]{/home/david/Documents/PhDScripts/SixS_2023_04_SXRD_Pt111/figures/l_scans_low_oxygen.pdf}
    \caption{
        Out-of plane measurements perpendicular to peaks observed in in-plane maps under \qty{5}{\milli\bar} of \dioxygen.
    }
    \label{fig:LScans05}
\end{figure}

\begin{table}[!htb]
    \centering
    \resizebox{\textwidth}{!}{%
    \begin{tabular}{@{}|l|l|lllllllllll|@{}}
        \toprule
        Structure & Interplanar & \multicolumn{11}{c|}{Oxygen pressure} \\
        \midrule
          & spacing (\unit{\angstrom}) & \multicolumn{2}{l|}{80 mbar} & \multicolumn{9}{l|}{5 mbar} \\
        \midrule
         & & \multicolumn{11}{c|}{Time since gas introduction (end of measurement)} \\
        \midrule
         & & 03h23 & 10h45 & \multicolumn{1}{|l}{00h34} & 04h03 & 08h00 & 15h57 & 22h56 & 24h08 & 25h43 & 26h36 & 27h28 \\
        \midrule % first struc split low O2
        RHSs & $3.010 \pm 0.012$ & \yes & \yes & \multicolumn{1}{|l}{\no} & \yes & \yes & \yes & \yes & \yes & \yes & \yes & \yes \\
        \midrule % first struc not split high O2
        RHSs & $2.919 \pm 0.030$ & \yes & \yes & \multicolumn{1}{|l}{\no} & \no & \no & \no & \no & \no & \yes & \no & \no \\
        \midrule % first struc not split low O2
        RHSs & $2.873 \pm 0.015$ & \no & \no & \multicolumn{1}{|l}{\yes} & \yes & \yes & \yes & \yes & \yes & \yes & \yes & \yes \\
        \midrule % first struc split low O2
        RHSs & $2.788 \pm 0.067$ & \no & \no & \multicolumn{1}{|l}{\no} & \no & \yes & \yes & \yes & \yes & \yes & \yes & \yes \\
        \midrule % hex struct
        $\alpha$-\ce{PtO_2} surface oxide & $2.688 \pm 0.022$ & \no & \yes & \multicolumn{1}{|l}{\no} & \no & \no & \no & \no & \yes & \yes & \yes & \yes \\
        \midrule % surf oxide
        RHSs & $2.312 \pm 0.040$ & \no & \yes & \multicolumn{1}{|l}{\no} & \no & \no & \yes & \yes & \yes & \yes & \yes & \yes \\
        \midrule % first struc not split high O2
        % NRHS & $1.550 \pm 0.000$ & \yes & nv & \multicolumn{1}{|l}{nv} & \no & nv & nv & nv & nv & nv & nv & nv \\
        % \midrule % first struc not split low O2
        $\alpha$-\ce{PtO_2} surface oxide & $1.528 \pm 0.000$ & \no & nv & \multicolumn{1}{|l}{nv} & \yes & nv & nv & nv & nv & nv & nv & nv \\
        % NRHS & $1.35 \pm 0.000$ & \no & nv & \multicolumn{1}{|l}{nv} & \yes & nv & nv & nv & nv & nv & nv & nv \\
        \bottomrule
    \end{tabular}
    }
    \caption{
        Interplanar spacing values observed in reciprocal maps for different oxygen pressure and exposition times (\yes).
        Non observed peaks are in red (\no), non visible peaks are represented by the acronym \text{nv}.
        The errors on the peak position are computed by considering the positions of similar peaks in q-space.
        RHSs stands for hexagonal rotated structures.
    }
    \label{tab:InterplanarSpacingsPt111Oxygen}
\end{table}

\begin{figure}[!htb]
    \centering
    \includegraphics[width=\textwidth]{/home/david/Documents/PhDScripts/SixS_2023_04_SXRD_Pt111/figures/T450_1.pdf}
    \caption{
        Time dependent partial pressures recorded from a leak in the reactor output by a residual gas analyser (RGA) during the SXRD experiment on the Pt(111) single crystal at \qty{450}{\degreeCelsius}.
        Vertical dotted lines indicate transitions between two conditions for which the \ce{NH_3} and \ce{O_2} flow is indicated in the legend.
        The RGA electron multiplier is off for all the masses.
    }
    \label{fig:RGA450Pt111_1}
\end{figure}

\begin{figure}[!htb]
    \centering
    \includegraphics[width=\textwidth]{/home/david/Documents/PhDScripts/SixS_2023_04_SXRD_Pt111/figures/T450_2.pdf}
    \caption{
        Time dependent partial pressures recorded from a leak in the reactor output by a residual gas analyser (RGA) during the SXRD experiment on the Pt(111) single crystal at \qty{450}{\degreeCelsius}.
        Vertical dotted lines indicate transitions between two conditions for which the \ce{NH_3} and \ce{O_2} flow is indicated in the legend.
        The RGA electron multiplier is on for all the masses besides oxygen and argon.
    }
    \label{fig:RGA450Pt111_2}
\end{figure}

\begin{figure}[!htb]
    \centering
    \includegraphics[width=\textwidth]{/home/david/Documents/PhDScripts/SixS_2023_04_SXRD_Pt111/figures/T450_3.pdf}
    \caption{
        Time dependent partial pressures recorded from a leak in the reactor output by a residual gas analyser (RGA) during the SXRD experiment on the Pt(111) single crystal at \qty{450}{\degreeCelsius}.
        Vertical dotted lines indicate transitions between two conditions for which the \ce{NH_3} and \ce{O_2} flow is indicated in the legend.
        The RGA electron multiplier is on for all the masses besides argon.
    }
    \label{fig:RGA450Pt111_3}
\end{figure}

\begin{figure}[!htb]
    \centering
    \includegraphics[width=\textwidth]{/home/david/Documents/PhDScripts/SixS_2023_04_SXRD_Pt111/figures/T450_4.pdf}
    \caption{
        Time dependent partial pressures recorded from a leak in the reactor output by a residual gas analyser (RGA) during the SXRD experiment on the Pt(111) single crystal at \qty{450}{\degreeCelsius}.
        Vertical dotted lines indicate transitions between two conditions for which the \ce{NH_3} and \ce{O_2} flow is indicated in the legend.
        The RGA electron multiplier is on for all the masses besides argon.
    }
    \label{fig:RGA450Pt111_4}
\end{figure}
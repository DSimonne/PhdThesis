\section{Surface x-ray diffraction on a Pt 111 single crystal} \label{sec:SXRD111}

\subsection{XPS results} \label{sec:XPS111}

Since the XPS measurements were performed at different total pressures, the raw data had first to be reduced in order to analyse the different spectra in a systematic way.
The adopted workflow for the analysis of XPS data is to first align the recorded spectra on the Fermi edge that corresponds to the kinetic energy of the first electron that escapes the sample.
By doing so, one can be confident that any shift in the peak positions is due to chemical changes, such as the oxidation state of the sample, and not to charging effects of the sample (cite).

Secondly, to be able to quantify and compare the evolution of the peak intensity, one must normalize the intensity of the detected electron beam since the electron mean free path depends on the pressure in the reaction chamber \parencite{Willmott}.
The range of kinetic energy just before the absorption edge of Pt 4f was chosen since it had the best signal to noise ratio and does not depend on any experimental parameter besides the pressure.

Finally, for the peaks that showed a good signal to noise ratio, the fitting of the peak shape was realised thanks to the \textit{lmfit} \parencite{Newville2016} package by the means of the Doniach-equation which is the best approximation of the asymmetric peak shape resulting from the convolution of the analyser function and the photoelectron process in metals \parencite{Doniach_1970}.

\subsection{Empirical Analysis}

\textcolor{red}{This chapter covers three areas: analysis of the data; discussion of the results of the analysis; and how your findings relate to the literature. The analysis of the data can be discussed here but the details of any analysis, such as statistical calculations, should be shown in the appendices. You should present any discussion clearly and logically and it should be relevant to your research questions/hypotheses or aims and objectives. Insert any tables or figures that you decide are important in a relevant part of the text not in the appendices, and discuss them fully. Make sure that you relate the findings of your primary research to your literature review. You can do this by comparison: discussing similarities and particularly differences. If you think your findings have confirmed some literature findings, say so and say why. If you think your findings are at variance with the literature, say so and say why.}


PtO2 plays a role \cite{McCabe1986, HANNEVOLD2005}
%%%% Modèle proposé par frederic.mazaleyrat@ens-paris-saclay.fr %%%%
%%%% 3 avril 2021 %%%%

\documentclass[french,12pt,a4paper]{book}
\usepackage[utf8]{inputenc}
\usepackage[T1]{fontenc}
\usepackage[french]{babel}
\usepackage{amsmath}
\usepackage{amsfonts}
\usepackage{fancyhdr}
\usepackage{amssymb}
\usepackage{color} % où xcolor selon l'installation
\definecolor{Prune}{RGB}{99,0,60}
\usepackage{mdframed}
\usepackage{multirow} %% Pour mettre un texte sur plusieurs rangées
\usepackage{multicol} %% Pour mettre un texte sur plusieurs colonnes
\usepackage{scrextend} %Forcer la 4eme  de couverture en page pair
\usepackage{tikz}
\usepackage{graphicx}
\usepackage[absolute]{textpos} 
\usepackage{colortbl}
\usepackage{array}
\usepackage{geometry}
\usepackage{hyperref}

\usepackage{lipsum} % à retirer!!!
\begin{document}

\begin{titlepage}


%\thispagestyle{empty}

\newgeometry{left=6cm,bottom=2cm, top=1cm, right=1cm}

\tikz[remember picture,overlay] \node[opacity=1,inner sep=0pt] at (-13mm,-135mm){\includegraphics{Bandeau_UPaS.pdf}};

% fonte sans empattement pour la page de titre
\fontfamily{fvs}\fontseries{m}\selectfont


%*****************************************************
%******** NUMÉRO D'ORDRE DE LA THÈSE À COMPLÉTER *****
%******** POUR LE SECOND DÉPOT                   *****
%*****************************************************

\color{white}

\begin{picture}(0,0)

\put(-110,-743){\rotatebox{90}{NNT: 2020UPASA000}}
\end{picture}
 
%*****************************************************
%**  LOGO  ÉTABLISSEMENT PARTENAIRE SI COTUTELLE
%**  CHANGER L'IMAGE PAR DÉFAUT **
%*****************************************************
\vspace{-10mm} % à ajuster en fonction de la hauteur du logo
\flushright \includegraphics[scale=1]{logo2.png}




%*****************************************************
%******************** TITRE **************************
%*****************************************************
\flushright
\vspace{10mm} % à régler éventuellement
\color{Prune}
\fontfamily{cmss}\fontseries{m}\fontsize{22}{26}\selectfont
  Propriétés catalytiques à l'échelle nanométrique sondées par diffraction des rayons X de surface et imagerie de diffraction cohérente.

\normalsize
\color{black}
\Large{\textit{Catalytic properties at the nanoscale probed by surface x-ray diffraction and coherent diffraction imaging}}
%*****************************************************

%\fontfamily{fvs}\fontseries{m}\fontsize{8}{12}\selectfont

\vspace{1.5cm}
\normalsize

\textbf{Thèse de doctorat de l'Université Paris-Saclay}


\vspace{15mm}

École doctorale n$^{\circ}$ 000, dénomination et sigle\\
\small Spécialité de doctorat: voir annexe\\
\footnotesize Unité de recherche: voir annexe\\
\footnotesize Référent: : voir annexe
\vspace{15mm}

\textbf{Thèse présentée et soutenue à .....,\\ le .... 202X, par}\\
\bigskip
\Large {\color{Prune} \textbf{David SIMONNE}}


%************************************
\vspace{\fill} % ALIGNER LE TABLEAU EN BAS DE PAGE
%************************************

%\flushleft \small \textbf{Composition du jury:}
\bigskip

\flushleft

\scriptsize
\begin{tabular}{|p{7cm}l}
\arrayrulecolor{Prune}
{\footnotesize \textbf{Composition du jury}}\\
& \\
\textbf{Prénom Nom} &   Président/e\\ 
Titre, Affiliation & \\
\textbf{Prénom Nom} &  Rapportrice \\ 
Titre, Affiliation   &   \\ 
\textbf{Prénom Nom} &  Rapporteur \\ 
Titre, Affiliation  &   \\ 
\textbf{Prénom Nom} &  Examinatrice \\ 
Titre, Affiliation   &   \\ 
\textbf{Prénom Nom} &  Examinateur \\ 
Titre, Affiliation   &   \\ 
\textbf{Prénom Nom} &  Examinateur \\ 
Titre, Affiliationt   &   \\ 

\end{tabular} 

\medskip
\begin{tabular}{|p{7cm}l}
\arrayrulecolor{Prune}
{\footnotesize \textbf{Direction de la thèse}}\\
& \\
\textbf{Alessandro Coati} &   Directeur\\
Dr., Affiliation & \\
\textbf{Andrea Resta} &   Codirecteur\\
Dr., Affiliation  &   \\
\textbf{Marie-Ingrid Richard} &   Coencadrante\\
Dr., Affiliation  &   \\


\end{tabular} 


\end{titlepage}

%%%%%%%%%%%%%%%%%%%%%%%%%%%%%%%%%%%%%%%%%%%%%%%%%%%%%%%%%%%%%%
% insérez ici les fichiers des chapitre

% \include{chapitre1}

\chapter{Quelques Conseils}
\section{Page de garde}
Ne changez rien dans le formatage, la taille des fontes, la couleur, l'alignement etc.

Les annexes, liste des ED, liste des référents etc, sont disponibles dans le modèle .docx sur le site de l'Université:

\url{https://www.universite-paris-saclay.fr/research/textes-de-reference/documents-de-reference-relatifs-la-soutenance-de-la-these#modelcover}

\section{Quatrième de couverture}

Les logos des ED sont à extraire du modèle .docx, même \verb!url!.

Ne pas réduire la police des résumés. Si ça ne rentre pas dans le cadre, résumez plus!

%%%%%%%%%%%%%%%%%%%%%%%%%%%%%%%%%%%%%%%%%%%%%%%%%%%%%%%%%%%%%%%
% 4eme de couverture
\ifthispageodd{\newpage\thispagestyle{empty}\null\newpage}{}
\thispagestyle{empty}
\newgeometry{top=1.5cm, bottom=1.25cm, left=2cm, right=2cm}
\fontfamily{rm}\selectfont

\lhead{}
\rhead{}
\rfoot{}
\cfoot{}
\lfoot{}

\noindent 
%*****************************************************
%***** LOGO DE L'ED À CHANGER IMPÉRATIVEMENT *********
%*****************************************************
\includegraphics[height=2.45cm]{EOBE}
\vspace{1cm}
%*****************************************************
\fontfamily{cmss}\fontseries{m}\selectfont

\small

\begin{mdframed}[linecolor=Prune,linewidth=1]

\textbf{Titre:} Titre de la thèse en français


\noindent \textbf{Mots clés:} Quelques mots-clé

\vspace{-.5cm}
\begin{multicols}{2}
\noindent \textbf{Résumé:} Le projet de recherche de doctorat fait partie d'un projet de cinq ans financé par l'ERC appelé CARINE (Coherent diffrAction foR a Look Inside NanostructurEs towards atomic resolution: catalysis and interfaces) pour développer et appliquer de nouvelles capacités d'imagerie par diffraction cohérente (CDI). Le principal objectif du projet est d'imager des nanostructures pour sonder les conditions in situ et operando ; mesurer la structure à l'échelle nanométrique et révéler également les effets de masse, de surface et d'interface ainsi que les défauts. Viser à terme à comprendre les phénomènes structurels importants pour les nanocatalyseurs et les relier à leur activité, sélectivité, réutilisabilité et durabilité. En complément des études cohérentes aux rayons X sur des particules individuelles, des techniques étudiants la moyenne des ensembles comme la diffraction des rayons X à incidence rasante seront employées pour voir si l'évolution des formes d'ensemble sont similaires à celles des nanoparticules uniques et sondent s'il y a une déconnexion entre les particules uniques et l'activité catalytique sur des billions de particules. Les catalyseurs jouent un rôle clé dans environ 90% des processus chimiques industriels. La catalyse des nanomatériaux est apparue comme un moyen efficace d'exposer une surface plus élevée et d'accélérer les processus catalytiques en maximisant le rapport surface-volume. Le développement d'une catalyse hétérogène avec une sélectivité ciblant les 100% est un défi constant ainsi que la compréhension de la durabilité et du vieillissement du catalyseur lui-même. Dans un procédé réel (réacteur d'usine de catalyseur, échappement de voiture, pile à combustible), l'évolution de la forme et de la déformation des nanoparticules catalytiques dans des conditions de réaction contribue au vieillissement du catalyseur et a un impact sur la durée de vie du dispositif. Cependant, le processus catalytique et les changements structurels associés restent encore mal compris. Comprendre comment la structure du catalyseur est affectée par la couche adsorbée dans des conditions de réaction est donc de la plus haute importance pour formuler des relations de performance de structure de catalyseur qui guident la conception de meilleurs catalyseurs.
\end{multicols}

\end{mdframed}

\begin{mdframed}[linecolor=Prune,linewidth=1]

\textbf{Title:} Thesis title in English

\noindent \textbf{Keywords:} Some keywords

\begin{multicols}{2}
\noindent \textbf{Abstract:} The PhD research project is part of a five-year ERC-funded project called CARINE (Coherent diffrAction foR a Look Inside NanostructurEs towards atomic resolution: catalysis and interfaces) to develop and apply new coherent diffraction imaging (CDI) capabilities. The main objective of the project is to image nanostructures to probe in situ and operando conditions; measure the structure at nanoscale and to reveal bulk, surface and interface effects, as well as defects. Ultimately aiming to understand the structural phenomena important for the working nanocatalysts and link them to their activity, selectivity, reusability and sustainability. In complement to coherent x-ray studies on individual particles, ensemble-averaging techniques like grazing incidence x-ray diffraction will be employed to see if the evolution of ensemble shapes is similar to the one of single nanoparticles and probe if there is a disconnect between single particles and the catalytic activity over trillions of particles. Catalysts play a key role in approximatively 90% of industrial chemical processes. Catalysis of nanomaterials has emerged as an efficient way to expose higher surface area and accelerate catalytic processes by maximizing the surface-volume ratio. The development of heterogeneous catalysis with selectivity targeting the 100% is a constant challenge as well as understanding the durability and ageing of the catalyst itself. In a real process (catalyst plant reactor, car exhaust, fuel cell) the shape and strain evolution of catalytic nanoparticles under reaction conditions contributes to the ageing of the catalyst and impact the lifetime of the device. However, the catalytic process and the associated structural changes remain poorly understood. Understanding how catalyst structure is affected by the adsorbed layer under reaction conditions is therefore of utmost importance to formulate catalyst structure performance relations that guide the design of better catalysts.
\end{multicols}
\end{mdframed}

%************************************
\vspace{\fill} % ALIGNER EN BAS DE PAGE
%************************************

\noindent
\color{Prune} \footnotesize Maison du doctorat de Université Paris-Saclay\\
2$^{\mathrm{e}}$ étage, aile ouest, École normale supérieure Paris-Saclay\\
4 avenue des Sciencs\\
91190 Gif-sur-Yvette, France

\end{document}